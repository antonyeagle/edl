%!TEX root = edl.tex

\section{Derivations and Semantic Arguments}

At the beginning of the previous chapter, we mentioned a number of considerations that motivate the development of formal object language derivation systems. Now that we have an example of such a system for \lone, we may reflect on how it compares to evaluating arguments by directly discussing \lone-structures.

One thing to note is that the tableau system can be considerably more efficient. Consider this argument: $$(P \wedge (Q_{1} \wedge (Q_{2} \wedge (Q_{3} \wedge (Q_{4} \wedge Q_{5})))) \vdash (P \vee (R_{1} \vee (R_{2} \vee (R_{3} \vee (R_{4} \vee R_{5})))).$$  We can see the tableau derivation demonstrating the correctness of this claim in \autoref{fig:short} closes vey quickly. Even if we finish the tableau by applying the conjunction and negated disjunction rules repeatedly, the whole thing will be fewer than 30 lines.

\begin{figure}[t]
	\centering 
	{
\leaf{$P$\\ $(Q_{1} \wedge (Q_{2} \wedge (Q_{3} \wedge (Q_{4} \wedge Q_{5})))$\\ $¬P$ \\ $¬(R_{1} \vee (R_{2} \vee (R_{3} \vee (R_{4} \vee R_{5})))$\\ $\otimes$}
\branch{1}{$P \wedge (Q_{1} \wedge (Q_{2} \wedge (Q_{3} \wedge (Q_{4} \wedge Q_{5}))))$\\ $¬(P \vee (R_{1} \vee (R_{2} \vee (R_{3} \vee (R_{4} \vee R_{5}))))$}
\qobitree	}\caption{A brief tableau. \label{fig:short}}
\end{figure}

It is easy to see – by giving a semantic argument that basically parallels this tableau derivation – that this argument is a valid entailment. But demonstrating this via truth tables would involve a truth table of $2^{11} = 2048$ rows, which is prohibitively unwieldy. The tableau procedure is no less mechanical than the truth table procedure, but is considerably more efficient in this case.

On the other hand, consider an invalid argument like $$(P \vee (¬P \vee (¬¬P \vee (¬¬¬P \vee ¬¬¬¬P)))) \nvDash (P \wedge (¬P \wedge (¬¬P \wedge (¬¬¬P \wedge ¬¬¬¬P)))).$$ The truth table is two lines, on each of which the premise is true and the conclusion false. But the tableau, with its many uses of the disjunction and negated conjunction rules, branches extensively and is much less convenient to work with. 

This pattern is fairly representative. To show an argument invalid, we need to demonstrate the existence of just one structure in which the premises are true and the conclusion false. But checking invalidity by the construction of an attempted derivation involves the construction of a finished open tableau. In bad cases, this construction will take a long time to complete. To show an argument valid involves demonstrating the non-existence of a counterexample structure, which may require consideration of a great many structures. But a direct derivation of the conclusion from the premises may be more efficient.

We cannot yet recommend that we use semantic techniques to demonstrate invalidity, and formal derivations to demonstrate validity. This is because we do not yet know whether \emph{every} valid argument has a corresponding derivation; nor do we know whether \emph{every} derivation corresponds to a valid argument. It turns out that every valid argument is derivable, and every correctly constructed derivation can be associated with a valid argument. Both of these facts will be proved later in this chapter (\autoref{sec:tsoundcomp}), when we discuss the soundness and completeness of the tableau derivation system with respect to the semantics of \lone. 


\section{Transforming Derivations} % (fold)

Sometimes when we have a tableau derivation, we can use purely formal manipulations on the tree to construct other derivations. While trees can and do have other trees occuring within them (Def. \ref{def:occurswithin}), the trees which occur within a given tableau will always not meet the conditions for being a tableau themselves. But often simple modifications will enable us to convert one tableau into another, taking sub-trees of a tableau and re-using them in a new tableau. 

\paragraph{Structural Rules} We proved some structural rules for $\vDash$ (\autoref{twostruct}). We can show that analogous rules apply to $\vdash$. I leave the demonstration of weakening for an exercise. \begin{theorem}[$\vdash$ Permutation]
	$\Gamma, \psi, \chi, \Delta \vdash \phi$ iff\, $\Gamma, \chi, \psi, \Delta \vdash \phi$. \begin{proof}
	Immediate consequence of \autoref{thmorder}.
	\end{proof}
\end{theorem}
\begin{theorem}[$\vdash$ Contraction]
$\Gamma, \psi, \psi \vdash \phi$ iff\, $\Gamma, \psi \vdash \phi$. \begin{proof}
	Any tableau generated by $\Gamma \cup \{\psi,\psi\} \cup \{¬\phi\}$ is generated by $\Gamma\cup \{\psi\} \cup \{¬\phi\}$, and vice versa, because those generating sets are identical (Def. \ref{def:gentabl}).
\end{proof}
\end{theorem}


\paragraph{Syntactic Deduction} Just as we proved the deduction theorem (\autoref{ded}) for the semantic turnstile, we can prove a syntactic deduction theorem for the syntactic turnstile.
\begin{theorem}[Syntactic Deduction]\label{thm:sdt} $\Gamma, \phi\vdash \psi$ iff\, $\Gamma \vdash \phi \to \psi$. \begin{proof} {\em If:} Assume $\Gamma \vdash \phi \to \psi$. Then every finished tableau $\mathbf{T}$ which is generated by $\Gamma \cup\{\neg(\phi \to \psi)\}$ is closed. Take one such tableau, where the first tableau rule applied was the negated conditional rule (\autoref{fig:stableaux}\subref{sf:nc}), so that each branch of the tableau begins $\langle \gamma_{1},\ldots,\gamma_{n}, \neg (\phi \to \psi), \phi, \neg \psi\rangle$. We have already dealt with $\neg (\phi \to \psi)$, so the sentences which close every branch on this tableau derive from $\Gamma$ or from $\phi$ or $\neg \psi$. Hence, we can construct a closed tableau where every branch begins $\langle \gamma_{1},\ldots,\gamma_{n}, \phi, \neg \psi\rangle$; hence the set $\Gamma\cup \{\phi, \neg \psi\}$ generates a closed tableau, hence $\Gamma, \phi \vdash \psi$. The only interesting case is if the original tableau had a branch which closed because of the presence of $(\phi \to \psi)$ on it. But then an application of the branching conditional rule to that node gives one branch with $\neg\phi$ occurring on it, which closes since $\phi$ occurs in the trunk;  and another branch with $\psi$ occurring on it, which closes since $\neg \psi$ occurs in the trunk. 

{\em Only if:} We assume that $\Gamma, \phi \vdash \psi$. Then there is a closed finished tableaux that is generated by $\Gamma\cup \{\phi, \neg \psi\}$. If we modify this tableau by inserting $\neg(\phi \to \psi)$ above $\phi$ on every branch, the modified tree is closed also. Since $\phi$ and $\neg \psi$ can come from $\neg (\phi \to \psi)$ (\autoref{fig:stableaux}\subref{sf:nc}), this modified tree in fact satisfies the conditions for being a tableau. Since the original tableau is finished, and the modified tableau contains the results of applying tableau rules to the only new node, this is a finished tableau generated by $\{\Gamma, \neg(\phi \to \psi)\}$, which means that $\Gamma \vdash \phi \to \psi$.
\end{proof}\end{theorem}

Now I'll prove some lemmas preliminary to proving a historically important theorem, Cut (\autoref{cut}). All of the following four lemmas involve showing that some tableau derivations can be converted into other tableau derivations, in such a way that the syntactic derivability mirrors what is intuitively provable in virtue of the meaning of the  sentences involved. 

\begin{lemma} \label{switch}
	$\Gamma \vdash \phi$ iff\, $\Gamma, \neg\phi \vdash$.
\end{lemma}
\begin{proof}
	$\Gamma \vdash \phi$ iff there is a finished closed tableau generated by $\Gamma \cup\{\neg\phi\}$ iff $\Gamma\cup\{\neg\phi\}$ is inconsistent (by Definition~\ref{defcontab}).
\end{proof}
\begin{lemma} \label{dne}
	If\, $\Gamma, \neg\neg\phi \vdash$, then\, $\Gamma,\phi \vdash$.
\end{lemma}
\begin{proof}
	If $\Gamma, \neg\neg\phi \vdash$, there is a finished closed tableau generated by $\Gamma \cup \{\neg\neg\phi\}$. Take such a tableau in which the first rule applied to the generating set was the $\neg\neg$ rule (\autoref{fig:stableaux}\subref{sf:dn}). Remove the node containing $\neg\neg\phi$: the tableau will still be closed. Why? Either $\Gamma$ itself was already closed; or the tableau closed because of some application of a tableau rule to $\phi$ (the result of applying the double negation rule to $\neg\neg\phi$); or the tableau closed because $\neg\neg\phi$ and its negation occurred on the tableau. In the first two cases, it is obvious that removing $\neg\neg\phi$ made no difference to whether the tableau is closed. So consider the third case. There are two subcases: either $\neg\neg\neg\phi$ occurs on the tableau, or $\neg\phi$ does (both are negations of $\neg\neg\phi$). If the latter, the tableau is closed because both $\phi$ and $\neg\phi$ occur. If the former, since the tableau is finished, some application of the double negation rule was made to $\neg\neg\neg\phi$ at some stage, so that $\neg\phi$ occurs as well as $\phi$, again closing the relevant branch.
\end{proof}


\begin{lemma} \label{negcon}
	If\, $\Gamma \vdash \neg(\phi \wedge\psi)$ then\, $\Gamma, \phi \vdash \neg\psi$.
\end{lemma}
\begin{proof}
	If $\Gamma \vdash \neg(\phi \wedge\psi)$, then every finished tableau generated by $\Gamma \cup \{¬¬(\phi\wedge\psi)\}$ is closed, and by Lemma \ref{dne}, so too is every finished tableau generated by $\Gamma \cup \{\phi\wedge\psi\}$. Consider such a tableau in which the first rule applied was the conjunction rule (\autoref{fig:stableaux}\subref{sf:and}); this is also a finished closed tableau generated by $\Gamma \cup \{\phi\wedge\psi,\phi,\psi\}$. Remove the node containing $\phi\wedge\psi$, and the tableau remains closed. Either the tableau is closed because of the results of applying the tableau rules to $\phi$ and $\psi$; or some branch of the tableau is closed because it contains $\neg(\phi\wedge\psi)$. But, because the tableau is finished, that branch also contains either $\neg \phi$ or $\neg\psi$; either way, it will then close because of the presence of $\phi$ and $\psi$ earlier.
\end{proof}

\begin{lemma} \label{con}
	If\, $\Gamma \vdash (\phi\wedge\psi)$, then\, $\Gamma \vdash \phi$ and\, $\Gamma \vdash \psi$.
\end{lemma}
\begin{proof}
	If $\Gamma \vdash (\phi\wedge\psi)$, there is a finished closed tableau generated by $\Gamma \cup\{\neg(\phi\wedge\psi)\}$. Suppose the first rule applied is the negated conjunction rule (\autoref{fig:stableaux}\subref{sf:nand}), giving two branches, both beginning with the members of $\Gamma$ and $\neg(\phi\wedge\psi)$, but one having $\neg\phi$ as a leaf, and the other with $\neg\psi$ as a leaf. Because the tableau is closed, every branch beginning like those two branches is closed. Suppose we delete the node containing $\neg(\phi\wedge\psi)$, but keep $\neg\phi$. We will also therefore prune off the branch with $\neg\psi$ occurring on it. But the pruned tableau remains closed, either because the tableau branches close due to the application of some tableau rule to $\neg\phi$, or because $\phi\wedge\psi$ occurs somewhere on the tableau, but since it is finished, so too then does $\phi$, which makes it closed. The same goes, \emph{mutatis mutandis}, for the other pruning.
\end{proof} 
Now we are in a position to prove the Cut theorem (or at least one theorem that commonly goes under that name.)
\begin{theorem}[Cut] \label{cut}
	If $\Gamma \vdash \phi$ and $\Gamma \vdash \neg\phi$ then $\Gamma \vdash$.
\end{theorem}
\begin{proof}
	We prove by (strong) induction on the complexity of $\phi$. The base case: suppose $\phi$ is a sentence letter. Then there is a finished closed tableau generated by $\Gamma \cup \{\neg\phi\}$ and a finished closed tableau generated by $\Gamma \cup \{\phi\}$. Suppose there is a finished open tableau $\mathbf{T}$ generated by $\Gamma$ alone. Inserting $\phi$ at the base of the trunk of $\mathbf{T}$ must render it closed, which means that each open branch must have contained $¬\phi$. But since each tableau generated by $\Gamma\cup\{¬\phi\}$ is also closed, those branches cannot have been open. So any tableaux generated by $\Gamma$ must be closed.
	
	The induction step: $\phi$ is complex, and the induction hypothesis is that for each less complex constituent $\psi$ of $\phi$, and any set $\Delta$, if $\Delta\vdash\psi$ and $\Delta\vdash\neg\psi$, then $\Delta \vdash$. I consider two cases explicitly, those where $\phi$ is either a negation or a conjunction: \begin{enumerate}
		\item $\phi = \neg\psi$. We assume $\Gamma \vdash \phi$ and $\Gamma \vdash \neg\phi$, i.e., by Lemma \ref{switch}, $\Gamma,\phi\vdash$ and $\Gamma,\neg\phi\vdash$. Because $\phi=\neg\psi$, we also have $\Gamma,\neg\psi\vdash$ and $\Gamma,\neg\neg\psi\vdash$. By Lemma \ref{dne}, $\Gamma,\psi \vdash$. By the induction hypothesis applied to the simpler constituent $\psi$, $\Gamma\vdash.$
	
\item $\phi = (\psi\wedge\chi)$. Again, we assume (a) $\Gamma \vdash (\psi\wedge\chi)$ and (b) $\Gamma \vdash \neg(\psi\wedge\chi)$. By Lemma \ref{con} applied to (a), we have $\Gamma\vdash\chi$. But if a tableau generated by $\Gamma\cup\{\neg\chi\}$ closes, so does one generated by $\Gamma\cup\{\psi,\neg\chi\}$, so $\Gamma,\psi\vdash\chi$. By Lemma \ref{negcon} applied to (b), we have $\Gamma,\psi\vdash\neg\chi$. By the induction hypothesis $\Gamma,\psi\vdash$. But from Lemma \ref{con} applied to (a), we also have $\Gamma\vdash\psi$; by Lemma \ref{switch}, $\Gamma,\neg\psi\vdash$; and by the induction hypothesis, $\Gamma\vdash$. 
\item Other cases left for exercises; you will also need to establish lemmas parallelling Lemmas \ref{negcon}–\ref{con} for disjunction, conditional, and biconditional. 
	\end{enumerate} 
\end{proof}

What does the Cut Theorem show? Basically this: if you can get a tableau generated by $\Gamma$ and $\neg\phi$ to close, and you can get a tableau generated by $\Gamma$ and $\phi$ to close, then you could already have got a tableau generated by $\Gamma$ to close: whatever appeal you made respectively to $\phi$ and $\neg\phi$, it was inessential that they appeared on the trunk. (Either they weren't used, or they were derivable from something else already in $\Gamma$.) The interest of Cut in derivation systems is in showing that our derivations can all be `direct' in a sense: if we can derive something with the assistance of either $\phi$ or $¬\phi$, then we can derive it without detouring through that assistance.\footnote{This is more important when we consider natural deduction derivations in Chapters \ref{c:l1nd}–\ref{c:ndmeta}, where the Cut theorem is used to show that in principle all derivations can have a particularly elegant form where sentences are first broken down into simpler constituents and then built back up to establish the desired conclusions.}
Here is another way of making the same point. \begin{theorem}[Cut Again]\label{cutagain}
	If\, $\Gamma, \phi\vdash \psi$ and\, $\Gamma\vdash \phi$ then\, $\Gamma\vdash \psi$. \begin{proof}
		Suppose $\Gamma,\phi\vdash \psi$; then $\Gamma\cup\{¬\psi\}, \phi\vdash$ by Lemma \ref{switch} (and the fact that what flanks the turnstile are sets in which order doesn't matter). Suppose $\Gamma\vdash \phi$; then $\Gamma,¬ \phi\vdash$, and $\Gamma\cup\{¬\psi\},¬ \phi\vdash$ (by Weakening for $\vdash$, proved in exercises). By \autoref{cut}, $\Gamma\cup\{¬ \psi\}\vdash$, i.e., $\Gamma\vdash \psi$.
	\end{proof}
\end{theorem} In fact, this theorem is equivalent to \autoref{cut}. Note the parallel with \autoref{semanticcut}.

\begin{theorem}[Transitivity of Derivability]
If $\Gamma \vdash \phi$ and $\phi \vdash \psi$, then $\Gamma \vdash \psi$.
\end{theorem}
\begin{proof}
Assume $\phi\vdash \psi$. Then by weakening $\Gamma,\phi \vdash \psi$. Assume $\Gamma\vdash \phi$. Then by \autoref{cutagain}, $\Gamma\vdash \psi$.
\end{proof}
This shows is that any two-step derivation has a short-cut: if we can derive $\psi$ from something we can derive from $\Gamma$, well, we could have derived $\psi$ directly.


\section{Finitude and Decidability}

Obviously not every finished tableau is finite, since any tableau which extends a finished tableau is also finished, and some infinite tableaux extend a finished finite tableau. 

\paragraph{Finitude of Some Finished Tableaux}  But when we begin with a finite generating set, the smallest finished tableaux - those that are finished and themselves extend only unfinished tableaux – are themselves finite. We begin by proving a couple of preliminary results.

\begin{lemma}[König's Lemma for Tableaux]\label{konig}
	Every tableau containing infinitely many sentences has an infinite branch. \begin{proof} (See also \citet[152]{bevpospa}.) Let us say that $\psi$ is a \emph{descendent} of $\phi$ iff there is a branch on which both occur and where $\psi$ occurs after $\phi$ in the sequence. If a tableaux contains infinitely many sentences, the root node $\rho=\nu_{0}$ must have infinitely many descendents. 

	Suppose a node $\nu_{n}$ on an infinite tableau has infinitely many descendents. Then there is a node $\nu_{n+1}$ immediately posterior to and on the same branch as $\nu_{n}$ with infinitely many descendents. For if every branch $B$ on which $\nu_{n}$ occurs is such that the next item in the sequence after $\nu_{n}$ has only finitely many descendents, then every branch on which $\nu_{n}$ occurs must be finite. Since $\nu_{n}$ has infinitely many descendents, there must be infinitely many branches $B_{i}$ agreeing with $B$ up through $\nu_{n}$ but disagreeing on the next item in the sequence: $B_{i} = \langle \rho,…,\nu_{n},\phi_{i},…,\lambda\rangle$. But this cannot be a tableau, since for any tableau formed in accordance with the definition, there are at most two distinct branches agreeing with $B$ up through $\nu_{n}$ – since the only way of extending a branch is to apply a list rule or a branch rule, which latter branches only twice. So at least one immediate descendent of $\nu_{n}$ must have infinitely many descendents; pick one and call it $\nu_{n+1}$.

	We have shown that the root $\nu_{0}$ has infinitely many descendents, and that if a node $\nu_{n}$ has infinitely many descendents, then there is an immediate descendent $\nu_{n+1}$ which also has infinitely many descendents. The sequence $\langle \nu_{0},…,\nu_{n},…\rangle$ is a branch, because it is a sequence of \lone\ sentences such that each follows from previous by the tableau rules applied to earlier members of the sequence. And it is an infinite sequence, because each member of the sequence has a successor. So it is an infinite branch.
	\end{proof}
\end{lemma}
Recall that the complexity of a sentence is the sum of the arities of the connectives occurring in it (Definition \ref{defcompl}). \begin{lemma}
	Every tableau rule applied to a sentence $\phi$ on a branch licenses only sentence(s) of lower complexity than $\phi$.
	\begin{proof}
		We prove this by direct inspection of the tableau rules.  Suppose that the complexity of a sentence $\sigma$ is $c(\sigma)$. Every tableau rule applies to a sentence of these three forms: a double negated sentence, a binary connective sentence, or a negated binary connective sentence. All tableau rules license writing down either constituents of the original sentences or negated constituents. So the maximum complexity of any sentence that is licensed by a tableau rule is $c(\phi)+1$, where $\phi$ is the most complex constituent of the original sentence. But the tableau rules apply to original sentences of complexity at least $c(\phi)+2$. So every tableau rule licenses only sentences of strictly lower complexity.
	\end{proof}
\end{lemma}


We make use of this lemma in proving that the smallest finished tableau generated by a finite set is itself finite. 
\begin{theorem}[Finitude of Tableaux]\label{fintab}
	The smallest tableaux generated by a finite set\, $\Gamma$ is finite. \begin{proof} (Sketch.)
		Suppose we begin with a finite generating set $\Gamma$, so that every branch in the finished tableau begins with $\langle \gamma_{1},…,\gamma_{n}\rangle$. Every sentence on every branch subsequent to this initial subsequence will be the result of finitely many applications of the tableau rules to some $\gamma_{i}$ and its further consequences. (Because the tableau is the smallest finished tableau, we do not ever apply the rules to any sentence on a branch more than once.)  But each tableau rule results in a subsequent sentence of lower complexity than the sentence to which it is applied, by the lemma we just proved. Since every \lone\ sentence is of finite complexity, beginning with some $\gamma_{i}$ ends up yielding sentence letter(s) or negated sentence letter(s) after finitely many applications of the tableau rules, having produced only finitely many intermediate sentences of decreasing complexity along the way. And at that point we cannot apply the tableau rules any further, since they do not permit us to extend a branch by applying a rule to a literal. After we have exhaustively applied every tableau rule possible to each sentences in $\Gamma$ and the resulting sentences, we will have produced only finitely many sentences, only finitely many times, before we run out of sentences to which we have not already applied the rules. So any finished branch beginning with $\Gamma$ is finite. By König's Lemma, an infinite tableau must have an infinite branch, so any tableau with only finite branches feature must be finite. 
	\end{proof}
\end{theorem}

\paragraph{Effective Production of Tableaux} This result may seem abstract, but it is crucial given our motivating remarks. For the finitude of a finished tableau is essential if there is to be any sort of effective procedure here. We need to be assured that, from a finite starting point, we can produce a finished tableau in a finite time. The finitude of the smallest finished tableau certainly assures us of this. 

Note too that we are able to effectively check whether a tableau is finished. At any stage of its construction, a tableau consists of finitely many finite branches. If the tableau is finished, we can simply exhaustively check each sentence occuring on each branch and see that extending the branch by the appropriate rule will result in a sentence already on that branch. So we can check if each branch is finished; and then we can apply that check finitely many times to see if the tableaux is finished. So if a tableaux is finished, this procedure will tell us that it is. And if it is not, that will be because some sentence $\phi$ on $B$ is such that applying the appropriate rule to $\phi$ extends $B$ with a sentence not already on $B$ – so the procedure should simply extend $B$ (either by adding sentences to the end of $B$, or bifurcating $B$ into two new branches and adding new sentences to the ends of each) and then check if the newly extended tableau just formed is finished. This process is thus an effective procedure for determining whether or not a tableau is finished. And since we can produce a finished tableau in finite time, and check that it is finished in a finite time, there is an effective procedure that, beginning with a finite generating set, terminates having produced a finished tableaux generated from that set after a finite time. The details do matter: we haven't said exactly how the procedure is to extend a tableau. (Do we preferentially apply list rules first or branch rules, given a branch containing sentences to which both sorts of rules apply?) 
But however we do it, we will be able to do so in an effective manner.

\paragraph{Decidability of Tableau Derivability}

XXX



\section{Soundness and Completeness} \label{sec:tsoundcomp}


A derivation system $P$ in a given language is \emph{sound} with respect to a semantic interpretation of the language just in case whenever there is a derivation which establishes $\Gamma \vdash_{P} \phi$, it is the case that $\Gamma \vDash \phi$.

A derivation system $P$ in a given language is \emph{complete} with respect to a semantic interpretation of the language just in case whenever it is the case that $\Gamma \vDash \phi$, there is a derivation which establishes $\Gamma \vdash_{P} \phi$.

Obviously we've already talked informally about the correspondence between $\vDash$ and $\vdash$ we are about to establish, and we designed our derivation systems with one eye on capturing all and only correct entailments in derivations. But those intentions in the design of these derivation systems could have gone awry, so it is important to check they have not. Today, we prove soundness for both tableaux and natural deduction derivation systems.


\section{Soundness of the Tableaux Derivation System}

We begin by proving soundness of the tableaux system: that is, every syntactic theorem is a semantic theorem. We start with an intermediate lemma.\footnote{You might wish to contrast the proof of the same theorem (25.10) in \citet[118--9]{hodges}, or that given in \citet[{\S}4.5]{bosintlo}.}

\begin{lemma}[Tableau preserve satisfiability]\label{lemma1} If a tableau is generated by a satisfiable negated sentence, there is at least one branch on that tableau such that every sentence on that branch is simultaneously satisfiable. That is, if there is an \lone\ structure such that $\val{\neg \phi}{A} =1$, there is some branch $B$ on a tableau generated by $\{\neg\phi\}$ such that for all sentences $\beta \in B$, $\val{\beta}{A} = 1$.\end{lemma}
\begin{proof} We prove the lemma by {\em induction on the length of tableau branch $B$}. In effect, we consider a sequence of tableaux, such that each member extends the preceding member by application of one of the tableau rules.
    
   \paragraph{Base} consider the smallest tableaux $\mathbf{T}$ generated by $\{\neg\phi\}$: 
    the single node $\langle\neg\phi\rangle$. Since this is the only
    member of the only branch on $\mathbf{T}$, and by hypothesis it is assigned 1 by $\mathscr{A}$, the lemma holds in this case.
    
 \paragraph{Induction} Assume that the lemma holds of a branch 
 $B$ on tableau $T_{n}$. Then we show the lemma holds of a branch $B^{+}$ on a tableau $T_{n+1}$ obtained from
    $T_{n}$ by one additional application of a rule in
    \autoref{fig:stableaux} to a sentence in $B$ on $T_{n}$. There are
    three cases.
    \begin{enumerate}
	\item We apply the Double Negation Rule (\autoref{fig:stableaux}\subref{sf:dn}).
	Then some sentence on $B$ is of the form $\neg\neg\chi$, and we add
	$\chi$ to the bottom of the branch to get $B^{+}$. Since the valuation function induced by the \lone\ structure $\mathscr{A}$ is classical valuation, $\val{\chi}{A} = \val{¬¬\chi}{A} = 1$, as required by the lemma.
	\item We apply a non-branching rule to some sentence  on $B$. Then we add two sentences to
	the bottom of $B$. If we applied, for example, the Negated
	Conditional rule (\autoref{fig:stableaux}\subref{sf:nc}) to $\neg(\psi \to \chi)$, we added $\psi$ and
	$\neg\chi$ to the bottom of $B$ to get $B^{+}$. By the rules on classical valuations, $\val{¬(\psi\to\chi)}{A} = 1$ iff $\val{\psi \to \chi}{A} = 0$ iff $\val{\psi}{A} = 1$ and $\val{\chi}{A} = 0$ iff $\val{\psi}{A} = 1$ and
	$\val{\neg\chi}{A} = 1$, as required. Similarly for the
 Conjunction rule: $\val{\psi\wedge\chi}{A} = 1$ iff $\val{\psi}{A} = 1$ and
	$\val{\chi}{A} = 1$. So the lemma holds of $B^{+}$. (The Negated Disjunction rule is left as an exercise.)

	\item We apply a branching rule to some sentence on $B$. We then
	get two new branches, $B^{+}_{1}$ and $B^{+}_{2}$, either of
	which can satisfy the lemma (but we only need one of them to satisfy it, since the lemma only says that there is at least one branch on a tableau where all the sentences on it are assigned $1$ by $\mathscr{A}$). For instance, if we apply 
	Disjunction (\autoref{fig:stableaux}\subref{sf:di}) to $\psi \vee \chi$, $B^{+}_{1} = B \cup \{\psi\}$ and
	$B^{+}_{2} = B \cup \{\chi\}$. Since by the rules for classical valuations,
	$\val{\psi\vee\chi}{A} = 1$ iff either
	$\val{\psi}{A} = 1$ or $\val{\chi}{A} = 1$; that 
	is, the lemma holds of at least one of $B^{+}_{1}$ or $B^{+}_{2}$.
	Similarly for the Negated Conjunction rule:	If $\neg(\phi \wedge \psi) \in B$, then $\neg\phi \in B^{+}_{1}$ and	$\neg\psi \in B^{+}_{2}$. $\val{¬(\phi \wedge \psi)}{A} =	1$ iff $\val{\phi \wedge \psi}{A} = 0$ iff	$\val{\phi}{A} = 0$ or $\val{\psi}{A} = 0$ iff
	$\val{¬ \phi}{A} = 1$ or $\val{¬ \psi}{A} =	1$. (The Conditional rule is left as an exercise.) 
	\end{enumerate}That completes the induction.\end{proof}

Lemma \ref{lemma1} shows that the tableau rules preserve satisfiability: any tableau, finished or otherwise, generated by a satisfiable sentence has at least one simultaneously satisfiable branch. Now we are in a position to prove soundness.





\begin{theorem}[Soundness] \label{thm:sound} If 
$\vdash \phi$ then $\vDash \phi$.\end{theorem}\begin{proof} Assume that $\vdash
\phi$. Then every finished tableau generated by $\{\neg
\phi\}$ is closed. 

Assume for \emph{reductio} that $\nvDash \phi$. Then there is an \lone structure $\mathscr{A}$ that makes $\neg \phi$ true: $\val{\neg\phi}{A}=1$. By Lemma \ref{lemma1}, there is a finished tableau $\mathbf{T}$ generated by $\neg
\phi$, with a branch $B$ such that for every $\beta \in B$,
$\val{\beta}{A} = 1$. Since the valuation function induced by $\mathscr{A}$ is classical, for any \lone\ sentence $\psi$
if $\psi \in B$, then $\neg\psi \notin B$ – since for any structure $A$, $\val{\psi}{A} ≠ \val{\neg\psi}{A}$, and every sentence on $B$ is satisfied in some structure and hence all have the same truth value in that structure. Hence $B$ cannot be closed;  hence $\mathbf{T}$ is not closed. But $\mathbf{T}$ is generated by $\{\neg \phi\}$, and our initial assumption was that any finished tableau generated by $\{¬\phi\}$ is closed. So our \emph{reductio} hypothesis must be wrong; that is, it must instead be true that $\vDash \phi$. But now we've shown $\vDash\phi$ on the assumption that $\vdash\phi$.
\end{proof}

Having proved the theorem, we can easily extend it to the general
case of an arbitrary argument. 

\begin{theorem}[General Soundness]
	If $\Gamma\vdash\phi$ then $\Gamma\vDash\phi$.
\end{theorem}
\begin{proof}
	Recall that $\Gamma\vdash\phi$ iff there is a finished closed tableau generated by $\Sigma\cup\{\neg\phi\}$, where $\Sigma$ is a finite subset of $\Gamma$. So  $\Gamma\vdash\phi$ iff there exists a finite $\Sigma\subseteq\Gamma$ such that $\Sigma\vdash\phi$. 
	
	Assume that $\Gamma\vdash\phi$. Then there is a finite set $\Sigma\cup\{¬\phi\} = \{\sigma_{1},…,\sigma_{n},\neg\phi\}$ which generates a finished closed tableau. By repeated applications of the Deduction theorem for tableaux (\autoref{thm:sdt}), $\vdash (\sigma_{1} \to (\sigma_{2} \to ( … (\sigma_{n}\to\phi))))$. By Soundness (Theorem \ref{thm:sound}), $\vDash (\sigma_{1} \to (\sigma_{2} \to ( … (\sigma_{n}\to\phi))))$. By repeated applications of the Deduction theorem for entailment (\autoref{ded}), $\Sigma \vDash\phi$. Since $\Sigma\subseteq\Gamma$, $\Gamma \vDash \phi$, by Weakening.\footnote{Actually (because Weakening as stated only applies to a single sentence) what we need is a stronger claim, that if $\Sigma\vDash\phi$, then for any set $\Delta$, $\Sigma,\Delta\vDash\phi$. And we then let $\Delta=\Gamma\setminus\Sigma$, so that $\Sigma\cup\Delta=\Gamma$, and the result follows.}
\end{proof}



% \exercise{\begin{ex}Prove the induction step of the derivation for Lemma 1 in the
%     cases where the branch $B$ is extended by (i) the negated disjunction
%     rule; and (ii) the conditional rule.
% \end{ex}

% \begin{ex} Assuming the  soundness theorem, prove that if $\Xi \vDash \phi$ 
% is an \emph{incorrect} sequent  then $\Xi \vdash \phi$ is also incorrect.
% \end{ex}

% \begin{ex} Recall the tableau rules involving the $\odot$ operator from exercise \ref{odot}.
%    Are those tableau rules sound?
%     \end{ex}

% }


\section{Completeness for Tableaux}

Make sure we are allowing for finite tableau demonstrating the unsatisfiability of infinite sets – the infinite case needs to be accomodated.



The proof of completeness for the tableaux derivation system is relatively straightforward. The idea is simple: we show that, if set of sentences $\Gamma$ is consistent, then $\Gamma$ is satisfiable. Recall that (Definition \ref{defcontab}) a set of sentences $\Gamma$ is consistent if any finished tableau generated by $\Gamma$ has an open branch. We will use this fact to show that there is a structure in which all the members of $\Gamma$ are true. To do this, we have to show an intermediate lemma, to the effect that an open branch can be used to determine a structure which makes each sentence on the branch true.


\begin{lemma}[Hintikka's Lemma]\label{lemm:hin} If $B$ is an open branch on a finished
    open tableau generated by a set of \lone\ sentences $\Gamma$, then there is an \lone\ structure $\mathcal{B}$ such that for every sentence $\beta$ on $B$, $\val{\beta}{B} = 1$.\end{lemma}
\begin{proof}
	
	Suppose there is an open branch on a finished tableau generated by $\Gamma$, $B$. Define an \lone\ structure $\mathscr{B}$ as follows: for every sentence letter $s$, \[\mathscr{B}(s) = \begin{cases}
		1 \quad\text{if $s$ occurs in some node on $B$;}\\
		0 \quad\text{if $\neg s$ occurs in some node on $B$;}\\
		1 \quad\text{otherwise.}
	\end{cases}\] Because $B$ is an open branch, this successfully defines an \lone\ structure: no sentence letter and its negation both occur on $B$, so this definition assigns one and only one value to each \lone\ sentence letter. (It is thus a classical structure, and induces a classical Boolean valuation function.)
	
We show by induction on complexity on sentences a the branch that  for every sentence $\beta$ on $B$, $\val{\beta}{B}=1$. Because this is a tableau construction, the nicest way to think about the induction on complexity is that the base case comprises those sentences to which no tableau rule applies, and the induction step involves consideration of sentences to which tableau rules apply. (Our base ‘case’ thus encompasses literals, rather than just sentence letters.) 

\paragraph{Base case:} Suppose $\beta$ occurs on $B$, and is a literal: either is a sentence letter or a negated sentence letter. If the former, by construction, $\mathscr{B}(\beta)=1$, so $\val{\beta}{B}=1$, by the definition of an \lone\ structure. If the latter, i.e., $\beta = \neg s$ for some $s$, $\val{\beta}{B} = 1 - \val{s}{B} = 1-\mathscr{B}(s) = 1-0=1$.

\paragraph{Induction step:} Suppose $\beta$ is a complex sentence, and the lemma holds of its less complex constituents. There are three cases of interest, corresponding to different tableau rules: $\beta$ is a
double negation; a branch rule can be applied to $\beta$; or a list
rule can be applied to $\beta$. \begin{enumerate}
	\item $\beta$ is a double negation, $\neg\neg\phi$. Then, because this is a finished branch, $\phi$ appears on $\beta$; because the lemma holds of less complex constituents, $\val{\phi}{B}=1$. So $\val{\neg\phi}{B}=0$, and $\val{\neg\neg\phi}{B}=1$, i.e., $\val{\beta}{B}=1$.
  \item $\beta$ can have a list rule applied to it. Let us choose Negated
    Conditional, so $\beta = \neg(\phi \to \psi)$. Then $\phi$ and $\neg \psi$
    appear on $B$, because the tableau is finished, and by the induction hypothesis $\val{\phi}{B} = \val{¬\psi}{B} =1$. Therefore by the rules on the valuation function, $\val{\phi\to\psi}{B}=0$, and $\val{\neg(\phi \to \psi)}{B}=1$, i.e., $\val{\beta}{B}=1$, as required. Analogous reasoning applies to the other list rules.
    \item $\beta$ can have a branch rule applied to it. Let $\beta = \phi
    \vee \psi$. Then either $\phi$ or $\psi$ appears on $B$; hence either
    $val{\phi}{B} = 1$ or $\val{\psi}{B} =1$, which by the rules on the valuation function, means that $\val{\phi\vee\psi}{B}=1$,  i.e., $\val{\beta}{B}=1$, as required. Analogous reasoning applies to the other branch rules.
\end{enumerate}
    That suffices to show the lemma.\end{proof}



\begin{theorem}[Completeness for tautologies]\label{thm:comp} If\/ $\vDash \phi$ then $\vdash
    \phi$.\end{theorem}\begin{proof} We prove the equivalent contrapositive, namely, that if
   it is {\em not} the case that $\phi$ is a syntactic theorem (which 
   we write `$\nvdash \phi$') then it is not the case that $\phi$ 
   is a semantic theorem (`$\nvDash \phi$').
    
Assume that $\nvdash \phi$. Then there is a finished open tableau generated by $\{\neg
\phi\}$ which has an open branch, $B$. Let $\mathscr{B}$ be the structure induced by $B$, in accordance with Hintikka's Lemma (Lemma \ref{lemm:hin}). By that lemma, every sentence occurring on $B$ is true in $\mathscr{B}$. Since $\neg\phi$ appears on $B$, as the root, $\val{\neg\phi}{B}=1.$ Hence there is a structure which makes $\phi$ false, so $\nvDash \phi$.
\end{proof}

Theorem \ref{thm:comp} states that every tautology is derivable. This can easily be extended to arguments with finitely many premises: \begin{theorem}[Completeness for finite arguments]\label{thm:fargc} When $\Gamma$ is finite, if\/ $\Gamma \vDash \phi$ then $\Gamma \vdash \phi$.	
\end{theorem}
\begin{proof}
	If $\Gamma \vDash \phi$, and $\Gamma$ is a finite set $\{\gamma_{1},\ldots,\gamma_{n}\}$, then (by repeated applications of the Deduction theorem for entailment (\autoref{ded}), $(\gamma_{1} \to (\gamma_{2} \to (\ldots(\gamma_{n} \to \phi)\ldots)))$ is a tautology. By Theorem \ref{thm:comp}, $\vdash (\gamma_{1} \to (\gamma_{2} \to (\ldots(\gamma_{n} \to \phi)\ldots)))$. By repeated applications of the deduction theorem for tableaux (\autoref{thm:sdt}), $\Gamma \vdash \phi$.
\end{proof}





\section{Compactness Revisited via Tableaux}

\begin{theorem}[Finite Tableaux Derivations]\label{ftd}
	If there is a tableau derivation establishing that $Γ ⊢ \phi$, then there is a finite tableau derivation of $\phi$ using only finitely many members of $\Gamma$.
\end{theorem} \begin{proof}
	Let $τ$ be a tableau derivation of $\phi$ from $Γ$. Every branch on $τ$ is finished and closed. That is, for each branch $B$, for some $\psi$, both $\psi$ and $\neg\psi$ occur on $B$. But because $\tau$ is a tableau, each sentence on it occurs in a node of some \emph{rank}; and each rank is finite. (There is no upper bound to ranks, but each rank is a natural number, by the construction of a tree.) So $\rho(\psi)$ and $\rho(\neg\psi)$ are both finite. For each such branch, the \emph{pruned} branch $B^{-}$ is the result of cutting any nodes from $B$ of rank greater than $\max(\rho(\psi),\rho(\neg\psi))$. The \emph{repaired} branch $B^{\dag}$ is the result of applying any tableaux rules to sentences on the pruned branch $B^{-}$ where the pruning has made that branch unfinished. Since each pruned branch is finite, each repaired branch is also finite. Let $\tau^{-}$ be the result of pruning and repairing every branch on $\tau$. By König's Lemma (\ref{konig}), $\tau^{-}$ is finite, since it is a binary branching tableau with only finite branches. Thus only finitely many members of $\Gamma$ appear on $\tau^{-}$; gather them together into a set  $\Sigma$. $\tau^{-}$ is a finished closed tableau generated by $\Sigma\cup\{\neg\phi\}$, i.e., showing $\Sigma\vdash\phi$ where $\Sigma$ is a finite subset of $\Gamma$.	
\end{proof}


\begin{theorem}[Compactness]
	If $\Gamma$ is finitely satisfiable, $\Gamma$ is satisfiable.
\end{theorem} \begin{proof}
	We prove the contrapositive. Assuming without loss of generality that $\Gamma$ is non-empty, then $\Gamma = \Delta \cup\{\phi\}$, and assume that $\Gamma=\Delta \cup\{\phi\}$ is unsatisfiable. That is, $\Delta \vDash \neg\phi$. By General Completeness (Theorem \ref{gc}), $\Delta \vdash \neg\phi$. By Theorem \ref{ftd}, there is a finite tableau derivation of $\neg\phi$ using only finitely many members of $\Delta$. Thus, some finite subset of $\Delta$, $\Delta_{0}$, is such that $\Delta_{0}\vdash\neg\phi$. By soundness, $\Delta_{0}\vDash\neg\phi$; i.e., $\Delta_{0}\cup\{\phi\}$ is unsatisfiable. But $\Delta_{0}\cup\{\phi\}$ is a finite subset of $\Gamma$, so $\Gamma$ is not finitely satisfiable, as needed to be shown.
 \end{proof}
	

\section{Alternate Tableau Systems}

labelled tableaux make semantics clear

Consider these two rules: \begin{center}
	{\leaf{$\neg\phi\vee\neg\psi$}\branch{1}{$\neg(\phi\wedge\psi)$}\qobitree} \qquad {\leaf{$\neg\phi\wedge\neg\psi$}\branch{1}{$\neg(\phi\vee\psi)$}\qobitree}
\end{center}

Let us note firstly, that these proposed rules can be derived in our existing system. That is demonstrated by these two schematic tableau proofs:
\begin{center}
	{\leaf{$\neg\phi\vee\neg\psi$}\branch{1}{$\neg(\phi\wedge\psi$)}\qobitree} \qquad {\leaf{$\neg\phi\wedge\neg\psi$}\branch{1}{$\neg(\phi\vee\psi)$}\qobitree}
\end{center}


What if we developed a new tableau system, in which we replaced our familiar rules for negated conjunction (\autoref{fig:stableaux}\subref{sf:nand}) and negated disjunction (\autoref{fig:stableaux}\subref{sf:ndsj}) by these rules?




{\small

\subsection*{Further Reading}
\addcontentsline{toc}{subsection}{Further Reading}
 \citet[ch. 4]{bosintlo} and \citet{jefforlos}  both discuss some of the same metatheoretic results proved above about our tableaux system. 

\subsection*{Exercises} \label{ex6a}
\addcontentsline{toc}{subsection}{Exercises}

\begin{enumerate}
\item Show Weakening for $\vdash$: that if $\Gamma\vdash \phi$ then $\Gamma\cup \Delta \vdash \phi$ for any $\Delta$.
\item Assuming the  soundness theorem, prove that if $\Gamma \vDash \phi$  is an \emph{incorrect} sequent  then $\Gamma \vdash \phi$ is also incorrect.

\item Consider  an operator $\odot$ defined by the following tableau rules: 
 \begin{center}
 {\leaf{$(\phi \odot \phi)$\\$(\psi \odot \psi)$}\branch{1}{$\qquad\qquad(\phi 
 \odot  \psi)\qquad\qquad$}\qobitree} \qquad
 {\leaf{$\phi$}\leaf{\quad$\psi$\qquad}\branch{2}{$\qquad\qquad((\phi \odot
\psi) \odot (\phi \odot \psi))\qquad\qquad$}\qobitree}\end{center}

\begin{enumerate}
 \item  Which truth-function does $\odot$ express? (I.e., which truth function is characterised by these tableau rules?)
 \item If these rules were the only rules for a system of tableau,
 under what conditions should a branch be counted as closed?
\item Are these tableau rules sound?
 \end{enumerate}
 \end{enumerate}


Answers to selected exercises on page \pageref{ans6a}.
}
