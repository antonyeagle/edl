%!TEX root = edl.tex



\section{Soundness for Natural Deduction}
\paragraph{Soundness and Completeness}

A proof system $P$ in a given language is \emph{sound} with respect to a semantic interpretation of the language just in case whenever there is a proof which establishes $\Gamma \vdash_{P} \phi$, it is the case that $\Gamma \vDash \phi$.

A proof system $P$ in a given language is \emph{complete} with respect to a semantic interpretation of the language just in case whenever it is the case that $\Gamma \vDash \phi$, there is a proof which establishes $\Gamma \vdash_{P} \phi$.

In this section and the next, we will show that the natural deduction system $ND_{\neg,\to}$ just introduced is \emph{sound and complete} for the standard semantics of $\mathcal{L}_{\neg,\to}$. Given the expressive adequacy of $\mathcal{L}_{\neg,\to}$, that also shows the completeness of the natural deduction system with respect to \lone; to show the soundness of that system, we also need to show that the other rules of the system $ND$ are also sound.


\paragraph{Soundness of $ND_{\neg,\to}$}

\begin{theorem} The Rules of $ND_{\neg,\to}$ are sound.
\begin{proof} Suppose that $\Gamma \vdash \phi$. We show by induction on the complexity of proofs that $\Gamma \vDash \phi$.

\emph{Base case}: The least complex proof is a single node, $\phi$, establishing $\phi \vdash \phi$. In this case, any structure which assigns $T$ to $\phi$ clearly assigns $T$ to $\phi$, so $\phi\vDash\phi$, i.e., $\Gamma\vDash\phi$.

\emph{Induction step}: Suppose $\Gamma\vdash\phi$, established by applying one of the natural deduction rules to some previously obtained proofs to obtain a proof of $\phi$. There are four cases: 
 \begin{description}
	\item [$\to$Elim] From a proof of $\psi$ on assumptions $\Delta$, and a proof of $\psi\to\phi$ on assumptions $\Theta$, obtain a proof of $\phi$ on assumption $\Gamma=\Delta\cup\Theta$. By the induction hypothesis, $\Delta \vDash \psi$ and $\Theta \vDash \psi\to\phi$, so every structure which makes all of the members of $\Gamma$ true, makes $\psi$ and $\psi\to\phi$ true, and therefore $\phi$ must be true in all such structures, showing that $\Gamma\vDash\phi$.
	\item [$\to$Intro] From a proof of $\phi$ on assumptions $\Gamma \cup \{\psi\}$, apply $\to$Intro to obtain a proof of $\psi\to\phi$ on assumption $\Gamma$ (discharging $\psi$). But since by hypothesis $\Gamma,\psi \vDash\phi$, by the deduction theorem, $\Gamma \vDash \psi\to\phi$.
	\item [$\neg$Elim] From a proof of $\psi$ on assumptions $\Gamma,\neg\phi$ and a proof of $\neg\psi$ on assumptions $\Gamma,\neg\phi$, obtain a proof of $\phi$ on assumptions $\Gamma$, discharging $\neg\phi$. Since $\Gamma,\neg\phi \vDash \psi$ and $\Gamma,\neg\phi\vDash\neg\psi$, it follows that $\Gamma,\neg\phi\vDash$; therefore $\Gamma\vDash\phi$.
		\item [$\neg$Intro] From a proof of $\psi$ on assumptions $\Gamma,\phi$ and a proof of $\neg\psi$ on assumptions $\Gamma,\phi$, obtain a proof of $\neg\phi$ on assumptions $\Gamma$, discharging $\phi$. Since $\Gamma,\phi \vDash \psi$ and $\Gamma,\phi\vDash\neg\psi$, it follows that $\Gamma,\phi\vDash$; therefore $\Gamma\vDash\neg\phi$.
\end{description}
Thus all the rules preserve soundness; no proof in $ND_{\neg,\to}$ can be constructed other than using these rules; so $ND_{\neg,\to}$ is sound. \end{proof}\end{theorem}




\section{Completeness}
\paragraph{Consistency}

A set of sentences $\Gamma$ is called \emph{(syntactically) consistent} iff $\Gamma\not\vdash\neg(\phi\to\phi)$. Otherwise it is \emph{inconsistent}, also written $\Gamma\vdash$. Equivalently, consider this proof:
\begin{equation*}
	\begin{prooftree}
		\[ \Gamma \leadsto \neg(\phi \to \phi)\] 
		\[[\phi] \justifies \phi\to\phi \using \to\text{Intro}\] 
		\justifies \psi \using \neg\text{Elim}
	\end{prooftree}
\end{equation*} If we have a proof of $\neg(\phi\to\phi)$, we have a proof of arbitrary $\psi$ from the assumption $\Gamma$; so we can say that $\Gamma$ is consistent iff there is a sentence \emph{not provable} from $\Gamma$. 

A set of sentences $\Gamma$ is \emph{maximally consistent} iff it is consistent and for any $\phi$, if $\phi\notin\Gamma$, then $\Gamma,\phi\vdash$ (one cannot add any more sentences while remaining consistent).
\paragraph{Properties of Maximal Consistent Sets: Deductive Closure}

\begin{lemma}[Deductive Closure]
	If $\Gamma$ is a maximal consistent set, then if $\Gamma \vdash \phi$, then $\phi \in \Gamma$. \begin{proof}
	{	 Suppose that $\Gamma\vdash\phi$, but $\phi\notin\Gamma$. Since $\Gamma$ is maximal consistent, $\Gamma,\phi\vdash \neg(\phi\to\phi)$. Using $\to$Intro and discharging $\phi$, we see that $\Gamma\vdash\phi\to(\neg(\phi\to\phi))$. But now use $\to$Elim, and obtain $\Gamma\vdash\neg(\phi\to\phi)$ i.e., $\Gamma$ is inconsistent.}
	\end{proof}
\end{lemma} 


{
 \paragraph{Properties of Maximal Consistent Sets: Negation Completeness}

\begin{lemma}[Negation Completeness]
	If $\Gamma$ is a maximal consistent set, then $\neg\phi \in \Gamma$ iff $\phi \notin\Gamma$. \begin{proof}
	{ L to R:	If both $\phi$ and $\neg\phi$ were in $\Gamma$, we could prove $\neg(\phi\to\phi)$ by an application of $\neg$Elim, showing $\Gamma$ inconsistent. 
		
		R to L: If $\phi\notin\Gamma$, then  $\Gamma,\phi\vdash\neg(\phi\to\phi)$: 
		 \begin{equation*}
			\begin{prooftree}
				\[\Gamma, [\phi] \leadsto \neg(\phi\to\phi)\]
				\[[\phi] \justifies \phi\to\phi\using \to\text{Intro}\]
				\justifies \neg\phi \using\neg\text{Intro}
			\end{prooftree}
		\end{equation*}  Therefore, $\Gamma\vdash\neg\phi$; by Deductive Closure Lemma, $\neg\phi\in\Gamma$.}
	\end{proof} 
\end{lemma}
 

\paragraph{Properties of Maximal Consistent Sets: Conditional Completeness}

\begin{lemma}[Conditional Completeness]
	If\, $\Gamma$ is any maximal consistent set, then	$\phi\to\psi\in\Gamma$ iff whenever $\phi\in\Gamma$, it follows that $\psi\in\Gamma$.
	\begin{proof}
{	L to R:	Suppose $\phi\to\psi\in\Gamma$. Then $\Gamma\vdash\phi\to\psi$; and if $\phi\in\Gamma$, then $\Gamma\vdash\phi$. Apply $\to$Elim, and establish $\Gamma\vdash\psi$, i.e., $\psi\in\Gamma$.
		
		R to L: Suppose that whenever $\phi\in\Gamma$, $\psi\in\Gamma$. Consider two cases: \begin{enumerate}
			\item $\phi\in\Gamma$. So $\Gamma=\Gamma\cup\{\phi\}$. Moreover, $\psi\in\Gamma$, so $\Gamma,\phi\vdash\psi$. Apply $\to$Intro and discharge $\phi$ to show $\Gamma\vdash \phi\to\psi$.
			\item $\phi\notin\Gamma$. Then $\Gamma,\phi\vdash\neg(\phi\to\phi)$. But then $\Gamma,\phi \vdash\psi$ (by a similar proof as just above); applying $\to$Intro and discharge $\phi$, $\Gamma\vdash\phi\to\psi$.
		\end{enumerate}
}		
	\end{proof} 
\end{lemma}

\paragraph{Properties of Maximal Consistent Sets: Satisfiability}

\begin{theorem}[Maximal Consistent Sets Satisfiable]
	Every maximal consistent set is satisfiable (i.e., has a model).\begin{proof}
	{	Suppose $\Gamma$ is a maximal consistent set in $\mathcal{L}_{\to,\neg}$. For $s$ a sentence letter, define a structure $\mathscr{A}$: {$\mathscr{A}(s) = T$ iff $s\in\Gamma$}.
			
			For any $\phi$, $\val{\phi}{A}=T$ iff $\phi\in\Gamma$. Proof by induction on complexity of sentences. The base case, $\phi$ a sentence letter, is immediate. 
		
	Suppose $\phi$ is complex. There are two cases: \begin{enumerate}
		\item $\phi=\neg\psi$. $\phi\in\Gamma$ iff $\psi\notin\Gamma$, by Negation Completeness; iff by the induction hypothesis, $\val{\psi}{A}=F$; iff by the clause on $\neg$, $\val{\phi}{A}=T$.
		\item $\phi=(\psi\to\chi)$. $\phi\in\Gamma$ iff if $\psi\in\Gamma$ then $\chi\in\Gamma$ (by Conditional Completeness); iff if $\val{\psi}{A}=T$ then $\val{\chi}{A}=T$; iff $\val{\phi}{A}=T$.
	\end{enumerate} }\end{proof}
\end{theorem} 

\paragraph{Completeness}

\begin{theorem}[Completeness for $\mathcal{L}_{\neg,\to}$]
	If $\Gamma\vDash\phi$, then $\Gamma\vdash\phi$. \begin{proof}
		{ We prove the \emph{contrapositive}. So assume that $\Gamma\not\vdash\phi$. (Assuming that $\Gamma$ itself is consistent; if it is not, then clearly the theorem holds fairly trivially.)
		
		We'll construct a maximal consistent set $\Gamma^{+}$, a \emph{superset} of $\Gamma$, such that $\Gamma^{+}\not\vdash\phi$.  By consistency of $\Gamma^{+}$, $\phi\notin\Gamma^{+}$; by the Negation Completeness lemma, $\neg\phi\in\Gamma^{+}$.
		
		But by the Satisfiability Theorem, $\Gamma^{+}$ has a model, $\mathscr{A}$. Since $\Gamma\subseteq\Gamma^{+}$, for all $\gamma\in\Gamma$, $\val{\gamma}{A}=T$; but since $\neg\phi\in\Gamma^{+}$, $\val{\neg\phi}{A}=T$, so $\val{\phi}{A}=F$. So $\Gamma\not\vDash\phi$, as required.
		
		We now show how to construct $\Gamma^{+}$ from $\Gamma$, completing the proof.
		}
	\end{proof}
\end{theorem}

\paragraph{The Construction of Maximal Consistent $\Gamma^{+}$}

Supposing $\Gamma\not\vdash\phi$, construct $\Gamma^{+}$ as follows. \begin{enumerate}
	\item Enumerate the sentences of $\mathcal{L}_{\neg,\to}$, $\{\sigma_{1},\sigma_{2},\ldots\}$. 
	\item Let $\Gamma_{0}=\Gamma$.
	\item \begin{equation*}
		\Gamma_{n+1} = \begin{cases}
			\Gamma_{n}\cup\{\neg\sigma_{n+1}\} &\text{if } \Gamma_{n},\sigma_{n+1}\vdash\phi;\\
			\Gamma_{n}\cup\{\sigma_{n+1}\} &\text{otherwise (i.e., if $\Gamma_{n},\sigma_{n+1}\not\vdash\phi$)}.
		\end{cases}
	\end{equation*}
	\item Let $\Gamma^{+}=\bigcup_{i} \Gamma_{i}$.
\end{enumerate}
It is clear that \emph{$\Gamma\subseteq\Gamma^{+}$}. It is also clear that \emph{$\Gamma^{+}\not\vdash\phi$}, for at no stage was a sentence added to any $\Gamma_{n}$ that permitted a proof of $\phi$. 

Finally, $\Gamma^{+}$ is a \emph{maximal consistent set}, since (by construction), for every $\sigma_{i}$, either $\sigma_{i}\in\Gamma^{+}$ or $\neg\sigma_{i}\in\Gamma^{+}$.  Thus $\Gamma^{+}$ has the properties required to prove Completeness stated just above.

}



{
\paragraph{Completeness for \lone}

We've shown the Completeness Theorem for the restricted language $\mathcal{L}_{\neg,\to}$. But by Expressive Adequacy, any sentence of \lone\ can be expressed in $\mathcal{L}_{\neg,\to}$. So we can assure ourselves that any truth function expressible in \lone\ can be expressed in $\mathcal{L}_{\neg,\to}$, and any valid argument in \lone\ has a corresponding provably correct sequent in $\mathcal{L}_{\neg,\to}$.

For completeness of \lone, we cannot rest there. What if the rules of \lone\ don't suffice to show the equivalence of some arbitrary \lone\ sentence $\phi$ with a sentence only involving $\neg$ and $\to$? In that case, while a sentence logically equivalent to $\phi$ is provable, $\phi$ may not be.

We thus need additional lemmas showing that maximal consistent sets \begin{itemize}
	\item Contain $\phi \wedge \psi$ iff they contain both $\phi$ and $\psi$;
	\item Contain $\phi \vee \psi$ iff they contain at least one of $\phi$ or $\psi$;
	\item Contain $\phi \bicond \psi$ iff they contain $\phi$ iff they contain $\psi$.
\end{itemize} 
 Then the Satisfiability Theorem must be extended to include these new cases; and the remainder of the theorem is proved as above.






\paragraph{Compactness  Again}\label{compagain}

Consideration of the nature of natural deduction proofs should convince you that any conclusion $\phi$ drawn on the basis of assumptions $\Gamma$ can be constructed by some finite application of the rules to finitely many sentences in $\Gamma$. Provability is essentially a finite notion: proofs are things that can be \emph{constructed}.

This observation, informal though it is, \emph{can} be made precise, to yield another proof of the \emph{Compactness theorem}. Since, by Completeness, every correct semantic sequent $\Gamma\vDash$ is provable, and every proof is a finite object using only finitely many members of $\Gamma$, then by Soundness there is a correct semantic sequent $\Gamma'\vDash$ where $\Gamma'$ is a finite subset of $\Gamma$. 






\section{Axioms}
\paragraph{Axiomatic Systems}

Another proof system we will now briefly consider is an \emph{axiomatic} (or Hilbert-style) proof system. 

This is familiar in mathematics: one begins with some \emph{axioms} (normally interpreted as `obvious' or `certain' truths), and by applications of some very simple truth-preserving rules of inference, one derives further truths.

If one's axioms are well chosen, one should be able to prove all and only truths about a given subject matter from the axioms characterising that subject matter. This was Euclid's method in the \emph{Elements}: he showed that all and only the truths of Euclidean geometry could be established on the basis of his well chosen axioms.

\paragraph{Axioms for Sentential Logic}

In an axiomatic system, one begins with theorems, the axioms, and applies the rules of inference to preserve theoremhood. 

Our axiom system \L\ is very simple. It has three \emph{axiom-schemata}, in the language $\mathcal{L}_{\neg,\to}$:
\begin{description}
	\item [A1] $\lvdash (\phi \to (\psi \to \phi))$;
	\item [A2]$\lvdash ((\phi \to (\psi \to \chi)) \to ((\phi \to \psi) \to (\phi \to \chi)))$;
	\item [A3] $\lvdash ((\neg \psi \to \neg \phi) \to (\phi \to \psi))$.
\end{description}
Any sentence of $\mathcal{L}_{\neg,\to}$ can be substituted into these axiom schemata to generate an instance of an axiom. Any axiom is a theorem.

The system has one rule (apart from substitution): \emph{\emph{Modus Ponens}}, then rule that if $\lvdash \phi$ and $\lvdash \phi\to\psi$, then $\lvdash\psi$. 

A \emph{proof} in \L\ is any sequence of theorems, each of which is either an axiom or follows by \emph{modus ponens} from earlier theorems.

\paragraph{A Proof in \L}

The system \L\ is  elegant, but proofs in it can be unnatural to say the least: (I annotate lines of the proof with the axiom scheme used.)
\begin{align*}
 1. &\lvdash (P\to((P\to P)\to P)) \to ((P\to(P\to P))\to(P\to P)) & \text{(A2)}\\
2. &\lvdash (P\to((P\to P)\to P)) &\text{(A1)}\\
3. &\lvdash ((P\to(P\to P))\to(P\to P)) &\text{(MP 1, 2)}\\
4. &\lvdash (P \to (P\to P)) & \text{(A1)}\\
5. &\lvdash (P\to P)&\text{(MP 3, 4)}
\end{align*} 

This is (apparently) the \emph{shortest} proof of $P \to P$ in \L!

\paragraph{Soundness of \L}

It is a trivial exercise (and hence left for an exercise) to show that \L\ is sound. That is, show: \begin{itemize}
	\item All the axioms of \L\ are tautologies.
	\item The rule of \L\ preserves tautologousness.
\end{itemize}

\paragraph{Equivalence of \L\ and $ND_{\neg,\to}$}

We now show that, perhaps surprisingly, the proof systems \L\ and $ND_{\neg,\to}$ are equivalent – everything that can be proved in one can be proved in the other. This shows, too, that \L\ is sound and complete.

\paragraph{Everything provable in \L\ is provable in $ND$}

\begin{theorem}
	If $\lvdash\phi$ then $\vdash\phi$.\end{theorem} \begin{proof}
		{ The proof is straightforward: show that each axiom is provable in $ND$, and show that the rule \emph{modus ponens} is a valid rule in $ND$.
		
		The second part is easy; if $\vdash\phi$ and $\vdash\phi\to\psi$, then  appending an instance of $\to$Elim to the preceding proofs is a proof of $\psi$.
		
		Then it is just a matter of proving the axioms. I show A1: \begin{equation*}
			\begin{prooftree}
				\[ [\phi] \justifies \psi\to\phi \using\to\text{Intro}\]
				\justifies \phi \to (\psi\to\phi) \using \to\text{Intro}
			\end{prooftree}
		\end{equation*}}
	\end{proof}



\paragraph{Proofs from Assumptions in \L}

The ugly proof of $\lvdash P\to P$ two paragraphs back shows the benefits of working with \emph{assumptions}. A proof of $\phi$ from assumptions $\Gamma$ in \L\ is a sequence of axioms, \emph{members of $\Gamma$}, or follows from preceding members by \emph{modus ponens}, and the last line of which is $\phi$.
This is a legitimate notion, because one can show: \begin{theorem}[Deduction Theorem for \L]
	 $\Gamma\lvdash\phi$ iff for some finite set of sentences $\gamma_{1},\ldots,\gamma_{n}$ actually used in the proof of $\phi$, $\lvdash (\gamma_{1}\to(\gamma_{2} \to \ldots (\gamma_{n} \to \phi)\ldots))$.
\end{theorem} It suffices to show that $\Gamma,\psi\lvdash\phi$ iff $\Gamma\lvdash\psi\to\phi$. I leave the proof for an exercise.
}


{
\paragraph{Everything Provable in $ND$ is provable in \L}
\begin{theorem}
	If $\vdash \phi$, then $\lvdash \phi$. \begin{proof}
		{ Induction on length of proofs: we show that the shortest proofs of $ND$ are provable in \L, and	that the rules are provable transitions.
		
	Certainly the shortest proof, the single node $\phi$, shows that $\phi\vdash\phi$; but $\phi\lvdash\phi$ is shown by the deduction theorem given $\lvdash \phi\to\phi$.
	
	Suppose we extend some existing  proofs by applying the rules of $ND$: \begin{description}
		\item[$\to$Elim] We have $\Gamma\vdash\phi$ and $\Gamma\vdash\phi\to\psi$, and extend by $\to$Elim. But this is obviously \emph{modus ponens}, so $\Gamma\lvdash\psi$.
	\end{description}
The case of $\to$Intro is shown by the deduction theorem. I leave you to show the cases for negation.}	\end{proof}
\end{theorem}
}

{\small
\subsection*{Further Reading}
\addcontentsline{toc}{subsection}{Further Reading}

Natural deduction was invented  by \citet{geninvinl}, the founder of proof theory. 
The proof theory of natural deduction systems discussed in section 4 of this chapter is based on \citet[39--41]{pranatde}; he uses Theorems \ref{altneg} and \ref{redcompl}, and the last problem, and proves the famous `cut elimination' theorem for natural deduction, unfortunately a topic beyond the scope of these notes.



 The axiom system here was developed by Jan \L ukasiewicz, simplifying Frege's system: see \citet[p. 25]{boslogwol}. (The Polish notation used in this article is initially confusing, but has the wonderful feature that it is entirely unambiguous without the use of parentheses.)
See also \citet[ch. 5--6]{bosintlo}; he gives a direct proof of the completeness of a system based on \L\ at pp.\ 217--9.


\subsection*{Exercises}
\addcontentsline{toc}{subsection}{Exercises}


\begin{enumerate}
\item Using the model soundness proof for the restricted natural deduction system $ND_{\neg,\to}$, show, by analysing proofs constructed using the remaining rules of the full Halbach system $ND$, that the system $ND$ is sound.
\item The Completeness proof for $\mathcal{L}_{\neg,\to}$ relied on various lemmas concerning the behaviour of maximal consistent sets with regard to  the connectives $\neg$ and $\to$. Show that \begin{enumerate}
	\item analogous lemmas hold for $\wedge$, $\vee$ and $\bicond$.
	\item the Satisfiability Theorem holds for the full language \lone.
	\item  the full Natural Deduction system is complete for \lone\ with respect to the intended semantics.
\end{enumerate}  
\item \begin{enumerate}
	\item Show that the axiomatic system \L\ introduced is sound.
	\item Show that the axioms A2 and A3 of \L\ are provable in natural deduction: \begin{enumerate}
		\item A2. $\lvdash ((\phi \to (\psi \to \chi)) \to ((\phi \to \psi) \to (\phi \to \chi)))$;
		\item A3. $\lvdash ((\neg \psi \to \neg \phi) \to (\phi \to \psi))$.
	\end{enumerate}
\end{enumerate}
\item Suppose that $\Gamma,\psi\lvdash\phi$, shown by a proof $\Pi$. Define the \emph{$\psi$-transform} of $\Pi$, written $\Pi^{\psi}$, as the sequence that results from prefixing `$\psi\to$' to the front of every sentence in $\Pi$. Show that each sentence in $\Pi^{\psi}$ is provable from assumptions $\Gamma$ in \L. Since obviously $\psi$ appears on $\Pi$, $\psi\to\psi$ appears on $\Pi^{\psi}$, and that is provable from $\Gamma$. So you need to show that \begin{enumerate}
		\item the prefixed axioms (i.e., those $\psi\to\delta$ in $\Pi^{\psi}$ where $\delta$ is an axiom) are provable from $\Gamma$ (hint: use A1);
		\item the prefixed assumptions (i.e., those $\psi\to\gamma$ in $\Pi^{\psi}$ where $\gamma \in\Gamma$) are provable from $\Gamma$;
		\item If $\chi$ followed by \emph{modus ponens} from early claims in $\Pi$, $\psi\to\chi$ follows from earlier prefixed claims in $\Pi^{\psi}$ (hint: use A2).
	\end{enumerate}
 Using these results, show the deduction theorem for \L.
\item Show the other cases for the equivalence of $ND$ and \L:\footnote{These proofs are quite elusive; I propose treating this question as optional.}  \begin{enumerate}
	\item Show that if $\Gamma,\phi\lvdash\psi$ and $\Gamma,\phi\lvdash\neg\psi$, then $\Gamma\lvdash \neg\phi$.
	\item Show that if $\Gamma,\neg\phi\lvdash\psi$ and $\Gamma,\neg\phi\lvdash\neg\psi$, then $\Gamma\lvdash \phi$. (Equivalently, show that if $\Gamma\lvdash\neg\neg\phi$, then $\Gamma\lvdash\phi$.)
\end{enumerate} 


\end{enumerate}





}





