%!TEX root = edl.tex

\section{Necessity and Possibility Revisited}

\section[Sentential Modal Logic]{Sentential Modal Logic: Syntax and Semantics of \sml}

\section{Tableaux Derivations for \sml}

\section{Predicate Modal Logic}

\section{Temporal Logic}

\section{Intuitionism and Modal Logic}

In \autoref{c:induct}, I talked about intuitionism as a philosophical approach to mathematics. Classical and intuitionistic mathematicians agree that mathematical proofs aspire to the rigour of formally valid arguments. So intuitionistic constraints on good mathematical proof cannot be adequately represented using classical entailment.

But with the additional expressive resources that modal logics provide, we can give a logical language that is both fundamentally classical, but also able to naturally delineate the characteristic patterns of intuitionistic proof. As was suggested in my earlier discussion, the key is to note that intuitionists require not just that valid arguments preserve truth in virtue of form, but that they preserve \emph{constructibility}.

Burgess §6.5

Glivenko's theorem? 

