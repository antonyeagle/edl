

\section{Natural Deduction Proofs}
\paragraph{Proof}

In the last chapter, we saw that using truth tables was a decidable method for evaluating sequents, but it is cumbersome and inelegant. Here, we consider elegant and powerful \emph{proof systems} for sentential logic.

A \emph{proof} in the informal sense is something which establishes the truth of some claim. Our proofs are structures made out of sentences of \lone, with the following desirable properties: \begin{itemize}
	\item Each step in the proof involves an obviously correct rule.
	\item Whenever the earlier sentences are true, the later sentences will be also; so the terminal conclusion of a proof should always be true whenever the assumptions are.
	\item All the logical truths should have a proof.
\end{itemize}

In this chapter, we'll see that there do exist proofs that meet these conditions for \lone; and thus that we can establish various claims in \lone\ without needing to consider all the \lone-structures.

\paragraph{Natural Deduction: Assumptions and $\vdash$}

The first proof system is the \emph{natural deduction} system introduced in The Logic Manual. 

In this system, every proof begins with \emph{assumptions}. \emph{Any sentence may be an assumption.} A proof terminates with a \emph{conclusion}. 

 Whenever there is a correctly constructed proof that $\phi$, on the basis of assumptions $\gamma_{1},\ldots,\gamma_{n}$, we write this \emph{syntactic sequent}: {$$\gamma_{1},\ldots,\gamma_{n}\vdash\phi.$$} This is to be read `$\phi$ is \emph{provable from} $\gamma_{1},\ldots,\gamma_{n}$', or `$\gamma_{1},\ldots,\gamma_{n}$ \emph{syntactically entail(s)} $\phi$'. If $\phi$ can be proved with no assumptions at all, we write $\vdash\phi$; in that case, we say that $\phi$ is a \emph{theorem}.

Any sentence $\phi$ may be assumed; applying no rules at all, the proof therefore terminates with $\phi$. So $\phi\vdash\phi$; not perhaps the most interesting result, but nevertheless intuitively correct.




\paragraph{Natural Deduction Rules}

 As any assumption is a proof on its own, our rules are rules for constructing new proofs from existing proofs, often (but not always) retaining the assumptions of the existing proofs.

Some rules permit an assumption to be \emph{discharged}. Suppose you have a proof of $\phi$ on the basis of assumption $\gamma$.  You thus show that \emph{if} you had a proof of $\gamma$, \emph{then} you could construct a proof of $\phi$; and this on its own is a proof of $\gamma\to\phi$, \emph{whether or not $\gamma$ is actually true}. The assumption $\gamma$ has been discharged.
 Note that $\gamma$ needn't be used, or even occur in the proof, to be discharged: the rules entitle one to discharge \emph{every} occurence of a sentence – even if there are none!

 We can consider two types of obvious rules: those which construct a proof of a new sentence of less complexity than sentence proved by the existing proof(s), and those which construct a proof of a sentence of increased complexity. We call the first type of rule \emph{elimination} rules, and the second type \emph{introduction} rules.




{
\paragraph{Natural Deduction Rules for $\mathcal{L}_{\neg,\to}$}
Let us consider for the moment the fragment of \lone\ involving only $\neg$ and $\to$, $\mathcal{L}_{\neg,\to}$. We have in this fragment four rules, laid out in Table \ref{tfour}, elimination rules and introduction rules for each connective. I write $[\gamma]$ to show that the rule permits the assumption $\gamma$ to be discharged.
 \begin{table}\label{tfour}
 	\centering
	\begin{tabular}{cc}
	
	\begin{prooftree}
	\[ [\gamma] \leadsto \phi
	\]
 \justifies \gamma\to\phi \using{{\to}\text{Intro}}
\end{prooftree} & \begin{prooftree}
	\gamma \to \phi \qquad \gamma \justifies \phi \using{{\to}\text{Elim}}
\end{prooftree}\\[30pt]
\begin{prooftree}
	\[ [\phi] \leadsto \psi\]
	\[ [\phi] \leadsto \neg\psi\] \justifies \neg\phi \using{\neg\text{Intro}} 
\end{prooftree} &
\begin{prooftree}
	\[ [\neg\phi] \leadsto \psi\]
	\[ [\neg\phi] \leadsto \neg\psi\] \justifies \phi \using{\neg\text{Elim}} 
\end{prooftree}
	\end{tabular} \caption{Natural Deduction Rules for $\mathcal{L}_{\neg,\to}$}
 \end{table}
Call this system of rules, plus the rules about assumptions, $ND_{\neg,\to}$.

\paragraph{Deduction Theorem for $ND_{\neg,\to}$}

\begin{theorem}[Deduction Theorem] $\Gamma \vdash \psi \to \phi$ iff $\Gamma,\psi \vdash \phi$.\end{theorem}
Suppose $\Gamma,\psi \vdash \phi$. Then we can apply $\to$Intro as follows: 
	\begin{equation*}
		\begin{prooftree}
\[\Gamma, [\psi] \leadsto \phi\] 
			\justifies \psi \to \phi \using{\to}\text{Intro}
				\end{prooftree}
	\end{equation*}
	
Therefore, $\Gamma\vdash\psi\to\phi$. Similarly, suppose $\Gamma\vdash\psi\to\phi$: \begin{equation*}
	\begin{prooftree}
		\[\Gamma \leadsto \psi\to\phi\] 
		\qquad \psi \justifies \phi \using {\to}\text{Elim}
	\end{prooftree}
\end{equation*} That is, $\Gamma,\psi\vdash\phi$. 


\paragraph{$\neg$Elim can be replaced by $\neg\neg$Elim}

Consider the rule: \begin{equation*}
	 \begin{prooftree}\neg\neg\phi \justifies \phi \using \neg\neg\text{Elim}
	\end{prooftree}
\end{equation*}
This rule can obviously be derived from our rules: the correctness of the syntactic sequent $\neg\neg\phi \vdash \phi$ is demonstrated by this proof: \begin{equation*}
	\begin{prooftree}
		\neg\neg\phi \quad [\neg\phi] \justifies \phi \using \neg\text{Elim} 
	\end{prooftree}
\end{equation*}   But with $\neg\neg$Elim we can show the rule $\neg$Elim is redundant, using $\neg$Intro: \begin{equation*}
	\begin{prooftree}
\[		\[[\neg \phi] \leadsto \psi\]\quad \[[\neg\phi] \leadsto \neg\psi\] \justifies \neg\neg\phi \using \neg\text{Intro} \] \justifies \phi\using\neg\neg\text{Elim}
	\end{prooftree}
\end{equation*} 


\section{Soundness}
\paragraph{Soundness and Completeness}

A proof system $P$ in a given language is \emph{sound} with respect to a semantic interpretation of the language just in case whenever there is a proof which establishes $\Gamma \vdash_{P} \phi$, it is the case that $\Gamma \vDash \phi$.

A proof system $P$ in a given language is \emph{complete} with respect to a semantic interpretation of the language just in case whenever it is the case that $\Gamma \vDash \phi$, there is a proof which establishes $\Gamma \vdash_{P} \phi$.

In this section and the next, we will show that the natural deduction system $ND_{\neg,\to}$ just introduced is \emph{sound and complete} for the standard semantics of $\mathcal{L}_{\neg,\to}$. Given the expressive adequacy of $\mathcal{L}_{\neg,\to}$, that also shows the completeness of the natural deduction system with respect to \lone; to show the soundness of that system, we also need to show that the other rules of the system $ND$ are also sound.


\paragraph{Soundness of $ND_{\neg,\to}$}

\begin{theorem} The Rules of $ND_{\neg,\to}$ are sound.
\begin{proof} Suppose that $\Gamma \vdash \phi$. We show by induction on the complexity of proofs that $\Gamma \vDash \phi$.

\emph{Base case}: The least complex proof is a single node, $\phi$, establishing $\phi \vdash \phi$. In this case, any structure which assigns $T$ to $\phi$ clearly assigns $T$ to $\phi$, so $\phi\vDash\phi$, i.e., $\Gamma\vDash\phi$.

\emph{Induction step}: Suppose $\Gamma\vdash\phi$, established by applying one of the natural deduction rules to some previously obtained proofs to obtain a proof of $\phi$. There are four cases: 
 \begin{description}
	\item [$\to$Elim] From a proof of $\psi$ on assumptions $\Delta$, and a proof of $\psi\to\phi$ on assumptions $\Theta$, obtain a proof of $\phi$ on assumption $\Gamma=\Delta\cup\Theta$. By the induction hypothesis, $\Delta \vDash \psi$ and $\Theta \vDash \psi\to\phi$, so every structure which makes all of the members of $\Gamma$ true, makes $\psi$ and $\psi\to\phi$ true, and therefore $\phi$ must be true in all such structures, showing that $\Gamma\vDash\phi$.
	\item [$\to$Intro] From a proof of $\phi$ on assumptions $\Gamma \cup \{\psi\}$, apply $\to$Intro to obtain a proof of $\psi\to\phi$ on assumption $\Gamma$ (discharging $\psi$). But since by hypothesis $\Gamma,\psi \vDash\phi$, by the deduction theorem, $\Gamma \vDash \psi\to\phi$.
	\item [$\neg$Elim] From a proof of $\psi$ on assumptions $\Gamma,\neg\phi$ and a proof of $\neg\psi$ on assumptions $\Gamma,\neg\phi$, obtain a proof of $\phi$ on assumptions $\Gamma$, discharging $\neg\phi$. Since $\Gamma,\neg\phi \vDash \psi$ and $\Gamma,\neg\phi\vDash\neg\psi$, it follows that $\Gamma,\neg\phi\vDash$; therefore $\Gamma\vDash\phi$.
		\item [$\neg$Intro] From a proof of $\psi$ on assumptions $\Gamma,\phi$ and a proof of $\neg\psi$ on assumptions $\Gamma,\phi$, obtain a proof of $\neg\phi$ on assumptions $\Gamma$, discharging $\phi$. Since $\Gamma,\phi \vDash \psi$ and $\Gamma,\phi\vDash\neg\psi$, it follows that $\Gamma,\phi\vDash$; therefore $\Gamma\vDash\neg\phi$.
\end{description}
Thus all the rules preserve soundness; no proof in $ND_{\neg,\to}$ can be constructed other than using these rules; so $ND_{\neg,\to}$ is sound. \end{proof}\end{theorem}




\section{Completeness}
\paragraph{Consistency}

A set of sentences $\Gamma$ is called \emph{(syntactically) consistent} iff $\Gamma\not\vdash\neg(\phi\to\phi)$. Otherwise it is \emph{inconsistent}, also written $\Gamma\vdash$. Equivalently, consider this proof:
\begin{equation*}
	\begin{prooftree}
		\[ \Gamma \leadsto \neg(\phi \to \phi)\] 
		\[[\phi] \justifies \phi\to\phi \using \to\text{Intro}\] 
		\justifies \psi \using \neg\text{Elim}
	\end{prooftree}
\end{equation*} If we have a proof of $\neg(\phi\to\phi)$, we have a proof of arbitrary $\psi$ from the assumption $\Gamma$; so we can say that $\Gamma$ is consistent iff there is a sentence \emph{not provable} from $\Gamma$. 

A set of sentences $\Gamma$ is \emph{maximally consistent} iff it is consistent and for any $\phi$, if $\phi\notin\Gamma$, then $\Gamma,\phi\vdash$ (one cannot add any more sentences while remaining consistent).
\paragraph{Properties of Maximal Consistent Sets: Deductive Closure}

\begin{lemma}[Deductive Closure]
	If $\Gamma$ is a maximal consistent set, then if $\Gamma \vdash \phi$, then $\phi \in \Gamma$. \begin{proof}
	{	 Suppose that $\Gamma\vdash\phi$, but $\phi\notin\Gamma$. Since $\Gamma$ is maximal consistent, $\Gamma,\phi\vdash \neg(\phi\to\phi)$. Using $\to$Intro and discharging $\phi$, we see that $\Gamma\vdash\phi\to(\neg(\phi\to\phi))$. But now use $\to$Elim, and obtain $\Gamma\vdash\neg(\phi\to\phi)$ i.e., $\Gamma$ is inconsistent.}
	\end{proof}
\end{lemma} 
}

{
 \paragraph{Properties of Maximal Consistent Sets: Negation Completeness}

\begin{lemma}[Negation Completeness]
	If $\Gamma$ is a maximal consistent set, then $\neg\phi \in \Gamma$ iff $\phi \notin\Gamma$. \begin{proof}
	{ L to R:	If both $\phi$ and $\neg\phi$ were in $\Gamma$, we could prove $\neg(\phi\to\phi)$ by an application of $\neg$Elim, showing $\Gamma$ inconsistent. 
		
		R to L: If $\phi\notin\Gamma$, then  $\Gamma,\phi\vdash\neg(\phi\to\phi)$: 
		 \begin{equation*}
			\begin{prooftree}
				\[\Gamma, [\phi] \leadsto \neg(\phi\to\phi)\]
				\[[\phi] \justifies \phi\to\phi\using \to\text{Intro}\]
				\justifies \neg\phi \using\neg\text{Intro}
			\end{prooftree}
		\end{equation*}  Therefore, $\Gamma\vdash\neg\phi$; by Deductive Closure Lemma, $\neg\phi\in\Gamma$.}
	\end{proof} 
\end{lemma}
 

\paragraph{Properties of Maximal Consistent Sets: Conditional Completeness}

\begin{lemma}[Conditional Completeness]
	If\, $\Gamma$ is any maximal consistent set, then	$\phi\to\psi\in\Gamma$ iff whenever $\phi\in\Gamma$, it follows that $\psi\in\Gamma$.
	\begin{proof}
{	L to R:	Suppose $\phi\to\psi\in\Gamma$. Then $\Gamma\vdash\phi\to\psi$; and if $\phi\in\Gamma$, then $\Gamma\vdash\phi$. Apply $\to$Elim, and establish $\Gamma\vdash\psi$, i.e., $\psi\in\Gamma$.
		
		R to L: Suppose that whenever $\phi\in\Gamma$, $\psi\in\Gamma$. Consider two cases: \begin{enumerate}
			\item $\phi\in\Gamma$. So $\Gamma=\Gamma\cup\{\phi\}$. Moreover, $\psi\in\Gamma$, so $\Gamma,\phi\vdash\psi$. Apply $\to$Intro and discharge $\phi$ to show $\Gamma\vdash \phi\to\psi$.
			\item $\phi\notin\Gamma$. Then $\Gamma,\phi\vdash\neg(\phi\to\phi)$. But then $\Gamma,\phi \vdash\psi$ (by a similar proof as just above); applying $\to$Intro and discharge $\phi$, $\Gamma\vdash\phi\to\psi$.
		\end{enumerate}
}		
	\end{proof} 
\end{lemma}

\paragraph{Properties of Maximal Consistent Sets: Satisfiability}

\begin{theorem}[Maximal Consistent Sets Satisfiable]
	Every maximal consistent set is satisfiable (i.e., has a model).\begin{proof}
	{	Suppose $\Gamma$ is a maximal consistent set in $\mathcal{L}_{\to,\neg}$. For $s$ a sentence letter, define a structure $\mathscr{A}$: {$\mathscr{A}(s) = T$ iff $s\in\Gamma$}.
			
			For any $\phi$, $\val{\phi}{A}=T$ iff $\phi\in\Gamma$. Proof by induction on complexity of sentences. The base case, $\phi$ a sentence letter, is immediate. 
		
	Suppose $\phi$ is complex. There are two cases: \begin{enumerate}
		\item $\phi=\neg\psi$. $\phi\in\Gamma$ iff $\psi\notin\Gamma$, by Negation Completeness; iff by the induction hypothesis, $\val{\psi}{A}=F$; iff by the clause on $\neg$, $\val{\phi}{A}=T$.
		\item $\phi=(\psi\to\chi)$. $\phi\in\Gamma$ iff if $\psi\in\Gamma$ then $\chi\in\Gamma$ (by Conditional Completeness); iff if $\val{\psi}{A}=T$ then $\val{\chi}{A}=T$; iff $\val{\phi}{A}=T$.
	\end{enumerate} }\end{proof}
\end{theorem} 

\paragraph{Completeness}

\begin{theorem}[Completeness for $\mathcal{L}_{\neg,\to}$]
	If $\Gamma\vDash\phi$, then $\Gamma\vdash\phi$. \begin{proof}
		{ We prove the \emph{contrapositive}. So assume that $\Gamma\not\vdash\phi$. (Assuming that $\Gamma$ itself is consistent; if it is not, then clearly the theorem holds fairly trivially.)
		
		We'll construct a maximal consistent set $\Gamma^{+}$, a \emph{superset} of $\Gamma$, such that $\Gamma^{+}\not\vdash\phi$.  By consistency of $\Gamma^{+}$, $\phi\notin\Gamma^{+}$; by the Negation Completeness lemma, $\neg\phi\in\Gamma^{+}$.
		
		But by the Satisfiability Theorem, $\Gamma^{+}$ has a model, $\mathscr{A}$. Since $\Gamma\subseteq\Gamma^{+}$, for all $\gamma\in\Gamma$, $\val{\gamma}{A}=T$; but since $\neg\phi\in\Gamma^{+}$, $\val{\neg\phi}{A}=T$, so $\val{\phi}{A}=F$. So $\Gamma\not\vDash\phi$, as required.
		
		We now show how to construct $\Gamma^{+}$ from $\Gamma$, completing the proof.
		}
	\end{proof}
\end{theorem}

\paragraph{The Construction of Maximal Consistent $\Gamma^{+}$}

Supposing $\Gamma\not\vdash\phi$, construct $\Gamma^{+}$ as follows. \begin{enumerate}
	\item Enumerate the sentences of $\mathcal{L}_{\neg,\to}$, $\{\sigma_{1},\sigma_{2},\ldots\}$. 
	\item Let $\Gamma_{0}=\Gamma$.
	\item \begin{equation*}
		\Gamma_{n+1} = \begin{cases}
			\Gamma_{n}\cup\{\neg\sigma_{n+1}\} &\text{if } \Gamma_{n},\sigma_{n+1}\vdash\phi;\\
			\Gamma_{n}\cup\{\sigma_{n+1}\} &\text{otherwise (i.e., if $\Gamma_{n},\sigma_{n+1}\not\vdash\phi$)}.
		\end{cases}
	\end{equation*}
	\item Let $\Gamma^{+}=\bigcup_{i} \Gamma_{i}$.
\end{enumerate}
It is clear that \emph{$\Gamma\subseteq\Gamma^{+}$}. It is also clear that \emph{$\Gamma^{+}\not\vdash\phi$}, for at no stage was a sentence added to any $\Gamma_{n}$ that permitted a proof of $\phi$. 

Finally, $\Gamma^{+}$ is a \emph{maximal consistent set}, since (by construction), for every $\sigma_{i}$, either $\sigma_{i}\in\Gamma^{+}$ or $\neg\sigma_{i}\in\Gamma^{+}$.  Thus $\Gamma^{+}$ has the properties required to prove Completeness stated just above.

}



{
\paragraph{Completeness for \lone}

We've shown the Completeness Theorem for the restricted language $\mathcal{L}_{\neg,\to}$. But by Expressive Adequacy, any sentence of \lone\ can be expressed in $\mathcal{L}_{\neg,\to}$. So we can assure ourselves that any truth function expressible in \lone\ can be expressed in $\mathcal{L}_{\neg,\to}$, and any valid argument in \lone\ has a corresponding provably correct sequent in $\mathcal{L}_{\neg,\to}$.

For completeness of \lone, we cannot rest there. What if the rules of \lone\ don't suffice to show the equivalence of some arbitrary \lone\ sentence $\phi$ with a sentence only involving $\neg$ and $\to$? In that case, while a sentence logically equivalent to $\phi$ is provable, $\phi$ may not be.

We thus need additional lemmas showing that maximal consistent sets \begin{itemize}
	\item Contain $\phi \wedge \psi$ iff they contain both $\phi$ and $\psi$;
	\item Contain $\phi \vee \psi$ iff they contain at least one of $\phi$ or $\psi$;
	\item Contain $\phi \bicond \psi$ iff they contain $\phi$ iff they contain $\psi$.
\end{itemize} 
 Then the Satisfiability Theorem must be extended to include these new cases; and the remainder of the theorem is proved as above.






\paragraph{Compactness  Again}

Consideration of the nature of natural deduction proofs should convince you that any conclusion $\phi$ drawn on the basis of assumptions $\Gamma$ can be constructed by some finite application of the rules to finitely many sentences in $\Gamma$. Provability is essentially a finite notion: proofs are things that can be \emph{constructed}.

This observation, informal though it is, \emph{can} be made precise, to yield another proof of the \emph{Compactness theorem}. Since, by Completeness, every correct semantic sequent $\Gamma\vDash$ is provable, and every proof is a finite object using only finitely many members of $\Gamma$, then by Soundness there is a correct semantic sequent $\Gamma'\vDash$ where $\Gamma'$ is a finite subset of $\Gamma$. 




\section{A Little More Proof Theory}

`Proof theory' is the mathematical examination of proofs conceived of as mathematical objects in their own right. In this section we explore a little of what this involves. We've already seen one some proof theory in section 1 of this chapter, where we showed the redundancy of the proof rule $\neg$Elim in the presence of $\neg\neg$Elim, and vice versa.

\paragraph{The Language $\mathcal{L}_{DP}$}

Consider the language $\mathcal{L}_{DP}$ which contains all sentence letters, and all sentences involving only the connectives $\wedge,\to$, and also contains a \emph{sentential constant} $\bot$. A sentential constant is syntactically like a sentence letter, but (unlike sentence letters) it gets a constant interpretation in every structure – in this case, for any structure $\mathscr{A}$, $|\bot|_{\mathscr{A}}=F$. The atomic sentences of $\mathcal{L}_{DP}$ thus include $\bot$ and all sentence letters; complex sentences are then built up in the usual way. 

\paragraph{Translating between $\mathcal{L}_{DP}$ and \lone} It's clear that the \lone\ sentence `$\neg \phi$' expresses the same truth function as the $\mathcal{L}_{DP}$ sentence `$\phi \to \bot$' (consideration of a truth table will suffice to show this). 
Using this translation, we can give $\mathcal{L}_{DP}$-versions of the natural deduction rules for $\neg$ in \lone, as seen in Table \ref{negldp}.
\begin{table}
	\centering
	\begin{tabular}{cc}
		\begin{prooftree}
			\[[\phi] \leadsto \psi\] \[[\phi] \leadsto \psi \to \bot\] \justifies \phi \to \bot \using{\neg\text{Intro}_{DP}}
		\end{prooftree}	&	\begin{prooftree}
					\[[\phi\to\bot] \leadsto \psi\] \[[\phi\to\bot] \leadsto \psi \to \bot\] \justifies \phi \using{\neg\text{Elim}_{DP}}
				\end{prooftree}		
	\end{tabular}\caption{Negation rules in $\mathcal{L}_{DP}$\label{negldp}}
\end{table}

With the negation rules as specified, we can prove this: \begin{theorem}[Alternative Negation Rules for $\mathcal{L}_{DP}$] \label{altneg}
 $\Pi$ is a proof of some sequent in the language $\mathcal{L}_{DP}$ that makes a use $u$ of one of the Table \ref{negldp} versions of Halbach's rules iff there is another proof $\Pi'$ of the same sequent which replaces that use $u$ by a construction which only uses this rule for $\neg$ (together with the  rules for  $\to$):
	\begin{equation*}
		\begin{prooftree}
	\[	[\phi \to \bot] \leadsto \bot\] \justifies \quad\phi\quad \using{\bot_{\mathsf{C}}}
		\end{prooftree}
	\end{equation*} \begin{proof}
	\emph{If:} There are two cases. Case 1: Suppose $\Pi$ is a proof that makes use of $\neg$Intro$_{DP}$. Then we can replace that use by the following construction, which yields another proof $\Pi'$ which uses only the rules for $\to$:
	\begin{equation*}
			\begin{prooftree}
					\[\[[\phi] \leadsto \psi\] \[[\phi] \leadsto \psi \to \bot\] \justifies\bot \using{\to\text{Elim}}\]\justifies\phi\to\bot\using{\to\text{Intro}} 
			\end{prooftree}
	\end{equation*}
	Case 2: Suppose $\Pi$ is a proof that makes use of $\neg$Elim$_{DP}$. Then we can replace that use by the following construction, which yields another proof $\Pi'$ which uses only the new rule $\bot_{\mathsf{C}}$ and the rule $\to$Elim:
\begin{equation*}
		\begin{prooftree}
				\[	\[[\phi\to\bot] \leadsto \psi\] \[[\phi\to\bot] \leadsto \psi \to \bot\] \justifies \bot \using{\to\text{Elim}}							
				\] \justifies \phi \using{\bot_{\mathsf{C}}}
				\end{prooftree}
\end{equation*}

\emph{Only if}: Suppose $\Pi$ is a proof which makes use of the new rule $\bot_{\mathsf{C}}$. Then we may replace that use by the following construction which uses only the old rules for $\to$ and $\neg$Elim$_{DP}$.
\begin{equation*}
	 \begin{prooftree}
		\[[\phi \to \bot] \leadsto \bot\] \[[\bot]\justifies\bot\to\bot \using{\to\text{Intro}}\]\justifies\phi\using{\neg\text{Elim}_{DP}} 
	\end{prooftree}
\end{equation*}								
	\end{proof}
\end{theorem} 


\paragraph{Reducing Complexity of Proofs}

Return now to the \lone-fragment containing only $\to,\wedge,\neg$. Let the \emph{(degree of) complication} of a sentence be the number of connectives occurring in the sentence (a sentence letter has degree of complication 0, if $\phi$ and $\psi$ have degree of complication $m$ and $n$ respectively, then $(\phi \wedge \psi)$ has degree of complication $m+n+1$, etc.) We can now show: \begin{theorem}[Reducing Complexity of Proofs] \label{redcompl}
	If $\Pi$ is a proof in Halbach's system for $\mathcal{L}_{\to,\wedge,\neg}$  that $\Gamma \vdash \phi$, in which the most complicated sentence resulting from an application of $\neg$Elim is $\psi$, then (when $\psi$ is not atomic) there is a distinct proof $\Pi'$ of $\Gamma \vdash \phi$ in which the results of all applications of $\neg$Elim are \emph{less complicated} than $\psi$. Indeed, there exists a proof $\Pi^{\dag}$ of that sequent, in which the result of every application of $\neg$Elim is atomic.
	\begin{proof}
		Suppose that the most complicated result of $\neg$Elim in $\Pi$ is $\psi$. So part of $\Pi$ looks like this: \begin{equation*}
			\begin{prooftree}
				\[\Gamma \quad [\neg \psi] \justifies \leadsto \chi\] \[\Delta \quad [\neg \psi] \leadsto \neg\chi\] \justifies \psi \using{\neg\text{Elim}}
			\end{prooftree}
		\end{equation*} $\psi$ is complicated (not a sentence letter), so given there are three connectives in this language,  there are three cases: \begin{enumerate}
			\item $\psi = \neg \pi$. In which case replace that part of the proof sketched above with this one: \begin{equation*}
				\begin{prooftree}
					\[\Gamma \[[\pi] \quad [\neg \pi] \justifies \neg\neg\pi \using{\neg\text{Intro}}\] \leadsto \chi\]
					\[\Delta  \[[\pi] \quad [\neg \pi] \justifies \neg\neg\pi \using{\neg\text{Intro}}\] \leadsto \neg\chi\] \justifies \neg\pi \using{\neg\text{Intro}}
				\end{prooftree}
			\end{equation*}
		\item $\psi = \pi \wedge \xi$. Proof left for Exercise.
			\item $\psi = \pi \to \xi$. In which case replace that part of the proof sketched above with this one: \begin{equation*}
	\hskip-1.5cm			\begin{prooftree}
				\[	\[\Gamma  \[\[ [\pi \to \xi] \quad [\pi] \justifies \xi \using{\to\text{Elim}}\] \quad [\neg \xi] \justifies \neg(\pi \to \xi) \using{\neg\text{Intro}}\] \leadsto \chi\]
				\[\Delta \[\[ [\pi \to \xi] \quad [\pi] \justifies \xi \using{\to\text{Elim}}\] \quad [\neg \xi] \justifies \neg(\pi \to \xi) \using{\neg\text{Intro}}\] \leadsto \neg\chi\]
					 \justifies  \xi \using{\neg\text{Elim}}\]\justifies \pi \to \xi \using{\to\text{Intro}}
				\end{prooftree}
			\end{equation*}
		\end{enumerate}	

Notice that applying these replacements repeatedly to uses of $\neg$Elim in a proof will eventually replace all complex sentences in uses of $\neg$Elim by simpler ones. All sentences of \lone\ are finitely complex, and all proofs involve only finitely many applications of a given rule, this process terminates after a finite time with only atomic sentences as the conclusions of uses of $\neg$Elim. 
	\end{proof}
\end{theorem}

It is clear that we can put the results of Theorems \ref{altneg} and \ref{redcompl} together, to establish that, in the system which involves the normal rules for $\wedge$ and $\to$, and has $\bot_{\mathsf{C}}$ as its only other rule, if $\Pi$ is a proof in the system $DP$ that $\Gamma \vdash_{DP}\phi$, then there exists a proof of that sequent, $\Pi^{\dag}$, in which the result of any application of $\bot_{\mathsf{C}}$ is atomic. (I leave the proof as a problem.)






\section{Axioms}
\paragraph{Axiomatic Systems}

Another proof system we will now briefly consider is an \emph{axiomatic} (or Hilbert-style) proof system. 

This is familiar in mathematics: one begins with some \emph{axioms} (normally interpreted as `obvious' or `certain' truths), and by applications of some very simple truth-preserving rules of inference, one derives further truths.

If one's axioms are well chosen, one should be able to prove all and only truths about a given subject matter from the axioms characterising that subject matter. This was Euclid's method in the \emph{Elements}: he showed that all and only the truths of Euclidean geometry could be established on the basis of his well chosen axioms.

\paragraph{Axioms for Sentential Logic}

In an axiomatic system, one begins with theorems, the axioms, and applies the rules of inference to preserve theoremhood. 

Our axiom system \L\ is very simple. It has three \emph{axiom-schemata}, in the language $\mathcal{L}_{\neg,\to}$:
\begin{description}
	\item [A1] $\lvdash (\phi \to (\psi \to \phi))$;
	\item [A2]$\lvdash ((\phi \to (\psi \to \chi)) \to ((\phi \to \psi) \to (\phi \to \chi)))$;
	\item [A3] $\lvdash ((\neg \psi \to \neg \phi) \to (\phi \to \psi))$.
\end{description}
Any sentence of $\mathcal{L}_{\neg,\to}$ can be substituted into these axiom schemata to generate an instance of an axiom. Any axiom is a theorem.

The system has one rule (apart from substitution): \emph{\emph{Modus Ponens}}, then rule that if $\lvdash \phi$ and $\lvdash \phi\to\psi$, then $\lvdash\psi$. 

A \emph{proof} in \L\ is any sequence of theorems, each of which is either an axiom or follows by \emph{modus ponens} from earlier theorems.

\paragraph{A Proof in \L}

The system \L\ is  elegant, but proofs in it can be unnatural to say the least: (I annotate lines of the proof with the axiom scheme used.)
\begin{align*}
 1. &\lvdash (P\to((P\to P)\to P)) \to ((P\to(P\to P))\to(P\to P)) & \text{(A2)}\\
2. &\lvdash (P\to((P\to P)\to P)) &\text{(A1)}\\
3. &\lvdash ((P\to(P\to P))\to(P\to P)) &\text{(MP 1, 2)}\\
4. &\lvdash (P \to (P\to P)) & \text{(A1)}\\
5. &\lvdash (P\to P)&\text{(MP 3, 4)}
\end{align*} 

This is (apparently) the \emph{shortest} proof of $P \to P$ in \L!

\paragraph{Soundness of \L}

It is a trivial exercise (and hence left for an exercise) to show that \L\ is sound. That is, show: \begin{itemize}
	\item All the axioms of \L\ are tautologies.
	\item The rule of \L\ preserves tautologousness.
\end{itemize}

\paragraph{Equivalence of \L\ and $ND_{\neg,\to}$}

We now show that, perhaps surprisingly, the proof systems \L\ and $ND_{\neg,\to}$ are equivalent – everything that can be proved in one can be proved in the other. This shows, too, that \L\ is sound and complete.

\paragraph{Everything provable in \L\ is provable in $ND$}

\begin{theorem}
	If $\lvdash\phi$ then $\vdash\phi$.\end{theorem} \begin{proof}
		{ The proof is straightforward: show that each axiom is provable in $ND$, and show that the rule \emph{modus ponens} is a valid rule in $ND$.
		
		The second part is easy; if $\vdash\phi$ and $\vdash\phi\to\psi$, then  appending an instance of $\to$Elim to the preceding proofs is a proof of $\psi$.
		
		Then it is just a matter of proving the axioms. I show A1: \begin{equation*}
			\begin{prooftree}
				\[ [\phi] \justifies \psi\to\phi \using\to\text{Intro}\]
				\justifies \phi \to (\psi\to\phi) \using \to\text{Intro}
			\end{prooftree}
		\end{equation*}}
	\end{proof}



\paragraph{Proofs from Assumptions in \L}

The ugly proof of $\lvdash P\to P$ two paragraphs back shows the benefits of working with \emph{assumptions}. A proof of $\phi$ from assumptions $\Gamma$ in \L\ is a sequence of axioms, \emph{members of $\Gamma$}, or follows from preceding members by \emph{modus ponens}, and the last line of which is $\phi$.
This is a legitimate notion, because one can show: \begin{theorem}[Deduction Theorem for \L]
	 $\Gamma\lvdash\phi$ iff for some finite set of sentences $\gamma_{1},\ldots,\gamma_{n}$ actually used in the proof of $\phi$, $\lvdash (\gamma_{1}\to(\gamma_{2} \to \ldots (\gamma_{n} \to \phi)\ldots))$.
\end{theorem} It suffices to show that $\Gamma,\psi\lvdash\phi$ iff $\Gamma\lvdash\psi\to\phi$. I leave the proof for an exercise.
}


{
\paragraph{Everything Provable in $ND$ is provable in \L}
\begin{theorem}
	If $\vdash \phi$, then $\lvdash \phi$. \begin{proof}
		{ Induction on length of proofs: we show that the shortest proofs of $ND$ are provable in \L, and	that the rules are provable transitions.
		
	Certainly the shortest proof, the single node $\phi$, shows that $\phi\vdash\phi$; but $\phi\lvdash\phi$ is shown by the deduction theorem given $\lvdash \phi\to\phi$.
	
	Suppose we extend some existing  proofs by applying the rules of $ND$: \begin{description}
		\item[$\to$Elim] We have $\Gamma\vdash\phi$ and $\Gamma\vdash\phi\to\psi$, and extend by $\to$Elim. But this is obviously \emph{modus ponens}, so $\Gamma\lvdash\psi$.
	\end{description}
The case of $\to$Intro is shown by the deduction theorem. I leave you to show the cases for negation.}	\end{proof}
\end{theorem}
}

{\small
\subsection*{Further Reading}
\addcontentsline{toc}{subsection}{Further Reading}

Natural deduction was invented  by \citet{geninvinl}, the founder of proof theory. 
The proof theory of natural deduction systems discussed in section 4 of this chapter is based on \citet[39--41]{pranatde}; he uses Theorems \ref{altneg} and \ref{redcompl}, and the last problem, and proves the famous `cut elimination' theorem for natural deduction, unfortunately a topic beyond the scope of these notes.



 The axiom system here was developed by Jan \L ukasiewicz, simplifying Frege's system: see \citet[p. 25]{boslogwol}. (The Polish notation used in this article is initially confusing, but has the wonderful feature that it is entirely unambiguous without the use of parentheses.)
See also \citet[ch. 5--6]{bosintlo}; he gives a direct proof of the completeness of a system based on \L\ at pp.\ 217--9.


\subsection*{Exercises}
\addcontentsline{toc}{subsection}{Exercises}


\begin{enumerate}
	\item \begin{enumerate}
\item Devise introduction and elimination rules for a natural deduction system for the language $\mathcal{L}_{\uparrow}$ with the Sheffer Stroke as its only connective, being sure to fully justify your answer. How many such rules do we need?
	\item Suppose in a system of natural deduction $ND$ for \lone\ we replaced the $\neg$Elim rule by the following, \emph{ex falso quodlibet}: \begin{equation*}
		\begin{prooftree}
			\psi \qquad \neg\psi \justifies \phi \using \text{EFQ}
		\end{prooftree}
	\end{equation*}
\begin{enumerate}
	\item Is the resulting system $ND'$ equivalent (in terms of what can be proved) to the original system? (Justify your answer; it is somewhat difficult to prove conclusively, but give some reasons for your view.)
	\item What happens if we, in addition, replace $\neg$Intro by the following rule, \emph{tertium non datur}, yielding the system $ND''$? \begin{equation*}
		\begin{prooftree}
			\[[\phi] \leadsto \psi\] \[[\neg\phi] \leadsto \psi\] \justifies \psi \using\text{TND}
		\end{prooftree}
	\end{equation*}
\end{enumerate}	
\item Consider the natural deduction system $ND_{\to,\vee}$ for the language $\mathcal{L}_{\to,\vee}$ in which the only connectives are $\to,\vee$. Suppose we introduce to the language the zero-place sentential constant $\bot$, and add the following rules to the proof system, yielding $ND_{\to,\vee,\bot}$: \begin{equation*}
	\begin{prooftree}
		\bot \justifies \phi \using\bot
	\end{prooftree}\qquad
	\begin{prooftree}
		\[[\phi \to \bot] \leadsto \bot\]
		\justifies \phi \using \text{C}
	\end{prooftree}
\end{equation*}\begin{enumerate}
	\item Give a natural translation between $\mathcal{L_{\to,\vee,\bot}}$  and $\mathcal{L}_{\to,\vee,\neg}$.
	\item Using that translation, can you show that the proof system $ND_{\to,\vee,\bot}$ just introduced is equivalent to the natural deduction system $ND_{\to,\vee,\neg}$ involving just Halbach's rules for $\to$, $\vee$ and $\neg$?
	\end{enumerate}
\item A mystery connective $\oplus$ has the following introduction and elimination rules: 

\begin{equation*}
	\begin{prooftree}
		\phi\oplus\psi \justifies \psi \using\oplus\text{Elim}
	\end{prooftree}\qquad \begin{prooftree}
		\phi \justifies \phi\oplus\psi \using\oplus\text{Intro}
	\end{prooftree}
\end{equation*}

\begin{enumerate}
	\item Show that no system containing a connective defined by these rules is sound. 
	\item What limits should be placed on the ability of introduction and elimination rules to define or characterise the inferential role of a connective in light on this result? (This question invites discussion, not a definitive answer. You may wish to consult \citet{prirunint}.)
	\end{enumerate}
	\item  Relatedly to 1.(d).ii: \label{fouronee}\begin{enumerate}
		\item Check, by means of truth tables, that $(((P\to Q)\to P)\to P)$ is a tautology.	
		\item Give a natural deduction proof of $(((P\to Q)\to P)\to P)$.
		\item Every valid argument involving only sentences containing $\wedge$ as their only connective can be proved valid using just the rules $\wedge$Intro and $\wedge$Elim. In that sense, $\wedge$ is completely characterised by its introduction and elimination rules. What do the previous results show about $\to$ in this connection?
\end{enumerate}
\end{enumerate}
\item Using the model soundness proof for the restricted natural deduction system $ND_{\neg,\to}$, show, by analysing proofs constructed using the remaining rules of the full Halbach system $ND$, that the system $ND$ is sound.
\item The Completeness proof for $\mathcal{L}_{\neg,\to}$ relied on various lemmas concerning the behaviour of maximal consistent sets with regard to  the connectives $\neg$ and $\to$. Show that \begin{enumerate}
	\item analogous lemmas hold for $\wedge$, $\vee$ and $\bicond$.
	\item the Satisfiability Theorem holds for the full language \lone.
	\item  the full Natural Deduction system is complete for \lone\ with respect to the intended semantics.
\end{enumerate}  
\item \begin{enumerate}
	\item Show that the axiomatic system \L\ introduced is sound.
	\item Show that the axioms A2 and A3 of \L\ are provable in natural deduction: \begin{enumerate}
		\item A2. $\lvdash ((\phi \to (\psi \to \chi)) \to ((\phi \to \psi) \to (\phi \to \chi)))$;
		\item A3. $\lvdash ((\neg \psi \to \neg \phi) \to (\phi \to \psi))$.
	\end{enumerate}
\end{enumerate}
\item Suppose that $\Gamma,\psi\lvdash\phi$, shown by a proof $\Pi$. Define the \emph{$\psi$-transform} of $\Pi$, written $\Pi^{\psi}$, as the sequence that results from prefixing `$\psi\to$' to the front of every sentence in $\Pi$. Show that each sentence in $\Pi^{\psi}$ is provable from assumptions $\Gamma$ in \L. Since obviously $\psi$ appears on $\Pi$, $\psi\to\psi$ appears on $\Pi^{\psi}$, and that is provable from $\Gamma$. So you need to show that \begin{enumerate}
		\item the prefixed axioms (i.e., those $\psi\to\delta$ in $\Pi^{\psi}$ where $\delta$ is an axiom) are provable from $\Gamma$ (hint: use A1);
		\item the prefixed assumptions (i.e., those $\psi\to\gamma$ in $\Pi^{\psi}$ where $\gamma \in\Gamma$) are provable from $\Gamma$;
		\item If $\chi$ followed by \emph{modus ponens} from early claims in $\Pi$, $\psi\to\chi$ follows from earlier prefixed claims in $\Pi^{\psi}$ (hint: use A2).
	\end{enumerate}
 Using these results, show the deduction theorem for \L.
\item Show the other cases for the equivalence of $ND$ and \L:\footnote{These proofs are quite elusive; I propose treating this question as optional.}  \begin{enumerate}
	\item Show that if $\Gamma,\phi\lvdash\psi$ and $\Gamma,\phi\lvdash\neg\psi$, then $\Gamma\lvdash \neg\phi$.
	\item Show that if $\Gamma,\neg\phi\lvdash\psi$ and $\Gamma,\neg\phi\lvdash\neg\psi$, then $\Gamma\lvdash \phi$. (Equivalently, show that if $\Gamma\lvdash\neg\neg\phi$, then $\Gamma\lvdash\phi$.)
\end{enumerate} 
\item A fellow student argues as follows: \begin{quotation}
	If the Natural Deduction system introduced by Halbach is complete, then whenever $\Gamma \vDash \phi$, there is a proof that shows $\Gamma \vdash \phi$. But proofs are finite; so if $\Gamma$ is infinite, there is no proof that $\Gamma \vdash \phi$ – so the system is not complete!
\end{quotation} How should you respond?
\item Prove the omitted case (where the complex sentence is a conjunction) in the proof of Theorem \ref{redcompl}.
\item Show  that, in the system which involves the normal rules for $\wedge$ and $\to$, and has $\bot_{\mathsf{C}}$ as its only other rule, if $\Pi$ is a proof in the system $DP$ that $\Gamma \vdash_{DP}\phi$, then there exists a proof of that sequent, $\Pi^{\dag}$, in which the result of any application of $\bot_{\mathsf{C}}$ is atomic.
\end{enumerate}





}





