%!TEX program = xelatex
%&encoding=UTF-8 Unicode
\documentclass[openany,leqno,11pt,draft]{book}
\usepackage{latexsym}
\usepackage{textcomp}
\usepackage{amsmath}
\usepackage[amsmath,amsthm,thmmarks,hyperref]{ntheorem}
\usepackage{amssymb}
\usepackage{fontspec}
\defaultfontfeatures{Mapping=tex-text,Ligatures={Common}}
\setmainfont[Numbers=OldStyle]{Linux Libertine O} % or DejaVu Sans
\setsansfont[Numbers=OldStyle]{Linux Biolinum O}
\setmonofont[Scale=MatchLowercase]{Menlo}
\usepackage[vargreek-shape=unicode]{unicode-math}
\setmathfont[Scale=MatchLowercase]{XITS Math} 
\usepackage{polyglossia}
\setdefaultlanguage[variant=australian]{english}

\usepackage{setspace}
\frenchspacing
\onehalfspacing
\raggedbottom





\usepackage{url}
	\urlstyle{sf} 
\usepackage[round,authoryear,sort]{natbib}
 	\bibpunct[:~]{(}{)}{;}{a}{,}{;~}
\usepackage[norule,bottom,multiple]{footmisc}
\usepackage{prooftree}
\usepackage[all]{xy}
\usepackage{phonetic}
\usepackage{metalogo}
\usepackage{stmaryrd}
\usepackage{tabulary}
\usepackage{booktabs}
\usepackage{multirow}
\usepackage{textcase}
\usepackage{fancyhdr}

\fancyhead{}
\fancyhead[CE]{\textsc{\lowercase{Elements of Deductive Logic}}}
\fancyhead[CO]{\textsc{\nouppercase{\rightmark}}}
\cfoot{\thepage} 
\renewcommand{\headrule}{}
\pagestyle{fancy}
\usepackage[lmargin=1.85cm,tmargin=2.425cm,paperwidth=6in,paperheight=9in,heightrounded=true,twoside]{geometry}	
\hfuzz=0.55pt

\setcounter{tocdepth}{3}
\setlength{\columnsep}{0.8cm}

\renewcommand{\refname}{Bibliography}


\theoremstyle{break}
\newtheorem{theorem}{Theorem}
\newtheorem{lemma}[theorem]{Lemma}
\newtheorem{corol}[theorem]{Corollary}
\theoremstyle{definition}
\newtheorem{definition}{Definition}
 \theoremstyle{remark}
\newtheorem{axiom}{Axiom}
\newcommand{\cquote}[1]{\ensuremath{\ulcorner #1 \urcorner}}

\makeatletter
\let\@oldquotation=\quotation
\let\@oldendquotation=\endquotation
\renewenvironment{quotation}
  {\@oldquotation\begin{singlespace}}
  {\end{singlespace}\@oldendquotation}
\makeatother

\newcommand{\lone}{\ensuremath{\mathcal{L}_{1}}}
\newcommand{\ltwo}{\ensuremath{\mathcal{L}_{2}}}
\newcommand{\lequ}{\ensuremath{\mathcal{L}_{=}}}
\newcommand{\val}[2]{\ensuremath{|#1|_{\mathscr{#2}}}}
\newcommand{\vals}[3]{\ensuremath{|#1|^{#3}_{\mathscr{#2}}}}
\newcommand{\valsa}[1]{\vals{#1}{A}{\alpha}}
\newcommand{\dd}{\text{\emph{\riota\,}}}
\newcommand{\lvdash}{\vdash_{\!\!\text{\vphantom{\normalsize|}\textnormal{Ł}}}}
\newcommand{\conc}{\mathop{\frown}}
\renewcommand{\labelenumi}{\textbf{\theenumi.}}%format enumerations
\newcommand{\eqdf}{=_{\mathrm{df}}}
\newcommand{\at}{\makeatletter{@}\makeatother}
\newcommand{\bicond}{\leftrightarrow}


\usepackage[unicode,pdfencoding=auto,pdfborder={0 0 0},plainpages=false,colorlinks=false,pdftitle={Elements of Deductive Logic},pdfauthor={Antony Eagle}]{hyperref}

% end of preamble



\begin{document}

\pagenumbering{roman}

% \renewcommand{\sectionmark}[1]{\markright{#1}}
\renewcommand{\sectionmark}[1]{}

\renewcommand{\chaptermark}[1]{%
 \markright{\MakeTextLowercase{#1}}{}}


\pagestyle{empty} 
\begin{center}
    ELEMENTS OF DEDUCTIVE LOGIC
\end{center}\newpage


\begin{titlepage}

    
 ~   \vskip 3cm
    {\Huge  Elements of Deductive Logic}
 \vskip 3cm

	{\large \textsc{an introduction to formal logic}

  \textsc{and elementary metatheory}\\[0.5cm] 

  \textsc{with exercises}}\\[5cm]
	

	
{\LARGE Antony Eagle}\\

\emph{University of Adelaide}
   

    
\end{titlepage}

\thispagestyle{empty}
\vspace{2cm}
\begin{center}
    
	\includegraphics{by-sa.png} \vspace{1cm}



{\small    	\emph{Elements of Deductive Logic} by Antony Eagle (\href{http://antonyeagle.org}{\nolinkurl{antonyeagle.org}}) is	licensed under the Creative Commons Attribution-ShareAlike 4.0 International License.\\[1cm]

To view a copy of this license, visit \href{http://creativecommons.org/licenses/by-sa/4.0/}{\nolinkurl{creativecommons.org/licenses/by-sa/4.0/}} or send a letter to Creative Commons, PO Box 1866, Mountain View, CA 94042, USA.\\[4cm]

This book is open source. \XeLaTeX\ source code for the book is available from my website, \href{http://antonyeagle.org}{\nolinkurl{antonyeagle.org}}, or by contacting me: \href{mailto:antony.eagle@adelaide.edu.au}{\nolinkurl{antony.eagle@adelaide.edu.au}}.

    Last updated  \today\\[1cm]
    


}

\end{center}


\pagestyle{fancy}

\chapter*{Preface}\markright{preface}

This is a textbook covering the basics of formal logic and elementary metatheory. Its distinguishing feature is that it has more emphasis on metatheory than comparable introductory textbooks. It was originally written to accompany lectures in the introductory-intermediate logic course at the University of Oxford, \emph{Elements of Deductive Logic}. It can be used independently of the Oxford lectures, of course. 

This book is made available under a Creative Commons license (CC BY-SA 4.0), which allows users to share the work as they like, and to create derivative and modified works for any purpose, as long as the original creator is appropriately credited, and derivative works are licensed in the same way. I hope this will encourage any users of the book to adapt it to their local environments.

\paragraph{Acknowledgements} Thanks especially to Ofra Magidor and Andrew Bacon for helpful feedback. Thanks also to James Studd, Brian King, Iain Atkinson, Katy Moe, Daniel Hoek, Bernhard Salow, Petra Staynova, Laura Castelli, and Syed Qader for comments on and corrections to the lectures and exercises. Several chapters of this book also formed the basis for part of the course \emph{Logic II} at the University of Adelaide in 2014; thanks to the students in that course for their feedback.


  

\newpage\tableofcontents\markright{contents}
\listoftables\listoffigures




\newpage \thispagestyle{empty} ~\\\newpage

\pagenumbering{arabic}

\setcounter{chapter}{-1}
\chapter{Introduction}

Introductory logic courses serve generally to tell you how to \emph{use} a new logical language. An introductory course will give you an introduction to the grammar, or \emph{syntax}, of the new formal language, tell you how to interpret the grammatical sentences by giving a \emph{semantics}, and most importantly from the perspective of applications, tell you how to translate from natural languages into the formal language. With a semantics and an translation scheme, you can now use a formal language to do various things more easily than you could otherwise: you can express claims precisely and without risk of ambiguity (No problem with `everybody loves somebody' in formal languages!), and check the validity or otherwise of arguments of certain forms, which – if you've done your translation well – reflects on the validity or otherwise of certain informal arguments you might be interested in as a philosopher or mathematician. If you have a formal proof system, giving rules for formal derivations in your languages, you can even construct arguments directly in the formal language in a more or less mechanical fashion, thus enabling you to prove various claims directly without having to detour through semantics.

Thinking about this for a second, though, we need some reassurance that the formal languages will do what we want them to. Is it true that if a formal proof is correct, then the argument it formalises is semantically valid? Is it the case that every valid argument has a corresponding formal derivation? Which things can and cannot be expressed in a given formal language? Is a formal language adequate to each argumentative use we might have for it in mathematics, or in philosophy? Are formal derivations really purely mechanical, or must there be human judgment involved too? These are questions \emph{about} formal logical languages, questions in what we might call the \emph{metatheory} of logic. Such questions are the focus of this book. Such questions aren't about how to use formal logics to do things, but about whether they are fit for purpose, whether they can in fact do the things we want them to do.\footnote{Such questions could easily be asked about natural languages too, though since the syntax and semantics of natural language are orders of magnitude harder to specify precisely, it is correspondingly harder to establish anything conclusively about the features of such languages.}

Answering such questions involves drawing a distinction between the \emph{object language}, the one we are talking about, and the \emph{metalanguage}, the language we are using to talk about the object language. For this book, the object languages are various more-or-less orthodox languages of classical logic: a propositional or sentential language \lone, a predicate language \ltwo, and a predicate language with identity \lequ. And the metalanguage is the mathematically enriched version of English we use to talk about formal languages. 

\paragraph{The Structure of this Book} I begin, in chapters \ref{c1} and \ref{c2}, by outlining some mathematical tools – specifically, some elementary \emph{set theory}, including the theory of functions – that will help us study and understand the formal languages which are our subject. In chapters \ref{c3} and \ref{c3a}, we'll briefly review the syntax and semantics of classical logic. Then we'll start proving some results in metatheory. We'll prove some small but still interesting results about the truth-functions, and about the intimate connections between conjunction (‘$\wedge$’), disjunction (‘$\vee$’), and negation (‘$¬$’). In chapter \ref{c4}, we'll prove our first `big' result: the so-called \emph{compactness} theorem for propositional logic. We'll take a break from metatheory in chapter \ref{c5}, to introduce and talk about \emph{formal derivations} in propositional logic, and introduce a particular kind of natural deduction derivation system. Then we'll prove two central results connecting derivability with validity: the \emph{soundness} and \emph{completeness} theorems for propositional logic. I also mention issues of computational implementation, in particular, the \emph{decidability} of propositional logic.
In chapter \ref{c6}, we'll review \emph{predicate} logic and (in chapter \ref{c7}) its natural deduction derivation system, and we'll prove soundness for standard predicate logic. I sketch the proof of completeness, and briefly consider issues of decidability. In chapter \ref{c8}, issues about predicate logic with identity are discussed, including the soundness of the natural deduction system we construct, and issues about numerical quantification and compactness are discussed. In the final chapter \ref{c9}, we turn from metatheory to issues about the relation between logical languages and natural languages, particularly focussing on issues about conditionals, entailment, designation and the theory of relations.

\paragraph{Exercises and Further Reading} This book also includes exercises, contained in a section entitled `Exercises' at the end of each chapter. These exercises are mostly review and consolidation of the material in the chapter, but some are intended to extend and stretch a student's knowledge of the material. No solutions are provided.

Many chapters also contain a section of `Further Reading', which contains references to books and articles that may provide additional information on particular topics of interest. Details of the items there referred to can be found in the Bibliography at the end of the book.

\paragraph{Appendices} The text makes free use of Greek letters as variables of various kinds; a quick refresher of the English names of these letters can be found in Appendix \ref{greek}. There are a number of defined expressions; a list of definitions can be found in Appendix \ref{defns}.

\paragraph{Some Other Useful Readings} The languages and derivation systems used in this book are those developed in Volker Halbach's \emph{Logic Manual} \citep{hallogma}. While the present book is self-contained, it may be helpful to review the material in Halbach's book as a useful adjunct. Another useful source of related material, that I drew on in a number of places in this book, is \citet{bosintlo}.  

\newpage

\chapter[Mathematical Induction and Proof]{Mathematical Induction and The Notion of Proof}\label{c1}
 

\section{Inductive Proofs}

\paragraph{Mathematical Proof} What is a proof? A proof of a theorem is just an argument, from assumptions that are known truths, that the theorem is true. In the mathematical case, the arguments involved are such that whenever the assumptions are true, the conclusions are true – they are supposed to be necessarily truth-preserving in English. And the assumptions in the mathematical case are the axioms of a particular mathematical theory: necessary mathematical truths that are collectively sufficient, when supplemented by appropriate definitions, to characterise a given body of mathematical truths. There are many issues surrounding the nature of proof that we will neglect; but basically, a proof is an absolutely \emph{conclusive} argument for a given claim, based on assumptions which are themselves known.


\paragraph{Mathematical Induction}
 We shall often use a powerful method of reasoning known as proof by \emph{mathematical induction}. 

The basic idea is simple. Assume we have a numbered sequence of things. Suppose (i) we can show that: the first member of the sequence has some property, and (ii) we can show that: if some member has the property, then then the subsequent member also has the property. Then we can conclude that \emph{every} member of the sequence has the property.

The easiest case is where the sequence \emph{is} the natural (counting) numbers $\mathbb{N}$, but the principle also works for sequences of other things that are ordered like the natural numbers.


\paragraph{The Weak Principle of Induction}

There are a number of formulations of induction. This one follows our intuitions closely. It is a principle about the legitimacy of a certain pattern of reasoning.

 \begin{definition}[Weak Principle of Induction]
 Assume there is a sequence of items $S$, ordered by the natural numbers. If both of the following conditions hold, \begin{enumerate}
   \item $P$ is true of the first
    member of $S$; and 
    \item  if $P$
    is true of the $n$-th member of $S$, then $P$ is true of $n+1$-th member of $S$;
 \end{enumerate} then, $P$ is true of every member of $S$.\end{definition}

To apply this principle, we first establish the \emph{base case} for condition (i); and then the \emph{induction case}, where we assume that (for some \emph{arbitrary} $n$) $P$ holds of the $n$-th member, and show it holds of the $n+1$-th member, to show condition (ii). (Condition (ii) is a conditional claim: it can hold even if nothing is $F$, as long as, if any $n$ is $F$, that suffices to establish that the subsequent thing is also $F$.) Having established those conditions, the principle of induction says, it is now legitimate to infer that every member of the sequence has the property in question. Here is an example.

\begin{theorem} For every natural number $n$, 
\begin{equation}0+1+2+\ldots+n = \frac{n(n+1)}{2}. \label{eqb}
\end{equation}

\begin{proof}
	We first prove the {\em base case}, that the
	property holds of $0$ (the first member of $\mathbb{N}$). Easy: $0 = \frac{0\cdot(0+1)}{2} = \frac{0}{2} = 0$. (Usually the base case is the easy part.)

	Now we wish to prove the {\em induction step}. That is, we assume the 
	property holds of $n$, and show it must hold of $n+1$. So we assume:
	$0+1+\ldots+n = (n(n+1))/2$.
  
\begin{eqnarray}
 (0+1+\ldots+n) & = & \frac{n(n+1)}{2}\\
    (0+1+\ldots+n)+n+1 & = & \left(\frac{n(n+1)}{2}\right) + n+1
    \label{eq:1}  \\
      & = & \frac{n(n+1)}{2} + \frac{2(n+1)}{2}
    \label{eq:2}  \\
      & = & \frac{(n+2)(n+1)}{2}
    \label{eq:3}  \\
      & = & \frac{(n+1)((n+1)+1)}{2}
    \label{eq:4}
\end{eqnarray} Equation \eqref{eq:4} is clearly just an instance of
Equation \eqref{eqb} with `$n+1$' in place of `$n$', as required. This
proves the induction step. \end{proof} \end{theorem} 





\paragraph{The Strong Principle of Induction}

\begin{definition}[Strong Principle of Induction] `For all $n$, if for all $m<n$, $P(m)$ then $P(n)$' implies `For
   all $n$, $P(n)$.' \end{definition}

The strong principle is just a conditional claim – there is no basis step. Why? Assume that for every $n$, if for all $m < n$ $F(m)$, then $F(n)$. Consider, in particular, $n=0$. Is it true that for all $m <0$, $F(m)$? Well, if it were false, there would be a number less than $0$ such that it isn't $F$. But there are no numbers less than zero, so in particular, there are none of them that are not $F$. So – in a degenerate or vacuous way – all numbers less than $0$ are $F$. So the antecedent of the conditional is satisfied, and since the conditional is true, the consequent must be true. The consequent is $F(0)$ – which is the basis step. Still, sometimes its useful to set out the basis step even when we're using strong induction.

The Weak Principle of Induction is equivalent to the Strong Principle. I show the left-to-right direction of the equivalence.

\begin{theorem}[Weak to Strong] The weak principle of induction entails the strong principle of induction.
  \begin{proof}
    We assume the conditional premise of the strong principle: that for every $n$, if for all $m < n$ $F(m)$, then $F(n)$. We will also assume the weak principle of induction. And we'll prove from those two assumptions that every $n$ is $F$. That is, assuming the validity of reasoning in accordance with the weak principle of induction, we can demonstrate the validity of reasoning in accordance with the strong principle.

    Here's how. We define a new condition, $G$, as follows: $$G(n) \eqdf \text{for all $m < n$, $F(m)$}.$$ (So a number has $G$ if all its predecessors have $F$.) We can re-write our conditional assumption using $G$ as follows: \begin{equation*}\tag{\dag}
      \text{For every $n$, if $G(n)$ then $F(n)$.}
    \end{equation*}

    \emph{Base case:} we show that $G(0)$. Trivial, by the argument just given.

    \emph{Induction step:} we want to show that, on the assumption that $G(n)$, it follows that $G(n+1)$. Assume $G(n)$. By (\dag), $F(n)$. By the definition of $G$, for all $m<n$, $F(m)$. So for all $m\leqslant n$, $F(n)$. But it is obvious that the collection of numbers less than or equal to $n$, is the same as the collection of numbers less than $n+1$. So we have shown that  for all $m< n+1$, $F(n)$. But by the definition of $G$ again, we've just shown $G(n+1)$. 

    Applying the weak principle of induction to this base case and this induction step, we conclude that for all $n$, $G(n)$. By (\dag) again, for all $n$, $F(n)$, as desired.
  \end{proof}
\end{theorem}



\paragraph{The Least Number Principle}

\begin{definition}[Least Number Principle] If $M$ is a subset of
  $\mathbb{N}$, and is non-empty, then $M$ has a least
  member.\end{definition}

On the assumption that reasoning in accordance with the LNP is good, then reasoning in accordance with the weak principle is also good, as is shown by the following theorem.
\begin{theorem}[LNP to Weak] The Weak Principle of Induction \label{lnpweak}
  follows from the Least Number Principle. \begin{proof} Assume
 that $P$ is a 
  property such that $P(0)$, and for every $n$, $P(n) \to P(n+1)$. We prove from the LNP and these assumptions that for every $n$, $P(n)$ (which is the Weak
  Principle).
  
  Let $M$ be the set of numbers which do \emph{not} satisfy $P$. By the LNP, $M$ has a least member if it is not
  empty – call this member $m$. $P(0)$ holds, by
  assumption, so $m>0$; and since $m$ is the least member of $M$, $P(m-1)$. But since if $P(n)$ then $P(n+1)$,
  $P(m)$ follows from $P(m-1)$. That contradicts our assumption that $m \in
  M$. So there can be no least member of $M$. A set only has no least member if it has no
  members at all; so there is no number 
  $n$
  such that $\neg P(n)$, hence for every number $n$,
  $P(n)$.\end{proof}\end{theorem}

To show that all three principles of induction we've considered are equivalent, we'd now need to show that the LNP follows from the Strong or Weak Principles of Induction; this is left for an exercise.


\section{Proof by \emph{Reductio}}
Another important proof strategy that you will commonly use is proof by \emph{reductio ad absurdum} (literally, `reduction to the absurd'). In mathematics, this is commonly known as \emph{proof by contradiction}. The idea of this proof is to show that some claim is true by showing that its negation cannot be true. In fact, we just used this proof strategy in the proof of Theorem \ref{lnpweak}.

\paragraph{Example of \emph{Reductio} Proof} Consider, for example this elementary number theoretic result: \begin{theorem} There is no largest prime number. \begin{proof}
Suppose – for the sake of argument – that $k$ is the largest prime. Take all the prime numbers less than or equal to $k$. Multiply all these numbers together, and add 1. The resulting number cannot be divided by $k$, nor by any of the primes smaller than $k$ – for if you were to divide by any of those numbers, the remainder would be one. Therefore it is either itself prime, or (by the fundamental theorem of arithmetic) it is divisible by a prime number greater than $k$. In either case, there is a prime greater than $k$, thus refuting our initial supposition that there is a greatest prime. So that initial supposition must be false, and theorem holds.
\end{proof}\end{theorem} I have made the role of the supposition explicit – we show that an absurdity, or a flat out contradiction, follows from our assumption by impeccable deductive reasoning. But the hallmark of impeccable deductive reasoning is that, if you start with a truth, you will end up with a truth. Since we have ended up with something untrue – an impossibility – we must not have started with a truth.

\section{Intuitionism} The foregoing line of reasoning is very persuasive, and its use is pervasive throughout mathematics and in this book. However, some mathematicians and philosophers, beginning with the influential topologist Brouwer, have wondered whether this proof strategy is legitimate. Everyone agrees that, in showing that our initial supposition of `not $S$' is untrue, we have shown that its negation, `not not $S$' is true. But these \emph{intuitionists} say that `not not $S$' is not equivalent to $S$. Why do they say this? It is because proof by contradiction permits us to show that there is an entity with a certain property (because to assume there is no such entity would lead to a contradiction), without telling us what that entity is. Such proofs are called \emph{non-constructive}, for they do not construct the entity which has the property in question.\footnote{Consider, for example, the classical proof that there are irrational numbers $a$ and $b$ such that $a^{b}$ is rational. We can show this by considering $q=\sqrt{2}^{\sqrt{2}}$, and show that either $q^{\sqrt{2}}$ is rational if $q$ is irrational, or that $q$ itself is rational – either way, there is a pair of irrational numbers (either $a = b = \sqrt{2}$, or $a=q, b=\sqrt{2}$) such that $a^{b}$ is rational. But we haven't shown \emph{which} of these numbers is in fact rational – we haven't constructed an example, we've merely shown that there must be an example. (In fact, $q$ is irrational, so that is an actual constructed example that verifies the theorem.)} 

\paragraph{Intuitionism on Mathematical Truth} What is the interpretation of mathematical truth that motivates intuitionists to reject the equivalence of `not-not-$P$' and `$P$'? It is this: \emph{mathematical truth is mathematical proof} \citep{shap}. Mathematical objects don't just exist, out there in some Platonic heaven, waiting for facts about them to be discovered. They need to be brought into existence by the activity of a mathematician. But bringing something into existence is a positive act – simply showing that it would be absurd if something didn't exist, is not yet to construct it. So intuitionists offer a reinterpretation of the language of proof – English, in our case – to reflect their philosophical commitments. \begin{itemize}
  \item The absurd sentence $\bot$ is not constructible.
\item A construction of `$P$ and $Q$' consists of a construction of $P$ and a construction of $Q$.
\item A construction of `$P$ or $Q$' consists of a construction of $P$ or a construction of $Q$.
\item A construction of `If $P$ then $Q$' is a technique which transforms any construction of $P$ into a construction of $Q$.
\end{itemize} They define `not-$P$' to mean: `If $P$ then $\bot$', i.e., there is a technique which turns a construction of $P$ into a construction of the absurdity. `not not $P$', accordingly, says that there is a technique which turns a technique turning a construction of $P$ into the construction of the absurdity, itself into a construction of an absurdity. But that is not at all obviously the same as a construction of $P$.

\paragraph{The Law of Excluded Middle} Characteristically, the rejection of the so-called \emph{law of double negation elimination} leads to the rejection of the \emph{law of excluded middle} (LEM): intuitionists reject the principle that for any $P$, either $P$ or not $P$. For them, that is equivalent to: for any $P$, there is (already) either a construction of $P$, or a construction turning a construction of $P$ into the construction of an absurdity. Why should we think, prior to any such construction being offered, that they antecedently exist?
Of course we can prove something related to LEM. \begin{theorem}[LEM\minus]
  For any $P$, not-not-($P$ or not-$P$)
 \begin{proof}
  Assume (i) that for some $P$, it is the case neither that $P$, nor that not-$P$. Assume (ii) $P$: it follows that $P$ or not-$P$, so our assumption (ii) must be false. If assumption (ii) were correct, in conjunction with assumption (i), we'd be able to construct a contradictory claim of the form $Q$ and not-$Q$. That is absurd, and not constructible. So, under assumption (i) alone, we can turn any construction of $P$ into a construction of the absurd, which is of course a construction of not-$P$ from assumption (i). But it follows from not-$P$, again trivially, that `$P$ or not-$P$'. So we can turn an assumption of (i) into a construction of the absurdity, which is a construction of not-not-($P$ or not-$P$).
\end{proof}
\end{theorem}
So we can show that it is absurd to deny LEM; but that is not yet tantamount to its being legitimate to assert it.


\paragraph{Resisting Intuitionism} To insist on constructive proofs, and to eschew the use of proof by contradiction, is to cripple mathematics. This would be acceptable – perhaps – if there were a powerful philosophical rationale for thinking that, in general `not not $S$' is not equivalent to $S$. But no persuasive examples have really been given, despite the best efforts of intuitionists to defend their view. So in this book we will adopt the standard highly plausible classical position, that proof by contradiction is perfectly acceptable.


{\small
\subsection*{Further Reading}
\addcontentsline{toc}{subsection}{Further Reading}

On mathematical induction, see  \citet{macsetthl}. Two philosophical attempts to motivate discussion of intuitionism are by \citet{heyintin} and \citet{dumphibai}. A useful discussion of the non-standard features of intuitionistic mathematics is by \citet{int}. 



\subsection*{Exercises}
\addcontentsline{toc}{subsection}{Exercises}

\begin{enumerate}
\item Prove that the Weak Principle of Induction entails the Least Number Principle.

\item \emph{Bivalence} is the principle that every meaningful declarative sentence is either true or false. What is the relationship between Bivalence and the Law of Excluded Middle? 


\end{enumerate}
}









	\newpage
	\chapter{Set Theory}\label{c2}
	 %!TEX root = edl.tex

\section{Sets}

A \emph{set} is a collection of objects. Like a collection, it is a single thing, not a plurality. If $X$ is a set, the objects in $X$ are called its \emph{elements} or \emph{members}. We write $x \in X$ for `$x$ is an element of $X$' (and $y \notin X$ for `$y$ is not an element of $X$').

 Sets are \emph{individuated} by their members; or as we sometimes say, sets are individuated purely \emph{extensionally}. There must be something to a set over and above its members – otherwise it would be simply a plurality identical to those members. But whatever that ‘something’ is, it is minimal enough that there cannot be two distinct sets which share the same members.Therefore, if $X$ and $Y$ name sets which have all and only the same members, $X = Y$. So we can specify a set by listing the members, conventionally in curly braces: $X = \{x_{1},...,x_{n},…\}$. 

 In this, sets differ from \emph{sequences}: two distinct sequences can have the same members as long as they occur in a different order in the two sequences. Some say that sets also differ from wholes in this respect: while you can make only one set from a given collection of members, you can make more than one object from a given collection of parts, if you arrange those parts in different ways. (This last example is however controversial: some think that appearances here are deceptive.) 

A set is not \emph{composed} of its members in any straightforward sense: it is not a whole with its members as parts. We can see this most easily in the case of a \emph{singleton} set, a set with just one member. Clearly $x \in \{x\}$; but since $x$ itself need not be a set, $x \notin x$. Since sets are individuated by their members, and since $\{x\}$ has a member that $x$ does not, $x\neq\{x\}$. So a set is not to be identified with the fusion of its members, even when that latter object exists.

\paragraph{Axioms} The standard approach to set theory is to take this informal conception of a set as a collection, and set down \emph{axioms} that characterise the behaviour of sets. We've already seen one of the standard axioms of so-called Zermelo-Fraenkel set theory : the axiom of \emph{extensionality}: \begin{axiom}[Extensionality]
  For any sets $X$ and $Y$, if for all objects $x$, $x\in X$ iff $x\in Y$, then $X=Y$.
\end{axiom} I will below mention various axioms, the basic starting points of modern set theory, but we will largely proceed in an informal way. In particular, I will not be offering proofs from the axioms of all the set theoretic principles I state below.

\paragraph{Sets in Sets} One thing is worth being explicit about at the outset. \emph{Sets can have other sets as members.} We might be used to thinking of collections as collections of material objects (rocks, china plates, etc.) But since a set is not identical to its members, and we can talk about and quantify over sets, they are also objects available to be the members of further sets. We might \emph{start} with the non-sets, or \emph{ur-elements} as they are sometimes known; but once we make sets of them, we can then make further sets which include sets as members, and then further sets involving sets of sets, etc. This is the so-called \emph{iterative conception of a set} \citep{boolos}.


\paragraph{Defining Sets}

We've seen one way to define a set: list the members. But what if there are too many to list? 

One thing we can do is define sets \emph{inductively}. This means: given a base case, and a \emph{generating relation}, we define a set that includes the base case, and anything related to a member of the set by the generating relation, and   nothing else (the \emph{closure condition}). Here's an example. \begin{definition}[Your Ancestors] 
   \emph{Base case:} Your mother and father are ancestors.\\ \emph{Generating Relation:} Any parent of an ancestor is an ancestor.\\ \emph{Closure Condition:} Nothing else is an ancestor of you.
 \end{definition}

Let $A_{ae}$ be the set of my ancestors. Any member of $A_{ae}$ is a parent of a parent of\ldots a parent of me, with $n >0$ occurences of `a parent of'. The property `is an ancestor of AE' is called the \emph{ancestral} of the property `is a parent of AE'. And more generally we can say that the relation $R$ `$x$ is an ancestor of $y$' is the ancestral of the relation `$x$ is a parent of $y$'. Note that the set $A_{ae} = \{x: R(x,ae)\}$

Alternatively, we can specify a set by stating a condition $F$ that the members all satisfy, typically in some formal mathematical language. Then we can specify the set of things $x$ such that each $x$ is $F$. We write this $\{x: F(x)\}$. If $Y = \{x:F(x)\}$, then $y \in Y$ iff $F(y)$. But in fact any collection of items forms a set: there needn't be a natural condition all the members satisfy.

 Not just any condition will define a set. Consider the condition $F = x \notin x$. This defines a set $\mathbf{R} = \{x: x\notin x\}$, the set of all non-self-membered things.

 Is $\mathbf{R}$ a member of $\mathbf{R}$? Assume $\mathbf{R} \in \mathbf{R}$. Then $\mathbf{R}$ must meet the defining condition $F$, so that $\mathbf{R} \notin \mathbf{R}$. But since $\mathbf{R}$ cannot both be and not be a member of itself, the assumption must be wrong. So $\mathbf{R}\notin \mathbf{R}$. But then $\mathbf{R}$ does meet $F$, and therefore $\mathbf{R} \in \mathbf{R}$. A contradiction follows. This is known as \emph{Russell's paradox}\label{russ}.

The problem is assuming that \emph{any} condition at all defines a set. The alternative is simple: to assume that, \emph{if we are given a set $X$}, then any condition at all will \emph{separate} the members of $X$ into those which meet the condition and those which do not. That is: for any set $X$, and any condition $F$, there will exist a set $Y$ which contains everything which is both a member of $X$ and satisfies $F$. This is the axiom of \emph{separation}, so called because it separates out those members of a set which have a certain specified feature.\begin{axiom}[Separation]
	If $X$ is a set that is already given, and $F$ is any condition, then there exists a set $Y$ consisting of all the members of $X$ which meet the condition $F$. (This may be trivial, in that all or no members of $X$ meet the condition.) That is, for any $X$ and $F$, this specification is well-defined and guaranteed to denote a set: $\{x\in X: F(x)\}$.
\end{axiom}


\paragraph{Relations Between Sets} Some important relations of inclusion and exclusion between sets can be defined, once we are assured that some sets exist.
\begin{definition}[Subset] A set $X$ is a \emph{subset} of $Y$, written
   $X \subseteq Y$ if every
    member of $X$ is also a member of $Y$: for every $x$, if $x \in X$
    then $x \in Y$. $X$ is a \emph{strict subset} (or \emph{proper
    subset}) of $Y$ if $X
    \subseteq Y$ and $Y \not\subseteq X$ (written `$X \subset
Y$').\end{definition}
Trivially, for all sets $X$, $X\subseteq X$. It also follows that if $X \subseteq Y$ and $Y \subseteq X$ then $X=Y$ (easy). And also if $X \subseteq Y$ and $Y \subseteq Z$ then $X \subseteq Z$.
Note that if $F$ is some condition, the set $\{x:x\in X \text{and } F(x)\}$, which is guaranteed to exist by the axiom of Separation, is always a subset of $X$.
\begin{definition}[Superset] A set $X$ is a \emph{superset} of $Y$, written   $X \supseteq Y$ if every    member of $Y$ is also a member of $X$: for every $x$, if $x \in Y$    then $x \in X$. $X$ is a \emph{strict superset} (or \emph{proper superset}) of $Y$ if $X    \supseteq Y$ and $Y \not\supseteq X$ (written `$X \supset Y$').\end{definition}
\begin{definition}[Disjoint] Sets $X$ and $Y$ are \emph{disjoint} iff there is no $x$ which is a member of both.	
\end{definition}


\paragraph{Operations on Sets} The foregoing relations are defined between existing sets. But there are also generating operations, that given some existing sets yield some new sets. For each of these generating operations, we need to give a definition of the properties of the operation. But we also need an axiom to ensure that that operation succeeds, that is, that there is a set which satisfies the definition we have set down. The axiom of Separation supports our first two definitions: 
\begin{definition}[Intersection] If $X$ and $Y$ are sets, the set $X
    \cap Y$ which contains only members of both is called the {\em
    intersection.} 

If $\mathbf{X}=X_{1},\ldots,X_{n},\ldots$ is a set of sets, $\bigcap \mathbf{X} = X_{1} \cap \ldots \cap X_{n} \cap \ldots$.
\end{definition}
 \begin{definition}[Relative Complement] If $X$ and $Y$
	are sets, then $X \setminus Y$ is the set of all members of $X$ which
	are \emph{not} members of $Y$, called the \emph{relative complement} of
	$Y$ with respect to $X$. \end{definition}
Because both of these definitions involve taking a set that is already given and constructing a subset, the existence of the resulting sets is supported by the axiom of Separation. In the case of intersection, if $X$ and $Y$ exist, $X \cap Y$ exists: $X\cap Y = \{x \in X:x\in Y\}$. Alternatively, we can denote this set as the set of any $x$ that meets the conjunctive condition of being in both $X$ and $Y$: $\{x: x\in X \text{and } x \in Y\}$. In the case of relatively complement, if $X$ and $Y$ exist, $X\setminus Y$ exists: $X \setminus Y = \{x \in Y: x\notin Y\}$.

What if we want to make a set which isn't a subset of an already given set?
\begin{definition}[Union] If $X$ and $Y$ are sets, then $X \cup Y$
    (read `the \emph{union} of $X$ and $Y$') is the set which contains all the members of $X$ and all the members of $Y$. 

If $\mathbf{X}=X_{1},\ldots,X_{n},\ldots$ is a set of sets, $\bigcup \mathbf{X} = X_{1} \cup \ldots \cup X_{n} \cup \ldots$.
\end{definition}
The existence of unions is supported by this axiom: \begin{axiom}[Unions]
  Suppose $\mathbf{X}$ is any set of sets (i.e., a set with only other sets as its members). For any such $\mathbf{X}$, there is a set $Y$ such that, for any $z$, $z\in Y$ iff $z$ is a member of one of the members of $\mathbf{X}$.  In short: if $X$ and $Y$ exist, $X\cup Y$ exists: $X \cup Y = \{x:x\in X \text{or } x \in Y\}$.  
\end{axiom} That is, $Y$ is the set containing all the things which are in any of the sets in $\mathbf{X}$. If $X$ and $Y$ are sets, there is a set $\{A,B\}$.\footnote{This is in fact another axiom, the axiom of Pairing.} It then follows from the axiom of Unions that there is a set which contains all the members of $X$ and all the members of $Y$. By extensionality, there is just one such set; we call it $A \cup B = \bigcup\{A,B\}$.

Note the links  between $\cap$ and `and', and $\cup$ and `or'.

\paragraph{The Empty Set}

If $X$ is a set, then by the definition of relative complement, $X\setminus X$ is a set. Suppose $X\setminus X$ has a member, $z$.  Then 	$z\in X\setminus X$ iff $z \notin X \text{ and } z \in X$. Contradiction; so $X\setminus X$ has no members. This set is the \emph{empty set}, written $\emptyset$. $Y\setminus Y$ is also empty; by extensionality, $X\setminus X=\emptyset=Y\setminus Y$; there is only one empty set.

The empty set is the collection which has nothing in it. It corresponds to the empty list. While the empty collection has no members, that doesn't mean that it itself is nothing: it is the unique thing which is a set and also has no members. 

 If $X$ is any set, then for any $x$, if $x \in \emptyset$ then $x\in X$ – for the empty set has no members at all. By the definition of subset, therefore, $\emptyset \subseteq X$, for any $X$.

 It is easy to show that when $X$ and $Y$ are disjoint, $X \cap Y = \emptyset$.

\paragraph{Absolute Complement and the Universal Set}

The empty set is the smallest possible set in terms of its members. Is there a largest set? If there is a \emph{universal set} $\Omega$ – a set which has everything (every individual \emph{and every set}) as a member – we could define something a bit closer to negation than relative complement. We could define: $-X = \Omega \setminus X$; it would be the set of all things which are not in $X$.

This is \emph{not} possible. Suppose there were a universal set. Then the axiom of separation, that if $X$ is a set then there is a set $Y$ which is a subset of $X$ and such that for all $x\in X$ if $F(x)$ then $x\in Y$, will entail a contradiction.\footnote{Hint for exercises: Apply the condition $x\notin x$ to the universal set $\Omega$ in the axiom of separation, and we will construct the Russell set $\mathbf{R}$ perfectly legitimately. Since the Russell set does not exist, something's gone wrong: the only candidate is the assumption that there is a set of everything.} So there is no universal set. (Philosophical puzzle: how do we quantify over – talk about – all sets, as we have been doing, if there is no collection big enough to contain them all? How indeed can we even talk about \emph{everything} if there are too many things to be collected together into a set?) 

\paragraph{Power Set}

So far you might think that sets are just collections of individuals (i.e., non-sets). (Well, we've already mentioned the axiom of Pairing, so you probably don't think that.) But set theory is much more general than that: anything at all, including other sets, can be gathered into a set (as long as we don't try to gather too many things).

One useful set of sets is the \emph{power set} of a set $X$, the set of all subsets of $X$:
 \begin{definition}[Power Set]Given a set $X$, the \emph{power set} of $X$, denoted $\wp(X)$, is
defined: $\wp(X) = \{Y: Y \subseteq X\}$. (It is another axiom of set theory that for any set, its power set exists.)\end{definition} 
Notice that, for any $X$, $\emptyset \in \wp(X)$; and $X \in \wp(X)$. The name `power set' comes from this fact: 
\begin{theorem} If $X$ has $n$ members, $\wp(X)$ has $2^{n}$ members.
	\begin{proof}
		A subset of $X$ is a set containing perhaps some but not necessarily all the members of $X$. We can specify a subset by saying, of each member of $X$, whether it is in the subset.  Suppose the members of $X$ are $x_{1},\ldots,x_{n}$. Then a finite binary sequence, consisting of $1$s and $0$s, of length $n$ suffices to determine a subset of $X$, by the following principle: let $x_{i}$ be a member of the subset associated with $s$ iff the $i$th place in the sequence $s$ is a $1$. By extensionality, each subset of $X$ corresponds to exactly one such sequence. So how many finite binary sequences of length $n$ are there? Each place $i$ in the sequence can be either $1$ or $0$, so there are $2^{n}$ such sequences. So there are $2^n$ distinct subsets of $X$, and hence $2^{n}$ members of $\wp(X)$. 
	\end{proof}
 \end{theorem}

\paragraph{Ordered Pairs and Sequences}

As mentioned above, sets are identical iff they have the same members. How those members happen to be listed is irrelevant: $\{x,y\}=\{y,x\}=\{x,x,y\}$. So if we want to capture the idea of an \emph{ordered sequence}, it cannot be a set – order doesn't matter for sets, and one can't have repeated elements in a set. Yet we can model, or represent, sequences just using sets.
\begin{definition}[Ordered Pair] An \emph{ordered pair} $\langle x, y \rangle$ is defined as the unordered set $\bigl\{x,\{x,y\}\bigr\}$.
\end{definition}
We need to check whether this definition works. We do so by establishing the fundamental principle of the identity of ordered pairs follows from this definition, namely, that ordered pairs are identical iff they have the same first member, and the same second member. What we show is that, if we define ordered pairs as above, they have all the characteristic features of ordered pairs. 
\begin{theorem}[Criteria of Identity for Ordered pairs]\label{cridorpa}
$\langle x,y\rangle = \langle u,v\rangle$ iff $x=u$ and $y=v$.
\begin{proof}
	$\langle x,y\rangle = \langle u,v\rangle$ iff $\left\{x,\{x,y\}\right\} = \left\{u,\{u,v\}\right\}$ iff (i) $x=u$ and $\{x,y\} = \{u,v\}$ or (ii) $x=\{u,v\}$ and $\{x,y\}=u$. Since (ii) is not possible – it would entail that $x$ is a member of one of its own members – it must be that (i), which in turn holds iff $x=u$ and $y=v$ (we need extensionality again to show that if $\{x,y\}=\{x,v\}$ then $y=v$).
\end{proof}
\end{theorem}
In the proof of this theorem, we had to appeal to the fact that no set is a member of a member of itself.\footnote{Kuratowski's original definition of an ordered pair was this: $« x,y» = \{\{x\},\{x,y\}\}$. In proving the criteria of identity for ordered pairs for this alternative definition, we need not appeal to the above-cited fact. (Exercise.) The choice to model ordered pairs as in the text, or by Kuratowski's alternative, is a choice of model – it has no significance for what is being modelled, namely, sequences of length $2$.} This obviously holds on the iterative conception of a set. It tacitly relies on this characteristic axiom of Zermelo-Fraenkel set theory: \begin{axiom}[Foundation]
	 Every non-empty set $X$ contains a member $y$  such that $X$ is disjoint from $y$: $X \cap y = \emptyset$.
\end{axiom} (This axiom is also known as \emph{regularity}.) \begin{theorem}
	No set is a member of itself. \begin{proof}
		Let $X$ be an arbitrary set. Clearly $X \in \{X\}$, which latter set exists by Pairing. By Foundation, since $\{X\}$ is non-empty, it has a member $y$ such that $\{X\} \cap y = \emptyset$. Since the only member of $\{X\}$ is $X$, it must be that $\{X\} \cap X = \emptyset$. But if $X\in X$, then $X$ and $\{X\}$ do have a member in common – namely, $X$ – and their intersection isn't empty. So $X \notin X$.
	\end{proof} 
\end{theorem} This theorem gives another route to avoiding Russell's paradox from earlier (page \pageref{russ}). But Foundation also enables us to rule out longer `cycles' of inclusion, since it entails that there are no infinite descending chains of set membership: every set bottoms out eventually in some members of its members of its members… that have no members themselves.

We can extend the idea of an ordered pair to an ordered sequence of any number of elements, as follows: \begin{definition}[Ordered $n$-tuple]
Let the notation  $⟪x_{1},\ldots,x_{n}⟫$ denote an ordered $n$-tuple, for $n\geqslant 2$. We offer an inductive definition. \begin{itemize}
	\item  \emph{Base case.} $⟪ x_{1},x_{2}⟫ \eqdf \langle x_{1},x_{2}\rangle$.
	\item \emph{Inductive step.} $⟪x_{1},…,x_{n+1}⟫ \eqdf \langle⟪ x_{1},…,x_{n}⟫,x_{n+1}\rangle$.
\end{itemize}\end{definition}
From now on, I will abuse notation, and use ‘$\langle$’ and ‘$\rangle$’ to delimit all ordered sequences, not only pairs. But, speaking strictly, when in the following I denote an ordered triple by $\langle x,y,z\rangle$, that expression should be interpreted as $⟪x, y, z⟫= \langle\langle x,y\rangle,z\rangle$. It doesn't really matter \emph{how} we define ordered tuples; all that matters is that they are well-defined set-theoretic entities which behave as expected. (So, strictly speaking, we'd need to prove a theorem analogous to Theorem \ref{cridorpa} showing that ordered $n$-tuples as just defined behave in the right way. We are not going to do that.)

One remaining case deserves attention. Our notation covers all ordered sequences of two or more members. What about the degenerate case of an ordered sequence with just one member? We shall adopt the following harmless convention (which will however be of use in §\ref{semltwo}): \begin{definition}[Ordered 1-tuple]\label{ordonetup}
	An ordered 1-tuple $\langle x\rangle$ is simply $x$ itself.
\end{definition}

\section{Relations and Functions}
Intuitively, a relation $R$ connects some things (its \emph{relata}). We will model a relation set theoretically as a set of \emph{pairs} of entities such that the first member of the pair bears the relation to the second.  So the set $$\{\langle \text{Melbourne},\text{Victoria}\rangle, \langle \text{Brisbane},\text{Queensland}\rangle,\ldots\}$$ is (part of) the relation of ` –  is the capital city of  – '.
\begin{definition}[Binary Relation]
	$X$ is a binary relation on a \emph{domain} $D$ iff it is a set that contains nothing except ordered pairs $\langle x,y\rangle$ where $x \in D$ and $y\in D$. ($\emptyset$ contains nothing at all, and so nothing other than ordered pairs, so is a binary relation).
\end{definition}
\begin{definition}[Cartesian Product]
	The \emph{Cartesian product} of sets $X$ and 
	$Y$, written‘$X \times Y$’, is the set of all pairs $\langle x,y\rangle$
	such that $x \in X$ and $y \in Y$.
\end{definition} A relation on $D$ is a subset of $D \times D$.

We discuss binary relations further, from a more metaphysical perspective, in Chapter \ref{c8}.

\paragraph{Functions}

Melbourne and Victoria are related by the `is the capital city of' relation iff `Melbourne' names the same thing as `the capital of Victoria'. The term-forming expression `the capital of  –' denotes a \emph{function}. We'll treat the notion of a function more precisely, as a special kind of relation.
\begin{definition}[Function]
	$f$ is an $n$-place function iff (i) it is relation on $D$ (ii) and for any sequence $x_{1},\ldots,x_{n}$ of objects from $D$, and any $y,z \in D$, if $\langle x_{1},\ldots,x_{n},y\rangle \in f$ and $\langle x_{1},\ldots,x_{n},z\rangle \in f$, then $y=z$ (i.e., each sequence of things stands in the relation to at most one thing). 
\end{definition}
 A $n+1$-place relation which meets this condition is an $n$-place function.

If $f$ is a unary function (i.e., a binary relation), we write $f(x) = y$ iff $\langle x,y\rangle \in f$. In this case we say $x$ is the \emph{argument} given to $f$, and $y$ the \emph{value} of $f$ given that argument. The fact that, for a given argument, a function has a unique value, justifies our talking of `\emph{the} value of the function for a given argument', and justifies the link to natural language definite descriptions like `the capital of South Australia' as corresponding to a function with an argument – see the discussion of definite descriptions in chapter \ref{c7} for more on this issue.
\begin{definition}[Domain]
The \emph{domain} (note: different to the domain of a relation) of a function  $f$ is the set $$\mathcal{D}(f)=\{x: \text{there is a $y$ such that } \langle x,y \rangle \in f\}.$$
\end{definition}
\begin{definition}[Range]
	The \emph{range} of $f$ is the set $$\mathcal{R}(f)=\{y: \text{there is a $x$ such that } \langle x,y \rangle \in f\}.$$
\end{definition}
\begin{definition}[Partial and Total]
	A function
$f$ is \emph{partial}  on $X$ if $\mathcal{D}(f) \subset X$; it is \emph{total} if $\mathcal{D}(f) =X$.
\end{definition}

\paragraph{Operators}
\begin{definition}[Operator]
	If $f$ is a function, and $\mathcal{R}(f) \subseteq \mathcal{D}(f)$, then $f$ is an \emph{operator}.
\end{definition}
Consider the numerical function $\mathsf{sq} = \left\{\langle x,y\rangle: x^{2} = y\right\}$, otherwise written $\mathsf{sq}(x) = x^2$. This generalises in the obvious way to binary functions: if the values of the function are appropriate arguments, it is an operator. An example is the addition function on the domain of integers, $\mathsf{plus} = \left\{\langle x, y, z\rangle : z = x + y\right\}$. This is an arithmetical operator.

Some properties of functions are conveniently illustrated by $\mathsf{plus}$.
\begin{definition}[Commutative]
	A binary function $f$ is commutative iff $$f(x,y) = f(y,x).$$  
\end{definition}
\begin{definition}[Associative]
	A binary operator $f$ is associative iff $$f(x,f(y,z)) = f(f(x,y),z).$$ (We need $f$ to be an operator to ensure that it is defined when its values are treated as arguments in this way.)
\end{definition}
It is obvious that $\mathsf{plus}$ is both commutative and associative: $x+y=y+x$ and $x+(y+z)=(x+y)+z$. A commutative but not associative operator is $\mathsf{mean}(x,y) = \tfrac{x+y}{2}$. Clearly $\mathsf{mean}(x,y)=\mathsf{mean}(y,x)$. (This follows from the commutativity of $\mathsf{plus}$.) But $\mathsf{mean}(x,\mathsf{mean}(y,z))$ is not in general equal to $\mathsf{mean}(\mathsf{mean}(x,y),z)$. An associative but non-commutative operator is the concatenation operator on strings $\conc$, which takes two strings and joins them into a longer string by appending the second to the first. Clearly $(x\conc y)\conc z = x \conc (y\conc z)$: `te'$\conc$`a' = `tea' = `t'$\conc$`ea'. But $x\conc y$ isn't necessarily $y\conc x$: `tea' isn't `ate'.



\paragraph{Properties of functions; enumerations}

Suppose $f$ is a total function, and $X$ and $Y$ are some other sets. \begin{definition}[Into, Onto, etc.]
\begin{description}
	\item [Into] $f$ is a function from $X$ \emph{into} $Y$ iff $\mathcal{D}(f) = X$ and $\mathcal{R}(f)\subseteq Y$.
	\item [Onto] $f$ is a function from $X$ \emph{onto} $Y$ iff $\mathcal{D}(f) = X$ and $\mathcal{R}(f) = Y$. An \emph{onto} function is also known as a \emph{surjection}.
	\item [One-One] $f$ is a \emph{one-one} function from $X$ to $Y$ iff $f$ is into and whenever $x\neq y$, $f(x)\neq f(y)$. A one-one function is also known as an \emph{injection}.
	\item [Bijection] $f$ is a \emph{bijection} from $X$ to $Y$ (or, is a \emph{one-one correspondence}) iff $f$ is one-one and onto.
\end{description}
\end{definition}
Some examples; these are functions from $\mathbb{N}$ to $\mathbb{N}$: \begin{itemize}
	\item The constant function $f(x) = 1$ is into.
	\item The function $g(x) = \begin{cases}
		x/2 &\text{ if $x$ is even}\\ (x+1)/2 &\text{ if $x$ is odd} \end{cases}$ is onto. (Every natural number is the value of this function for two arguments.)
	\item The function $h(x)=x^3$ is one-one. (Every natural number has a unique cube, but not every natural number has a natural number cube root.)
 	\item The identity function $i(x) = x$ is a bijection.
\end{itemize}

\begin{definition}[Inverse] If $g$ is a function, it is the \emph{inverse} of $f$ iff  $\mathcal{D}(f)=\mathcal{R}(g)$, $\mathcal{R}(f)=\mathcal{D}(g)$, and for all $x$, $g(f(x))=x$. In that case, we often write $f^{-1}$ for $g$.
 \end{definition} 
\begin{theorem}
	If a function has an inverse, it is one-one.
\end{theorem}
\begin{definition}[Partial function]
 $f$ is a partial function from $X$ to $Y$ iff $\mathcal{D(f)} \subset X$. If $\mathcal{D}(f)=X$, $f$ is a total function from $X$.
\end{definition}
\begin{theorem}
	If $f$ is one-one but not a bijection, $f^{-1}$ is a partial function.
\end{theorem}


\section{Size}

\begin{definition}[Finite]
  A set $X$ is finite iff there exists a set of natural numbers $N = \{1,\ldots,n\}$ and a bijection $f$ from $X$ to $N$.
\end{definition} A finite set can be put into one-one correspondence with some initial fragment of the sequence of natural numbers. A set is infinite iff it is not finite. Where there is a bijection between $X$ and $\{1,\ldots,n\}$, $X$ has $n$ members – we say that $X$ has cardinality $n$, $|X|=n$.

Now we will define some further notions useful for talking about \emph{size}.
\begin{definition}[Enumeration] $f$ is an \emph{enumeration} of $X$ iff $\mathcal{D}(f) \subseteq \mathbb{N}$ (the natural numbers), $\mathcal{R}(f) = X$, and $f$ is onto.
\end{definition} An enumeration associates each member of $X$ with some number – it corresponds to the intuitive notion of counting a set. (Though since we only require that an enumeration be onto, all sorts of weird `counting' procedures are included: e.g, the constant function from $\mathbb{N}$ onto $\{a\}$ is an enumeration of $\{a\}$.) This gives us the following idea: 
\begin{definition}[{[Un]}Countable] $X$ is \emph{countable} iff there exists a function which enumerates it; it is \emph{denumerable} iff it is countable and infinite; it is \emph{uncountable} if it is not countable.
\end{definition}
Obviously, all finite sets are enumerable, since if $f$ is a bijection showing $X$ to be finite, $f^{-1}$ exists and is an enumeration of $X$. But some infinite sets are countable too: obviously, $\mathbb{N}$ itself (there is a no bijection between $\mathbb{N}$ and some initial subset of $\mathbb{N}$, but the identity function enumerates it).

The notion of cardinality can be extended to infinite sets too \citep[chs. 3, 6]{macsetthl}. \begin{definition}[Equinumerosity]
	$X$ and $Y$ are \emph{equinumerous} iff there exists a bijection from $X$ to $Y$.
\end{definition}
\begin{definition}[Cardinality]\label{card}
	For each set, $X$ – whether finite or infinite – there exists its cardinality, $|X|$, meeting the following condition: for any $X$ and $Y$, $|X|=|Y|$ iff $X$ and $Y$ are equinumerous.
\end{definition}
Certainly treating the natural numbers as cardinals of finite sets meets the condition in Definition \ref{card}. But Definition \ref{card} applies to any sets; it's just that, for the infinite sets, we have no pre-theoretical grasp on what their cardinalities might be. That's okay though: we can characterise them. \begin{definition}[$\leq$]\label{leq}
	If $\lambda$ and $\mu$ are cardinals, such that $|X|=\lambda$ and $|Y|=\mu$, then $\lambda \leq \mu$ iff there is an injection from $X$ to $Y$. 
\end{definition}
\begin{theorem}
	Let $|Y|=\mu$. Then $\lambda\leq\mu$ iff there exists an $X$ such that $X \subseteq Y$ and $|X|=\lambda$. \begin{proof}
		Obvious: by Definition \ref{leq}, $\lambda\leq\mu$ iff there is an injection from $X$ to $Y$ iff there is a bijection from $X$ to a subset of $Y$.\label{cardleq}
	\end{proof}
\end{theorem}
\begin{theorem}[Schröder-Bernstein]
	$\leq$ partially orders the cardinals \citep[38–41]{macsetthl}. 
\end{theorem}

With the notion of a cardinal, we can prove that there are more sizes of set than finite sizes, because there are more cardinals than just the natural numbers, which are the cardinalities of finite sets. Since there is no bijection from any finite subset of natural numbers to the whole set of natural numbers, the cardinality of the set of natural numbers is not any finite natural number. We introduce $\aleph_{0}$ to name $|\mathbb{N}|$. 
\begin{theorem}[Cantor]\label{cantor}
For any set $X$, $|X| < |\wp(X)|$.
\begin{proof}
	Define a function $f$ such that for all $x \in X$, $f(x) = \{x\}$. Since if $x \in X$, $\{x\} \subseteq X$, $f$ is an injection from $X$ into $\wp(X)$, and hence $|X|\leq|\wp(X)|$ by Definition \ref{leq}.
	
	To show that $|X| < |\wp(X)|$, we need to show that $|X| \neq |\wp(X)|$, i.e. (by Definition \ref{card}), that $X$ and $\wp({X})$ are not equinumerous. Let $g$ be any function from $X$ to $\wp(X)$. The domain of $g$ is the members of $X$, and the range are subsets of $X$, precisely the kinds of things members of $X$ can be members of. But not every member of $X$ needs to be a member of the subset of $X$ picked out by $g(x)$. Let us define the set of such things, the $x\in X$ which aren't members of $g(x)$: $$D = \{x \in X : x \notin g(x)\}.$$
	    Since $D$ consists solely of members of $X$, $D \subseteq X$ and hence $D \in \wp(X)$. If $g$ is a bijection, then there must be some $d \in X$ such that $g(d) = D$. Obviously, then $d \in D$ iff $d \in g(d)$. But by definition of $D$, $d \in D$ iff $d \notin g(d)$. So if $g$ is a bijection, there is some $d$ such that $d \in g(d)$ iff $d \notin g(d)$, and since there obviously is no such $d$, then $g$ cannot be a bijection. Since $g$ was arbitrary, \emph{no} function from $X$ to $\wp(X)$ is a bijection, and therefore $X$ and $\wp(X)$ are not equinumerous.
  \end{proof}
\end{theorem} 
Cantor's theorem shows us that the powerset of a set is always a strictly larger cardinality than the set. We already know that there is one countable infinite set, $\mathbb{N}$. Are there uncountable sets? Yes, obviously: $\wp(\mathbb{N})$: the set of all sets of natural numbers. This allows us to show that another set is uncountable: the real numbers, numbers with arbitrarily many decimal places. Indeed, we can show the following, from which it follows trivially that the set of reals is uncountable. 
\begin{theorem}[Uncountability of the Unit Interval]
The set of real numbers between $0$ and $1$ inclusive, $[0,1]$, is uncountable. \begin{proof}
	(Sketch.) Each subset of $\mathbb{N}$ can be associated with an infinite binary sequence (sequence of $0$s and $1$s): the sequence corresponding to $X \in \wp(\mathbb{N})$ is the one which has a $1$ at position $n$ iff $n \in X$. Each infinite binary sequence corresponds to exactly one real number in $[0,1]$. So there is a bijection from $\wp(\mathbb{N})$ to $[0,1]$, so $|\wp(\mathbb{N})| = |[0,1]|$, and since the former is uncountable, so is the latter.
\end{proof}
\end{theorem}


A final theorem linking countability and set theory, useful – perhaps – for some exercise or other.
\begin{theorem}[Countable Unions]\label{countu}
	If $\mathbf{X} = \{X_{1},\ldots,X_{n},\ldots\}$ is countable, and each $X_{i}$ is countable, then $\bigcup \mathbf{X}$ is countable.
\end{theorem}
I sketch the proof.
Let $P$ be the set of positive prime numbers. This set is in one-one correspondence with $\mathbb{N}$ (for every $n$, there is one and only one $n$-th prime). Since $\mathbf{X}$ is countable, there is a one-one function from $\mathbf{X}$ into the natural numbers, so there is a one-one function $f$ from $\mathbf{X}$ into $P$.

Since each $X_{i}$ is countable, there is a function $g_{i}$ that is a one to one correspondence between $X_{i}$ and a subset of $\mathbb{N}$. Define the function $h_{i}(x) = f(X_{i})^{g_{i}(x)}$. Since every integer has a unique prime decomposition, if $h_{i}\neq h_{j}$ then $\mathcal{R}(h_{i})\cap\mathcal{R}(h_{i})=\emptyset.$

Now define: $h(x) = h_i(x)$ where $x\in X_i$ and $x$ is not a member of any $X_{j}$ where $j < i$. Clearly $\mathcal{D}(h) = \bigcup\mathbf{X}$. Obviously $\mathcal{R}(h) \subseteq \mathbb{N}$. And $h$ is one-one, since each $h_{i}$ is a function and their ranges are disjoint. So $\bigcup\mathbf{X}$ is in one-one correspondence with a subset of $\mathbb{N}$ so is countable. 

{\small
\subsection*{Further Reading}
\addcontentsline{toc}{subsection}{Further Reading}


Another presentation of the set theory needed for this book can be found in
 \citet{bevpospa}. A presentation which goes well beyond what we need, but is philosophically nuanced, is
 \citet{potsetthi}. The iterative conception of a set is discussed by \citet{boolos}.



\subsection*{Exercises} \label{ex2}
\addcontentsline{toc}{subsection}{Exercises}

\begin{enumerate}

\item Prove: \begin{enumerate}
	\item If $X \subseteq Y$ and $Y \subseteq X$ then $X=Y$.
	\item If $X \subseteq Y$ and $Y \subseteq Z$ then $X \subseteq Z$.
	\item $X \not\subset X$.
	\item If $X \subset Y$ then $Y \not\subset X$.
	\item If $X \subset Y$ then $X \subseteq Y$.
\end{enumerate}

\item Prove: \begin{enumerate}
	\item $X\cap X=X\cup X=X$.
	\item $X \cap \emptyset = \emptyset$.
	\item $X \cup \emptyset = X$.
	\item $X \subseteq Y$ iff $X \cap Y = X$ iff $X \cup Y = Y$.
	\item $X \cap (Y \cup Z) = (X \cap Y) \cup (X \cap Z)$.
	\item $X\setminus (X \cap Y) = X\setminus Y$.
	\item $X \setminus (Y \cap Z) = (X\setminus Y)\cup(X\setminus Z)$.
\end{enumerate}

\item Prove that if there is a universal set $\Omega$, then the axiom of separation entails a contradiction.

\item Let us define a \emph{Kuratowski ordered pair} $« x,y»$ as the set $\{\{x\},\{x,y\}\}$. Prove that \begin{enumerate} 
	\item $« x,y» = « u,v»$ iff $x=u$ and $y=v$.
	\item If $x=y$ then $« x,y» = \{\{x\}\}$.
\end{enumerate}

\item \begin{enumerate}
	\item Prove that if $X$ has $n$ members, $\wp(X)$ has $2^{n}$ members.
	\item Prove that $X \subseteq Y$ iff $\wp(X) \subseteq \wp(Y)$.
	\item Show, by providing a counterexample, that it is not always true that $\wp(X) \cup \wp(Y) = \wp(X \cup Y)$.
\end{enumerate}

\item \begin{enumerate}
	\item Show that $f$ has an inverse function iff it is one-one.
	\item Give an example of a function that \begin{enumerate}
		\item is into but is not onto;
		\item is one-one but is not a bijection;
		\item has no inverse;
		\item is its own inverse. 
	\end{enumerate}
	\item Under exactly what conditions is the union of two functions itself a function? (I.e., state necessary and sufficient conditions.)
\item Two sets are said to be \emph{equipollent} iff there is a one-one correspondence between them. Show that $X$ and $\wp(X)$ are not equipollent. 
	\item Give an example of a set which is \begin{enumerate}
		\item Countable;
		\item Denumerable;
		\item Uncountable. (Hint: use 6.(d) and 6.(e).ii.)
	\end{enumerate} 
\end{enumerate} 



\item \begin{enumerate}
	\item Show that if $X$ is countable, then if $Y \subseteq X$, $Y$ is countable.
\item Show that if $X$ and $Y$ are both countable, $X \times Y$ is countable. (Hint: show that the set of ordered pairs of natural numbers is countable.)

\end{enumerate}





\end{enumerate}

Answers to selected exercises on page \pageref{ans2}.
}


		\newpage
		
\chapter{The Syntax of $\mathcal{L}_{1}$}\label{c3}
\input{edl-3.tex}	
\newpage

\chapter{The Semantics of $\mathcal{L}_{1}$}\label{c3a}
%!TEX root = edl.tex



\section{Semantics for \texorpdfstring{\lone}{L1}}

To complete our specification of \lone, we need to provide some account of the intended meanings of sentences of \lone. In natural languages each expression has a specific meaning or meanings, and each use makes a specific contribution to the utterances in which it appears. There is some non-specificity due to ambiguity, but even that is tightly circumscribed.  Our language will differ from natural language in this respect. For we will not insist on any one fixed interpretation of the sentence letters of the language.\footnote{We resist the assignment of any fixed meaning to the sentence letters in part for practical reasons. If we wish to use our logical languages to model natural language arguments, it would be tedious indeed to have to figure out which sentence letter translates the natural language sentences we are considering. (Is it $P_{371}$?) So we allow the meaning of the sentence letters to be fixed anew in each application.} However, the meanings we assign to the logical connectives will resemble more closely the kinds of meanings in natural languages: the meanings of connectives will make a fixed contribution, alongside the syntax, to determining the meanings of complex sentences.

In a sense, \lone\ and other formal languages are not really devices for communication of meaning at all. They provide ways of representing structure within sentences and arguments, and it is those parts of the language which are assigned a fixed meaning which determine which aspects of structure are able to be captured by a given formal language. \lone\ gives the connectives which combine with sentence letters to create complex sentences a fixed meaning – so it is able to represent those aspects of the structure of a sentence that it possesses in virtue of the presence of those connectives. And it can be used to model the structure of natural language sentences which feature analogues of those connectives. But it is only in a particular application that the sentence letters of \lone\ are given any meaning. A temporary assignment of meanings to those components of a language without fixed meanings is known as an \emph{interpretation} of the language.


\paragraph{Meaning and Truth Values} 

One common approach to the theory of meaning, or \emph{semantics}, for natural languages is to identify one primary aspect of meaning of a sentence with its \emph{truth conditions}, the circumstances under which the sentence is true. So in important respects, the meaning of `Jonquil is walking' is given by specifying that it is true under conditions when Jonquil is walking, and false under other conditions. The actual truth value of the sentence is thus part of the meaning of the sentence, because whether a sentence is true in the actual circumstances is part of its truth conditions. And the same is true for the truth value of the sentence in other, non-actual, circumstances.

\lone\ adopts a similar approach, because it assigns a truth value to each of its sentences under each interpretation. Note that the approach is similar but nevertheless distinct: natural language semantics is concerned with truth in various actual and possible circumstances, while in formal logic we are interested in truth under various interpretations and reinterpretations of the constituents of the language – this difference has particular significance when we consider validity and entailment below.
 Furthermore, the theory of meaning for \lone\ states that the truth value assigned exhausts the meaning of the sentence, under that interpretation – there are no further aspects of meaning.

In practice, an intepretation of \lone\ associates a \emph{truth value} – either True or False – with  every sentence letter of \lone. Having interpreted (or re-interpreted) the sentence letters, the remainder of the semantics is constructed to be such that the truth-value of a sentence depends on its semantically relevant structure and the truth-values of its constituents. \lone\ thus has what is known as a \emph{compositional semantics}: having assigned meanings to the most basic constituents of the language, the semantics specifies how those meanings are to be extended to every sentence in the language by showing how the semantic values (which are just truth values, in \lone) of a complex sentence are determined by its structure and the semantic values of its parts. 

\paragraph{Translation} When translating between natural languages, it is best (if possible) to translate a sentence in the home language by a sentence in the target language with the same truth conditions. Such a guideline cannot be applied to translation between natural and formal languages. What we can do is make sure that we pick an intepretation of the formal language such that the actual truth value of the natural language sentence is reflected in the truth value assigned to the formal language sentence letter. (And, of course, make sure you translate only those natural language sentences without internal logical structure that can be represented in your target formal language by sentence letters of that language.) 

\paragraph{Classical Valuations} In classical logic, as in life, there are two truth values: True $T$ and False $F$ (sometimes written $1$ and $0$, sometimes written $\top$ and $\bot$). Recall our discussion of Bivalence in \autoref{c1}; it is a presumption of classical logic that there are only two truth values, and that each meaningful sentence possesses one and only one of them. \lone\ is a classical logic, and any semantics for \lone\ will support a total pattern of assignments of truth values to the sentences of the language that respects Bivalence, once we have a Bivalent interpretation of the sentence letters. Such a assignment of truth values is known as a valuation:
\begin{definition}[Valuation]
	A \emph{valuation} of a language is an assignment of values (of some sort of other) to the sentences of that language. A \emph{classical valuation} is a valuation in which the possible values are the classical truth values $T$ and $F$, and in which every sentence has exactly one value assigned to it.
\end{definition}
We can express what we said about classical valuations more concisely using the language of functions: a classical valuation on some language is a total function from the set of sentences of a language into the set of truth values. We may also use it to define a new notion: 
	\begin{definition}[\lone-Structure]
		An \emph{\lone-structure} is a (total) function from the set of sentence letters of \lone\ into the set of classical truth values $\{T,F\}$.
	\end{definition} An \lone-Structure is a formal mathematical rendering of what we informally termed an interpretation of \lone.

A valuation need not be compositional, in the sense that there need not be any rules governing which complex sentences get which truth values. But ours will be, and we can use the rules governing the construction of complex sentences to extend an \lone\ structure to a full classical valuation for \lone.
\begin{definition}[\lone-valuation]\label{value}
		When $\mathscr{A}$ is any \lone-structure, $\val{\cdot}{A}$ is a valuation function from the set of \lone\ sentences to the set of truth values $\{T, F\}$ iff it meets these conditions: \begin{enumerate}
			\item If $\phi$ is a sentence letter, $\val{\phi}{A} = \mathscr{A}(\phi)$.
			\item $\val{\neg \phi}{A} = T$ if and only if (iff) $\val{\phi}{A}= F$.
			\item $\val{\phi \wedge \psi}{A} = T$ iff $\val{\phi}{A}=T$ and $\val{\psi}{A}=T$.
			\item $\val{\phi \vee \psi}{A} = T$ iff $\val{\phi}{A}=T$ or $\val{\psi}{A}=T$ (or both).
			\item $\val{\phi \to \psi}{A} = T$ iff $\val{\phi}{A}=F$ or $\val{\psi}{A}=T$ (or both).
			\item $\val{\phi \bicond \psi}{A} = T$ iff $\val{\phi}{A}=\val{\psi}{A}$.
		\end{enumerate}
	\end{definition} It is easy to see that for any \lone-structure $\mathscr{A}$ the valuation function $\val{\cdot}{A}$ is a classical valuation. Every sentence letter is assigned exactly one of $T$ or $F$, and every complex sentence is assigned exactly one of $T$ or $F$. (This depends on the result proved in the exercises to \autoref{c3} – see \pageref{ans3} – that each sentence of \lone\ has exactly one main connective, so that every sentence of \lone\ falls under one and only one of the clauses in the definition of an \lone-valuation.)

What is also clear is that the definition of the valuation function specifies the meaning of the logical expressions of \lone, the connectives. For the meaning of a complex sentence is determined by the semantic value assigned to its constituents, and the valuation function. The meaning of `$\wedge$' just is its functional role, the operator that makes a complex sentence from simpler constituents which is true iff both constituents are true. We'll make this precise when we talk about truth-functions as the meanings of connectives below \autoref{truthfuncs}.

\begin{definition}[Agreement of Structures]\label{agreestr}
	When $S$ is a set of \lone\ sentence letters, let us say that two \lone\ structures $\mathscr{A}$ and $\mathscr{B}$ \emph{agree} on $S$ iff for each $\phi$ in $S$, $\mathscr{A}(\phi) = \mathscr{B}(\phi)$. 
\end{definition}
Since the value of a sentence is determined by the values of its constituents, it is obvious that if $\psi$ is any \lone\ sentence, and if two \lone\ structures $\mathscr{A}$ and $\mathscr{B}$ agree on the sentence letters in $\psi$, then $\val{\psi}{A}=\val{\psi}{B}$.  

\paragraph{Falsity clauses} Here is another example of proof by induction on complexity of sentences. Recall the definition of a valuation. It told us under what circumstances a sentence of a given form was true in a valuation. Why didn't we also need to state when a sentence was false? Because a sentence is false iff it is not true.

\begin{theorem}[Falsity is Untruth]\label{fisut}
		$\val{\phi}{A}=F$ iff $\val{\phi}{A}≠T$.
		\begin{proof}
			\emph{Base case:} $\phi$ is a sentence letter. Because $\mathscr{A}$ is a function, if $\val{\phi}{A}=F$ then $\val{\phi}{A} ≠T$. Because $\mathscr{A}$ is into and total, if $\val{\phi}{A}≠T$ then $\val{\phi}{A}=F$.

			\emph{Induction step:} Suppose $\phi$ is a sentence, but is complex, and that the theorem holds for the constituents of $\phi$. We show two illustrative cases: \begin{enumerate}
				\item Suppose $\phi = \neg \psi$. Then $\val{\phi}{A}=F$ iff $\val{\neg \psi}{A}=F$, iff $\val{\psi}{A} = T$, iff $\val{\psi}{A}≠F$ iff $\val{\neg\psi}{A}≠T$ iff $\val{\phi}{A}≠T$.
				\item Suppose $\phi = (\psi \vee \chi)$. Then $\val{\phi}{A} = F$ iff $\val{\psi \vee \chi}{A}=F$ iff $\val{\psi}{A}=F$ and $\val{\chi}{A}=F$ iff $\val{\psi}{A}≠T$ and $\val{\chi}{A} ≠T$ iff  $\val{\psi\vee\chi}{A}≠T$ iff $\val{\phi}{A}≠T$.
			\end{enumerate} The cases of the other connectives are left for exercises.
		\end{proof}
	\end{theorem}

\section{Truth Tables}

One way of representing \lone\ structures – or at least, representing as much of them as matters in some particular application – is using a \emph{truth table}. Consider any finite set of \lone\ sentences,  $\Gamma$. Since each sentence in $\Gamma$ is also only finitely long, there are at most finitely many sentence letters occurring in $\Gamma$. The values assigned to each of these sentence letters, in accordance with Definition \ref{value}, determines the values assigned to every sentence in $\Gamma$.

Since there are only finitely many sentence letters in $\Gamma$, each of which is assigned either $T$ or $F$ by our structures, it is clear there are only finitely many ways of assigning truth values to the sentence letters. (In fact, if there are $n$ sentence letters, there are $2^n$ ways of assigning them truth values.) So we can write down – in principle – each of those ways of assigning truth values in a \emph{truth table}. This table summarises the truth values of the sentences in $\Gamma$ across all possible \lone-structures.

\paragraph{Constructing Truth Tables} For each sentence letter that appears in some sentence within $\Gamma$, inscribe a column. To be precise, inscribe those columns in the standard order: $P,Q,R,P_{1},Q_{1},R_{1}…$. For each way of assigning truth values to sentence letters, inscribe a row, by putting a truth value under the appropriate column so that every possible combination of independent assignments of classical truth values to the sentence letters is represented in some row. Now for any way of assigning truth-values to sentence letters in $\Gamma$, there is a row of the truth table representing that assignment. It is obvious that each row of the truth table corresponds to a class of structures: namely, a class of structures that all agree on the sentence letters in $\Gamma$, as per Definition \ref{agreestr}. (Each distinct structure disagrees over some sentence letter, but the structures corresponding to a single row in a truth table for $\Gamma$ only disagree over sentence letters not occuring in $\Gamma$ –  since they are not relevant to determining the truth value of any sentence in $\Gamma$, such distinctions do not matter for drawing up the truth table.) Now add a column for each sentence in $\Gamma$, and for each row, inscribe under the sentence the value which is assigned to that sentence by any structure corresponding to that row. (Since all such structures agree on the sentence letters in the sentence, they will all agree on what they assign to the sentence too, so it doesn't matter which one we pick.) 

So, for instance, suppose $\Gamma$ is this interesting set of \lone\ sentences: $$\left\{\neg P, (P\wedge Q), (P\vee Q), (P \to Q), (P\bicond Q)\right\}.$$ The truth table for this set of sentences is pictured in Table \ref{tt}. 

\begin{table}[t]
	\centering
	\begin{tabular}{cc|ccccc}
\toprule
$P$ & $Q$ & $\neg P$ & $(P\wedge Q)$ & $(P\vee Q)$ & $(P \to Q)$ & $(P\bicond Q)$	\\
\midrule
T & T & F & T & T & T & T \\
T & F & F & F & T & F & F \\
F & T & T & F & T & T & F \\
F & F & T & F & F & T & T \\
\bottomrule
	\end{tabular}
	\caption{Truth Table for the Standard Connectives\label{tt}}
\end{table}


\section{Satisfaction, Entailment, and other Semantic Notions}

Many conceptions of logic have it that logic is fundamentally about \emph{consequence}: what it is for some sentences to follow from one another, in virtue of the logical form of the sentences involved \citep{brlc}. But what is consequence? We may characterise it semantically as follows: some sentences of \lone\ have another sentence as a consequence if the former sentences \emph{entail} the latter sentence. And we may characterise entailment, and a number of other semantic relations between sentences, using the notions we've already introduced.
\begin{definition}[Satisfaction]
	Suppose $\Gamma$ is any (possibly empty, possibly finite, possibly infinite) set of sentences of \lone, and $\mathscr{A}$ is a \lone-structure, such that $\val{\gamma}{A}=T$ for every sentence $\gamma\in\Gamma$. In that case, $\mathscr{A}$ \emph{satisfies} $\Gamma$, or $\mathscr{A}$ is a \emph{model} of $\Gamma$.	
	\end{definition}
\begin{definition}[Entailment]
	A set of sentences $\Gamma$ \emph{(semantically) entails} a sentence $\phi$ iff every \lone-structure which satisfies $\Gamma$ also satisfies $\phi$. Notation: $\Gamma \vDash \phi$.
\end{definition}
\begin{definition}[Tautology]		
	$\phi$ is a \emph{tautology} iff every \lone\ structure satisfies $\{\phi\}$. A tautology is sometimes called a \emph{logical truth}. 
\end{definition}
\begin{theorem}
	If $\phi$ is a tautology, then for any set of sentences $\Gamma$, $\Gamma \vDash \phi$ – even if \,$\Gamma$ is the empty set, which contains no sentences at all.
\end{theorem}
This theorem explains the fact that we often write `$\vDash \phi$' to mean that $\phi$ is a tautology.

If no \lone-structure satisfies $\Gamma$,  $\Gamma$ is \emph{unsatisfiable}, or \emph{semantically inconsistent}, which we write $\Gamma\vDash$.\footnote{We use this notation to make the notation for unsatisfiability mirror the notation for tautologousness – don't worry about trying to understand unsatisfiability as entailment of the empty set or anything like that – just remember that `$\Gamma \vDash$' is shorthand for `$\Gamma$ is unsatisfiable'.} Accordingly, a set of sentences is \emph{semantically consistent} iff it is satisfiable. If $\Gamma = \{\phi\}$ and $\Gamma \vDash$ then $\phi$ is a \emph{contradiction}.
\begin{theorem}
	$\phi$ is a tautology iff $\neg\phi$ is a contradiction.
	\begin{proof}
		$\phi$ is a tautology iff $\val{\phi}{A} =T$ in every \lone\ structure $\mathscr{A}$. By Definition \ref{value}, this is the case iff $\val{\neg\phi}{A}=F$ in every \lone\ structure $\mathscr{A}$; iff no \lone\ structure satisfies $\{\neg\phi\}$, i.e., $\neg\phi$ is a contradiction. 
	\end{proof}
\end{theorem}

\begin{theorem}[Entailment and Unsatisfiability]\label{entuns}
	$\Gamma\vDash\phi$ iff $\Gamma,¬\phi\vDash$.
	\begin{proof}
		$\Gamma\vDash \phi$ iff every \lone-structure which makes all of $\Gamma$ true makes $\phi$ true; iff every \lone-structure which makes all of $\Gamma$ true makes $¬\phi$ false; iff there is no \lone-structure which makes all of $\Gamma$ true along with $¬\phi$; iff $\Gamma\cup\{¬\phi\}$ is unsatisfiable.
	\end{proof}
\end{theorem}
This elementary theorem is nevertheless rather useful. One thing it shows is that \emph{reductio} reasoning is good in \lone: for if we have an unsatisfiable set, we can pick any member of that set, negate it,  and conclude that the remaining members entail it. (Remember that $\Gamma,\phi\vDash$ just means that the set $\Gamma\cup\{\phi\}$ is unsatisfiable – order does not matter.) As we have already seen, it is often easier in practice to prove $\Gamma\cup\{¬\phi\}$ unsatisfiable than to come up with a constructive argument from $\Gamma$ to $\phi$.


\paragraph{Entailment, Validity and Necessity}An \emph{argument} involves a set of sentences $\Gamma$, its \emph{premises}, and a single sentences $\phi$ as its conclusion.
 \begin{definition}[Properties of arguments]
 	An argument from premises $\Gamma$ to conclusion $\phi$ is \emph{(logically) valid} iff  $\Gamma$  entails $\phi$. A valid argument is \emph{sound (under a given interpretation)} iff each of its premises is actually true under the interpretation.
 \end{definition} 

 Sometimes one sees a purported definition of validity (or entailment, or [logical] consequence) along these lines: that an argument is valid iff it is \emph{impossible} for its premises all to be true while its conclusion is false \citep[19]{sweetreas}. That is, if there is no possible situation in which the premises are true and the conclusion false. While there are some similarities between \lone-structures and possible situations, this sort of definition of validity should be resisted. We can think of a possible situation as specified by a description of a \emph{way things could have been}. Importantly, the description will be in a certain actual language, which we need to hold fixed in order for it to do its descriptive job. But an \lone-structure is, in some ways, precisely the opposite of this. It is a way of interpreting the sentence letters of \lone, while holding fixed the world at which those reinterpreted sentences are to be evaluated. In many cases it makes no difference whether we consider a sentence $P$ to be evaluated at a possible situation in which what $P$ says obtains, or to be evaluated at actuality under an interpretation which makes it true. But there are cases in which it matters. Consider this argument: \begin{quotation}
	Sylvester is a child;

	Therefore, Sylvester is not an adult.
\end{quotation}
The premise \emph{necessitates} the conclusion, for there is no possible world in which a child is an adult. But this argument isn't logically valid; for under a reinterpretation of the non-logical vocabulary ‘adult’ on which it means ‘child’, the premise is actually true and the (reinterpreted) conclusion actually false. The premise guarantees the truth of the conclusion, in part because of what the non-logical expressions ‘adult’ and ‘child’ actually mean. But in logic, we are interested in arguments where the premises guarantee the truth of the conclusion in virtue only of the meanings of the logical expressions involved \citep{tarski}.

Why does logic concern itself with what follows from a sentence under every possible reintepretation of the non-logical expressions (involved (i.e., those without a fixed meaning - the sentence letters, in the case of \lone)? This is because logic is primarily concerned with that special case of consequence in which the conclusion follows from the premises in virtue of the syntactic form of the sentences involved. To focus on syntactic structure, we ought to neglect the actual meaning of the basic constituents of the language, and focus only on how those constituents are deployed into a complex syntactic structure. It is in a sense like replacing those basic constituents with nonsense – but once that replacement is made, we can see that the resulting argument still might be good in virtue of its structure. `Fleebles bork and greebles gork' entails `Fleebles bork', no matter what those constituent expressions happen to mean, just as long as `and' has \emph{its} actual meaning. What the above example shows is that there are arguments where the premises necessitate the conclusion, but not in virtue of their logical form – at least, not in virtue of those aspects of logical form able to be represented in \lone. Perhaps in some logic, we could analyse `child' as `non-adult', and in such a logic the argument above would be logically valid. This shows us something else interesting: that logical form depends on which expressions one takes to be logical expressions, and that may vary between different logical languages.

These remarks also reflect on the notion of a tautology. Our conception is this: a tautology is true in virtue of its logical structure, no matter what the meaning of its non-logical constituents. This suffices for necessity, but the converse is not true. There are necessary truths which are not logical truths – for example, `Adults are not children' is necessary but not a tautology of \lone.

\section{Meaning, Possibility, and Time} The picture we have of meaning in our language \lone\ is this. A sentence letter has as its meaning, relative to an \lone\ structure, a truth value. If a possible scenario is which has a coherent description in a given language, then we can define a \emph{\lone-logically possible scenario} as one described by a satisfiable set of \lone\ sentences. I will say something briefly about the philosophical issues about meaning involved in this picture.

There are several respects in which this language differs from natural language. We have just noted, in effect, that being true in every possible scenario relative to \lone\ is to be distinguished from being genuinely necessary. It is not really possible to make each sentence in this set jointly true: $\{$‘Sylvester is a child’, ‘Sylvester is an adult’$\}$. But the only possible translations of these sentences into \lone\ turn that set into a satisfiable one, so the scenario is logically possible. (This probably shows that the terminology of ‘logical possibility’ is misleading, since it is not a type of possibility at all – perhaps ‘logically coherent’ or `formally coherent' are preferable.)  On the other hand, it is to be hoped that any genuinely possible scenario is also logically coherent.\footnote{Even this is complicated by the fact that different logical languages might be able to express different things, as we'll see later in this book, and so there might not be any single notion of logical coherence that is a feature of any genuine possibility.}

The treatment of possibility for \lone\ sentences is not as controversial as the treatment of time. For many natural language sentences change their truth value while keeping their meaning fixed – so truth value isn't even part of the meaning (though it might be determined by the meaning in conjunction with the circumstances the sentence is taken to be describing). So the English sentence `It's raining in Munich' is true today, but false tomorrow, even though it seems to mean the same thing on both days. Three proposals might enable us to bring the treatment of the logical language and natural language closer on this issue. \begin{itemize}
	\item First, you might consider the logical language adequate to handle only that fragment of natural language which concerns sentences which have a constant truth value over time. 
	\item Second, you might think that \lone\ sentences concern just what is \emph{presently true}; if you wanted to handle claims about what \emph{was true} (which we handle with past tense sentences in natural language), or what \emph{will be true} (natural language future tense and related constructions), then you would need to extend the language of \lone\ to include something like tense. This is the subject of what is called, naturally enough, \emph{tense logic}. The basic idea is to treat a structure for a sentential tense logic as a  sequence of \lone\ structures, with each structure in the sequence telling us what sentence letters would be true if that structure corresponded to the present moment. So if ‘It is raining in Munich’ is true at time $t_{1}$ then false at time $t_{2}$, we could model that by translating the sentence by the sentence letter $P$, letting $t_{1}$ correspond to some \lone\ structure in which $P$ is assigned $T$, and let $t_{2}$ correspond to some \lone\ structure in which $P$ is assigned $F$. A whole possible world, on this view, is then modelled by the sequence of structures – because a possible situation, to accomodate tense as an important feature of reality, must include other times in addition to the present moment. 
	\item Alternatively, we could take it that natural language tense is eliminable in some way, so that in fact the meanings of natural language sentences have constant truth value. Surprisingly enough, this is a fairly orthodox view in natural language semantics~\citep{king,partee}. The idea is that each utterance of a sentence like ‘It's raining in Munich’ expresses a proposition like \emph{It is raining in Munich on such-and-such date} – and those propositions are permanently true if true at all, because they are about a specific time. If this is right, then there is no problem assigning as the meaning of a sentence a truth value relative to a possible scenario, since the meanings of natural language sentences determine propositions of constant truth value. To do full justice to this approach, however, involves delicate issues about both natural language and about the `untensed' nature of reality. 
\end{itemize} For practical reasons, I'll simply adopt the first proposal, and ignore tense in discussions of \lone\ (and \ltwo). As we'll see when we move to talking about \ltwo\ later in this book, there are already natural language constructions which cannot be handled by \lone, so there is precedent for simply taking the language of sentential logic to be able to adequately translate only some fragment of natural language. 





 \section{Entailment and the Connectives}

Entailment is a relation between a set of sentences and a single sentence; and a claim like $\{P, Q\} \vDash P \vee Q$ is a claim about the sentences of \lone – it is not a statement in \lone. But there are interesting relationships between some statements about entailment (and other semantic notions, like satisfaction), and some sentences of \lone. We see these relationships manifest in how certain statments about entailment license further entailments between sentences involving our logical connectives in distinctive ways. For example:
\begin{theorem}[Conjunction and Satisfaction] Some sentences are satisfied in a structure iff their conjunction is satisfied in that structure. That is, if\, $\Gamma$ is some set of sentences $\{\gamma_{1},\ldots,\gamma_{n}\}$, then for every \lone\ structure $\mathscr{A}$, $\mathscr{A}$ satisfies\, $\Gamma$ iff\, $\val{(\gamma_{1}\wedge\ldots\wedge\gamma_{n})}{A}=T$.\label{cone}
\end{theorem}

\begin{definition}[Equivalence]
	$\phi$ and $\psi$ are \emph{logically equivalent} iff $\phi \vDash \psi$ and $\psi\vDash\phi$.
\end{definition}
\begin{theorem}[Equivalence and Biconditional]	
		$\phi$ and $\psi$ are logically equivalent iff\, $\vDash \phi \bicond \psi$.
		\begin{proof}
		$\phi\vDash\psi$ iff every structure in which $\phi$ has the value $T$ is one in which $\psi$ also has the value $T$, and \emph{vice versa}. That is, iff in every structure $\mathscr{A}$, $\val{\phi}{A}=\val{\psi}{A}$. That holds in turn, by Definition \ref{value}, iff $\val{\phi\bicond\psi}{A}=T$ in every structure, and $\phi\bicond\psi$ is a tautology.
		\end{proof}
	\end{theorem}		

\paragraph{Deduction Theorem}
	Let the notation `$\Gamma,\phi \vDash \psi$' abbreviate $\Gamma \cup\{\phi\} \vDash \psi$. \begin{theorem}[Deduction]\label{ded} $\Gamma,\phi \vDash \psi$ iff $\Gamma \vDash (\phi \to \psi)$. 
\begin{proof}
$\Gamma, \phi \vDash \psi$ iff  every \lone-structure which satisfies $\Gamma\cup\{\phi\}$ also satisfies $\{\psi\}$. That holds iff there is no structure $\mathscr{A}$ in which $\Gamma$ is satisfied and in which $\val{\phi}{A}=T$ and $\val{\psi}{A}=F$. By Definition \ref{value}, there is no structure $\mathscr{A}$ in which $\Gamma$ is satisfied while $\val{\phi\to\psi}{A}=F$, i.e., $\Gamma\vDash\phi\to\psi$.
\end{proof}
\end{theorem}
Repeated applications of the deduction theorem permits us to say that an argument $\gamma_{1},\ldots,\gamma_{n} \vDash \phi$ is valid iff $\vDash (\gamma_{1} \to (\gamma_{2} \to …(\gamma_{n}\to \phi)…))$ – i.e., that an argument with finitely many premises is valid iff the corresponding nested conditional is a tautology.\begin{corol}
	An argument from premises $\Gamma$ to conclusion $\phi$ is valid iff the conditional with the conjunction of the premises in $\Gamma$ as antecedent, and $\phi$ as consequent, is a tautology. That is $\Gamma\vDash\phi$ iff $\vDash (\gamma_{1}\wedge\ldots\wedge\gamma_{n}) \to \phi$. \begin{proof}
		Obvious from Theorem \ref{cone} and Theorem \ref{ded}.
	\end{proof}
\end{corol} The deduction theorem is the source of considerable confusion on the part of first-year logic students and others, for it seems to suggest a correspondence between entailment and the conditional. It does not: the correspondence is between entailment and a \emph{tautologous} conditional. A conditional is true just in case – in fact – if the antecedent is true, then the consequent is also true. A tautologous conditional is one that is true on grounds of logic alone, and it is only this sort of conditional that expresses in \lone\ something analogous to entailment between sentences of \lone. The conditional `If we return our rental car late, we will be charged' is true, but not tautologously true – compare the genuine tautology `If we return our rental car and we return it late, then we return our rental car', which cannot help but be true in virtue of its form ($P\wedge Q \vDash P$). We return to the topic of conditionals in \autoref{c9}.




\section{Structural Rules}\label{twostruct}

The expression of entailment ‘$\Gamma \vDash \phi$’ is known as a \emph{sequent}. This sequent is correct if $\Gamma$ does entail $\phi$. There are a set of rules, each of which can be justified from the definitions above, that governs correctness-preserving transformations of one sequent into another. These rules are known as \emph{structural rules}. Theorems justifying three standard structural rules follow (where $\Gamma, \Delta$ are sets of premises): \begin{theorem}[Permutation] $\Gamma, \psi,\chi,\Delta \vDash \phi$ iff $\Gamma, \chi,\psi,\Delta \vDash \phi$ (premise order doesn't matter).\end{theorem}
\begin{theorem}[Contraction]$\Gamma, \psi,\psi, \Delta \vDash \phi$ iff $\Gamma, \psi,\Delta \vDash \phi$ (duplicate premises don't matter).
\end{theorem}
\begin{theorem}
	[Weakening] If $\Gamma \vDash \phi$, then $\Gamma, \psi \vDash	\phi$.
	\begin{proof}
		Note that every structure which satisfies all of $\Gamma$ and also $\{\psi\}$ also satisfies $\Gamma$; as every structure satisfying $\Gamma$ satisfies $\phi$, every structure satisfying $\Gamma, \psi$ satisfies $\phi$.
	\end{proof}
\end{theorem}
The structural rules are so-called because they don't depend on the meaning of any connectives, but only on the definition of entailment and the underlying set theory governing the behaviour of sets of premises. Interesting non-classical logics can arise when one varies the structural rules, which may even be done while retaining classical valuations \citep{restsub} – of course one must vary the set-theoretic conception of entailment as a relation between sets of sentences and a sentence to vary the structural rules, in light of the theorems above.\footnote{For example, a \emph{multiset} (sometimes known as a \emph{bag}) is a generalisation of a set where the number of times an element occurs is significant (though order still doesn't matter). So the multisets $⟅a,a,b⟆$ and $⟅a,b⟆$ differ from one another, though $\{a,a,b\}=\{a,b\}$. If we think of entailment as a relation between multisets of premises and a conclusion, then we are unlikely to want to accept contraction: maybe the resources that the multiset $⟅P,P⟆$ provides are strictly more than those provided by $⟅P⟆$. The status of these \emph{substructural} logical formalisms as logic – in the sense of having something to do with entailment and validity – is problematic, but their formalism has applications, often in computer science. For example, logics that do not permit contraction (\emph{linear logics}) are often used to model computational processes where keeping track of resource use is important for implementation.}

\paragraph{Cut, Transitivity and Contraposition}

\begin{theorem}[Cut] If $\Gamma, \psi \vDash \phi$ and $\Gamma \vDash \psi$, then $\Gamma \vDash \phi$. \begin{proof}
	Assume that $\Gamma, \psi \vDash \phi$. If every structure in which $\Gamma$ is satisfied is one in which $\{\psi\}$ is satisfied, then it is clear that the set of structures
	which make $\Gamma, \psi$ true is just the set of structures which makes 
	$\Gamma$ true. Given that all the former satisfy $\phi$, so must all the latter: $\Gamma \vDash \phi$.
\end{proof}
\end{theorem}
\begin{theorem}[Transitivity]
	If $\Gamma \vDash \psi$ and $\psi \vDash \phi$, then $\Gamma \vDash \phi$.
\end{theorem}

\begin{theorem}[Contraposition]
$\phi \vDash \psi$ iff $\neg \psi \vDash \neg \phi$.
	
\end{theorem}




\section{Substitution}

\paragraph{Formality of Logic} The notions of logical validity and entailment in \lone\ are \emph{formal}: it is in virtue of the form of the premises that they entail the conclusion of a valid argument. But what does it mean to say that logic is formal? Here's one thing it could mean: take a sentence of the language, and replace some of its constituents with others of the same category, while preserving the logical structure of the sentence with respect to some particular logic. These two sentences have the same form, with respect to that logic. If we can prove some sort of equivalence between the two sentences, we will have shown that what matters in the language is form, rather than the particular constituents of a given sentence. 

This is something you will have relied on implicitly in earlier logic courses: how could it be that for example $P$ and $P\to Q$ entail $Q$, yet $R$ and $R \to P_{1}$ fail to entail $P_{1}$? But we should prove it explicitly, to make sure that our implicit assumptions aren't leading us astray.

In the case of \lone, the substituable constituents are sentences (including sentence letters), and the logical structure is given by the logical connectives. \begin{definition}[Constituent sentence]
	A \emph{constituent sentence}  of $\phi$ is any well-formed \lone\ sentence which occurs within $\phi$ as a substring. If one decomposes the sentence successively in accordance with the syntactic rules, every constituent sentence appears at some stage. 
\end{definition}
 We begin by considering the substitution of sentences for sentence letters.
\begin{definition}[Uniform Substitution]If $\Gamma$ is a set of sentences, $\theta$ a sentence, and $s$ a sentence letter, then write $\Gamma[\theta/s]$ for the set of sentences that results by \emph{uniformly substituting} $\theta$ for every occurrence of constituent sentence $s$ in every sentence in $\Gamma$. (If $s$ doesn't occur in $\Gamma$, then $\Gamma[\theta/s]=\Gamma$.) Likewise, let $\phi[\theta/s]$ be the sentence that results from replacing every occurrence of $s$ in $\phi$ by $\theta$.
\end{definition}

\begin{definition}[Substitution Instance]
	If  $\phi = \psi[\theta/\chi]$, for some $\theta,\chi$, then $\phi$ is called a \emph{substitution instance} of $\psi$.
\end{definition}

We now show a useful lemma. \begin{lemma}[Substitution]\label{sublem}
	Suppose $\phi$ and $\theta$ are sentences, and $s$ a sentence letter. For any structure $\mathscr{A}$, define a structure for every sentence letter $\alpha$: $$\mathscr{A}^{\star}(\alpha) = \begin{cases} \mathscr{A}(\alpha) &\text{ iff } \alpha \neq s \\
	\val{\theta}{A} &\text{ iff } \alpha = s.
		\end{cases}$$ Then $\val{\phi}{A^{\star}}= \val{\phi[\theta/s]}{A}$. 
\begin{proof}	The new structure $\mathscr{A}^{\star}$ is just like $\mathscr{A}$, with the possible exception that it assigns to $s$ the value that $\theta$ has in $\mathscr{A}$.

\emph{Base case}: If $\phi$ is a sentence letter, then the result follows by construction of $\mathscr{A}^{\star}$.\\
		\emph{Induction step}: If $\phi$ is complex, and the theorem holds for less complex claims, then the result follows. I show one case: where $\phi$ is a conjunction of two simpler constituents. The induction hypothesis is that  $\val{\phi_{i}}{A^{\star}} = \val{\phi_{i}[\theta/s]}{A}$ for each $\phi_{i}$ which is a constituent of $\phi$.

By the semantic rules for the valuation function, and the induction hypothesis, \begin{align*}
	\val{\phi}{A^{\star}} = T &\text{ iff } \val{(\phi_{1} \wedge \phi_{2})}{A^{\star}} = T\\ &\text{ iff } \val{\phi_{1}}{A^{\star}}=\val{\phi_{2}}{A^{\star}}=T\\ &\text{ iff } \val{\phi_{1}[\theta/s]}{A}=\val{\phi_{2}[\theta/s]}{A}=T\\ &\text{ iff } \val{\phi_{1}[\theta/s] \wedge \phi_{2}[\theta/s]}{A}=T\\ &\text{ iff } \val{(\phi_{1} \wedge \phi_{2})[\theta/s]}{A}=T.
\end{align*}
The desired result follows: $\val{\phi}{A^{\star}} = \val{\phi[\theta/s]}{A}$. \emph{Mutatis mutandis} for the other connectives. 
\end{proof} 
\end{lemma}


With this lemma in hand, we can establish further results about substitution.
\begin{theorem}[Substitution of Material Equivalents]
	If $\val{\theta}{A} = \mathscr{A}(s)$, then $\val{\phi}{A} = \val{\phi[\theta/s]}{A}$. \begin{proof}
		If $\val{\theta}{A} = \mathscr{A}(s)$, then $\mathscr{A}=\mathscr{A}^{\star}$; applying Theorem \ref{sublem}, the result follows immediately.
	\end{proof}
\end{theorem}
\begin{theorem}[Substitution of Sentence Letters]\label{substitution} If\, $\Gamma$ entails $\phi$, then so too\, $\Gamma[\theta/s] \vDash \phi[\theta/s]$, for any $\theta, s$.
\begin{proof}
	Now suppose for \emph{reductio} that $\Gamma \vDash \phi$, but $\Gamma[\theta/s] \nvDash \phi[\theta/s]$. In that case, there is a structure $\mathscr{B}$ such that for each $\gamma \in \Gamma$, $\val{\gamma[\theta/s]}{B}=T$, but $\val{\phi[\theta/s]}{B}=F$.

	By the Substitution theorem \ref{sublem}, there is therefore a structure $\mathscr{B}^{\star}$ such that for each $\gamma \in \Gamma$, $\val{\gamma}{B^{\star}}=T$ and $\val{\phi}{B^{\star}}=F$. But in that case, $\Gamma \nvDash \phi$ after all, since there is an \lone\ structure where the premises are true and the conclusion false, contradicting our \emph{reductio} supposition, which must have been wrong.
\end{proof}
\end{theorem}
\paragraph{Generalised Substitition and Formal Logic} A generalisation of Theorem \ref{substitution} is also provable. The notation ‘$\phi[\beta/\alpha][\gamma/\beta]$’ is to represent the uniform substitution, first, of $\beta$ for $\alpha$ throughout $\phi$, and \emph{then} the substitution of $\gamma$ for $\beta$ throughout the resulting sentence. It is clear that $\phi[\beta/\alpha][\gamma/\beta] = \phi[\gamma/\alpha]$, as long as $\beta$ didn't already occur as a constituent sentence of $
\phi$. \begin{theorem}[General Substitution]\label{generalsub}
	When $\Gamma$ is finite, if $\Gamma \vDash \phi$, then also $\Gamma[\theta/\chi] \vDash \phi[\theta/\chi]$, for any $\theta, \chi$. \begin{proof}
		Suppose for \emph{reductio} that $\Gamma \vDash \phi$, but there is a sentence letter $s$ \emph{not occurring in $\Gamma$ or $\phi$} such that $\Gamma[s/\chi] \nvDash\phi[s/\chi]$. There must be a structure $\mathscr{B^{\star}}$ such that for each $\gamma \in \Gamma$, $\val{\gamma[s/\chi]}{B^{\star}}=T$, but $\val{\phi[s/\chi]}{B^{\star}}=F$.
		By theorem \ref{sublem}, there is therefore a structure $\mathscr{B}$ such that for each $\gamma \in \Gamma$, $\val{\gamma}{B}=T$ and $\val{\phi}{B}=F$.\footnote{This is because for any sentence $\gamma$ in which $s$ does not occur, $\gamma = \gamma[s/\chi][\chi/s]$.} But in that case, $\Gamma \nvDash \phi$ after all, contradiction: so our supposition must have been wrong. So in fact for any sentence letter $s$ not occurring in $\Gamma$ or $\phi$, when $\Gamma \vDash \phi$, $\Gamma[s/\chi] \vDash \phi[s/\chi]$. 
But since there are infinitely many sentence letters and each of the finitely many members of $\Gamma$ and $\phi$ contains only finitely many sentence letters, we may always find such a `new' $s$.  

	Now we may apply Theorem \ref{substitution} to show that $\Gamma[s/\chi][\theta/s] \vDash \phi[s/\chi][\theta/s]$. But this latter sequent just is $\Gamma[\theta/\chi] \vDash \phi[\theta/\chi]$.
	\end{proof}
\end{theorem}
Theorem \ref{generalsub} gives us another route to the formality of logic.\footnote{The reasoning is actually a little subtle: what happens when $\Gamma$ is infinite? Then we cannot be assured that there is some new sentence letter occurring in neither $\phi$ nor $\Gamma$. The Compactness Theorem (\autoref{compact}), which we will prove in \autoref{c4}, assures us that: $\Gamma \vDash \phi$ iff there is some finite set $\Gamma^{-} \subseteq \Gamma$ such that $\Gamma^{-} \vDash \phi$. We can then apply Theorem \ref{generalsub} to $\Gamma^{-}$. But don't worry about this wrinkle until then; we rarely come across arguments with infinitely many premises in everyday life….} We might propose that $\phi$ shares a logical form with $\psi$ if each is a substitution instance of the other.\footnote{So $(P\vee Q)$ shares a logical form with $((R\wedge ¬R) \vee Q)$, because $(P\vee Q)[(R \wedge \neg R)/P] = ((R\wedge ¬R) \vee Q)$, and $((R\wedge ¬R) \vee Q)[P/(R \wedge ¬R] = (P\vee Q)$. Note however that even though $(P\vee Q)[Q/P] = (Q \vee Q)$, $(Q \vee Q)$ does not share a logical form with $(P \vee Q)$, since we cannot uniformly substitute for $Q$ in $(Q\vee Q)$ and obtain $(P\vee Q)$.} Any valid argument, when we uniformly and systematically substitute one constituent for another throughout every sentence involved in premises and conclusion, will transform into another argument of the same logical form; and that second argument will also be valid. Since $P, P\to Q \vDash Q$, accordingly so too $P, (P \to \neg P) \vDash \neg P$, which is a substitution instance of the former argument. But as this example shows, while we retain validity, we need not retain \emph{soundness}. In this case no matter what interpretation we give to $P$, the premises of the post-substitition argument are inconsistent with each other, so we know that they cannot all be true under any interpretation. But we could have interpreted $P$ and $Q$ in the original argument so that they came out simultaneously true, so the argument could have been sound while its substitution instance could not. Soundness is not a matter of form, but the particular interpretation of a sentence of a given form. 

\paragraph{Equivalence Revisited}
There is one case where soundness under an interpretation is \emph{guaranteed} to be preserved: if what is substituted for one another are logically equivalent sentences. \begin{theorem}[Equivalents] \label{equivalents}
	 If $\Gamma \vDash \phi$ is sound under an interpretation, then so is any sequent $\Gamma[\theta/\chi] \vDash \phi[\theta/\chi]$ where $\theta$ and $\chi$ are logically equivalent, under the same interpretation. \begin{proof}
Left for exercise.\end{proof}
\end{theorem}
\begin{theorem}[Equivalence]\label{tequiv}
	Suppose $\phi$ is a constituent sentence of $\chi$, and $\chi' = \chi[\psi/\phi]$. Then $\phi \bicond \psi \vDash \chi \bicond \chi'.$
	\begin{proof} \emph{Base case}: Suppose $\chi=\phi$. Then $\chi'=\psi$, and obviously $\phi\bicond\psi\vDash\phi\bicond\psi$.
	
	\emph{Induction step}: Suppose $\chi$ is complex, e.g., $\chi = \neg \gamma$, and the induction hypothesis holds for simpler constituents: $\phi\bicond\psi \vDash \gamma\bicond\gamma'$. For any structure, $\val{\alpha\bicond\beta}{A} = \val{\neg\alpha \bicond \neg\beta}{A}$, so $\phi\bicond\psi \vDash \neg\gamma\bicond\neg(\gamma')$. But since $\neg\gamma=\chi$, and $\neg(\gamma')=(\neg\gamma)'=\chi'$, the result follows. (The other cases are left for exercises.)
	\end{proof}
\end{theorem}

\section{Truth-functions}\label{truthfuncs}


\begin{definition}[Truth Function]A \emph{truth function} $f_{n}$ is any total function from ordered sequences of $n$ truth values into the set of truth values $\{T,F\}$. \end{definition} 
Since an $n$-place truth-function is a function from sequences of truth values to truth values, it can be represented by an unlabelled truth-table of $2^{n}$ rows (one row for each sequence of truth values of length $n$). It is unlabelled because the columns on the left are not headed by a sentence letter, so there is no association in the table between these strings of truth values and \lone-structures. One such truth-function table is depicted in \autoref{ttft}, for the truth-function $\mathbf{c}$. This truth-function can also be represented perhaps more compactly as follows: $$\mathbf{c}(x,y) = \begin{cases} F &\text{if } x = T \text{ and } y = F;\\
T &\text{otherwise}.\end{cases}$$ 
\begin{table}[t]
	\centering
	\begin{tabular}{cc|c}
		\toprule
		& & $\mathbf{c}$ \\
		\midrule
		$T$ & $T$ & $T$  \\
		$T$ & $F$ & $F$ \\
		$F$ & $T$ & $T$ \\
		$F$ & $F$ & $T$ \\
		\bottomrule
	\end{tabular}
	\caption{A truth-function table for $\mathbf{c}$ \label{ttft}}
\end{table} 


\begin{theorem}
	There are $2^{2^{n}}$ $n$-place truth functions.\label{ntf}
	\begin{proof}
		There are $2^{n}$ distinct sequences of truth values of length $n$, each of which is an input to an $n$-place truth function. Each of these sequences can be given $T$ or $F$ as its value. If one function assigns a different value to a given input than another function, they are different functions. So there are at least $2^{2^{n}}$ different truth functions (one for each way of assigning $F$s and $T$s to sequences of truth values of length $n$). And if $f$ and $g$ have the same value for every input, then $f$ and $g$ are the same function. (By extensionality of the underlying sets.) So there are exactly $2^{2^{n}}$ different $n$-place truth functions.
	\end{proof}
\end{theorem}

Consider the 1-place truth functions. By Theorem \ref{ntf}, there are 4 such functions, outlined in Table \ref{t1tf}. They are the \emph{constant} functions $\mathbf{t}$ and $\mathbf{f}$, which yield the same value for every argument; the \emph{identity} function $\mathbf{i}$, which yields as value the argument; and the \emph{negation} function $\mathbf{n}$, which yields the opposite truth value from the argument. 
\begin{table}[b]
	\centering
	\begin{tabular}{c|cccc}
		\toprule
		& $\mathbf{t}$ & $\mathbf{f}$ & $\mathbf{i}$ & $\mathbf{n}$ \\
		\midrule
		$T$ & $T$ & $F$ & $T$ & $F$ \\
		$F$ & $T$ & $F$ & $F$ & $T$ \\
		\bottomrule
	\end{tabular}
	\caption{The four 1-place truth functions\label{t1tf}}
\end{table}

\paragraph{Expression} We saw that we can represent a truth-function by an unlabelled truth-table. The truth-function table for $\mathbf{c}$ in \autoref{ttft} bears some resemblance to the truth table for $(P\to Q)$, but how can we characterise the relationship between truth-functions and labelled truth-tables for sentences more precisely? What we want to say is that a sentence $\phi$ is related to a truth-function $f$ if, once we label the columns on the truth-function table in the standard way, we get the truth-table for $\phi$. Here is the official definition.  
\begin{definition}[Expression]
A sentence $\phi$ \emph{expresses} a truth function $f_{n}$ iff if $\phi$ contains exactly the sentence letters $s_{1},…,s_{n}$ (ordered in the standard way)  then for every \lone-structure $\mathscr{A}$, $$f(\mathscr{A}(s_{1}),\ldots,\mathscr{A}(s_{n}))= \val{\phi}{A}.$$ 
\end{definition} 
So, for example, $(P\to Q)$ expresses the truth-function $\mathbf{c}$, because it contains the sentence letters $P,Q$; the standard order on sentence letters places $P$ before $Q$, and in every structure, $\mathbf{c}(\mathscr{A}(P),\mathscr{A}(Q)) = \val{(P\to Q)}{A}$. Likewise, $(Q \to R)$ also expresses $\mathbf{c}$. But $(Q\to P)$ does not express $\mathbf{c}$, because in any structure $\mathscr{B}$ where $\mathscr{B}(P)=F,\mathscr{B}(Q)=T$, $\val{(Q \to P)}{B}=F$ but $\mathbf{c}(\mathscr{B}(P),\mathscr{B}(Q)) = \mathbf{c}(F,T) = T$. 
 

Since every \lone\ sentence has a truth-table produced in the orthodox way, every \lone\ sentence expresses some truth-function.   The converse claim, that every truth-function is expressed by some sentence, will be proved in chapter \ref{c4} when we talk about \emph{expressive adequacy} (page \pageref{expressiveadeq}).




\paragraph{Truth-Functional Connectives} We've already seen some sentences which express interesting truth functions above: those  sentences expressing the truth functions associated with the connectives of \lone\ (\autoref{tt}). 
\begin{definition}[Truth-functional Connective]\label{tfc}
	A \emph{truth-functional connective} is an $n$-place connective $\oplus$ such that there exists a $n$-place truth function $f$ such that, for any $\phi_{1},\ldots,\phi_{n}$, $$\val{\oplus(\phi_{1},\ldots,\phi_{n})}{A} = f\left(\val{\phi_{1}}{A},\ldots,\val{\phi_{n}}{A}\right),$$ i.e., if $\oplus(\phi_{1},\ldots,\phi_{n})$ expresses some truth function $f$. 
\end{definition}
It is evident from Table \ref{tt}  that each of the connectives of \lone\ are truth-functional connectives, so \lone\ is a \emph{truth-functional language}. We've already encountered the 1-place truth function corresponding to the negation connective: it is $\mathbf{n}$ from Table \ref{t1tf}. We also encountered the 2-place connective $\mathbf{c}$ corresponding to the conditional connective. Here are three other pertinent two-place truth functions:
\begin{align*}
	\mathbf{a}(x,y) &= \begin{cases} F &\text{if } x =  y = F;\\ 
T &\text{otherwise}.\end{cases}\\
 \mathbf{k}(x,y) &= \begin{cases} T &\text{if } x = y = T;\\
F &\text{otherwise}.\end{cases}\\
\mathbf{e}(x,y) &= \begin{cases} T &\text{if } x = y;\\
F &\text{otherwise}.\end{cases}
\end{align*}
It is easy to verify which connectives in Table \ref{tt} express these truth-functions.

\paragraph{Non-Truth Functional Connectives} Consider the natural language connective `necessarily', i.e., the connective that, given a sentence $\phi$, yields a sentence \cquote{\text{Necessarily }\phi}. If $\phi$ is false in a structure, that obviously guarantees that \cquote{\text{Necessarily }\phi} is false in that structure too (assuming the standard meaning for ‘Necessarily’). But if $\phi$ is true in a structure, that doesn't tell us whether \cquote{\text{Necessarily }\phi} is true in that structure: for some truths are necessary, others merely contingent. So to deal with the `Necessarily' operator, we will have to introduce some further technology than we have available in \lone\ structures and classical valuations. That is the topic taken up by \emph{modal logic} – see the discussion at the end of chapter \ref{c8}. We already can see, however, one expressive limitation of \lone\ as compared to English: for English includes non-truth-functional connectives like `necessarily', `probably', `typically', etc., and \lone\ does not. Every sentence in \lone\ has its truth value determined by the valuation rules, and the valuation rules only make use of truth functions.

\paragraph{Revisiting the Semantics} We could use our truth-functions to express the semantics for our language in another form, giving this alternative definition of a valuation. \begin{definition}[Alternative \lone-valuation]
	When $\mathscr{A}$ is any \lone-structure, the valuation function generated by $\mathscr{A}$ is this function\label{val2}: $$\val{\phi}{A} = \begin{cases}
		\mathscr{A}(\phi) &\text{if $\phi$ is a sentence letter};\\
		\mathbf{n}\left(\val{\psi}{A}\right) &\text{if } \phi = \neg\psi;\\
		\mathbf{k}\left(\val{\psi}{A},\val{\chi}{A}\right) &\text{if } \phi = (\psi \wedge \chi);\\ 
		\mathbf{a}\left(\val{\psi}{A},\val{\chi}{A}\right) &\text{if } \phi = (\psi \vee \chi);\\
		\mathbf{c}\left(\val{\psi}{A},\val{\chi}{A}\right) &\text{if } \phi = (\psi \to \chi);\\
		\mathbf{e}\left(\val{\psi}{A},\val{\chi}{A}\right) &\text{if } \phi = (\psi \bicond \chi).\\ 
	\end{cases}$$
\end{definition} This definition is obviously equivalent to that in Definition \ref{value}, in that both definitions determine the same valuation of all the sentences of the language given the same initial structure.





\begin{definition}[Base] \citep[27]{bevpospa}
Let $\val{\cdot}{A}$ be a classical valuation. A \emph{base} $\mathfrak{B}$ for $\val{\cdot}{A}$ is a triple $\langle V, \mathbb{O}, c\rangle$ where $V$ is a set of elements, $\mathbb{O}$ a set of operators on $V$ (functions such that their domain and range are both $V$ – recall Definition \ref{operator}), and $c$ is a function from the set of connectives of \lone, $\{\neg,\wedge,\vee,\to,\bicond\}$, into $\mathbb{O}$, such that \begin{itemize}
	\item $\val{\phi}{A} \in V$, for any \lone\ sentence $\phi$;
	\item for any sentences $\phi_{1},\ldots,\phi_{n}$, and any $n$-ary connective $\oplus$, $$\val{\oplus\left(\phi_{1},\ldots,\phi_{n}\right)}{A,L} = c(\oplus)\left(\val{\phi_{1}}{A,L},\ldots,\val{\phi_{n}}{A,L}\right).$$
\end{itemize} 
\end{definition}
Informally, a base for a valuation is a set of values $V$, a set of truth-functions $\mathbb{O}$, and a mapping that takes connectives of the language to the truth-functions they express (Definition \ref{tfc}). A base for \lone\ is this: let $V = \{T,F\}$, $\mathbb{O} = \{\mathbf{n},\mathbf{k},\mathbf{a},\mathbf{c},\mathbf{e}\}$, and let $c$ be the function that maps $\neg$ to $\mathbf{n}$, $\wedge$ to $\mathbf{k}$, \emph{etc}.




{\small


\subsection*{Exercises} \label{ex4}
\addcontentsline{toc}{subsection}{Exercises}


\begin{enumerate}
	\item Show that $\Gamma \vDash \phi$ iff $\Gamma \cup \{\neg \phi\} \vDash$.

	\item State what features the truth table for $\Gamma \cup \{\phi,\psi\}$ will possess when \begin{enumerate}
		\item $\Gamma$ entails $\phi$;
		\item $\Gamma$ is consistent;
		\item $\phi$ is a contradiction;
		\item $\phi$ and $\psi$ are logically equivalent.
	\end{enumerate}
	
\item \begin{enumerate}
	\item Prove Transitivity.
	\item Prove Contraposition.
	\item Prove this rule, related to Cut: if $\Gamma \vDash \phi$, and $\phi,\Delta \vDash \psi$, then $\Gamma,\Delta \vDash \psi$.
\end{enumerate} 

\item Show the remaining clauses of the induction step for the proof of Theorem \ref{fisut} for the cases where the complex sentence is of the forms: \begin{enumerate}
	\item $(\phi \wedge \psi)$;
	\item $(\phi \to \psi)$;
	\item $(\phi \bicond \psi)$.
\end{enumerate}


\item Prove Equivalents (Theorem \ref{equivalents}).

\item \begin{enumerate}
	\item Prove the remaining induction cases in the proof of Equivalence (Theorem \ref{tequiv}).
	\item As a corollary of the Equivalence Theorem, prove that if $\vDash \phi\bicond\psi$ and $\phi$ is a constituent sentence of $\chi$, then $\vDash \chi \bicond \chi[\psi/\phi]$.
\end{enumerate}

\item Show that, if $\phi,\psi \vDash \chi$ is a false sequent,
it has a substitution instance $\phi',\psi' \vDash \chi'$, in which 
$\phi'$ and $\psi'$ are tautologies and $\chi'$ is
inconsistent. (Note: by `substitution instance' in the question, you are to understand a sequent that results from one \emph{or more} uniform substitutions of sentence letters. It follows from General Substitution [Theorem \ref{generalsub}] that successive chains of uniform substitutions never allow one to transform a correct sequent to an incorrect one.)

\item  Show that if there are $n$ sentence letters in $\Gamma$, the truth table for $\Gamma$ has $2^{n}$ lines.

\item Suppose $\phi$ and $\psi$ contain the same sentence letters. Prove that $\phi$ and $\psi$ are logically equivalent iff they express the same truth function.

\item Consider the deprived language which has the same syntactic formation rules and semantics as $\mathcal{L}_{1}$, but only contains the connectives $\neg$ and $\bicond$. Show that, in this language, every sentence is logically equivalent to a sentence in which no occurrence of $\neg$ has $\bicond$ in its scope.
\end{enumerate}

Answers to selected exercises on page \pageref{ans4}.
}  
\newpage

\chapter{Some Metatheorems Concerning $\mathcal{L}_{1}$}\label{c4}
 %!TEX root = edl.tex

\section{Disjunctive Normal Form}

\begin{definition}[Arbitrary Conjunction] An inductive definition: \begin{itemize}
	\item $\left[\bigwedge_{i=1}^{1} \phi_{i}\right] \eqdf \phi_{1}$;
	 \item $\left[\bigwedge_{i=1}^{n+1}\phi_{i}\right]\eqdf \left(\left[\bigwedge_{i=1}^{n} \phi_{i}\right]\wedge \phi_{n+1}\right).$
\end{itemize} 
\end{definition} The definition of arbitrary disjunction, $\bigvee_{i=1}^{n}$, is wholly parallel to that for arbitrary conjunction. (Giving it explicitly is left as an exercise.) \begin{definition}[Literal]
	A \emph{literal} is any sentence letter or negated sentence letter. 
\end{definition}
\begin{definition}[Disjunctive Normal Form (\textsc{\lowercase{DNF}})]
	An \lone\ sentence of $\phi$ is in \emph{disjunctive normal form} iff there exist $n,m_{1},\ldots,m_{n}$ such that\[\phi = \left[\bigvee_{i=1}^{n}\left[\bigwedge_{j=1}^{m_{i}} \pm s_{i,j}\right]\,\right],\]where each $\pm s_{i,j}$ is a literal.
\end{definition} That is: a sentence is in disjunctive normal form iff it is a (perhaps degenerate) disjunction of (perhaps degenerate) conjunctions of (perhaps negated) sentence letters. One clear example is $((P\wedge \neg Q) \vee (P \wedge Q))$. But since $P$ is a literal, and it is a degenerate arbitrary conjunction, and a degenerate arbitrary disjunction, it is in \textsc{\lowercase{DNF}}also. Another way of characterising which \lone\ sentences are in \textsc{\lowercase{DNF}}is as follows: a sentence is \textsc{\lowercase{DNF}}iff it contains at most only connectives drawn from $\{\neg,\wedge,\vee\}$, and such that $\vee$ never occurs in the scope of $\wedge$ or $\neg$, and $\wedge$ never occurs in the scope of $\neg$.


Our interest in disjunctive normal form lies in the following theorem: 
\begin{theorem}[\textsc{dnf}]\label{dnf}
	Every truth function is expressed by a \lone\ sentence in \textsc{\lowercase{DNF}}.
\begin{proof}	
Suppose $f$ is an $n$-place truth function. Let $s_{1},\ldots,s_{n}$ be  sentence letters, and let $\mathscr{A}$ be any structure.  Define an \emph{$f$-agreeing} structure $\mathscr{A}$ as one where $f(\val{s_{1}}{A},\ldots,\val{s_{n}}{A}) = T$. Assume there is at least one $f$-agreeing structure. Then define $\mathfrak{l}_{\mathscr{A},i} = s_{i}$ if $\mathscr{A}(s_{i})=T$; otherwise let $\mathfrak{l}_{\mathscr{A},i} = \neg s_{i}$. 
For any structure $\mathscr{B}$ that agrees with $\mathscr{A}$ on $s_{1},\ldots,s_{n}$, let $\mathfrak{l}_{\mathscr{B},i}=\mathfrak{l}_{\mathscr{A},i}$. It is obvious that $\val{\mathfrak{l}_{\mathscr{A,i}}}{A} = T$ for each $\mathscr{A},i$.

For every structure $\mathscr{A}$, define $\mathfrak{c}_{\mathscr{A}} = \mathfrak{l}_{\mathscr{A},1}\wedge\ldots\wedge \mathfrak{l}_{\mathscr{A},n} = \left[\bigwedge_{i=1}^{n}\mathfrak{l}_{\mathscr{A},i}\right]$. Again, it is obvious that $\val{\mathfrak{c}_{\mathscr{A}}}{A}=T$. Since there are only finitely many structures that differ from one another in their assignments to $s_{1},\ldots,s_{n}$, there are only finitely many distinct sentences $\mathfrak{c}_{\mathscr{A}}$ for various $\mathscr{A}$.  Therefore, there are only finitely many $\mathfrak{c}_{\mathscr{A}}$ \emph{such that $\mathscr{A}$ is an $f$-agreeing structure}; let them be enumerated $\mathfrak{c}_{1},\ldots,\mathfrak{c}_{m}.$ It is obvious that each $\mathfrak{c}_{i}$ is true in one and only one structure, and that each such structure is $f$-agreeing.

 Define $\mathfrak{d} = \mathfrak{c}_{1} \vee \ldots \vee \mathfrak{c}_{m} = \left[\bigvee_{j=1}^{m} \mathfrak{c}_{j}\right]$,  \emph{unless} there were no $f$-agreeing structures, in which case let $\mathfrak{d} = P_{1} \wedge \neg P_{1}$. It is obvious that $\mathfrak{d}$ is true in any structure which is among those such that some $\mathfrak{c}_{i}$ is true, i.e., $\mathfrak{d}$ is true in any $f$-agreeing structure.

$\mathfrak{d}$ is in DNF, by construction; and $\val{\mathfrak{d}}{A} = f(\val{s_{1}}{A},\ldots,\val{s_{n}}{A})$  for all $\mathscr{A}$, because it captures all and only the $f$-agreeing structures. So $\mathfrak{d}$ expresses $f$. \end{proof}\end{theorem}



The truth-table rationale for this proof is clear: first, find the rows of the truth table on which the truth-function gets $T$ (those are the $f$-agreeing rows). Specify a sentence true in exactly one row by conjoining the relevant literals (so if the row is $\val{P}=T$, $\val{Q}=F$, the relevant conjunction is $P \wedge \neg Q$). Then disjoin those conjunctions which correspond to the $f$-agreeing rows; the result is a \textsc{\lowercase{DNF}} sentence true in exactly the $f$-agreeing rows. 

We can obviously use this result to show that, for any sentence $\phi$, there is a sentence $\phi'$ which is logically equivalent to $\phi$ and which is in \textsc{\lowercase{DNF}} form. This follows immediately from the \textsc{\lowercase{DNF}} theorem and the fact that every sentence expresses a truth function.


\paragraph{\textsc{\lowercase{CNF}} and Positive Truth Functions}

A formula is said to be in \emph{Conjunctive Normal Form} iff it is a (possibly degenerate) conjunction of (possibly degenerate) disjunctions of literals. This is obviously a dual notion to \textsc{\lowercase{CNF}}.
In the case of \textsc{\lowercase{DNF}}, the \textsc{\lowercase{DNF}} sentence expressing some truth function $f$ is the disjunction of conjunctions, each of which specifies an $f$-agreeing structure. Each conjunct is therefore \emph{sufficient} for $f$ to have the value $T$ in the corresponding structure; the \textsc{\lowercase{DNF}} sentence is the disjunction of all of these sufficient conditions. For \textsc{\lowercase{CNF}}, we want the dual notion: the \textsc{\lowercase{CNF}} expression of $f$ will be a sentence which is a conjunction of disjunctions, each of which specifies a  necessary condition for a structure to be $f$-agreeing. And what is a necessary condition for a structure to be $f$-agreeing? It must avoid being \emph{$f$-disagreeing} – so each disjunct will be a disjunction of sentence letters and negated sentence letters, the truth of any of which is sufficient to ensure that some $f$-disagreeing structure is avoided.  (A problem asks you to make the foregoing remarks into a more precise proof of the \textsc{\lowercase{CNF}} theorem.)


Call a truth function $f$ \emph{positive}  iff $f(T,\ldots,T)=T$ \citep[exercise 2.9.3]{bosintlo}. (Equivalently, if the top row of its truth table is $T$.) We can show that  all truth-functions which can be expressed using only $\to$ and $\wedge$ are positive (left for problem 9). Of interest is the converse theorem: \begin{theorem}
	All positive truth functions can be expressed by $\to$ and $\wedge$. \label{positive} \begin{proof}
	I sketch the proof.	Suppose the contrary, for \emph{reductio}. Then there is a truth function $f$ which is positive but can't be expressed by $\wedge,\to$. There is a \textsc{\lowercase{CNF}} sentence $\phi_{c}$ which expresses $f$; this will be a conjunction of disjunctions of (negated) sentence letters. By the constructive proof of the \textsc{\lowercase{CNF}} theorem, it can be shown (by induction on complexity of sentences, and using the result that $\phi\vee \psi\Dashv\vDash (\psi \to \phi)\to \phi$) that each conjunct of the resulting \textsc{\lowercase{CNF}} sentence will be equivalent to either (i) an arrow sentence \emph{without} negation, or (ii)  a conjunct  of the form $\neg \phi \vee \neg \psi$ (problem 10). In the second case, the \textsc{\lowercase{CNF}} sentence as a whole can't be positive. So the first case must obtain for $\phi_{c}$. By substituting each conjunct of $\phi_{c}$ for the arrow-sentence equivalent, we obtain a sentence $\phi$, which is a conjunction of arrow sentences without negation. But $\phi$ expresses $f$ and contains only $\to$ and $\wedge$.
	\end{proof}
\end{theorem} 




\section{Functional Completeness, Expressive Adequacy}
\paragraph{Functional Completeness and Expressive Adequacy}

A language is said to be \emph{functionally complete} iff there is a formula in that language that, under the intended semantics, expresses every truth function. The \textsc{\lowercase{DNF}} theorem shows that every truth function is expressed by a formula in \lone\ (as \textsc{\lowercase{DNF}} claims clearly are), and so \lone\ is functionally complete.

A set of connectives is \emph{expressively adequate} iff there is a sentence containing only those connectives which expresses any truth function. The \textsc{\lowercase{DNF}} theorem shows that $\{\vee,\wedge,\neg\}$ is expressively adequate, and therefore that the set of connectives of \lone\ is expressively adequate.

What this last result shows is that, in some sense, $\to$ and $\bicond$ are \emph{dispensible} – every truth function expressible using them can be expressed by a sentence without them (i.e., a \textsc{\lowercase{DNF}} one).

There is another way to show this same result. Since two sentences are logically equivalent iff they express the same truth-function, the \textsc{\lowercase{DNF}} theorem, together with the fact that every \lone\ sentence expresses a truth-function, suffices to show that every \lone\ sentence is logically equivalent to one in \textsc{\lowercase{DNF}} form. Hence sentences involving $\to$ and $\bicond$ are logically equivalent to sentences in \textsc{\lowercase{DNF}} form, and therefore in a sense dispensible.



\paragraph{The De Morgan Equivalences}

Let `$\phi\Dashv\vDash\psi$' abbreviate `$\phi \vDash \psi$ and $\psi \vDash \phi$'.
\begin{theorem}[De Morgan Equivalences]
		\begin{align*}
			(\phi \vee \psi) &\Dashv\vDash \neg(\neg\phi \wedge \neg \psi),\\
	(\phi \wedge \psi) &\Dashv\vDash\neg(\neg\phi \vee \neg \psi).
		\end{align*}
\begin{proof}
	These equivalences can be easily demonstrated using a truth table: Table \ref{tone}.
\end{proof}
\end{theorem} \begin{table} \label{tone}
    \centering
    \begin{tabular}{cc|cc|cc}
    \toprule
        $\phi$ & $\psi$ &$\phi \wedge \psi$ & $\neg(\neg \phi \vee \neg \psi)$ & $\phi \vee \psi$ & $\neg (\neg \phi \wedge \neg \psi)$\\
        \midrule    
    $T$\ & $T$\ & $T$\ & $T$\ & $T$\ & $T$\ \\
    $T$\ & $F$ & $F$& $F$ & $T$\ & $T$\  \\
    $F$ & $T$& $F$ &$F$ & $T$\ & $T$\ \\
    $F$ & $F$& $F$ & $F$& $F$& $F$\\
    \bottomrule
    \end{tabular}
\caption{De Morgan Equivalences}
\end{table}
These equivalences show that $\{\neg,\vee\}$ and $\{\neg,\wedge\}$ are expressively adequate, given the \textsc{\lowercase{DNF}} theorem.

\paragraph{Inadequacy and New Connectives}

A set of connectives is expressively \emph{inadequate} iff there is some truth function which cannot be expressed by sentences involving only those connective. Accordingly, $\{\neg\}$  is expressively inadequate: quite apart from whether it can express a 2- or more place truth function, consider the 1-place function $f_{T}$ which is constantly true. Since $f_{T}(T)=f_{T}(F)$,  we would need a sentence $\phi$ involving only negation such that in even one case $\val{\phi}{A}=\val{\neg\phi}{A}$; this clearly cannot be.

Given our definition of a connective, we can imagine extending our language \lone\ by adding new sentence-forming operators that express different connectives. We could add the $0$-place connective $\top$ which expresses $f_{T}$. As we know from the \textsc{\lowercase{DNF}} theorem, adding $\top$ doesn't add to the expressive power of the language: $\top$ is logically equivalent to $(P \vee \neg P)$, and so adding it would be redundant.  Adding $\wedge$ to the language $\mathcal{L}_{\neg}$, in which the only connective is negation, is not redundant: $\mathcal{L}_{\neg,\wedge}$ is functionally complete and $\mathcal{L}_{\neg}$ is not. 

\paragraph{Sheffer Stroke and Peirce Arrow}

The connectives $\uparrow$ and $\downarrow$ can be added to a language, with the truth tables in Table \ref{ttwo}: 
\begin{table}\label{ttwo}
\centering
\begin{tabular}{cc|c|c}
\toprule
	$\phi$ & $\psi$ & $(\phi \uparrow \psi)$ & $(\phi \downarrow \psi)$\\
	\midrule
	$T$ & $T$ & $F$ & $F$\\
	$T$ & $F$ & $T$ & $F$\\
	$F$ & $T$ & $T$ & $F$\\
	$F$ & $F$ & $T$ & $T$\\	
	\bottomrule
\end{tabular}	\caption{Sheffer Stroke and Peirce Arrow}
\end{table}The $\uparrow$, also written $|$, is called the \emph{Sheffer stroke}; the $\downarrow$ has no standard name, but is sometimes called the \emph{Peirce arrow}. 

We can show that both of these connectives are expressively adequate. The  Sheffer stroke is, since $\neg \phi$ is logically equivalent to $(\phi\uparrow\phi)$, and $(\phi \wedge \psi)$ is logically equivalent to  $((\phi\uparrow\psi)\uparrow(\phi\uparrow\psi))$.


\section{Duality}
\paragraph{Duality}

$\wedge$ and $\vee$, as well as $\uparrow$ and $\downarrow$, are \emph{duals} of each other. 

\begin{definition}[Duality]
The \emph{dual} of a truth-functional connective is that connective whose truth table
results from that of the given connective by replacing \emph{every} occurrence
of $T$ by $F$ and every occurrence of $F$ by $T$ (see Table \ref{tthree}).
\end{definition}
\begin{table}\centering\begin{tabular}{ccc}
{    \begin{tabular}{cc|c|c}
\toprule
$\phi$ & $\psi$ & $\phi \wedge \psi$ & $\phi \uparrow \psi$ \\
\midrule
$T$ & $T$ & $T$ &$F$\\
$T$ & $F$ & $F$&$F$\\
$F$ & $T$ & $F$ &$F$\\
$F$ & $F$ & $F$ &$T$\\
\bottomrule
	\end{tabular}} &
$\Leftrightarrow$ &
{	\begin{tabular}{cc|c|c}
$\phi$ & $\psi$ & $\phi \vee \psi$ & $\phi\downarrow\psi$ \\
\hline
$F$ & $F$ & $F$ &$T$\\
$F$ & $T$ & $T$ &$T$\\
$T$ & $F$ & $T$ &$T$\\
$T$ & $T$ & $T$	&$F$
\end{tabular}
}    \end{tabular}\caption{Duality Illustrated Using Truth Tables\label{tthree}}\end{table}
Other connectives have duals too: $\phi \bicond \psi$ is dual to the operator $\nleftrightarrow$ (where $\phi \nleftrightarrow\psi$ is true just
in case $\phi$ and $\psi$ have different truth values).


\paragraph{The Duality Lemma for $\wedge,\vee$}

Suppose that $\phi$ is a sentence of \lone\ which involves only connectives in $\{\wedge,\vee,\neg\}$ (any sentence of \lone\ will be equivalent to some such $\phi$). Now we define two operations on sentences of this form: \begin{enumerate}
	\item Let $\overline{\phi}$ be the sentence that
	results from writing `$\neg$' directly in front of every sentence letter in
	$\phi$ (even those already negated).
	\item Let $\phi^{\star}$ be the sentence that results from substituting each connective in $\phi$ by its dual.
\end{enumerate}
\begin{lemma}[Duality for Conjunction and Disjunction]
	For any $\phi$, $\phi^{\star}\Dashv\vDash\neg\overline{\phi}$. 
\begin{proof}
	\emph{Base case:}  $\phi$ is just a sentence
	letter: so that $\phi^{\star} = p$, and $\neg\overline{\phi} =
	\neg\neg p$, which are equivalent (by truth tables).

	\emph{Induction step:} \begin{enumerate}
		\item $\phi$ is $\neg \psi$, and the hypothesis holds for $\psi$. Then $\phi^{\star} = (\neg \psi)^{\star} = \neg
	    (\psi^{\star})$. Because the theorem holds of $\psi$, $\neg (\psi^{\star})
	    \Dashv\vDash \neg\neg\overline{\psi}$; but $\neg \neg\overline{\psi} =
	   \neg \overline{\neg\psi} = \neg\overline{\phi}$, which proves the
	   case.
	   	\item $\phi$ is $(\psi \wedge \chi)$, and the hypothesis holds for $\psi,\chi$. $\phi^{\star} = (\psi \wedge \chi)^{\star} = \psi^{\star} \vee
	   \chi^{\star}$. By the induction hypothesis, $\psi^{\star} \vee \chi^{\star}
	   \Dashv\vDash \neg\overline{\psi} \vee \neg \overline{\chi}$. By De
	   Morgan, $\neg\overline{\psi} \vee \neg \overline{\chi} \Dashv\vDash
	   \neg (\overline{\psi} \wedge \overline{\chi})$, which is
	   $\neg(\overline{\psi \wedge \chi}) =\neg(\overline{\phi})$.
		\item $\phi$ is $(\psi \vee \chi)$; much as for $\wedge$. 
	\end{enumerate}
\end{proof}\end{lemma}


\paragraph{Duality Theorem}

 \begin{theorem}[Duality for Conjunction and Disjunction]
	If $\phi^{\star}$ is dual to $\phi$ and $\psi^{\star}$ is dual to $\psi$, then $\phi \vDash \psi$ iff $\psi^{\star}\vDash \phi^{\star}$.\begin{proof}
		Suppose $\phi\vDash\psi$. Then, by Substitution of each sentence letter by its negation, $\overline{\phi}\vDash\overline{\psi}$. By Contraposition, $\neg\overline{\psi}\vDash\neg\overline{\phi}$. By the Duality Lemma, $\psi^{\star}\vDash\phi^{\star}$. 
	\end{proof}
\end{theorem}
\begin{corol} If $\phi \Dashv\vDash \psi$, $\phi^{\star}\Dashv\vDash\psi^{\star}$.
\end{corol}

\begin{theorem}[Duality for Tautologies] If $\vDash \phi$, then $\vDash \neg(\phi^{\star})$. \begin{proof}
	Suppose $\vDash\phi$. Since $\phi$ is true in every structure, $\overline{\phi}$ is also true in every structure (obvious by substitution). So $\vDash \overline{\phi}$. By the Duality Lemma, $\overline{\phi}\Dashv\vDash\neg\phi^{\star}$. So $\vDash \neg\phi^{\star}$.
\end{proof}
\end{theorem}

\paragraph{Duality Generalised}

Every truth functional connective $c$ – in any truth-functional language – has a dual, which we denote $c^{\star}$ by extending our previous notation. A set of truth functors $\mathbb{C}=\{c_{1},\ldots,c_{n}\}$ is \emph{self-dual} iff $\mathbb{C} = \{c_{1}^{\star},\ldots,c_{n}^{\star}\}$. It is obvious that $\{\neg\}$ is self-dual. Given the result above, the set $\{\wedge,\vee\}$ is self-dual, as $\vee = \wedge^{\star}$ and $\wedge = \vee^{\star}$.

Define a function $\star$ on truth-values such that $T^{\star}=F$ and $F^{\star}=T$. If $\mathscr{A}$ is a structure, let $\mathscr{A}^{\star}$ be the structure such that for all $\phi$, $\mathscr{A}^{\star}(\phi)=T^{\star}$ iff $\mathscr{A}(\phi)=T$. For any language $\mathcal{L}$, let a connective $c$ be in $\mathcal{L}$ iff $c^{\star}$ is in $\mathcal{L}$. The dual $\phi^{\star}$ of a sentence $\phi$ of $\mathcal{L}$ results from replacing \emph{every} connective in $\phi$ by its dual. Then we can show: $$\val{\phi}{A} = (\val{\phi^{\star}}{A^{\star}})^{\star}.$$ 





\section{Interpolation}


\paragraph{Craig Interpolation Theorem for Sentential Logic}

\begin{theorem}[Craig Interpolation]\label{thmcraig}If $\phi 
\vDash \psi$, and there is a non-empty set $B$ of sentence letters occurring in both $\phi$ and
$\psi$, then there is a sentence $\chi$ (an \emph{interpolant}) such that both $\phi \vDash \chi$ and
$\chi \vDash \psi$, and each sentence letter in $\chi$ is in $B$. \end{theorem}
\begin{proof} Let the members of $B$ be enumerated $b_{1},\ldots,b_{n}$.
Let $\mathscr{A}$ be a \emph{$B$-variant} of $\mathscr{B}$ iff $\mathscr{A}(b)=\mathscr{B}(b)$ for all $b\in B$. 

 Let $\{\mathscr{B}_{1},\ldots,\mathscr{B}_{2^{n}}\}$ be a set of structures every pair of which differ on their assignment of a truth value to at least one $b \in B$. Define a $n$-place truth function $f_{\chi}$, for every structure $\mathscr{B}_{i}$: \begin{equation*}
f_{\chi}\left(\left\langle\left|{b_{1}}\right|_{\mathscr{B}_{i}},\ldots,\left|{b_{n}}\right|_{\mathscr{B}_{i}}\right\rangle\right) =
 	 \begin{cases} F & \parbox[c]{.4\textwidth}{if there is a $B$-variant $\mathscr{A}$ of $\mathscr{B}_{i}$ in which $\val{\psi}{A}=F$;}\\
T & \parbox[c]{.4\textwidth}{otherwise.}\end{cases}
 \end{equation*}
 By construction, $f_{\chi}$ is a total truth function. 
The sentence $\chi$ that (by \textsc{\lowercase{DNF}}) expresses $f_{\chi}$ is our needed interpolant.
\begin{itemize}
	\item $\chi$ clearly 
	entails $\psi$, since by construction no structure that makes $\chi$ true has a $B$-variant that makes $\psi$ false.
	\item  Moreover, $\chi$ is clearly entailed by $\phi$,
	since if there were a structure $\mathscr{C}$ which made $\phi$ true but $\chi$
	false,  $\mathscr{C}$ would be a $B$-variant of some $\mathscr{B}_{i}$ that makes
	$\psi$ true; but by construction $\chi$ would be true in $\mathscr{B}_{i}$
	and in every $B$-variant of it, including  $\mathscr{C}$. There can
	therefore be no such  $\mathscr{C}$ that both makes $\chi$ true and false.
\end{itemize}  \end{proof}

The Craig Interpolation Theorem actually names the version of this result proved for predicate logic  \citep{crathrush}.


\section{Compactness}


\paragraph{Compactness}

If $\Gamma$ is a inconsistent set of \lone\ sentences ($\Gamma\vDash$), then it needn't be that there is a proper subset $\Gamma' \subset \Gamma$ such that $\Gamma'\vDash$ – while $P,\neg P\vDash$, clearly $\{P\}$ and $\{\neg P\}$ are both consistent.

But, it turns out, when $\Gamma$ is \emph{infinite} and inconsistent, there is always some \emph{finite} subset of $\Gamma$ which is inconsistent. This property is known as \emph{compactness}. 

It has the immediate consequence that if $\Gamma\vDash\phi$, then there is some finite set of premises $\Gamma'$ such that $\Gamma'\vDash\phi$; every valid argument in \lone\ can be captured by a finite sequent.

\paragraph{Compactness Theorem}

A set $\Gamma$ is \emph{finitely satisfiable} iff every finite subset of $\Gamma$ is satisfiable (i.e., has a model).
 
\begin{theorem}[Compactness]\label{compact}
If $\Gamma$ is finitely satisfiable then $\Gamma$ is satisfiable. (The contrapositive of this is Compactness in the sense introduced just above.)	
\end{theorem}

This theorem can be proved a number of ways. It is a fairly quick consequence of the Completeness theorem for the natural deduction formalism for sentential logic, which we  prove in the next chapter. Here I will prove it directly. The proof comes in several stages, marked out below.

Note that the \emph{converse} of Compactness – that satisfiability entails finite satisfiability – is trivial from the structural rules.

\paragraph{Proof of Compactness, Stage 1: Definition of $\Gamma_{i}$s}

Suppose that $\Gamma$ is finitely satisfiable. The set of sentence letters occurring in $\Gamma$ is enumerable, so let its members be enumerated $\phi_{1},\phi_{2},\ldots$.

We define a sequence of supersets of $\Gamma$ as follows:
 \begin{align*}
		\Gamma_{0} &= \Gamma\\
	&\vdots \\
	\Gamma_{n+1} &= \begin{cases}
		\Gamma_{n}\cup\{\phi_{n+1}\} &\text{ if } \Gamma_{n} \cup \{\phi_{n+1}\} \text{ is finitely satisfiable;}\\
		\Gamma_{n}\cup\{\neg\phi_{n+1}\} &\text{ otherwise.}
	\end{cases}
\end{align*}



\paragraph{Proof of Compactness, Stage 2:  $\Gamma_{i}$s are finitely satisfiable}

\begin{lemma}
Each $\Gamma_{i}$ is finitely satisfiable.
\begin{proof}
{\emph{Base case:} $\Gamma_{0}=\Gamma$, which is finitely satisfiable by hypothesis.
	
	\emph{Induction step:} Suppose that each $\Gamma_{i}$, $i\leqslant n$, is finitely satisfiable. Now suppose $\Gamma_{n+1}$ is not. Then, by definition of $\Gamma_{n+1}$, neither $\Gamma_{n}\cup\{\phi_{n+1}\}$ nor $\Gamma_{n}\cup\{\neg\phi_{n+1}\}$ is finitely satisfiable.
	
	So there must be finite subsets $\Delta$ and $\Theta$ of $\Gamma_{n}$ such that both $\Delta,\phi_{n+1}\vDash$ and $\Theta,\neg\phi_{n+1}\vDash$. By weakening, both $\Delta,\Theta,\phi_{n+1}\vDash$ and $\Delta,\Theta,\neg\phi_{n+1}\vDash$. 
	
	But $\Delta\cup\Theta$ is a finite subset of a finitely satisfiable set, so is consistent. In any structure $\mathscr{A}$ in which all the members of $\Delta\cup\Theta$ are true, either $\val{\phi_{n+1}}{A}=T$ or $\val{\neg\phi_{n+1}}{A}=T$. So  either $\Delta,\Theta,\phi_{n+1}\not\vDash$ or $\Delta,\Theta,\neg\phi_{n+1}\not\vDash$. Contadiction; so $\Gamma_{n+1}$ is finitely satisfiable. 
}\end{proof}
\end{lemma}

\paragraph{Proof of Compactness, Stage 3: Define a structure from the $\Gamma_{i}$s}

For all $\phi_{i}$, define \begin{equation*}
	\mathscr{G}(\phi_{i}) = \begin{cases}
		T & \text{if } \phi_{i} \in\Gamma_{i};\\
		F & \text{otherwise (i.e., if $\neg\phi_{i}\in\Gamma_{i}$)}.
	\end{cases}
\end{equation*}

For any $k$, $\Gamma_{k}$ is finitely satisfiable; since for all $i\leqslant k$, either $\phi_{i}\in\Gamma_{k}$ or $\neg\phi_{i}\in\Gamma_{k}$, the finite set of literals $$\Phi_{k} = \{\phi_{i}\in\Gamma_{k}:i\leqslant k\}\cup\{\neg\phi_{i}\in\Gamma_{k}:i\leqslant k\}$$
is a subset of $\Gamma_{k}$ and is consistent. 

By construction, for any $k$, and for all $\psi\in\Phi_{k}$, $\val{\psi}{G}=T$. Moreover, \emph{every} structure $\mathscr{G}'$ which agrees with $\mathscr{G}$ on the sentence letters in $\Phi_{k}$ also assigns $T$ to all members of $\Phi_{k}$.  (And obviously, by construction, only those structures which agree on these sentence letters can satisfy $\Phi_{k}$.)

\paragraph{Proof of Compactness, Stage 4: $\mathscr{G}$ satisfies $\Gamma$}

Suppose $\mathscr{G}$ does not satisfy $\Gamma$.
Then for some $\gamma\in\Gamma$, $\val{\gamma}{G}=F$.

$\gamma$ has finitely many sentence letters occurring in it; let $\phi_{k}$ be the highest numbered. Since every $\psi\in\Phi_{k}$  is true in every structure which agrees with $\mathscr{G}$ on $\phi_{i}$, $i\leqslant k$, $\gamma$ is false in every such structure.

So $\Phi_{k}\cup\{\gamma\}$ is unsatisfiable. But it is a finite subset of $\Gamma_{k}$, which is finitely satisfiable, so $\Phi_{k}\cup\{\gamma\}$ satisfiable. Contradiction; so $\mathscr{G}$ must satisfy $\Gamma$. $\Gamma$ is satisfiable if all of its finite subsets are.

An alternative proof of compactness can be found in Appendix \ref{altcomp}.

\section{Decidability}
\paragraph{Decidability}

An \emph{effective procedure} for determining some query is an automatic and mechanical algorithm that terminates in a \emph{finite} time with a correct `yes' or `no' answer. 

A property is \emph{decidable} iff there is an effective procedure which tests for it.

\begin{theorem}[Decidability of Finite Validity]
	If $\Gamma$ is a finite set of sentences of \lone, then it is decidable whether $\Gamma \vDash$.
\end{theorem}

This has the immediate consequence that it is decidable whether $\Gamma\vDash\phi$ (check whether $\Gamma,\neg\phi\vDash$), and thus that \emph{validity is decidable}.

(The notion of an effective procedure is introduced informally, but it can be made precise by investigating \emph{Turing machines} and  \emph{recursive functions} – see \citet[chs.\ 1--8]{bbjcomlo}.)

\paragraph{Finite Decidability Using Truth Tables}

Let $\Gamma$ be a finite set of \lone\ sentences. Each $\gamma\in\Gamma$ is only finitely long, so there are $n$ sentence letters occurring in $\Gamma$, for some finite $n$.

Construct a truth-table with $2^{n}$ lines such that each line corresponds to some structure assigning truth values to each sentence letter in $\Gamma$; this can be done mechanically in a finite time. For each $\gamma \in \Gamma$, determine its truth value in each row; this can be done mechanically in a finite time for each $\gamma$ in each row, and there are finitely many rows, so the truth value of each $\gamma$ in every structure can be determined mechanically in a finite time. Since $\Gamma$ is finite, we can determine the truth value of each sentence in $\Gamma$ in every structure in a finite time. 

Check each row of the truth table to see if all sentences in $\Gamma$ have been assigned $T$: If they all have, in some row, then stop, and answer `No' to the query, `Is $\Gamma\vDash$?'. If no row has all $\gamma\in\Gamma$ assigned $T$, then answer `Yes'. There are only finitely many rows, so this final check can be done in a finite time. All steps can be finitely mechanised, and there are finitely many steps: inconsistency is decidable.

\paragraph{Positive Infinite Decidability}

A query is \emph{positively decidable} iff an effective procedure will terminate and correctly answer `Yes' in a finite time. A query is \emph{negatively decidable} iff an effective procedure will terminate and correctly answer `No' in a finite time. It is decidable iff it is both positively and negatively decidable.

\begin{theorem}[Positive Decidability of Inconsistency]
If $\Gamma$ is an infinite set of sentences of \lone, then if $\Gamma\vDash$, there is an effective procedure that will demonstrate that; but there is no effective procedure that will
decide if arbitrary $\Gamma$ is consistent.\end{theorem}

We first show a lemma.
\paragraph{An Effective Procedure Generates Finite Subsets of $\mathbb{N}$}

\begin{lemma}
	For every finite subset of the natural numbers $\mathbb{N}$, there is an effective procedure which produces all and only members of that set. \begin{proof}
		{ \textbf{Proof sketch}: Begin by
		announcing $1$. Then proceed inductively: if $n$ is the highest
		number announced so far, list every set which contains only
		numbers $\leqslant n$.
		Since at any point we will only have announced finitely many
		numbers, there are only finitely many such sets, so this is
		finitely achievable. Once this is complete, proceed to announce
		$n+1$. It's clear that any
		finite set of natural numbers will be generated at some finite
		time from the beginning of this series – since each such set has a 
		largest member $n$, it will be listed before $n+1$ is announced.}
	\end{proof}
\end{lemma}

\paragraph{Proof of Positive Infinite Decidability}

\begin{proof}
 	 The set of sentences of \lone\ is enumerable, so
 $\Gamma = \{\gamma_{1},\ldots,\gamma_{n},\ldots\}$ where each $\gamma_{i}$ is
assigned some natural number as an index. If $\Gamma \vDash$, then  
by the Compactness theorem,
there is a finite subset $\Delta \subseteq \Gamma$ such that $\Delta 
\vDash$. 

By the Lemma, we can effectively generate each finite subset
 $\Delta$  by
generating the indices $i$ on $\gamma_{i}\in\Delta$, and since finite consistency is decidable (by Finite Decidability), if $\Delta$ is
inconsistent that will be decided at some finite point. So if
$\Gamma$ is inconsistent, this test will indicate that some finite
subset is inconsistent after some finite time.

Yet if $\Gamma$ is consistent, no finite subset is inconsistent,
and this procedure will continue indefinitely checking consistent 
sets and showing that they are consistent. It never halts after a finite time,
as there remain infinitely many other consistent finite subsets of
$\Gamma$ not yet checked. So inconsistency is only positively decidable, and infinite consistency is undecidable.  \end{proof}


{\small

\subsection*{Exercises}
\addcontentsline{toc}{subsection}{Exercises}

\begin{enumerate}
\item Give, explicitly, the inductive definition of arbitrary disjunction.
\item \begin{enumerate}
	
	\item Show that if there are $n$ sentence letters in $S$, there are $2^{n}$ sentences of the form $\mathfrak{c}_{\mathscr{A}}$ defined in the proof of the \textsc{\lowercase{DNF}} theorem.
\item Explain clearly why, in the proof of the \textsc{\lowercase{DNF}} theorem, the formula $\mathfrak{d}$ there constructed expresses the truth function $f$.
\end{enumerate}
\item  Prove, by an argument analogous to the \textsc{\lowercase{DNF}} theorem, this claim: \begin{theorem}[\textsc{\lowercase{CNF}}] 	Every truth function is expressed by a formula in Conjunctive Normal Form.\end{theorem}
\item Show that  all truth-functions which can be expressed using only $\to$ and $\wedge$ are positive. (Prove by induction on complexity of formulae.)

\item The proof of theorem \ref{positive} was only sketched. Fill in these two crucial gaps.\begin{enumerate} 
	\item Show that $\phi\vee \psi$ is logically equivalent to $(\psi \to \phi)\to \phi$;
	\item Show (by induction on complexity of sentences, and using the above result) that each conjunct of a \textsc{\lowercase{CNF}} sentence will be equivalent to either (i) an arrow sentence \emph{without} negation, or (ii)  a conjunct  of the form $\neg \phi \vee \neg \psi$.
		\item	 Show that a necessary and sufficient condition for a sentence $\phi$
	 to be logically equivalent to a sentence involving just the truth functional connectives $\to$ and 
	$\wedge$ is that the sentence has the value $T$ in any structure that assigns the
		    value $T$ to each sentence letter in $\phi$. \end{enumerate}

\item \begin{enumerate}

	\item Show that $\{\vee,\neg\}$ is expressively adequate.
	\item Is $\{\to,\neg\}$ expressively adequate?
	\item Is $\{\bicond,\wedge,\to,\vee\}$ expressively adequate?
	\item Is any connective in \lone\ expressively adequate by itself?
	\item Prove that $\downarrow$ is expressively adequate.
	\item How many 2-place truth-functional connectives are expressively adequate by themselves?
	\item Consider the 0-place connective $\bot$ that expresses the constant 0-place function $f: \emptyset \mapsto F$. Is this expressively adequate, or can you add a connective to get an expressively adequate set (where the added connective is not itself expressively adequate)? 
\end{enumerate}

\item \begin{enumerate}
	\item Consider the self-dual set of connectives $\{\to,\to^{\star}\}$. Is this set of connectives expressively adequate? Is self-duality either necessary or sufficient for expressive adequacy of a set of connectives?  
\item Show that if a two-place truth functional connective $\oplus$ is self-dual,
	then the function that expresses it, $f_{\oplus}$, must be such that
	$f_{\oplus}(T, F) \neq  f_{\oplus}(F,T)$ and
	$f_{\oplus}(F,F) \neq f_{\oplus}(T,T)$. Make use of this
	result, establish how many self-dual two-place truth functors there
	are.
	\item Let $\phi$ and $\psi$ be sentences of $\mathcal{L}_{\neg,\wedge,\vee}$. We say that a sentence $\phi$ is \emph{dualable} iff for all $\mathscr{A}$, $(\val{\phi}{A^{\star}})^{\star}=\val{\phi^{\star}}{A}$. \begin{enumerate}
		\item Prove that if $\phi$ and $\psi$ are dualable, so too are $\neg\phi$, $(\phi\wedge\psi)$, and $(\phi\vee\psi)$.
		\item Prove using this result that all sentences of $\mathcal{L}_{\neg,\wedge,\vee}$ are dualable.
	\end{enumerate}
	\item If $\phi$ is an \lone\ sentence, let $\phi^{\ddag}$ be the sentence that results from uniformly substituting in $\phi$ any subsentence of the form 
	\begin{itemize} \item $(\psi \wedge \chi)$ by $(\psi \vee \chi)$;
	\item $(\psi \vee \chi)$ by $(\psi \wedge \chi)$;
	\item $(\psi \to \chi)$ by $\neg(\chi\to\psi)$; and
	\item $(\psi\bicond\chi)$ by $\neg(\psi\bicond\chi)$.
\end{itemize} \begin{enumerate}
	\item Show that $\vDash \phi \bicond \psi$ iff $\vDash \phi^{\ddag} \bicond \psi^{\ddag}$.
	\item Letting the dual operator $\star$ apply to all \lone\ connectives in the natural way, show that $\phi^{\ddag} \Dashv\vDash \phi^{\star}$. 
\end{enumerate}
\end{enumerate}

	\item Let $\mathcal{L}_{\uparrow,\downarrow}$ be the language which has $\uparrow$ and $\downarrow$ as its only connectives. \begin{enumerate}
		\item  For any $\phi$ in $\mathcal{L}_{\uparrow,\downarrow}$, and any structure $\mathscr{A}$, show that $\val{\phi}{A} = (\val{\phi^{\star}}{A^{\star}})^{\star}$.	
	 \item For any $\mathcal{L}_{\uparrow,\downarrow}$-sentence $\phi$, let $\phi^{\star}$ be the sentence which results from replacing every connective in $\phi$ by its dual, and let $\overline{\phi}$ be the result of substituting $(P_{i}\uparrow P_{i})$ for every sentence letter $P_{i}$ occurring in $\phi$.	
	\begin{enumerate}
		\item Prove that for every $\mathcal{L}_{\uparrow,\downarrow}$ sentence $\phi$, $\phi^{\star}$ is logically equivalent to $(\overline{\phi}\uparrow\overline{\phi})$.
		\item Prove that if $\phi$ and $\psi$ are $\mathcal{L}_{\uparrow,\downarrow}$ sentences, $\phi \vDash \psi$ iff $\psi^{\star} \vDash \phi^{\star}$.
	\end{enumerate}\end{enumerate}


	\item Find an interpolant for the following sequents. Be sure in each case to give the simplest interpolant (i.e., find  the most elegant sentence that is equivalent to your chosen interpolant).\begin{enumerate}
		\item $((Q\vee P)\to R)\vDash ((P_{1}\wedge \neg R)\to\neg Q)$;
			\item $(\neg(P \vee Q)
		\wedge (P \bicond R)) \vDash ((R \to P) \wedge \neg (P_{1} \wedge R))$;
		 \item $((Q_{2} \bicond Q) \wedge
		    \neg ((R \to \neg P_{1}) \vee \neg (P \to Q))) \vDash
		    (R_{1} \to (P_{2} \to (\neg
		    P \vee Q)))$;
		    % \item $(P \to (Q \to (R \vee (P_{1} \wedge Q_{1}))), R_{1} \vee P_{2} \vDash P_{1} \to (R \wedge \neg P)$.
	\end{enumerate}
	




\item \begin{enumerate}

	\item Let $\Gamma$ be a possibly infinite set of sentences of \lone\ such that $\Gamma \vDash$. Show that there is a finite disjunction, $\delta$, each disjunct of which is the negation of a sentence in $\Gamma$, and such that $\vDash \delta$.
	\item Consider the following relation holding between sets of sentences: \begin{quote} Where $\Gamma$ and $\Delta$ are any sets of sentences,
		$\Gamma \vDash^{\!\star} \Delta$ is correct iff every structure which satisfies \emph{every} $\gamma \in \Gamma$ is also one which satisfies \emph{at least one} $\delta \in \Delta$.
	\end{quote}Show that if $\Gamma \vDash^{\!\star} \Delta$, there is a finite conjunction of \lone\ sentences in $\Gamma$, $$\Phi = (\phi_{1}\wedge \ldots \wedge \phi_{n}),$$ and a finite disjunction of \lone\ sentences in $\Delta$, $$\Psi = (\psi_{1} \vee \ldots\vee \psi_{n}),$$ such that $\Phi \vDash \Psi$. (You may assume the Compactness theorem.) 
	\item A set of sentences of English is \emph{compossible} just in case it is possible for them all to be true together. An analogue for the compactness theorem in English would be: for every infinite set $E$ of English sentences not all compossible, there is a finite subset of $E$ whose members are not all compossible. \begin{enumerate}
		\item Show that this analogue of compactness fails for English.
		\item What does this show about translations from English into \lone?
	\end{enumerate}
\end{enumerate}
\item Intuitively, any effective procedure must be able to be written down by a finite string of sentences in some language – say, English. \begin{enumerate}
	\item Give an argument that the set of effective procedures is countable.
	\item Let a \emph{recipe} be any finite set of English sentences.
		Consider the set $E$ of recipes. (It is obvious that the set of effective procedures in English which compute one-place functions whose domain is (a subset of) the natural numbers will be a subset of $E$.)  This set is countable; let $f_{n}$ be the function – if there is one – computed by the $n$-th recipe in $E$ under some enumeration of $E$. Define $$d(n)=\begin{cases}2 &\text{if $f_{n}(n)$ is defined and equal to $1$};\\
	1 & \text{otherwise}	
	\end{cases}$$ \begin{enumerate}
		\item Show that there is no effective procedure for computing $d$.  
		\item Show that there is no effective procedure for deciding if a recipe is an effective procedure, and therefore that while the set of recipes $E$ can be effectively produced, some subsets of $E$ (in particular, the one corresponding to the set of effective procedures) cannot be effectively produced.
			\end{enumerate} 
\end{enumerate}

\end{enumerate}

}








\newpage

\chapter[Natural Deduction in \lone]{Natural Deduction Proofs in $\mathcal{L}_{1}$; Soundness and Completeness}\label{c5}
  

\section{Natural Deduction Proofs}
\paragraph{Proof}

In the last chapter, we saw that using truth tables was a decidable method for evaluating sequents, but it is cumbersome and inelegant. Here, we consider elegant and powerful \emph{proof systems} for sentential logic.

A \emph{proof} in the informal sense is something which establishes the truth of some claim. Our proofs are structures made out of sentences of \lone, with the following desirable properties: \begin{itemize}
	\item Each step in the proof involves an obviously correct rule.
	\item Whenever the earlier sentences are true, the later sentences will be also; so the terminal conclusion of a proof should always be true whenever the assumptions are.
	\item All the logical truths should have a proof.
\end{itemize}

In this chapter, we'll see that there do exist proofs that meet these conditions for \lone; and thus that we can establish various claims in \lone\ without needing to consider all the \lone-structures.

\paragraph{Natural Deduction: Assumptions and $\vdash$}

The first proof system is the \emph{natural deduction} system introduced in The Logic Manual. 

In this system, every proof begins with \emph{assumptions}. \emph{Any sentence may be an assumption.} A proof terminates with a \emph{conclusion}. 

 Whenever there is a correctly constructed proof that $\phi$, on the basis of assumptions $\gamma_{1},\ldots,\gamma_{n}$, we write this \emph{syntactic sequent}: {$$\gamma_{1},\ldots,\gamma_{n}\vdash\phi.$$} This is to be read `$\phi$ is \emph{provable from} $\gamma_{1},\ldots,\gamma_{n}$', or `$\gamma_{1},\ldots,\gamma_{n}$ \emph{syntactically entail(s)} $\phi$'. If $\phi$ can be proved with no assumptions at all, we write $\vdash\phi$; in that case, we say that $\phi$ is a \emph{theorem}.

Any sentence $\phi$ may be assumed; applying no rules at all, the proof therefore terminates with $\phi$. So $\phi\vdash\phi$; not perhaps the most interesting result, but nevertheless intuitively correct.




\paragraph{Natural Deduction Rules}

 As any assumption is a proof on its own, our rules are rules for constructing new proofs from existing proofs, often (but not always) retaining the assumptions of the existing proofs.

Some rules permit an assumption to be \emph{discharged}. Suppose you have a proof of $\phi$ on the basis of assumption $\gamma$.  You thus show that \emph{if} you had a proof of $\gamma$, \emph{then} you could construct a proof of $\phi$; and this on its own is a proof of $\gamma\to\phi$, \emph{whether or not $\gamma$ is actually true}. The assumption $\gamma$ has been discharged.
 Note that $\gamma$ needn't be used, or even occur in the proof, to be discharged: the rules entitle one to discharge \emph{every} occurence of a sentence – even if there are none!

 We can consider two types of obvious rules: those which construct a proof of a new sentence of less complexity than sentence proved by the existing proof(s), and those which construct a proof of a sentence of increased complexity. We call the first type of rule \emph{elimination} rules, and the second type \emph{introduction} rules.




{
\paragraph{Natural Deduction Rules for $\mathcal{L}_{\neg,\to}$}
Let us consider for the moment the fragment of \lone\ involving only $\neg$ and $\to$, $\mathcal{L}_{\neg,\to}$. We have in this fragment four rules, laid out in Table \ref{tfour}, elimination rules and introduction rules for each connective. I write $[\gamma]$ to show that the rule permits the assumption $\gamma$ to be discharged.
 \begin{table}\label{tfour}
 	\centering
	\begin{tabular}{cc}
	
	\begin{prooftree}
	\[ [\gamma] \leadsto \phi
	\]
 \justifies \gamma\to\phi \using{{\to}\text{Intro}}
\end{prooftree} & \begin{prooftree}
	\gamma \to \phi \qquad \gamma \justifies \phi \using{{\to}\text{Elim}}
\end{prooftree}\\[30pt]
\begin{prooftree}
	\[ [\phi] \leadsto \psi\]
	\[ [\phi] \leadsto \neg\psi\] \justifies \neg\phi \using{\neg\text{Intro}} 
\end{prooftree} &
\begin{prooftree}
	\[ [\neg\phi] \leadsto \psi\]
	\[ [\neg\phi] \leadsto \neg\psi\] \justifies \phi \using{\neg\text{Elim}} 
\end{prooftree}
	\end{tabular} \caption{Natural Deduction Rules for $\mathcal{L}_{\neg,\to}$}
 \end{table}
Call this system of rules, plus the rules about assumptions, $ND_{\neg,\to}$.

\paragraph{Deduction Theorem for $ND_{\neg,\to}$}

\begin{theorem}[Deduction Theorem] $\Gamma \vdash \psi \to \phi$ iff $\Gamma,\psi \vdash \phi$.\end{theorem}
Suppose $\Gamma,\psi \vdash \phi$. Then we can apply $\to$Intro as follows: 
	\begin{equation*}
		\begin{prooftree}
\[\Gamma, [\psi] \leadsto \phi\] 
			\justifies \psi \to \phi \using{\to}\text{Intro}
				\end{prooftree}
	\end{equation*}
	
Therefore, $\Gamma\vdash\psi\to\phi$. Similarly, suppose $\Gamma\vdash\psi\to\phi$: \begin{equation*}
	\begin{prooftree}
		\[\Gamma \leadsto \psi\to\phi\] 
		\qquad \psi \justifies \phi \using {\to}\text{Elim}
	\end{prooftree}
\end{equation*} That is, $\Gamma,\psi\vdash\phi$. 


\paragraph{$\neg$Elim can be replaced by $\neg\neg$Elim}

Consider the rule: \begin{equation*}
	 \begin{prooftree}\neg\neg\phi \justifies \phi \using \neg\neg\text{Elim}
	\end{prooftree}
\end{equation*}
This rule can obviously be derived from our rules: the correctness of the syntactic sequent $\neg\neg\phi \vdash \phi$ is demonstrated by this proof: \begin{equation*}
	\begin{prooftree}
		\neg\neg\phi \quad [\neg\phi] \justifies \phi \using \neg\text{Elim} 
	\end{prooftree}
\end{equation*}   But with $\neg\neg$Elim we can show the rule $\neg$Elim is redundant, using $\neg$Intro: \begin{equation*}
	\begin{prooftree}
\[		\[[\neg \phi] \leadsto \psi\]\quad \[[\neg\phi] \leadsto \neg\psi\] \justifies \neg\neg\phi \using \neg\text{Intro} \] \justifies \phi\using\neg\neg\text{Elim}
	\end{prooftree}
\end{equation*} 


\section{Soundness}
\paragraph{Soundness and Completeness}

A proof system $P$ in a given language is \emph{sound} with respect to a semantic interpretation of the language just in case whenever there is a proof which establishes $\Gamma \vdash_{P} \phi$, it is the case that $\Gamma \vDash \phi$.

A proof system $P$ in a given language is \emph{complete} with respect to a semantic interpretation of the language just in case whenever it is the case that $\Gamma \vDash \phi$, there is a proof which establishes $\Gamma \vdash_{P} \phi$.

In this section and the next, we will show that the natural deduction system $ND_{\neg,\to}$ just introduced is \emph{sound and complete} for the standard semantics of $\mathcal{L}_{\neg,\to}$. Given the expressive adequacy of $\mathcal{L}_{\neg,\to}$, that also shows the completeness of the natural deduction system with respect to \lone; to show the soundness of that system, we also need to show that the other rules of the system $ND$ are also sound.


\paragraph{Soundness of $ND_{\neg,\to}$}

\begin{theorem} The Rules of $ND_{\neg,\to}$ are sound.
\begin{proof} Suppose that $\Gamma \vdash \phi$. We show by induction on the complexity of proofs that $\Gamma \vDash \phi$.

\emph{Base case}: The least complex proof is a single node, $\phi$, establishing $\phi \vdash \phi$. In this case, any structure which assigns $T$ to $\phi$ clearly assigns $T$ to $\phi$, so $\phi\vDash\phi$, i.e., $\Gamma\vDash\phi$.

\emph{Induction step}: Suppose $\Gamma\vdash\phi$, established by applying one of the natural deduction rules to some previously obtained proofs to obtain a proof of $\phi$. There are four cases: 
 \begin{description}
	\item [$\to$Elim] From a proof of $\psi$ on assumptions $\Delta$, and a proof of $\psi\to\phi$ on assumptions $\Theta$, obtain a proof of $\phi$ on assumption $\Gamma=\Delta\cup\Theta$. By the induction hypothesis, $\Delta \vDash \psi$ and $\Theta \vDash \psi\to\phi$, so every structure which makes all of the members of $\Gamma$ true, makes $\psi$ and $\psi\to\phi$ true, and therefore $\phi$ must be true in all such structures, showing that $\Gamma\vDash\phi$.
	\item [$\to$Intro] From a proof of $\phi$ on assumptions $\Gamma \cup \{\psi\}$, apply $\to$Intro to obtain a proof of $\psi\to\phi$ on assumption $\Gamma$ (discharging $\psi$). But since by hypothesis $\Gamma,\psi \vDash\phi$, by the deduction theorem, $\Gamma \vDash \psi\to\phi$.
	\item [$\neg$Elim] From a proof of $\psi$ on assumptions $\Gamma,\neg\phi$ and a proof of $\neg\psi$ on assumptions $\Gamma,\neg\phi$, obtain a proof of $\phi$ on assumptions $\Gamma$, discharging $\neg\phi$. Since $\Gamma,\neg\phi \vDash \psi$ and $\Gamma,\neg\phi\vDash\neg\psi$, it follows that $\Gamma,\neg\phi\vDash$; therefore $\Gamma\vDash\phi$.
		\item [$\neg$Intro] From a proof of $\psi$ on assumptions $\Gamma,\phi$ and a proof of $\neg\psi$ on assumptions $\Gamma,\phi$, obtain a proof of $\neg\phi$ on assumptions $\Gamma$, discharging $\phi$. Since $\Gamma,\phi \vDash \psi$ and $\Gamma,\phi\vDash\neg\psi$, it follows that $\Gamma,\phi\vDash$; therefore $\Gamma\vDash\neg\phi$.
\end{description}
Thus all the rules preserve soundness; no proof in $ND_{\neg,\to}$ can be constructed other than using these rules; so $ND_{\neg,\to}$ is sound. \end{proof}\end{theorem}




\section{Completeness}
\paragraph{Consistency}

A set of sentences $\Gamma$ is called \emph{(syntactically) consistent} iff $\Gamma\not\vdash\neg(\phi\to\phi)$. Otherwise it is \emph{inconsistent}, also written $\Gamma\vdash$. Equivalently, consider this proof:
\begin{equation*}
	\begin{prooftree}
		\[ \Gamma \leadsto \neg(\phi \to \phi)\] 
		\[[\phi] \justifies \phi\to\phi \using \to\text{Intro}\] 
		\justifies \psi \using \neg\text{Elim}
	\end{prooftree}
\end{equation*} If we have a proof of $\neg(\phi\to\phi)$, we have a proof of arbitrary $\psi$ from the assumption $\Gamma$; so we can say that $\Gamma$ is consistent iff there is a sentence \emph{not provable} from $\Gamma$. 

A set of sentences $\Gamma$ is \emph{maximally consistent} iff it is consistent and for any $\phi$, if $\phi\notin\Gamma$, then $\Gamma,\phi\vdash$ (one cannot add any more sentences while remaining consistent).
\paragraph{Properties of Maximal Consistent Sets: Deductive Closure}

\begin{lemma}[Deductive Closure]
	If $\Gamma$ is a maximal consistent set, then if $\Gamma \vdash \phi$, then $\phi \in \Gamma$. \begin{proof}
	{	 Suppose that $\Gamma\vdash\phi$, but $\phi\notin\Gamma$. Since $\Gamma$ is maximal consistent, $\Gamma,\phi\vdash \neg(\phi\to\phi)$. Using $\to$Intro and discharging $\phi$, we see that $\Gamma\vdash\phi\to(\neg(\phi\to\phi))$. But now use $\to$Elim, and obtain $\Gamma\vdash\neg(\phi\to\phi)$ i.e., $\Gamma$ is inconsistent.}
	\end{proof}
\end{lemma} 
}

{
 \paragraph{Properties of Maximal Consistent Sets: Negation Completeness}

\begin{lemma}[Negation Completeness]
	If $\Gamma$ is a maximal consistent set, then $\neg\phi \in \Gamma$ iff $\phi \notin\Gamma$. \begin{proof}
	{ L to R:	If both $\phi$ and $\neg\phi$ were in $\Gamma$, we could prove $\neg(\phi\to\phi)$ by an application of $\neg$Elim, showing $\Gamma$ inconsistent. 
		
		R to L: If $\phi\notin\Gamma$, then  $\Gamma,\phi\vdash\neg(\phi\to\phi)$: 
		 \begin{equation*}
			\begin{prooftree}
				\[\Gamma, [\phi] \leadsto \neg(\phi\to\phi)\]
				\[[\phi] \justifies \phi\to\phi\using \to\text{Intro}\]
				\justifies \neg\phi \using\neg\text{Intro}
			\end{prooftree}
		\end{equation*}  Therefore, $\Gamma\vdash\neg\phi$; by Deductive Closure Lemma, $\neg\phi\in\Gamma$.}
	\end{proof} 
\end{lemma}
 

\paragraph{Properties of Maximal Consistent Sets: Conditional Completeness}

\begin{lemma}[Conditional Completeness]
	If\, $\Gamma$ is any maximal consistent set, then	$\phi\to\psi\in\Gamma$ iff whenever $\phi\in\Gamma$, it follows that $\psi\in\Gamma$.
	\begin{proof}
{	L to R:	Suppose $\phi\to\psi\in\Gamma$. Then $\Gamma\vdash\phi\to\psi$; and if $\phi\in\Gamma$, then $\Gamma\vdash\phi$. Apply $\to$Elim, and establish $\Gamma\vdash\psi$, i.e., $\psi\in\Gamma$.
		
		R to L: Suppose that whenever $\phi\in\Gamma$, $\psi\in\Gamma$. Consider two cases: \begin{enumerate}
			\item $\phi\in\Gamma$. So $\Gamma=\Gamma\cup\{\phi\}$. Moreover, $\psi\in\Gamma$, so $\Gamma,\phi\vdash\psi$. Apply $\to$Intro and discharge $\phi$ to show $\Gamma\vdash \phi\to\psi$.
			\item $\phi\notin\Gamma$. Then $\Gamma,\phi\vdash\neg(\phi\to\phi)$. But then $\Gamma,\phi \vdash\psi$ (by a similar proof as just above); applying $\to$Intro and discharge $\phi$, $\Gamma\vdash\phi\to\psi$.
		\end{enumerate}
}		
	\end{proof} 
\end{lemma}

\paragraph{Properties of Maximal Consistent Sets: Satisfiability}

\begin{theorem}[Maximal Consistent Sets Satisfiable]
	Every maximal consistent set is satisfiable (i.e., has a model).\begin{proof}
	{	Suppose $\Gamma$ is a maximal consistent set in $\mathcal{L}_{\to,\neg}$. For $s$ a sentence letter, define a structure $\mathscr{A}$: {$\mathscr{A}(s) = T$ iff $s\in\Gamma$}.
			
			For any $\phi$, $\val{\phi}{A}=T$ iff $\phi\in\Gamma$. Proof by induction on complexity of sentences. The base case, $\phi$ a sentence letter, is immediate. 
		
	Suppose $\phi$ is complex. There are two cases: \begin{enumerate}
		\item $\phi=\neg\psi$. $\phi\in\Gamma$ iff $\psi\notin\Gamma$, by Negation Completeness; iff by the induction hypothesis, $\val{\psi}{A}=F$; iff by the clause on $\neg$, $\val{\phi}{A}=T$.
		\item $\phi=(\psi\to\chi)$. $\phi\in\Gamma$ iff if $\psi\in\Gamma$ then $\chi\in\Gamma$ (by Conditional Completeness); iff if $\val{\psi}{A}=T$ then $\val{\chi}{A}=T$; iff $\val{\phi}{A}=T$.
	\end{enumerate} }\end{proof}
\end{theorem} 

\paragraph{Completeness}

\begin{theorem}[Completeness for $\mathcal{L}_{\neg,\to}$]
	If $\Gamma\vDash\phi$, then $\Gamma\vdash\phi$. \begin{proof}
		{ We prove the \emph{contrapositive}. So assume that $\Gamma\not\vdash\phi$. (Assuming that $\Gamma$ itself is consistent; if it is not, then clearly the theorem holds fairly trivially.)
		
		We'll construct a maximal consistent set $\Gamma^{+}$, a \emph{superset} of $\Gamma$, such that $\Gamma^{+}\not\vdash\phi$.  By consistency of $\Gamma^{+}$, $\phi\notin\Gamma^{+}$; by the Negation Completeness lemma, $\neg\phi\in\Gamma^{+}$.
		
		But by the Satisfiability Theorem, $\Gamma^{+}$ has a model, $\mathscr{A}$. Since $\Gamma\subseteq\Gamma^{+}$, for all $\gamma\in\Gamma$, $\val{\gamma}{A}=T$; but since $\neg\phi\in\Gamma^{+}$, $\val{\neg\phi}{A}=T$, so $\val{\phi}{A}=F$. So $\Gamma\not\vDash\phi$, as required.
		
		We now show how to construct $\Gamma^{+}$ from $\Gamma$, completing the proof.
		}
	\end{proof}
\end{theorem}

\paragraph{The Construction of Maximal Consistent $\Gamma^{+}$}

Supposing $\Gamma\not\vdash\phi$, construct $\Gamma^{+}$ as follows. \begin{enumerate}
	\item Enumerate the sentences of $\mathcal{L}_{\neg,\to}$, $\{\sigma_{1},\sigma_{2},\ldots\}$. 
	\item Let $\Gamma_{0}=\Gamma$.
	\item \begin{equation*}
		\Gamma_{n+1} = \begin{cases}
			\Gamma_{n}\cup\{\neg\sigma_{n+1}\} &\text{if } \Gamma_{n},\sigma_{n+1}\vdash\phi;\\
			\Gamma_{n}\cup\{\sigma_{n+1}\} &\text{otherwise (i.e., if $\Gamma_{n},\sigma_{n+1}\not\vdash\phi$)}.
		\end{cases}
	\end{equation*}
	\item Let $\Gamma^{+}=\bigcup_{i} \Gamma_{i}$.
\end{enumerate}
It is clear that \emph{$\Gamma\subseteq\Gamma^{+}$}. It is also clear that \emph{$\Gamma^{+}\not\vdash\phi$}, for at no stage was a sentence added to any $\Gamma_{n}$ that permitted a proof of $\phi$. 

Finally, $\Gamma^{+}$ is a \emph{maximal consistent set}, since (by construction), for every $\sigma_{i}$, either $\sigma_{i}\in\Gamma^{+}$ or $\neg\sigma_{i}\in\Gamma^{+}$.  Thus $\Gamma^{+}$ has the properties required to prove Completeness stated just above.

}



{
\paragraph{Completeness for \lone}

We've shown the Completeness Theorem for the restricted language $\mathcal{L}_{\neg,\to}$. But by Expressive Adequacy, any sentence of \lone\ can be expressed in $\mathcal{L}_{\neg,\to}$. So we can assure ourselves that any truth function expressible in \lone\ can be expressed in $\mathcal{L}_{\neg,\to}$, and any valid argument in \lone\ has a corresponding provably correct sequent in $\mathcal{L}_{\neg,\to}$.

For completeness of \lone, we cannot rest there. What if the rules of \lone\ don't suffice to show the equivalence of some arbitrary \lone\ sentence $\phi$ with a sentence only involving $\neg$ and $\to$? In that case, while a sentence logically equivalent to $\phi$ is provable, $\phi$ may not be.

We thus need additional lemmas showing that maximal consistent sets \begin{itemize}
	\item Contain $\phi \wedge \psi$ iff they contain both $\phi$ and $\psi$;
	\item Contain $\phi \vee \psi$ iff they contain at least one of $\phi$ or $\psi$;
	\item Contain $\phi \bicond \psi$ iff they contain $\phi$ iff they contain $\psi$.
\end{itemize} 
 Then the Satisfiability Theorem must be extended to include these new cases; and the remainder of the theorem is proved as above.






\paragraph{Compactness  Again}

Consideration of the nature of natural deduction proofs should convince you that any conclusion $\phi$ drawn on the basis of assumptions $\Gamma$ can be constructed by some finite application of the rules to finitely many sentences in $\Gamma$. Provability is essentially a finite notion: proofs are things that can be \emph{constructed}.

This observation, informal though it is, \emph{can} be made precise, to yield another proof of the \emph{Compactness theorem}. Since, by Completeness, every correct semantic sequent $\Gamma\vDash$ is provable, and every proof is a finite object using only finitely many members of $\Gamma$, then by Soundness there is a correct semantic sequent $\Gamma'\vDash$ where $\Gamma'$ is a finite subset of $\Gamma$. 




\section{A Little More Proof Theory}

`Proof theory' is the mathematical examination of proofs conceived of as mathematical objects in their own right. In this section we explore a little of what this involves. We've already seen one some proof theory in section 1 of this chapter, where we showed the redundancy of the proof rule $\neg$Elim in the presence of $\neg\neg$Elim, and vice versa.

\paragraph{The Language $\mathcal{L}_{DP}$}

Consider the language $\mathcal{L}_{DP}$ which contains all sentence letters, and all sentences involving only the connectives $\wedge,\to$, and also contains a \emph{sentential constant} $\bot$. A sentential constant is syntactically like a sentence letter, but (unlike sentence letters) it gets a constant interpretation in every structure – in this case, for any structure $\mathscr{A}$, $|\bot|_{\mathscr{A}}=F$. The atomic sentences of $\mathcal{L}_{DP}$ thus include $\bot$ and all sentence letters; complex sentences are then built up in the usual way. 

\paragraph{Translating between $\mathcal{L}_{DP}$ and \lone} It's clear that the \lone\ sentence `$\neg \phi$' expresses the same truth function as the $\mathcal{L}_{DP}$ sentence `$\phi \to \bot$' (consideration of a truth table will suffice to show this). 
Using this translation, we can give $\mathcal{L}_{DP}$-versions of the natural deduction rules for $\neg$ in \lone, as seen in Table \ref{negldp}.
\begin{table}
	\centering
	\begin{tabular}{cc}
		\begin{prooftree}
			\[[\phi] \leadsto \psi\] \[[\phi] \leadsto \psi \to \bot\] \justifies \phi \to \bot \using{\neg\text{Intro}_{DP}}
		\end{prooftree}	&	\begin{prooftree}
					\[[\phi\to\bot] \leadsto \psi\] \[[\phi\to\bot] \leadsto \psi \to \bot\] \justifies \phi \using{\neg\text{Elim}_{DP}}
				\end{prooftree}		
	\end{tabular}\caption{Negation rules in $\mathcal{L}_{DP}$\label{negldp}}
\end{table}

With the negation rules as specified, we can prove this: \begin{theorem}[Alternative Negation Rules for $\mathcal{L}_{DP}$] \label{altneg}
 $\Pi$ is a proof of some sequent in the language $\mathcal{L}_{DP}$ that makes a use $u$ of one of the Table \ref{negldp} versions of Halbach's rules iff there is another proof $\Pi'$ of the same sequent which replaces that use $u$ by a construction which only uses this rule for $\neg$ (together with the  rules for  $\to$):
	\begin{equation*}
		\begin{prooftree}
	\[	[\phi \to \bot] \leadsto \bot\] \justifies \quad\phi\quad \using{\bot_{\mathsf{C}}}
		\end{prooftree}
	\end{equation*} \begin{proof}
	\emph{If:} There are two cases. Case 1: Suppose $\Pi$ is a proof that makes use of $\neg$Intro$_{DP}$. Then we can replace that use by the following construction, which yields another proof $\Pi'$ which uses only the rules for $\to$:
	\begin{equation*}
			\begin{prooftree}
					\[\[[\phi] \leadsto \psi\] \[[\phi] \leadsto \psi \to \bot\] \justifies\bot \using{\to\text{Elim}}\]\justifies\phi\to\bot\using{\to\text{Intro}} 
			\end{prooftree}
	\end{equation*}
	Case 2: Suppose $\Pi$ is a proof that makes use of $\neg$Elim$_{DP}$. Then we can replace that use by the following construction, which yields another proof $\Pi'$ which uses only the new rule $\bot_{\mathsf{C}}$ and the rule $\to$Elim:
\begin{equation*}
		\begin{prooftree}
				\[	\[[\phi\to\bot] \leadsto \psi\] \[[\phi\to\bot] \leadsto \psi \to \bot\] \justifies \bot \using{\to\text{Elim}}							
				\] \justifies \phi \using{\bot_{\mathsf{C}}}
				\end{prooftree}
\end{equation*}

\emph{Only if}: Suppose $\Pi$ is a proof which makes use of the new rule $\bot_{\mathsf{C}}$. Then we may replace that use by the following construction which uses only the old rules for $\to$ and $\neg$Elim$_{DP}$.
\begin{equation*}
	 \begin{prooftree}
		\[[\phi \to \bot] \leadsto \bot\] \[[\bot]\justifies\bot\to\bot \using{\to\text{Intro}}\]\justifies\phi\using{\neg\text{Elim}_{DP}} 
	\end{prooftree}
\end{equation*}								
	\end{proof}
\end{theorem} 


\paragraph{Reducing Complexity of Proofs}

Return now to the \lone-fragment containing only $\to,\wedge,\neg$. Let the \emph{(degree of) complication} of a sentence be the number of connectives occurring in the sentence (a sentence letter has degree of complication 0, if $\phi$ and $\psi$ have degree of complication $m$ and $n$ respectively, then $(\phi \wedge \psi)$ has degree of complication $m+n+1$, etc.) We can now show: \begin{theorem}[Reducing Complexity of Proofs] \label{redcompl}
	If $\Pi$ is a proof in Halbach's system for $\mathcal{L}_{\to,\wedge,\neg}$  that $\Gamma \vdash \phi$, in which the most complicated sentence resulting from an application of $\neg$Elim is $\psi$, then (when $\psi$ is not atomic) there is a distinct proof $\Pi'$ of $\Gamma \vdash \phi$ in which the results of all applications of $\neg$Elim are \emph{less complicated} than $\psi$. Indeed, there exists a proof $\Pi^{\dag}$ of that sequent, in which the result of every application of $\neg$Elim is atomic.
	\begin{proof}
		Suppose that the most complicated result of $\neg$Elim in $\Pi$ is $\psi$. So part of $\Pi$ looks like this: \begin{equation*}
			\begin{prooftree}
				\[\Gamma \quad [\neg \psi] \justifies \leadsto \chi\] \[\Delta \quad [\neg \psi] \leadsto \neg\chi\] \justifies \psi \using{\neg\text{Elim}}
			\end{prooftree}
		\end{equation*} $\psi$ is complicated (not a sentence letter), so given there are three connectives in this language,  there are three cases: \begin{enumerate}
			\item $\psi = \neg \pi$. In which case replace that part of the proof sketched above with this one: \begin{equation*}
				\begin{prooftree}
					\[\Gamma \[[\pi] \quad [\neg \pi] \justifies \neg\neg\pi \using{\neg\text{Intro}}\] \leadsto \chi\]
					\[\Delta  \[[\pi] \quad [\neg \pi] \justifies \neg\neg\pi \using{\neg\text{Intro}}\] \leadsto \neg\chi\] \justifies \neg\pi \using{\neg\text{Intro}}
				\end{prooftree}
			\end{equation*}
		\item $\psi = \pi \wedge \xi$. Proof left for Exercise.
			\item $\psi = \pi \to \xi$. In which case replace that part of the proof sketched above with this one: \begin{equation*}
	\hskip-1.5cm			\begin{prooftree}
				\[	\[\Gamma  \[\[ [\pi \to \xi] \quad [\pi] \justifies \xi \using{\to\text{Elim}}\] \quad [\neg \xi] \justifies \neg(\pi \to \xi) \using{\neg\text{Intro}}\] \leadsto \chi\]
				\[\Delta \[\[ [\pi \to \xi] \quad [\pi] \justifies \xi \using{\to\text{Elim}}\] \quad [\neg \xi] \justifies \neg(\pi \to \xi) \using{\neg\text{Intro}}\] \leadsto \neg\chi\]
					 \justifies  \xi \using{\neg\text{Elim}}\]\justifies \pi \to \xi \using{\to\text{Intro}}
				\end{prooftree}
			\end{equation*}
		\end{enumerate}	

Notice that applying these replacements repeatedly to uses of $\neg$Elim in a proof will eventually replace all complex sentences in uses of $\neg$Elim by simpler ones. All sentences of \lone\ are finitely complex, and all proofs involve only finitely many applications of a given rule, this process terminates after a finite time with only atomic sentences as the conclusions of uses of $\neg$Elim. 
	\end{proof}
\end{theorem}

It is clear that we can put the results of Theorems \ref{altneg} and \ref{redcompl} together, to establish that, in the system which involves the normal rules for $\wedge$ and $\to$, and has $\bot_{\mathsf{C}}$ as its only other rule, if $\Pi$ is a proof in the system $DP$ that $\Gamma \vdash_{DP}\phi$, then there exists a proof of that sequent, $\Pi^{\dag}$, in which the result of any application of $\bot_{\mathsf{C}}$ is atomic. (I leave the proof as a problem.)






\section{Axioms}
\paragraph{Axiomatic Systems}

Another proof system we will now briefly consider is an \emph{axiomatic} (or Hilbert-style) proof system. 

This is familiar in mathematics: one begins with some \emph{axioms} (normally interpreted as `obvious' or `certain' truths), and by applications of some very simple truth-preserving rules of inference, one derives further truths.

If one's axioms are well chosen, one should be able to prove all and only truths about a given subject matter from the axioms characterising that subject matter. This was Euclid's method in the \emph{Elements}: he showed that all and only the truths of Euclidean geometry could be established on the basis of his well chosen axioms.

\paragraph{Axioms for Sentential Logic}

In an axiomatic system, one begins with theorems, the axioms, and applies the rules of inference to preserve theoremhood. 

Our axiom system \L\ is very simple. It has three \emph{axiom-schemata}, in the language $\mathcal{L}_{\neg,\to}$:
\begin{description}
	\item [A1] $\lvdash (\phi \to (\psi \to \phi))$;
	\item [A2]$\lvdash ((\phi \to (\psi \to \chi)) \to ((\phi \to \psi) \to (\phi \to \chi)))$;
	\item [A3] $\lvdash ((\neg \psi \to \neg \phi) \to (\phi \to \psi))$.
\end{description}
Any sentence of $\mathcal{L}_{\neg,\to}$ can be substituted into these axiom schemata to generate an instance of an axiom. Any axiom is a theorem.

The system has one rule (apart from substitution): \emph{\emph{Modus Ponens}}, then rule that if $\lvdash \phi$ and $\lvdash \phi\to\psi$, then $\lvdash\psi$. 

A \emph{proof} in \L\ is any sequence of theorems, each of which is either an axiom or follows by \emph{modus ponens} from earlier theorems.

\paragraph{A Proof in \L}

The system \L\ is  elegant, but proofs in it can be unnatural to say the least: (I annotate lines of the proof with the axiom scheme used.)
\begin{align*}
 1. &\lvdash (P\to((P\to P)\to P)) \to ((P\to(P\to P))\to(P\to P)) & \text{(A2)}\\
2. &\lvdash (P\to((P\to P)\to P)) &\text{(A1)}\\
3. &\lvdash ((P\to(P\to P))\to(P\to P)) &\text{(MP 1, 2)}\\
4. &\lvdash (P \to (P\to P)) & \text{(A1)}\\
5. &\lvdash (P\to P)&\text{(MP 3, 4)}
\end{align*} 

This is (apparently) the \emph{shortest} proof of $P \to P$ in \L!

\paragraph{Soundness of \L}

It is a trivial exercise (and hence left for an exercise) to show that \L\ is sound. That is, show: \begin{itemize}
	\item All the axioms of \L\ are tautologies.
	\item The rule of \L\ preserves tautologousness.
\end{itemize}

\paragraph{Equivalence of \L\ and $ND_{\neg,\to}$}

We now show that, perhaps surprisingly, the proof systems \L\ and $ND_{\neg,\to}$ are equivalent – everything that can be proved in one can be proved in the other. This shows, too, that \L\ is sound and complete.

\paragraph{Everything provable in \L\ is provable in $ND$}

\begin{theorem}
	If $\lvdash\phi$ then $\vdash\phi$.\end{theorem} \begin{proof}
		{ The proof is straightforward: show that each axiom is provable in $ND$, and show that the rule \emph{modus ponens} is a valid rule in $ND$.
		
		The second part is easy; if $\vdash\phi$ and $\vdash\phi\to\psi$, then  appending an instance of $\to$Elim to the preceding proofs is a proof of $\psi$.
		
		Then it is just a matter of proving the axioms. I show A1: \begin{equation*}
			\begin{prooftree}
				\[ [\phi] \justifies \psi\to\phi \using\to\text{Intro}\]
				\justifies \phi \to (\psi\to\phi) \using \to\text{Intro}
			\end{prooftree}
		\end{equation*}}
	\end{proof}



\paragraph{Proofs from Assumptions in \L}

The ugly proof of $\lvdash P\to P$ two paragraphs back shows the benefits of working with \emph{assumptions}. A proof of $\phi$ from assumptions $\Gamma$ in \L\ is a sequence of axioms, \emph{members of $\Gamma$}, or follows from preceding members by \emph{modus ponens}, and the last line of which is $\phi$.
This is a legitimate notion, because one can show: \begin{theorem}[Deduction Theorem for \L]
	 $\Gamma\lvdash\phi$ iff for some finite set of sentences $\gamma_{1},\ldots,\gamma_{n}$ actually used in the proof of $\phi$, $\lvdash (\gamma_{1}\to(\gamma_{2} \to \ldots (\gamma_{n} \to \phi)\ldots))$.
\end{theorem} It suffices to show that $\Gamma,\psi\lvdash\phi$ iff $\Gamma\lvdash\psi\to\phi$. I leave the proof for an exercise.
}


{
\paragraph{Everything Provable in $ND$ is provable in \L}
\begin{theorem}
	If $\vdash \phi$, then $\lvdash \phi$. \begin{proof}
		{ Induction on length of proofs: we show that the shortest proofs of $ND$ are provable in \L, and	that the rules are provable transitions.
		
	Certainly the shortest proof, the single node $\phi$, shows that $\phi\vdash\phi$; but $\phi\lvdash\phi$ is shown by the deduction theorem given $\lvdash \phi\to\phi$.
	
	Suppose we extend some existing  proofs by applying the rules of $ND$: \begin{description}
		\item[$\to$Elim] We have $\Gamma\vdash\phi$ and $\Gamma\vdash\phi\to\psi$, and extend by $\to$Elim. But this is obviously \emph{modus ponens}, so $\Gamma\lvdash\psi$.
	\end{description}
The case of $\to$Intro is shown by the deduction theorem. I leave you to show the cases for negation.}	\end{proof}
\end{theorem}
}

{\small
\subsection*{Further Reading}
\addcontentsline{toc}{subsection}{Further Reading}

Natural deduction was invented  by \citet{geninvinl}, the founder of proof theory. 
The proof theory of natural deduction systems discussed in section 4 of this chapter is based on \citet[39--41]{pranatde}; he uses Theorems \ref{altneg} and \ref{redcompl}, and the last problem, and proves the famous `cut elimination' theorem for natural deduction, unfortunately a topic beyond the scope of these notes.



 The axiom system here was developed by Jan \L ukasiewicz, simplifying Frege's system: see \citet[p. 25]{boslogwol}. (The Polish notation used in this article is initially confusing, but has the wonderful feature that it is entirely unambiguous without the use of parentheses.)
See also \citet[ch. 5--6]{bosintlo}; he gives a direct proof of the completeness of a system based on \L\ at pp.\ 217--9.


\subsection*{Exercises}
\addcontentsline{toc}{subsection}{Exercises}


\begin{enumerate}
	\item \begin{enumerate}
\item Devise introduction and elimination rules for a natural deduction system for the language $\mathcal{L}_{\uparrow}$ with the Sheffer Stroke as its only connective, being sure to fully justify your answer. How many such rules do we need?
	\item Suppose in a system of natural deduction $ND$ for \lone\ we replaced the $\neg$Elim rule by the following, \emph{ex falso quodlibet}: \begin{equation*}
		\begin{prooftree}
			\psi \qquad \neg\psi \justifies \phi \using \text{EFQ}
		\end{prooftree}
	\end{equation*}
\begin{enumerate}
	\item Is the resulting system $ND'$ equivalent (in terms of what can be proved) to the original system? (Justify your answer; it is somewhat difficult to prove conclusively, but give some reasons for your view.)
	\item What happens if we, in addition, replace $\neg$Intro by the following rule, \emph{tertium non datur}, yielding the system $ND''$? \begin{equation*}
		\begin{prooftree}
			\[[\phi] \leadsto \psi\] \[[\neg\phi] \leadsto \psi\] \justifies \psi \using\text{TND}
		\end{prooftree}
	\end{equation*}
\end{enumerate}	
\item Consider the natural deduction system $ND_{\to,\vee}$ for the language $\mathcal{L}_{\to,\vee}$ in which the only connectives are $\to,\vee$. Suppose we introduce to the language the zero-place sentential constant $\bot$, and add the following rules to the proof system, yielding $ND_{\to,\vee,\bot}$: \begin{equation*}
	\begin{prooftree}
		\bot \justifies \phi \using\bot
	\end{prooftree}\qquad
	\begin{prooftree}
		\[[\phi \to \bot] \leadsto \bot\]
		\justifies \phi \using \text{C}
	\end{prooftree}
\end{equation*}\begin{enumerate}
	\item Give a natural translation between $\mathcal{L_{\to,\vee,\bot}}$  and $\mathcal{L}_{\to,\vee,\neg}$.
	\item Using that translation, can you show that the proof system $ND_{\to,\vee,\bot}$ just introduced is equivalent to the natural deduction system $ND_{\to,\vee,\neg}$ involving just Halbach's rules for $\to$, $\vee$ and $\neg$?
	\end{enumerate}
\item A mystery connective $\oplus$ has the following introduction and elimination rules: 

\begin{equation*}
	\begin{prooftree}
		\phi\oplus\psi \justifies \psi \using\oplus\text{Elim}
	\end{prooftree}\qquad \begin{prooftree}
		\phi \justifies \phi\oplus\psi \using\oplus\text{Intro}
	\end{prooftree}
\end{equation*}

\begin{enumerate}
	\item Show that no system containing a connective defined by these rules is sound. 
	\item What limits should be placed on the ability of introduction and elimination rules to define or characterise the inferential role of a connective in light on this result? (This question invites discussion, not a definitive answer. You may wish to consult \citet{prirunint}.)
	\end{enumerate}
	\item  Relatedly to 1.(d).ii: \label{fouronee}\begin{enumerate}
		\item Check, by means of truth tables, that $(((P\to Q)\to P)\to P)$ is a tautology.	
		\item Give a natural deduction proof of $(((P\to Q)\to P)\to P)$.
		\item Every valid argument involving only sentences containing $\wedge$ as their only connective can be proved valid using just the rules $\wedge$Intro and $\wedge$Elim. In that sense, $\wedge$ is completely characterised by its introduction and elimination rules. What do the previous results show about $\to$ in this connection?
\end{enumerate}
\end{enumerate}
\item Using the model soundness proof for the restricted natural deduction system $ND_{\neg,\to}$, show, by analysing proofs constructed using the remaining rules of the full Halbach system $ND$, that the system $ND$ is sound.
\item The Completeness proof for $\mathcal{L}_{\neg,\to}$ relied on various lemmas concerning the behaviour of maximal consistent sets with regard to  the connectives $\neg$ and $\to$. Show that \begin{enumerate}
	\item analogous lemmas hold for $\wedge$, $\vee$ and $\bicond$.
	\item the Satisfiability Theorem holds for the full language \lone.
	\item  the full Natural Deduction system is complete for \lone\ with respect to the intended semantics.
\end{enumerate}  
\item \begin{enumerate}
	\item Show that the axiomatic system \L\ introduced is sound.
	\item Show that the axioms A2 and A3 of \L\ are provable in natural deduction: \begin{enumerate}
		\item A2. $\lvdash ((\phi \to (\psi \to \chi)) \to ((\phi \to \psi) \to (\phi \to \chi)))$;
		\item A3. $\lvdash ((\neg \psi \to \neg \phi) \to (\phi \to \psi))$.
	\end{enumerate}
\end{enumerate}
\item Suppose that $\Gamma,\psi\lvdash\phi$, shown by a proof $\Pi$. Define the \emph{$\psi$-transform} of $\Pi$, written $\Pi^{\psi}$, as the sequence that results from prefixing `$\psi\to$' to the front of every sentence in $\Pi$. Show that each sentence in $\Pi^{\psi}$ is provable from assumptions $\Gamma$ in \L. Since obviously $\psi$ appears on $\Pi$, $\psi\to\psi$ appears on $\Pi^{\psi}$, and that is provable from $\Gamma$. So you need to show that \begin{enumerate}
		\item the prefixed axioms (i.e., those $\psi\to\delta$ in $\Pi^{\psi}$ where $\delta$ is an axiom) are provable from $\Gamma$ (hint: use A1);
		\item the prefixed assumptions (i.e., those $\psi\to\gamma$ in $\Pi^{\psi}$ where $\gamma \in\Gamma$) are provable from $\Gamma$;
		\item If $\chi$ followed by \emph{modus ponens} from early claims in $\Pi$, $\psi\to\chi$ follows from earlier prefixed claims in $\Pi^{\psi}$ (hint: use A2).
	\end{enumerate}
 Using these results, show the deduction theorem for \L.
\item Show the other cases for the equivalence of $ND$ and \L:\footnote{These proofs are quite elusive; I propose treating this question as optional.}  \begin{enumerate}
	\item Show that if $\Gamma,\phi\lvdash\psi$ and $\Gamma,\phi\lvdash\neg\psi$, then $\Gamma\lvdash \neg\phi$.
	\item Show that if $\Gamma,\neg\phi\lvdash\psi$ and $\Gamma,\neg\phi\lvdash\neg\psi$, then $\Gamma\lvdash \phi$. (Equivalently, show that if $\Gamma\lvdash\neg\neg\phi$, then $\Gamma\lvdash\phi$.)
\end{enumerate} 
\item A fellow student argues as follows: \begin{quotation}
	If the Natural Deduction system introduced by Halbach is complete, then whenever $\Gamma \vDash \phi$, there is a proof that shows $\Gamma \vdash \phi$. But proofs are finite; so if $\Gamma$ is infinite, there is no proof that $\Gamma \vdash \phi$ – so the system is not complete!
\end{quotation} How should you respond?
\item Prove the omitted case (where the complex sentence is a conjunction) in the proof of Theorem \ref{redcompl}.
\item Show  that, in the system which involves the normal rules for $\wedge$ and $\to$, and has $\bot_{\mathsf{C}}$ as its only other rule, if $\Pi$ is a proof in the system $DP$ that $\Gamma \vdash_{DP}\phi$, then there exists a proof of that sequent, $\Pi^{\dag}$, in which the result of any application of $\bot_{\mathsf{C}}$ is atomic.
\end{enumerate}





}







\newpage

\chapter{The Syntax and Semantics of $\mathcal{L}_{2}$}\label{c6}
 \input{edl-7.tex}

\newpage

\chapter[Natural Deduction in \ltwo]{Natural Deduction Proofs in \ltwo; Soundness}\label{c7}
 

\section{Proofs in \ltwo}
\paragraph{Proofs in \ltwo}

In the case of \lone, a proof procedure was not strictly speaking necessary. Every sentence of \lone\ has only finitely many sentence letters, so a truth table will need only to specify finitely many rows to capture what is true in every \lone-structure.

But in \ltwo\ we don't have anything like a truth-table: one cannot simply survey a finite number of cases to represent all the structures. (Intuitively, this is because of the introduction of quantification – now a sentence may depend for its truth on the way infinitely many things are, so surveying only finitely many of them won't suffice.) Brute force checking of semantic sequents is not possible. 

Thus the introduction of a proof technique, to enable the derivation of some \ltwo\ claims from others, is crucial. Once more, we use natural deduction; again, we abuse notation at let $\vdash$ stand for \emph{provability in \ltwo}.

\paragraph{Natural Deduction}

The rules for the natural deduction system $ND_{2}$ include all of those from the original system $ND$. As \ltwo\ includes all of \lone, everything provable in $ND$ is also provable in $ND_{2}$. But we include new rules to cover the quantifiers, laid out in Table \ref{tfive}. Throughout, let $\phi[\tau/\upsilon]$ be the result of replacing \emph{free} occurrences of variable $\upsilon$ in $\phi$ by a constant $\tau$.
\begin{table}\label{tfive}
 	\centering
	\begin{tabular}{ccp{4cm}}
	\begin{prooftree}
	\forall \upsilon \phi
 \justifies \phi[\tau/\upsilon] \using{{\forall}\text{Elim}}
\end{prooftree} & \begin{prooftree}
	\[\leadsto \phi[\tau/\upsilon]\] \justifies \forall \upsilon \phi \using{{\forall}\text{Intro}}
\end{prooftree} &{\footnotesize Provided $\tau$ doesn't occur in $\phi$ or in any undischarged assumption in the proof of $\phi[\tau/\upsilon]$.}\\[10pt]
\begin{prooftree}
	\phi[\tau/\upsilon] \justifies \exists \upsilon \phi \using{\exists\text{Intro}} 
\end{prooftree} &
\begin{prooftree}
	\[ [\phi[\tau/\upsilon]] \leadsto \psi\]
	\[  \leadsto \exists \upsilon\phi\] \justifies \psi \using{\exists\text{Elim}} 
\end{prooftree} &{\footnotesize Provided $\tau$ doesn't occur in $\exists\upsilon\phi$, in $\psi$, or in any undischarged assumption other than $\phi[\tau/\upsilon]$ in the proof of $\psi$.}
	\end{tabular} \caption{New Natural Deduction Rules for \ltwo}
 \end{table}

\paragraph{Some Elementary Theorems About $ND_{2}$}

Many of the theorems for $ND$ continue to hold. For example, the proof of the  \emph{deduction theorem} in $ND$ relied only on the arrow rules, so carry directly over to $ND_{2}$.

\begin{theorem}[Cut]
	If $\Gamma \vdash \phi$ and $\phi,\Delta\vdash \psi$ then $\Gamma, \Delta\vdash\psi$. \begin{proof}
		We cannot, as in $ND$, simply chain the proofs together, because $\Gamma$ may contain a constant $\tau$ which, in the proof that $\phi,\Delta\vdash\psi$, is used in an application of $\exists$Elim or $\forall$Intro (of course $\tau$ cannot appear in $\phi$ or $\Delta$ given the correctness of $\Delta,\phi\vdash\psi$). So we need to show that there is \emph{another}  proof of $\Delta,\phi\vdash\psi$ which doesn't involve any such $\tau$. I leave this for an exercise.
	\end{proof}
\end{theorem}	

\paragraph{Proofs and Connectives}

 The natural deduction proofs of many sequents involving sentences with given connectives involve only the rules for those connectives. For example, the sequent $\forall x\exists y(Px \wedge Qx) \vdash \exists y \forall x (Px\wedge Qx)$ can be proved entirely using the rules for conjunction and the quantifiers.

 But this is not true in general. Consider this: $\forall	x (P \vee Qx)\vdash P \vee \forall x Qx$. (Proof below.)

This proof cannot be carried out using just the rules for disjunction and the quantifiers. 

The situation is very similar to a case in an earlier problem (Chapter 4, 1.(e), page \pageref{fouronee}) where the natural deduction rules for $\to$ did not suffice to prove every tautology involving  $\to$ as its only connective. 



\paragraph{Proof of $\forall	x (P \vee Qx)\vdash P \vee \forall x Qx$}

~\\[6pt]

\hskip-0.75cm
 {\footnotesize\begin{prooftree}
 \[
\[
 \[ 
 \[
 \forall x (P \vee Qx) \justifies P \vee Qa \using \forall\text{Elim}\]
  \[[P]	\[   [\neg (P \vee \forall x Qx)] 
 \[ [P] \justifies P \vee \forall x Qx \using \vee\text{Intro}\]
 	\justifies \neg P \using \neg\text{Elim}\]
\justifies Qa \using \neg\text{Elim}\]  
[Qa]
 \justifies Qa \using \vee\text{Elim}\]
 \justifies \forall x Qx \using \forall\text{Intro} \]
 \justifies P \vee \forall Qx \using \vee\text{Intro}\] 
 [\neg (P \vee \forall x Qx)] \justifies P \vee \forall x Qx \using \neg\text{Elim}
\end{prooftree}}


\paragraph{Some Proof Theory: Uniqueness of $\exists$}

Some philosophers have wanted to maintain that there are at least two senses of the English word `exists'. (They have wanted to do this typically to distinguish a committal from a non-committal sense, so they can accept the truth of claims like `there are at lest two prime numbers less than 8' and `there are at least two cities smaller than London', while being committed to cities, and avoiding committment to such mysterious entities as numbers.) But these philosophers have typically also desired to speak a language including both quantifiers;  or at least to be able to translate between the two languages.
There is however a fairly straightforward theorem which shows that this situation is not possible: if there are two quantifiers which both meet the minimal requirements to be an existential quantifier, those quantifiers are logically equivalent. \begin{theorem}[\citealt{harwhasol}] If there are two quantifiers $\exists_{1}$ and $\exists_{2}$ in a language, both governed by the standard introduction and elimination rules for the quantifiers, then $\exists_{1}\upsilon\phi$ is logically equivalent to $\exists_{2}\upsilon\phi$. \begin{proof}
	Assuming that we have introduction and elimination rules for the two quantifiers, we can give the following natural deduction proof:
	\begin{equation*}
		\begin{prooftree}
			\exists_{1}\upsilon\phi \[ [\phi[\tau/\upsilon]] \justifies \exists_{2}\upsilon\phi \using \exists_{2}\text{Intro}\] \justifies \exists_{2}\upsilon\phi \using \exists_{1}\text{Elim}
		\end{prooftree}	
	\end{equation*}
	
	Obviously a precisely similar proof will show that $\exists_{2}\upsilon\phi \vdash\exists_{1}\upsilon\phi$. 
\end{proof}
\end{theorem}

Unless, then, we wish to introduce two different sorts of names, two different sorts of predicates, etc., we cannot simply introduce two quantifiers into a language – not, at least, if they are to meet the minimal conditions that quantifiers should, namely, being characterised by our introduction and elimination rules.

\section{Soundness of \ltwo}
\paragraph{Soundness}

We are now in possession of a proof system $ND_{2}$, which characterises a provability turnstile $\vdash$. We now wish to show that this system is \emph{sound} with respect to the semantics for \ltwo\ we introduced in the last chapter. That is, we wish to prove \begin{theorem}[Soundness]
	If $\Gamma \vdash \phi$ then $\Gamma \vDash \phi$.
\end{theorem}
As $ND_{2}$ extends $ND$, and the rules of $ND$ remain sound (verification of this is left to an exercise), we just need to show that the new rules – the quantifier rules – are sound. We also still know that the base case $\phi\vdash\phi$ is sound. So we just consider proofs extended by the quantifier rules.
\paragraph{Soundness of \ltwo: $\forall$Elim and $\exists$Intro}

Suppose $\Gamma \vdash \phi$, where $\phi$ is obtained from an earlier proof by the use of the $\forall$Elim rule, and so is of the form $\psi[\tau/\upsilon]$. Then there was a proof on the assumption $\Gamma$ of $\forall \upsilon \psi$.

By the induction hypothesis, $\Gamma \vDash \forall \upsilon \psi$. So every structure which makes all the members of $\Gamma$ true, makes $\forall \upsilon \psi$ true. By theorem \ref{ueei} (proved on page \pageref{ueei}), $\forall \upsilon \psi \vDash \psi[\tau/\upsilon]$. Since $\psi[\tau/\upsilon]=\phi$, $\phi$ will be true in every structure which makes $\Gamma$ true, i.e., $\Gamma \vDash \phi$.

 The proof for $\exists$Intro is similar. I leave it for an exercise.


\paragraph{Soundness of \ltwo: $\forall$Intro and $\exists$Elim}



The trickier case is if $\Gamma\vdash\phi$ which was obtained from $\exists$Elim. From a proof of $\exists \nu \psi$ on assumptions $\Delta$, and a proof of $\phi$ on assumptions $\Theta,\psi[\tau/\nu]$, we obtain a proof of $\phi$ on assumption $\Gamma=\Delta\cup\Theta$ (with appropriate restriction on where $\tau$ occurs). 
Since $\Delta\subseteq\Gamma$, we have $\Gamma\vdash \exists \nu \psi$, and by the induction hypothesis, $\Gamma \vDash \exists \nu \psi$. 

Since $\Theta\subseteq\Gamma$, we have $\Gamma,\psi[\tau/\nu] \vdash \phi$, and by the induction hypothesis $\Gamma,\psi[\tau/\nu] \vDash \phi$.
By theorem \ref{uiee} (proved on page \pageref{uiee}), $\Gamma, \exists \nu\psi \vDash \phi$. And by the Cut theorem, $\Gamma\vDash \phi$.
 
The soundness of $\forall$Intro is similarly demonstrated; I leave it for an exercise.



\section{Completeness of \ltwo}
\paragraph{Completeness}

 \begin{theorem}[Completeness of \ltwo]
	If $\Gamma \vDash \phi$, $\Gamma \vdash \phi$.
\end{theorem}The proof extends the proof of completeness for \lone, roughly like this: \begin{itemize}
	\item First, show that, e.g., if $\Gamma$ is a maximal consistent set,  then $\exists \upsilon\phi \in \Gamma$ iff $\phi[\tau/\upsilon]\in \Gamma$ for some constant $\tau$. (This is analogous to the other closure properties we showed.)
	\item Second, show that every maximal consistent set of \ltwo-sentences is satisfiable.
	\item Finally, show that if $\Gamma \not\vdash\phi$, there is a maximal consistent set $\Gamma^{+}$ such that $\Gamma \cup \{\neg\phi\}\subseteq \Gamma^{+}$ – since it is maximal, $\Gamma^{+}$ is satisfiable, so there is a model where all of $\Gamma$ is true and $\phi$ is false, so $\Gamma\not\vDash\phi$. By contraposition, this shows completeness.
\end{itemize}
We won't prove this theorem here; the details are involved.

\paragraph{Compactness}

\begin{theorem}[Compactness]
	If $\Gamma$ is any set of \ltwo\ sentences, then $\Gamma$ is satisfiable iff every finite subset of $\Gamma$ is satisfiable.
\end{theorem}Again, compactness is  a consequence of the finitude of proofs and the Completeness Theorem. A direct proof, like that we gave for \lone\ in Chapter 3, is more difficult in \ltwo, and again beyond the scope of this course.


Compactness has the consequence that certain relations are undefinable. In English, the infinitely many sentences of the form `$x$ is \emph{not} a predecessor of\ldots a predecessor of $y$' entail `$x$ is not less than $y$', but no finite subset does – but their translation into the compact language \ltwo\ must therefore fail. This is because the relation between predecessor and less than is not definable in \ltwo.


% {
% \paragraph{Proof of Compactness Using K\"onig's Lemma}
% 
%  (following notes on the formalization of logic, pp.\ 154--5.)
% }




\section{Decidability and Undecidability}
\paragraph{Completeness and Decidability}



\begin{theorem}[\ltwo\ is positively decidable]
 If $\vDash\phi$, then there is an effective procedure demonstrating that.  \begin{proof}
		{Completeness shows that if $\phi$ is a theorem, there is a proof of it. By brute force, we can generate finite proofs and spit out a proof of $\phi$ if one is to be had. Since the set of sentences of \ltwo\ is countable, and since every proof is of finite length, consider the proofs of length $n$ which involve only the first $m$ \ltwo\ sentences. For each pair $\langle m,n\rangle$ there are only finitely many such proofs; we can effectively produce them. Since the set of pairs of integers are countable (exercise), there is an enumeration of those pairs, hence we can effectively produce any finite proof.
		But every theorem $\phi$ is proved by some such finite proof; so at some finite point in the enumeration, we will reach an $m$ greater than the numbers of all the members of a proof of $\phi$, and at some finite point spit out a proof of $\phi$ of length less than $n$.}
	\end{proof}
\end{theorem}



\paragraph{Undecidability}

\begin{theorem}
	\ltwo\ is \emph{negatively undecidable}: if $\phi\not\vDash$, no effective procedure exists which will show that for any $\phi$ in a finite time.
\end{theorem} Considering the effective procedure for positive decidability, we can see why this won't stop after a finite time (the proof of $\neg\phi$ might always be achieved using the next pair to be considered); but maybe there is another method that will terminate?  

There is not.  One way of helping to make this thought persuasive is to consider a particular incorrect sequent:  $\forall x \exists y Pxy \vDash Paa$. One can certainly give a counterexample to this (let the domain be the natural numbers, and `$P$' be interpreted as `$<$'). But is there an effective procedure for producing such counterexamples?


\paragraph{The Halting Problem}

If $S$ is a finite set of instructions, can we give an effective procedure that determines whether  it is the case that those instructions, carried out on some acceptable input, will \emph{terminate} after a finite time with some output?

 Consider some enumeration of the set of finite sets of instructions; the halting problem, if soluble, will entail the existence of an effectively computable function $h(i)$ that yields $1$ if the $i$-th set of instructions halts with a defined output on input $i$, and $0$ otherwise (i.e., $h(i) =0$ if the $i$-th set of instructions determines a function which is undefined on input $i$). Can there be such a function $h$?


\paragraph{Sketch of Insolubility of the Halting Problem}

If there were such a function $h$, we could effectively compute the function $g(i) = 0$ if the $i$-th set of instructions is undefined on input $i$, and undefined otherwise. We could define $g$ by adding to the instructions that compute $h$ the last instruction: if $h(i)$ outputs $1$, then check if $h(i)$ outputs $1$. This last instruction will send give an infinite loop if $h(i)$ is equal to 1, rendering $g(i)$ undefined; but if $h(i)=0$, then $g(i)=0$ too. 

But $g$, if effectively computable, will be specified by a finite set of instructions; let it be $j$-th in our list. Then $g(j)=0$ iff $g(j)$ is undefined, so it is undefined; but then $g(j)$ must be defined. Contradiction; so there is no such $g$, and therefore no such $h$. The \emph{self-halting problem} is insoluble, and therefore so is the halting problem.

For a more humorous proof sketch, see \citet{pulscolos}.

\paragraph{Undecidability and the Halting Problem}

It turns out that the undecidability of \ltwo\ is closely related to the Halting problem. For there is a way of associating to each set of instructions $i$ an \emph{argument} in \ltwo\ such that the $i$-th set of instructions halts on input $i$ iff the argument is valid. So 	if there was an effective test for invalidity of an argument in \ltwo, the halting problem would be solvable. Since it is insoluble, we have
 \begin{theorem}[Undecidability of \ltwo]
\ltwo\ is not decidable.
\end{theorem}
The translation essentially involves \emph{binary} predicates (Jeffrey \S 8.7) and sentences in which (sometimes) a universal quantifier is essentially in the scope of an existential quantifier. So we might wonder: are there fragments of \ltwo\ that are decidable? 



\paragraph{Prenex Normal Form}

A sentence $\phi$ of \ltwo\ is in \emph{prenex normal form} iff no quantifier in $\phi$ occurs in the scope of any truth-functor (all quantifiers are in a block). 
\begin{theorem}[PNF]
	Every sentence of \ltwo\ is equivalent to a sentence in prenex normal form.
\begin{proof}
	{The proof will be work by showing that if $\phi$ is some sentence, any quantifier which occurs somewhere in the middle of $\phi$ can be `moved' to the left. The idea is to show these equivalences: \begin{enumerate}
		\item $\neg\forall\upsilon\phi \equiv \exists \upsilon\neg \phi$; (basically shown by Theorem \ref{qi})
		\item $\neg \exists\upsilon\phi \equiv \forall \upsilon\neg\phi$; (basically shown by Theorem \ref{qi})
		\item $\psi \wedge \forall \upsilon \phi \equiv \forall\upsilon(\psi\wedge\phi)$ ($\upsilon$ not free in $\psi$).
		\item $\psi \wedge \exists \upsilon \phi \equiv \exists\upsilon(\psi\wedge\phi)$ ($\upsilon$ not free in $\psi$).
	\end{enumerate} Etc. The full proof is left for an exercise.}
\end{proof}\end{theorem}


\paragraph{Decidability of $\forall\exists$-sentences}

An \emph{$\forall\exists$-sentence} is a sentence in PNF in which all universal quantifiers precede all existential quantifiers.

\begin{theorem}[Decidability of $\forall\exists$-sentences]
	There exists an effective procedure which establishes the validity of any valid $\forall\exists$-sentence, and the invalidity of any invalid $\forall\exists$-sentence.
\end{theorem} 

The proof is as follows. We show how to reduce the question of validity of a $\forall\exists$-sentence to the question of validity of a related quantifier-free sentence; and such quantifier-free sentences of \ltwo\ are decidable. For convenience, I abbreviate $\forall x_{1} \ldots \forall x_{n}$ as $\forall x_{1}\ldots x_{n}$, and similarly for $\exists$.

\paragraph{Lemma on Quantifier-Free Sentences}

\begin{lemma}[Quantifier-Free Sentences] \label{qfs}
	If $\phi$ is a sentence of \ltwo, which is not a (substitution instance of a) truth-functional tautology, then there is a structure $\mathscr{A}$ in which $\val{\phi}{A}=F$ and either (i) if $\phi$ contains no constants, the domain of $\mathscr{A}$ has one element; or (ii) otherwise, $\mathscr{A}$ has an element in its domain for each distinct constant occurring in $\phi$ and no other elements, and each constant in $\phi$ is assigned a distinct element of the domain.  \begin{proof}
		{Hint: if $\phi$ is not a truth-functional tautology, there is an assignment of truth-values to atomic sentences in $\phi$ that mimics an \lone-structure; and  we can construct an \ltwo-structure $\mathscr{A}$ which agrees with that pseudo-\lone-structure on its assignment to atomic sentences.}
	\end{proof}
\end{lemma}
A corollary is that quantifier-free sentences of \ltwo\ are decidable, since the truth-table test decides truth-functional tautologies. 

\paragraph{$\exists$-sentences}

\begin{theorem}[Constant-free $\exists$-sentences]
	If $\phi$ is a quantifier-free formula which contains only the variables $\nu_{1},\ldots, \nu_{n}$ and which contains no constants, then $\vDash \exists \nu_{1}\ldots\nu_{n} \phi$ iff $\vDash \phi[\tau/\nu_{1},\ldots,\tau/\nu_{n}]$. \begin{proof}
		{\emph{R to L:} Repeated applications of Existential Introduction (Theorem \ref{ueei}) yield $\phi[\tau/\nu_{1},\ldots,\tau/\nu_{n}] \vDash \exists \nu_{1}\ldots\nu_{n} \phi$. By Cut, the result follows.
		
		 \emph{L to R:} If $\not\vDash \phi[\tau/\nu_{1},\ldots,\tau/\nu_{n}]$, then by Lemma \ref{qfs}, there is a structure with a  one-element domain $\mathscr{A}$ where it is false.  If $\val{\exists \nu_{1},\ldots\nu_{n} \phi}{A}=T$, there is a variable assignment $\alpha$ such that $\valsa{\phi}=T$; but this variable assignment must assign $\valsa{\tau}$ to each variable (there is only one element in the domain), so $\valsa{\phi[\tau/\nu_{1},\ldots,\tau/\nu_{n}]}=T$, contradiction. (Using Theorem \ref{ssev}.)}
	\end{proof}
\end{theorem}
These $\exists$-sentences are \emph{equi-decidable} with quantifier-free sentences.

\begin{theorem}[Arbitrary $\exists$-sentences]
	If $\phi$ is a quantifier free formula containing only $\tau_{1},\ldots,\tau_{n}$ and $\nu_{1},\ldots,\nu_{m}$, then $$\vDash\exists \nu_{1}\ldots\nu_{m}\phi \quad\text{ iff }\quad \vDash \bigvee (\phi[\tau_{i}/\nu_{j}]),$$ where $\bigvee (\phi[\tau_{i}/\nu_{j}])$ is the disjunction of \emph{all} the ways of substituting the constants $\tau_{1},\ldots,\tau_{n}$ for the variables $\nu_{1},\ldots,\nu_{m}$.
\end{theorem}

I leave the proof of this for an exercise; it is along the same lines as the proof of the proof for constant-free $\exists$-sentences.

Now we have shown that an arbitrary sentence involving only existential quantification is valid iff some quantifier-free sentence is valid, and since the latter is decidable, so is the former. Now we just need to show that any $\forall\exists$-sentence is valid iff some $\exists$-sentence is valid.


\paragraph{$\forall\exists$-Sentences}

\begin{theorem}[$\forall\exists$-sentences and $\exists$-sentences]
	If $\tau_{1},\ldots,\tau_{m}$ don't occur in $\phi$, then $$\vDash \forall \upsilon_{1}\ldots\upsilon_{m}\exists\nu_{1}\ldots\nu_{n}\phi \quad\text{ iff }\quad \vDash\exists\nu_{1}\ldots\nu_{n}\phi[\tau_{1}/\upsilon_{1},\ldots,\tau_{m}/\upsilon_{m}].$$ \begin{proof} 
		{\emph{L to R:} By repeated applications of $\forall$-Elimination (Theorem \ref{ueei}), we have \begin{equation*}
			\forall \upsilon_{1}\ldots\upsilon_{m}\exists\nu_{1}\ldots\nu_{n}\phi \vDash \exists\nu_{1}\ldots\nu_{n}\phi[\tau_{1}/\upsilon_{1},\ldots,\tau_{m}/\upsilon_{m}],
		\end{equation*}from which the result follows by Cut.
		
		\emph{R to L:} Because $\Gamma$ is empty, and $\tau_{1},\ldots,\tau_{m}$ don't occur in $\phi$, we can use $\forall$-Introduction (Theorem \ref{uiee}) repeatedly to show the theorem.}
	\end{proof}
\end{theorem}
So any $\forall\exists$-sentence is equi-valid with some quantifier-free sentence.

\paragraph{Elementary Quantifications and Monadic Sentences}

A sentence is an \emph{elementary quantification} iff no quantifier occurs in the scope of any other. Some sentences won't be equivalent to any elementarily quantified sentence – e.g., $\forall x\exists y Rxy$ can't be put into that form.

An \ltwo\ sentence is \emph{monadic} iff it contains at most unary (one-place) predicates.

\begin{lemma}[Monadic and Elementary Sentences]
	Every monadic sentence of \ltwo\ is equivalent to some elementary quantification. \begin{proof}
		Left as a problem. The general idea is: since we have no binary predicates, each bound variable in a sentence is associated with a unique quantifier, and we can `drive in' the quantifiers so that it binds only its own variable.
	\end{proof}
\end{lemma}

\paragraph{Corollary: Monadic \ltwo\ decidable}

\begin{theorem}[Monadic \ltwo\ sentences]
	If $\phi$ is a monadic sentence of \ltwo, it is equivalent to a $\forall\exists$-sentence. \begin{proof}
		Suppose $\phi$ is a monadic sentence.  We can effectively produce a an elementary quantification $\phi'$, logically equivalent to $\phi$.   But switching the order of any quantifiers in that driven in sentence can't change the truth-value (exercise). And so we can reorder the quantifiers in $\phi'$ to get another logically equivalent sentence $\phi''$ that is $\forall\exists$. Each step is effective: monadic \ltwo\ is decidable.
	\end{proof}
\end{theorem}

{\small
\subsection*{Further Reading}
\addcontentsline{toc}{subsection}{Further Reading}

Further discussion of the Harris Theorem is in \citet{mcgtherue}.

The completeness proof was first established by G\"odel in his 1929 doctoral thesis; the proof sketched here follows the later proof of \citet{hencomfio}.  The technique of associating (in)valid arguments with (un)defined computable functions mentioned in the proof of undecidability is involved. An elementary version is in \citet[ch. 7--8]{jefforlos}. A more sophisticated version is in \citet[ch. 1--8]{bbjcomlo}. The connection between decidability and the halting problem was shown by Church and Turing, independently: Turing's result is in \citet{turoncon}.  This technique for proving decidability of $\forall\exists$-sentences is in \citet[\S 3.9]{bosintlo}. He provides an alternative decision procedure for monadic \ltwo\ in \S 3.8.



\subsection*{Exercises}
\addcontentsline{toc}{subsection}{Exercises}

\begin{enumerate}
\item Prove, using the hints in the chapter, the Cut Theorem: that if $\Gamma\vdash\phi$ and $\Delta,\phi\vdash\psi$, then $\Gamma,\Delta\vdash\psi$.
\item Explain why we could have used this rule instead of our $\forall$Intro: \begin{center}
	\begin{tabular}{cp{5cm}}
		\begin{prooftree}
			\[\leadsto \phi\] \justifies \forall \upsilon \phi[\upsilon/\tau] \using{{\forall}\text{Intro}}
		\end{prooftree} &{\footnotesize Provided $\tau$ doesn't occur in any undischarged assumption in the proof of $\phi$.}
	\end{tabular}
\end{center}  
\item Let $\psi$ be the result of substituting $\upsilon$ for \emph{all} occurrences of $\tau$ in $\phi$. Show that this rule is not sound (i.e., not truth-preserving) \begin{quotation}
	If $\Gamma,\phi \vdash \chi$ then $\Gamma,\exists \upsilon \psi \vdash \chi$, provided $\tau$ does not occur in $\Gamma$ or in $\chi$.
\end{quotation} Can you think of a way of amending the rule to make it sound? \emph{(Hint: Theorem \ref{fivesubcod}}.)
\item Verify that all the rules of $ND$ are sound with respect to the \ltwo\ semantics.
\item \begin{enumerate}
	\item Show that the rule $\exists$Intro is sound with respect to the \ltwo\ semantics.
	\item Show that the rule $\forall$Intro is sound with respect to the \ltwo\ semantics.
\end{enumerate}
\item Explain why one cannot define the relation `$x$ is an ancestor of $y$' in terms of the relation `$x$ is a parent of $y$'. What resources could be added to the language to allow this to be defined?
\item Sketch the proof that for any finite set of \ltwo\ sentences, we can effectively produce every proof that uses only sentences in that set.
\item Assuming that $\wedge,\neg,\vee$ are the only truth functional connectives in $\phi$: \begin{enumerate}
	\item Show that, where $\upsilon$ is not free in $\psi$, that $\psi \wedge \forall\upsilon\phi \equiv \forall\upsilon(\psi \wedge \phi)$.
	\item Show that, where $\upsilon$ is not free in $\psi$, that $\psi \wedge \exists\upsilon\phi \equiv \exists\upsilon(\psi \wedge \phi)$.
	\item Show that, where $\upsilon$ is not free in $\psi$, that $\psi \vee \forall\upsilon\phi \equiv \forall\upsilon(\psi \vee \phi)$.
	\item Show that, where $\upsilon$ is not free in $\psi$, that $\psi \vee \exists\upsilon\phi \equiv \exists\upsilon(\psi \vee \phi)$.	
	\item Show that $\forall \upsilon \phi \wedge \forall \upsilon\psi \equiv \forall\upsilon(\phi\wedge\psi)$.
	\item Show that $\exists \upsilon \phi \vee \exists \upsilon\psi \equiv \exists\upsilon(\phi\vee\psi)$.
	\end{enumerate}
	\item \begin{enumerate}
		\item Using the results in problem 8, plus the results on the interdefinability of quantifiers from week 5, prove the PNF theorem for the fragment of \ltwo\ without arrow or biconditional.
		\item Prove the PNF in the following stronger form: \begin{quotation}
			For every sentence $\phi$ in the arrow and biconditional-free fragment of \ltwo, there is a logically equivalent sentence $\phi'$ in PNF \emph{that has no more connectives than $\phi$}.\end{quotation} 
		\item Can we extend the stronger result to the full language \ltwo?
	\end{enumerate}
	\item Prove the Lemma on Quantifier-Free Sentences.
	\item Prove the Theorem for Arbitrary $\exists$-Sentences.
	\item Suppose that $\phi$ is a monadic \ltwo\ sentence containing only $\wedge,\vee,\neg$ as connectives in addition to quantifiers.\begin{enumerate}
		\item Show that we can construct a sentence $\phi'$ logically equivalent to $\phi$ in which no quantifier occurs in the scope of any other – i.e., an elementary quantification. (Consider exercises 8.(a--f), and recalling facts about DNF and CNF, thinking particularly about giving a CNF `equivalent' of an open formula in DNF and \emph{vice versa}.)
		\item Show that $\forall \upsilon \phi \wedge \forall \upsilon \psi \equiv \forall \upsilon \phi \wedge \forall \nu \psi[\nu/\upsilon]$ (and similarly for disjunction, and other combinations of universal and existential quantifiers).
		\item Show that for any elementary quantification $\phi'$ we can construct a logically equivalent sentence $\phi''$ in which any quantifier in $\phi'$ is brought to the front and has scope over the others. (Tip: remember to use the result proved in 12.(b).)
		\item Show, using these results, that there is a $\forall\exists$ sentence equivalent to any such monadic $\phi$.
	\end{enumerate}
\item Prove that there is no sentence $\phi$ which contains just one occurrence of some two-place connective, just one occurrence of any quantifier, which is in PNF and is equivalent to $P \bicond \forall x Qx$. Is there any connective other than $\bicond$ for which this also holds? Why?
\item The \emph{Cantor Pairing Function} is this function from $\mathbb{N}\times \mathbb{N}$ to $\mathbb{N}$: \begin{equation*}
	\pi(\langle x,y\rangle) = \dfrac{(x+y)(x+y+1)}{2}+y.
\end{equation*} Show that this function has an inverse (i.e., a function from natural numbers to pairs of natural numbers). Show therefore that $\pi$ is one-one and onto, and that the set of pairs of natural numbers is countable. (Hint: show that, given some $z$, we can reconstruct a unique $x$ and $y$ from it such that $\pi(\langle x,y\rangle)=z$.)
 \end{enumerate}

}

	




\chapter{Syntax, Semantics, and Proofs in $\mathcal{L}_{=}$}\label{c8}
 


	\section{Identity}

	
	\paragraph{Identity: Syntax and Semantics}

	The language \ltwo\ doesn't have a privileged predicate for identity. Yet such a predicate is very useful in discussing binary relations. 

	We \emph{add} to \ltwo\ a binary predicate `='. We add  new atomic formulae: where $\tau_{1}$ and $\tau_{2}$ are any terms, `$\tau_{1}=\tau_{2}$' is an atomic formula.  We keep the same structures as \ltwo, but give this new predicate a constant semantic value, so that in every \ltwo-structure $\mathscr{A}$, \begin{equation}
		I_{\mathscr{A}}({=}) = \{\langle x,x\rangle : x \in D_{\mathscr{A}}\} \tag{=}
	\end{equation}
	Of course the property that we intend this predicate to express could exist in an \ltwo-structure already; the change is that we now  give it a logically privileged meaning.
	The satisfaction conditions are: \begin{itemize}
	\item	$\valsa{\tau_{1}=\tau_{2}}=T$ iff $\langle \valsa{\tau_{1}},\valsa{\tau_{2}}\rangle \in \valsa{{=}}$, i.e., iff $\valsa{\tau_{1}}=\valsa{\tau_{2}}$; i.e., if $\valsa{\tau}$ is the very same object as $\valsa{\kappa}$.
	\end{itemize}

	Call the language which keeps all the old atomic formulae and formation rules of \ltwo, but adds these atomic formulae, \emph{\lequ}. 
	
\paragraph{Some Theorems About Identity}

\begin{theorem}[Theoremhood of Identity]
	For any constant $\tau$, $\vDash \tau=\tau$. \begin{proof}
		Suppose $\not\vDash\tau=\tau$ for some $\tau$. Then there is a structure $\mathscr{A}$ such that $\val{\tau=\tau}{A}=F$, iff $\valsa{\tau}\neq\valsa{\tau}$; impossible.
	\end{proof}
\end{theorem} \begin{theorem}[Substitution of co-designating terms II]
	$\tau=\kappa\vDash \phi[\tau/\upsilon] \bicond \phi[\kappa/\upsilon]$. \begin{proof}Fairly immediate from the Substitution of Co-Designating Terms theorem (chapter 5, page \pageref{fivesubcod}). \label{scdc}
	\end{proof}
\end{theorem}

\paragraph{The Relation of Identity}

\begin{theorem}[Identity an Equivalence Relation]
	\begin{itemize}
		\item $\vDash \forall x x=x$;
		\item $\vDash \forall x \forall y (x=y \to y=x)$;
		\item $\vDash\forall x \forall y \forall z ((x=y \wedge y=z) \to x=z)$.
	\end{itemize} 
\end{theorem} \begin{theorem}[`Leibniz' Law']
For any $\phi$ in which at most $\upsilon$ occurs free, $\vDash \forall \upsilon \forall \nu (\upsilon=\nu \to \phi \bicond \phi[\nu/\upsilon])$. 
 \end{theorem}

		Proofs left for exercises. \emph{Leibniz' Law} usually refers to this \emph{definition} of identity in second order logic: $\forall \upsilon \forall \nu (\upsilon = \nu \bicond \forall \Phi (\Phi(\upsilon) \bicond \Phi[\nu/\upsilon](\nu)))$.

\paragraph{Proofs in \lequ}

To obtain a natural deduction system for \lequ, $ND_{=}$, we \emph{add} to the rules of $ND_{2}$ the rules in Table \ref{tsix}, keeping all other rules. 
\begin{table}\label{tsix}\begin{center}
	\begin{tabular}{cc}
	\begin{prooftree}
		[\tau=\tau]\justifies \vdots \using \text{=Intro} 
	\end{prooftree}& \\ \begin{prooftree}
		\[ \leadsto \phi[\tau/\upsilon]\] \[\leadsto \tau=\kappa\] \justifies \phi[\kappa/\upsilon] \using \text{=Elim-r}
	\end{prooftree}	& \begin{prooftree}
			\[ \leadsto \phi[\tau/\upsilon]\] \[\leadsto \kappa=\tau\] \justifies \phi[\kappa/\upsilon] \using \text{=Elim-l}
		\end{prooftree}	
	\end{tabular} \end{center}	\caption{New Natural Deduction Rules for \lequ}
\end{table}
In =Elim-l and =Elim-r, $\phi$ is a formula in which at most $\upsilon$ occurs free. 

\paragraph{Dispensibility of =Elim-l/=Elim-r}

The two rules =Elim-l and =Elim-r obviously offer very similar resources in proofs. If we could show that $\tau=\kappa \vdash \kappa=\tau$ using just \emph{one} of these rules, we could show the other to be dispensible.

\begin{theorem}[Dispensibility of =Elim-r (or =Elim-l)]
	In the presence of the rule =Elim-l (respectively, =Elim-r), the functionality of =Elim-r (=Elim-l) can be derived. \begin{proof}{
		\begin{prooftree} \phi[\tau/\upsilon]\quad \[[\kappa=\kappa] \quad \tau=\kappa \justifies \kappa=\tau \using \text{=Elim-l}.\] \justifies \phi[\kappa/\upsilon] \using \text{=Elim-l}
		\end{prooftree}
		
 This is obviously equivalent to =Elim-r; a similar proof would show the dispensibility of =Elim-l. (Note the use of =Intro.)}
	\end{proof}
\end{theorem}

\paragraph{Soundness of \lequ}

\begin{theorem}[\lequ\ is Sound]
If $\phi$ is a sentence of \lequ, then if $\vdash_{\!\mathcal{L}_{=}} \phi$ then $\vDash \phi$. \begin{proof}
	 We only need show that proofs extended by the new rules are sound. \begin{itemize}
		\item We have a proof of $\phi$ on assumption $\tau=\tau$, and apply =Intro to discharge the assumption. Since by the Theoremhood of Identity, $\val{\tau=\tau}{A}=T$ in every structure, and by the induction hypothesis $\val{\phi}{A}=T$ if $\val{\tau=\tau}{A}=T$, clearly $\val{\phi}{A}=T$.
		\item We have a proof of $\tau=\kappa$ on assumptions $\Gamma$, and a proof of $\phi[\tau/\upsilon]$ on assumptions $\Delta$, and we extend to a proof of $\phi[\kappa/\upsilon]$ using =Elim-r on assumptions $\Gamma\cup\Delta$. By the induction hypothesis, for any $\mathscr{A}$ where $\Gamma\cup\Delta$ is satisfied,  $\val{\tau=\kappa}{A}=T$, and similarly $\val{\phi[\tau/\upsilon]}{A}=T$. By the Substitution of co-designating constants theorem of page \pageref{scdc} and logic, in any such $\mathscr{A}$, $\val{\phi[\kappa/\upsilon]}{A}=T$. 
	\end{itemize} 
\end{proof}
\end{theorem}

\paragraph{Further metalogical results}

It turns out that the proof of completeness of \ltwo\ can be extended to provide a completeness proof for \lequ. \begin{theorem}[Completeness of \lequ]
	If $\vDash \phi$ then $\vdash \phi$.
\end{theorem}
In the now familiar way, this can be adapted to a compactness proof.

Since \ltwo\ is not decidable, and \lequ\ contains \ltwo\ as a part, \lequ\ is not decidable either. Obviously the monadic fragment of \lequ\ is decidable (as `$=$' is binary, the monadic fragment is the same as that of \ltwo). A more interesting fragment is \emph{the pure theory of identity} – the fragment where the only atomic formulae are of the form $\tau=\kappa$. This, it turns out, is decidable \citep[329--31]{bosintlo}. 


% {
% \paragraph{The Pure Theory of Identity}
% 
% bostock 329--31
% }
% 
% {
% \paragraph{Decidability of the Pure Theory of Identity}
% 
% 
% }
% 
% {
% \paragraph{Decidability of \lequ}
% 
% 
% }


\section{Numerical Quantification and the Theory of Definite Descriptions}
\paragraph{Numerical Quantification: `At Least'}

The ordinary interpretation of the sentence $\exists x Px$ is that there are one or more things which satisfy $P$. Obviously, $\exists x \exists y Px \wedge Py$ does not express that there are two or more things which satisfy $P$, because $x$ and $y$ may take the same values under a variable assignment. To express that there are at least two $P$s, we need to guarantee the \emph{distinctness} of the values of the variables. We can do this using identity: \begin{equation*}
	\exists x \exists y ((Px \wedge Py) \wedge \neg x=y).
\end{equation*}
  Similarly, \begin{equation*}
  	\exists x \exists y \exists z ((Px \wedge Py \wedge Pz)\wedge (\neg x=y \wedge \neg y=z \wedge \neg x=z))
  \end{equation*} expresses that there are \emph{at least three} distinct $P$s. Let $\exists_{n}\phi$ express that there are at least $n$ things which satisfy $\phi$.

\paragraph{Numerical Quantification: `At Most', `Exactly'}

Just as for `at least', so we can formalise `there are at most $n$ $P$s'. Under what circumstances are there at most $n$ things? Just in case, if we chose $n+1$ times, we would have to have chosen some of the things twice: some of things chosen would have to be identical.  So we formalise `there are at most $2$ $P$s' as \begin{equation*}\forall x \forall y \forall z ((Px \wedge Py \wedge Pz) \to (x=y \vee x=z \vee y=z)).
	\end{equation*}
Formalise `there are at most $n$ $\phi$s' as $\forall_{n} x \phi$. Then we can say that \emph{there are exactly $n$ $\phi$} as \begin{equation*}
	\exists_{n} x \phi \wedge \forall_{n} x \phi.
\end{equation*} There will be more or less concise ways of expressing this claim.

\paragraph{Definite Descriptions}

The \lequ\ sentence $Pa$ is often a good translation of simple English sentences of subject-predicate form. Yet some sentences, with this apparent form, cannot be translated in this way. Consider `The fourth-oldest college is a college'. While `the fourth-oldest college' is a referring expression, denoting a particular item in the domain, it seems a bad idea to formalise this by a constant $a$. For one thing, the English sentence is a tautology, but the proposed formalisation is not. So some English  expressions which, like constants, denote an object in the domain, shouldn't be formalised by constants. It is notable that `the fourth-oldest college' doesn't name a particular college (Exeter) – it \emph{describes} it. A \emph{definite description} designates a particular thing by giving a description that the thing – and \emph{only} that thing – satisfies. We will look again at the relation between constants of \ltwo/\lequ\ and English referring expressions in Chapter 8.

So `the strongest man in the world' designates that person who satisfies the property of being a man stronger than any other. `Strongest' entails uniqueness, and this is generally true for definite descriptions: they fail to refer if more than one thing satisfies the description (so `the college on Turl St' fails, because there are three such). \emph{Indefinite} descriptions, such as `a strong man', do not require uniqueness.

Definite descriptions also seem to require existence of the thing they describe: the definite description `the present king of France' fails because \emph{nothing} satisfies that description.

\paragraph{Definite Descriptions and Numerical Quantification}

Putting these ideas together, we might say that the definite description `the $P$' refers to the unique existing thing that is $P$. But we know how to express that there is one and only one $P$, using numerical quantification – so we can say that `the $P$ is $Q$', because we can say that there is one and only one $P$, and it is $Q$. Where $\phi,\psi$ are expressions in which at most $\upsilon$ occurs free, let us introduce the notation $\dd \upsilon: \phi$ to denote `the $\upsilon$ such that $\phi$', or simply, `the $\phi$'. One could permit this expression can take the place of a constant, so if $\psi\tau$ is well-formed, so is $\psi(\dd \upsilon: \phi)$; the latter says `the $\phi$ is $\psi$'. Given our resources, though, we needn't add this to the language, or deal with the complications that ensue, for this expression can be defined:  \begin{definition}[Definite Descriptions]
	$\psi(\dd \upsilon: \phi) \eqdf \exists \upsilon (\psi \wedge \forall \nu \phi[\nu/\upsilon] \bicond \upsilon=\nu)$.
\end{definition}
This, essentially, is \emph{Russell's analysis of the logical form of definite descriptions}. 

\paragraph{Descriptions and Scope}

One reason to suspect that definite descriptions aren't just ordinary names – and a corresponding reason to reject the descriptivist account of the meaning of the constants of natural language – is that descriptions, unlike names, have \emph{scope}. Consider \begin{description}
	\item [(*)]  The prime minister has always been Australian.
\end{description} In \lequ, there is no non-truth-functional operator `it always has been that $\phi$'. We can mimic this (or maybe it is not mimicking) by \emph{quantifying over times}. Let $t$ be `now', $\prec$ be `is earlier than', $P$ be `is prime minister at', and $A$ be `is Australian'. (*) has two readings: \begin{align*}
	\forall x (x \prec t \to \exists y ((Pyx \wedge \forall z (Pzx \bicond y=z)))\wedge Ay);\\
	\exists y ((Pyt \wedge \forall z (Pzt \bicond y=z))\wedge \forall x (x \prec t \to Ay)).
\end{align*} The first says: always the prime minster, whoever they've been, has been Australian at the time of their prime ministership; the second says the present prime minister has always been Australian.

(Very similar results can be seen in \emph{modal} contexts of possibility and necessity. We can again mimic the behaviour of the operator `possibly' as an existential quantifier over `possible worlds', and `necessarily' (like `always') as a universal quantifier.)

The problem is that in the case of names we seem \emph{not} to see these scope problems. For this claim has only one reading: \begin{description}
	\item [(\dag)] Kevin Rudd has always been Australian.
\end{description}The existence of scopes for definite descriptions, is some evidence for the fact that they are really quantifier expressions. 

Yet there are some disanalogies. Consider \begin{description}
	\item [(\ddag)] The present king of France is bald.
\end{description} According to Russell's theory, this is \emph{false} – existence fails. But we may not share that intuition; this claim may strike us as neither true nor false, or having a false presupposition and thus unassessable.


\paragraph{Alternative Theories of Descriptions}

Strawson used our uncertain intuitions about (\ddag) to argue that descriptions are often used referentially, \begin{quotation}
	to mention 	or refer to some individual person or single object\ldots, in the course 	of doing what we should normally describe as making a statement 
	about that person [or] object. 
\end{quotation} When uttering a referring expression, one \emph{presupposes} the existence of the thing. But when our intentions go awry, and we have presupposed something false, the utterance fails to have a truth value.

One truly weird way of dealing with (\ddag) is due to Meinong. This is to suppose that definite descriptions always pick out something, but sometimes the things picked out do not exist! Rather, according to Meinong, they \emph{subsist}. Meinongianism apaprently handles one type of problematic definition description with ease:
\begin{description}
	\item [(\P)] The present king of France does not exist.
\end{description} 
But one might legitimately query whether a real success has been achieved here. For Meinong, apparently empty definite descriptions, like the present king of France, do function referentially: they pick out merely subsisting entities. But then Meinong offers us an explanation as to why we cannot use existential generalisation (the English analogue of the \ltwo\ rule $\exists$-intro) on such singular terms. For we manifestly cannot use it, else the true claim (\P) would entail this false claim:
\begin{description}
	\item [(!)] There is something that does not exist.
\end{description} 



\section{Compactness and Cardinality}
\paragraph{`There are $n$ things' and infinity}

We saw before that there is a set of \lequ\ sentences satisfiable only in infinite domains.  If $R$ is a transitive asymmetric relation (both of which properties are expressible by an \lequ\ sentence), then $\forall x \exists y Rxy$ will be satisfiable in no finite domain. Can we give a set of sentences satisfiable in \emph{all and only} infinite domains?

Using the resources for numerical quantification, define $\mathbf{n}$ as $\exists_{n} x (x=x)$. Then $\mathbf{3}$ says there are at least three things.

 Consider the set $\mathbf{N} = \{\mathbf{n} : n \in \mathbb{N}\}$. $\mathbf{N}$ is not satisfiable in any finite domain: suppose it were satisfiable in domain of size $m$, then where $n=m+1$, $\mathbf{n}$ would be false and  because $\mathbf{n} \in \mathbf{N}$, $\mathbf{N}$ is not satisfied. So $\mathbf{N}$ is satisfiable only in infinite domains. Moreover, in every infinite domain, $\mathbf{N}$ is satisfied. (exercise). 

$\mathbf{N}$ is thus an `axiomatisation of infinity'. It is a consequence of compactness that there is no single \lequ\ sentence which is an axiom of infinity (exercise).




\paragraph{Finitude Undefinable}

Since there is at least one sentence of \lequ\ true only in infinite domains, perhaps there is a sentence of \lequ\ true only in \emph{finite} domains? Perhaps surprisingly, there is not.\begin{theorem}[Finitude Undefinable]
There is no set of \lequ-sentences satisfiable in all and only finite domains. \begin{proof}
	{Suppose $FIN$ is a set of \lequ\ sentences true in all and only finite domains. $FIN \cup \mathbf{N}$ must be unsatisfiable; by compactness, where $\Gamma$ is a finite subset of $FIN$ and $\Delta$ is a finite subset of $\mathbf{N}$, $\Gamma \cup \Delta$ is unsatisfiable. As $\Delta$ is a finite subset of $\mathbf{N}$, there is some greatest $\mathbf{n}\in\Delta$; in any model $\mathscr{A}$ with a domain of size greater than $n$, $\Delta$ is satisfiable in $\mathscr{A}$. Since $\Gamma\cup\Delta$ is unsatisfiable, $\Gamma$ is unsatisfiable in any such model $\mathscr{A}$, so there is a model with a finite domain (any such $\mathscr{A}$) in which $FIN$ is unsatisfiable, contrary to assumption. So there is no such $FIN$.}\end{proof}\end{theorem}


\paragraph{L\"owenheim-Skolem Theorem}

\begin{theorem}[The L\"owenheim-Skolem Theorem]
	If $\Gamma$ is a set of sentences of \lequ\ which has an infinite model $\mathscr{A}$ (of size $\omega_{\alpha}$), then \begin{description}
			\item [Upward] $\Gamma$ has a model for every infinite size $\omega_{\beta} > \omega_{\alpha}$.
			\item [Downward] $\Gamma$ has a countable model.
	\end{description} \begin{proof}
		{\emph{Upward:} Extend the language for $\Gamma$ to $\mathcal{L}^{+}$ by adding new constants $C$ such that size of $C$ is $\omega_{\beta}$.  Consider the set of $\mathcal{L}^{+}$ sentences $\Gamma^{*} = \Gamma \cup {c_{i}\neq c_{j}: i\neq j, c_{i},c_{j}\in C}$. Since $\Gamma$ is satisfiable, every finite subset of $\Gamma^{*}$ is satisfiable – by compactness (for this uncountable language!), $\Gamma^{*}$ has a model, and the model must have cardinality $\omega_{\beta}$.
	
		\emph{Downward:} too complex to be covered here, unfortunately – see, e.g., \citet[chs.\ 12--13]{bbjcomlo}.}
			\end{proof}
\end{theorem}


% {
% \paragraph{Downward L\"owenheim-Skolem Theorem}
% 
% An \lequ-sentence is in \emph{Skolem Normal Form} (SNF) iff it is a prenex sentence with only universal quantifiers. \begin{lemma}[Skolemization]
% 	Every sentence of \lequ\ is equisatisfiable with a SNF sentence. \begin{proof}
% 		For every sentence of \lequ\ $\phi$, there is a $\exists\forall$ sentence $\psi$ such that $\vDash\phi$ iff $\vDash\psi$ (exercise). By many uses of $\exists$Elim, there is a $\psi'$ that is satisfiable iff $\psi$ is. $\psi'$ is the \emph{Skolemisation} of $\phi$.
% 	\end{proof}
% \end{lemma}
% \begin{theorem}[Downward L\"owenheim-Skolem]
% 	\begin{proof}
% 		{\em $\Gamma$ is satisfiable. Let $\Gamma'$ be the Skolemisation of $\Gamma$. Then $\Gamma'$ is satisfiable. Proof to follow that $\Gamma'$ and thus $\Gamma$ has a countable model.}
% 	\end{proof}
% \end{theorem}
% 
% 
\paragraph{Overspill}

\begin{theorem}[Overspill]
	If a set of sentences has an arbitrarily large finite model, it has a countable model. \begin{proof}
		{Let $\Gamma$ have arbitrarily large finite models. Let $\Gamma' = \Gamma \cup \mathbf{N}$. Any finite subset of $\Gamma'$ is a subset of $\Gamma\cup\Delta$, where $\Delta$ is a finite subset of $\mathbf{N}$, with a greatest $\mathbf{n} \in \Delta$. Since $\Gamma$ has arbitrarily large models, $\Gamma\cup\Delta$ has a model. But then every finite subset of $\Gamma'$ has a model; by Compactness, $\Gamma'$ has a model. And by the downward L\"owenheim-Skolem theorem, it has a countable model \citep[147]{bbjcomlo}.)
		}
	\end{proof}
\end{theorem}

\paragraph{Skolem's Paradox}

The L\"owenheim-Skolem theorem says that no first-order theory can restrict itself to models of one infinite cardinality.  But set theory is a  first-order theory: the L\"owenheim-Skolem theorm says it has a (presuambly unintended) countable model, but set theory itself entails that there are uncountable sets! This is \emph{Skolem's paradox}.

The resolution of the `paradox' is to note that set theory \begin{quotation}
	 only ``says'' that [some set] $S$ is nondenumerable in a \emph{relative} sense: the sense that the members of $S$ cannot 
	be put in one-to-one correspondence with a subset of $\mathbb{N}$ by any $R$ \emph{in the model}. 
	A set $S$ can be ``nondenumerable'' in this \emph{relative} sense and yet be denumerable 
	``in reality''. This happens when there \emph{are} one-to-one correspondences between 
	$S$ and $\mathbb{N}$ but all of them lie outside the given model. What is a ``countable'' set 
	from the point of view of one model may be an uncountable set from the point of 
	view of another model.  \citep[465]{putmodre}
\end{quotation}



% {
% \paragraph{}
% 
% The key here is the fact that "There is no weakest Axiom of Infinity" is a
% corollary of Trakhtenbrot's theorem that the set of first-order formulas
% valid in FINITE models (i.e. whose negations are not FINITELY satisfiable)
% is undecidable. This follows from G\"{o}del 1931. There is no complete
% axiomatization of the Pi-1-1 sentences of arithmetic (i.e., the set of
% Pi-1-1 truths is not r.e.). A proper initial segment of the natural
% numbers, however, can be thought of as a finite model. So, (*), if finite
% satisfiability was decidable, we'd have have a way of proving arbitrary
% Pi-1-1 truths of arithmetic: just carry out the decision procedure and note
% that no finite model knocks out the candidate sentence! (The fiddly bit,
% (*), has to do with a bit of change to the sentence: finite segments of
% omega are not closed under addition and multiplication, so what we would
% have to test in finite models was a variant: put a bound on the initial
% universal quantifiers of the Pi-1-1 sentence (using a new constant), and
% check that the bounded sentence holds in models that go enough higher than
% the bound to include denotations for all the terms appearing.)
% }

\section{Further Directions of Research and Study in Logic}

We could now proceed to extend, or alter our logic, in quite a few ways:
\begin{description}
	\item [Enriching] We could add still further to \lequ\ or \lone, introducing \emph{modal operators} like `necessarily' and `possibly', \emph{temporal operators} like `it was the case that', or even moral operators like `it ought to be that', or stronger conditionals like the \emph{counterfactual} `If it had been the case that $\phi$, it would have been the case that $\psi$'. Then we can prove results about these enriched systems.
	\item [Altering]  Some complain that $\neg\neg \phi \not \vDash \phi$; these \emph{intuitionists} want to revise our logic. \emph{Free logic} permits constants to lack a designation in a model, and \emph{universally free logic} abandons the requirement that the domain be non-empty; these of course will also revise our logic as given.
	\item [Advancing] We could augment our languages in another way too, perhaps introducing \emph{function symbols} – operators on terms that yield terms – like $+$ or $\times$. We can then formalise languages of interest – most famously \emph{arithmetic}, in the \emph{Peano axioms}, and prove various things about them. In this way we can build up to one of the most celebrated results of twentieth century logic: \begin{theorem}[G\"odel Incompleteness Theorems]
	If $T$ is a consistent, finitely axiomatisable theory able to express elementary arithmetic, then \begin{itemize}
		\item $T$ cannot prove every arithmetical truth;
		\item $T$ cannot prove its own consistency (i.e., cannot prove an arithmetical sentence that is true iff $T$ is consistent). (This second result obviously entails the first given that $T$ is consistent.)
	\end{itemize}
\end{theorem}
The proof is sadly beyond us. 
\end{description}

In the next chapter, we will look at some of the philosophical issues arising for the logic we have introduced. In particular, we will look at how well the operator $\to$ matches the English conditional, and how well the semantics of designators and relations for \ltwo\ matches  similar constructions in English. We'll also reexamine the sequent $\phi,\neg\phi\vdash\psi$.

{\small
\subsection*{Further Reading}
\addcontentsline{toc}{subsection}{Further Reading}

Russell's famous paper on definite descriptions is \citet{rusonde}; the nearly as famous response by Strawson is \citet{stronre}.  A good overall source is \citet{sep-descriptions}. 

 One much-discussed philosophical analysis of the consequences of the L\"owenheim-Skolem theorem is \citet{putmodre}. This article is very controversial! One response is \citet{lewputpa}.

On modal, temporal, etc., logics, a good treatment is by \citet{bevpospa}; a very elegant recent examination of these topics an the topicof non-classial logic generally is \citealt{burphilo}.  Free logics are treated by \citet[\S\S 8.4--8.7]{bosintlo}, and \citet{lamfrelo}.


 A good text on the further development of metamathematics is \citealt{bbjcomlo}.





\subsection*{Exercises}
\addcontentsline{toc}{subsection}{Exercises}


\begin{enumerate}
\item Prove the theorem called `Substitution of co-designating constants' from page \pageref{scdc}.
\item Prove the following \begin{enumerate}
	\item $\vDash \forall x x=x$;
		\item $\vDash \forall x \forall y (x=y \to y=x)$;
		\item $\vDash\forall x \forall y \forall z ((x=y \wedge y=z) \to x=z)$.
\end{enumerate}	
\item \begin{enumerate}
	\item Show that in the presence of {=}Elim, $\exists$Elim and $\exists$Intro, the rule {=}Intro can be replaced by this rule (where $\tau$ is a constant): \begin{equation*}
		\begin{prooftree}
			[\exists \upsilon \upsilon = \tau] \justifies \vdots \using {=}\text{Intro\dag} 
		\end{prooftree}
	\end{equation*}
\item What would be the effect of adopting, instead of our rule =Intro, this rule (in the presence of the other rules of $ND_{2}$): \begin{equation*}
	\begin{prooftree}
	\phi	\justifies \tau = \tau \using \text{=Intro*}
	\end{prooftree}
\end{equation*}
\item Can we replace both {=}Intro and {=}Elim by this pair of rules, in the presence of the other rules of $ND_{2}$ ($\phi$ is a formula in which at most $\upsilon$ occurs free): \begin{center}
	\begin{tabular}{cc}
		\begin{prooftree}
			\phi[\tau/\upsilon] \justifies \exists \upsilon (\upsilon=\tau \wedge \phi) \using {=}\text{Intro\ddag}
		\end{prooftree}
		&
		\begin{prooftree}
		 \exists \upsilon (\upsilon=\tau \wedge \phi) \justifies \phi[\tau/\upsilon]  \using {=}\text{Elim\ddag}
		\end{prooftree}
	\end{tabular}
\end{center}
\end{enumerate}
\item Suppose $\phi$ is a formula in which at most $\upsilon$ occurs free, and in which $\nu$ does not occur at all.  \begin{enumerate}
	\item Show that $\vDash \forall \upsilon \forall \nu (\upsilon=\nu \to \phi \bicond \phi[\nu/\upsilon])$. 
	\item Suppose for \emph{every} formula $\phi$ with just $\upsilon$ free, not itself involving identity, and any constants $\tau$, $\kappa$, we could show that $\val{\phi[\tau/\upsilon]}{A} = \val{\phi[\kappa/\upsilon]}{A}$. Could we then show that $\val{\tau=\kappa}{A}=T$? (This is intended to mimic as best we can the second-order quantification over properties in the definition of identity: the question is, does it work in the right way?) 
\end{enumerate}
\item \begin{enumerate}
\item  Show that  $\exists_{n+1}x Px \equiv \exists x (Px \wedge \exists_{n} y (Py \wedge \neg x=y))$.
	\item Define $\forall_{\tilde{n}}x\phi$ as $\neg\exists_{n}x\neg\phi$. Give an English translation of $\forall_{\tilde{n}}xPx$.
\end{enumerate} 

\item In the analysis of definite descriptions, $\exists \upsilon (\psi \wedge (\forall \nu \phi[\nu/\upsilon] \bicond \upsilon=\nu))$,  could we replace the `$\bicond$' by `$\to$'? If not, why? What about in this formulation: $\exists \upsilon (\psi \wedge (\forall \nu \upsilon=\nu \bicond \phi[\nu/\upsilon]))$?
\item Give appropriate analyses in \lequ\ of these claims, being sure to point out any cases of ambiguity, and commenting on other issues: \begin{enumerate}
	\item The number of planets is necessarily eight.
	\item The prime minister must be an honest person.
	\item There used to be  a Labor prime minister.
\end{enumerate}
\item What is the best way to formalise `$x$ exists' in \lequ? What are the difficulties with a Russellian analysis of (\P) `The present king of France does not exist'?
\item Can the Russellian analysis of descriptions deal with the following sentences? \begin{enumerate}
	\item `The beer is all gone.'
	\item `He is tall, handsome, and the love of my life.'
	\item `She is the proud owner of a Rolls Royce.'
\end{enumerate}
\item Can we use identity and quantification to express `there are exactly as many $P$s as $Q$s'?
\item \begin{enumerate}\item Show that the infinite set of sentences $\mathbf{N}$ is satisfied in every infinite domain.
\item Show that there is no single \lequ-sentence which is satisfied in $\mathscr{A}$ iff $\mathbf{N}$ is satisfied in $\mathscr{A}$.
\end{enumerate}
\item A \emph{permutation} on $S$ is a one-one function $\pi$ from $S$ to $S$. If $\mathscr{A}$ is a model, where $\pi$ is a function from the domain of $\mathscr{A}$ to itself, let the permutation $\pi(\mathscr{A})$ be this model \begin{itemize}
	\item $D_{\mathscr{A}}=D_{\pi(\mathscr{A})}$.
	\item For any term $\tau$, $I_{\pi(\mathscr{A})}(\tau) = \pi(I_{\mathscr{A}}(\tau)$.
	\item For any $n$-ary predicate $\Phi$, $$I_{\pi(\mathscr{A})}(\Phi) = \{\langle \pi(x_{1}),\ldots,\pi(x_{n})\rangle:\langle x_{1},\ldots, x_{n}\rangle \in I_{\mathscr{A}}(\Phi)\}.$$
\end{itemize}  \begin{enumerate}
	\item Show the \emph{Permutation Theorem}: for any sentence $\phi$, and permutation $\pi$, $\val{\phi}{A} = \val{\phi}{\pi(A)}$.
	\item Consider this passage: \begin{quotation}
		the Permutation Theorem shows that, even if we can somehow discount the problems raised by the [Skolem's paradox and the L\"owenheim-Skolem Theorem], and settle on models with one fixed domain as those which can truly claim to be abstractions of the world – even then, there are as many different reference relations holding between the subsentential expressions of $T$'s language and the world as there are nontrivial permutations of the fixed domain, all of which underwrite the truth of exactly the same sentences, and, hence, equally subserve the truth of the theses of $T$. There being no basis for preference between these reference relations, realism's doctrine that there is a unique, privileged such relation is apparently discredited. \citep[53--4]{taymodtrr}
	\end{quotation}Assess and evaluate this argument. Do you accept the conclusion? If not, why not?
\end{enumerate}
 \end{enumerate}


}






\newpage
\chapter[Logic and Natural Language]{Logic and Natural Language: Conditionals, Entailment, Designators, and Relations}\label{c9}


\section{Conditionals}

\paragraph{`If $\phi$, $\psi$' and $\to$}
So far we have rendered the English conditional construction `If $\phi$, $\psi$' using the \emph{material conditional} $\to$. You may be worried, as others have worried before you, that this is not an accurate rendering. For, you may think, it is implausible that `If $\phi$, $\psi$' is true whenever $\psi$ is true. Consider: 
\begin{quote}
	\begin{exe}
	\ex The Tigers  will not lose every game this season;
	\ex The Tigers will make it to the finals;
	\ex Therefore: If the Tigers lose every game this season, they will make it to the finals.
 \end{exe}
\end{quote}
If `if $\phi$, $\psi$' is $\phi \to \psi$, this argument should be valid (it is an instance of the valid form $\phi,\psi \vDash \neg\phi \to \psi$). But this argument does not strike us as valid. 

\paragraph{Can we do better in our logic?}
The first question we should address is: is there a better option in \lone? And it seems that there is not. Suppose that `If $\phi$, $\psi$' is expressed by some truth function $f(|\phi|,|\psi|)$. One thing is definitely true; if $|\phi|=T$ and $|\psi|=F$, we want $f(|\phi|,|\psi|)=F$. This leaves us with 8 possible two-place truth-functions. Another thing we want is \emph{asymmetry}: there exists at least one structure $\mathscr{A}$ where $f(|\phi|_{\mathscr{A}},|\psi|_{\mathscr{A}})\neq f(|\psi|_{\mathscr{A}},|\phi|_{\mathscr{A}})$. This rules out four further functions (it rules out all those where $f(F,T)=F$), leaving us with four. Since we've just suggested in our discussion of the boxed argument above that it is undesirable for the truth of a conditional to follow from the truth of the consequent or the falsity of the antecedent, it would surely be even worse if the conditional were \emph{equivalent} to the consequent or the negation of the antecedent. This rules out two of our remaining truth functions, leaving us with just two: $\to$ and a function $g$ (expressed by $\neg(\psi\to\phi)$). But $g(T,T)=F$, hardly what we want from a conditional. So the only truth-function which satisfies minimal conditions necessary to count as a conditional is $\to$.

\paragraph{Meaning as characterised by proof rules}
Of course, this is not a strong argument – that $\to$ is better than other candidates is no argument that $\to$ is itself adequate! But we can offer better considerations. Given that our proof system for \lone\ is sound and complete, there is a sense in which those rules – collectively – govern or fix the meaning of the connectives. No other rules are necessary to ensure the right things are provable, the things which correspond exactly to the meaning of the connectives.

Things may not be much different in natural languages. Of course we don't have soundness and completeness proofs, because we have neither a fully spelled out semantics nor a formal proof system for natural language. But we do have characteristic patterns of inference for the connectives of English. Some of these patterns of inference look remarkably like those for the formal connectives of \lone, notably the rules governing `and'. (This is some motivation behind the nomenclature of `natural deduction' – it is natural in the sense that it formalises intuitively fundamental characteristic patterns of inference for the corresponding operators.)

Whatever else we might want to say about the English conditional connective `if $\phi$, $\psi$', it seems it should obey these rules: \begin{description}
	\item [\emph{Modus Ponens}]  If you have established $\phi$, and you have established  `if $\phi$, $\psi$', then you can on that basis establish $\psi$.
	\item [Conditional Proof] If $\psi$ can be established, conditional on the assumption that $\phi$, then you can establish `if $\phi$, $\psi$' without need of that assumption. 
\end{description}
It is obvious that something which didn't obey these rules would be quite unlike a conditional. A conditional helps us neatly summarise reasoning from assumptions, store it, and then use it when those assumptions come true – conditional proof captures the first part, and \emph{modus ponens} the last.
But if, as seems very plausible, these two rules capture the English conditional, then we can offer an argument that the English conditional is $\to$. More precisely, the argument is that ‘If $\phi$, $\psi$’ is true iff $\phi \to \psi$ is:  \begin{quote}
	\emph{Only If}: Suppose that if $\phi$, $\psi$. Assume $\phi$. By \emph{modus ponens}, $\psi$. By $\to$Intro, $\phi\to\psi$, discharging the assumption.

\emph{If}: Suppose that $\phi \to \psi$. This is equivalent to $\neg \phi \vee \psi$. Assume $\phi$. By disjunctive syllogism, $\psi$. By conditional proof, it follows that if $\phi$, $\psi$, without need of the assumption.
\end{quote}

`Disjunctive syllogism' is the rule that from $\phi \vee \psi$ and $\neg \phi$ one can infer $\psi$. (A problem asks you to show that this is valid derived rule in the system \lone; it is obviously a good rule for the English `or'.) 

\paragraph{Replying to the apparent counterexamples – Grice}
If this argument works, we're wrong to think that the cases where $\to$ doesn't seem to behave like `If\ldots then' are genuine counterexamples. But we need to say something about those cases.

The explanation offered is that, even though these sentences are true, they \emph{conversationally implicate} something false, and that is what we are responding to. The notion of conversational implicature was introduced by Grice, who noted that many utterance seem to communicate more than what is strictly said, and gave a series of principles that he argued govern what is communicated by an utterance, based on what strictly it means \citep{grilogco}. One classic example is this: `some philosophers have beards'. An utterance of this sentence conversationally implicates that not all philosophers have beards, and that is what something most people will assume the speaker to be committed to in their utterance. Yet this is not a consequence of the utterance: `some philosophers have beards; in fact, all of them do!' can perfectly well be true, and is no contradiction. 

\paragraph{The Maxim of Quantity}
The principle Grice invokes to explain this implicature is his `Maxim of Quantity' – simply put, this is the principle that cooperative speakers be as informative as they can be. A hearer, without good reason to think otherwise, will assume that a speaker is being cooperative, and hence that when the speaker utters a claim, it is also the most informative claim the speaker is in a position to responsibly utter. One measure of informativeness is this: If $\phi$ entails $\psi$, then $\phi$ is at least as informative than $\psi$ (for it carries the information that $\psi$, and perhaps additional information too if $\psi$ does not entail $\phi$). Since `All philosophers have beards' entails `some philosophers have beards', someone who utters the latter utters something less informative than they could utter, if in fact they thought that all philosophers have beards. Since they are cooperative and did not utter the stronger claim, therefore, the hearer infers that the speaker did not believe the stronger claim. So the hearer now knows that the speaker must believe that some but not all philosophers have beards, and the speaker therefore communicates that not all philosophers have beards to the hearer.

\paragraph{Applying Quantity to conditionals}
  
The material conditional is equivalent to a disjunction $\neg \phi \vee \psi$. Any disjunct of a disjunction entails the disjunction, so contains more information. So an utterance of a disjunction implicates that neither disjunct is believed by the speaker. So if it is apparent that the speaker \emph{does} believe either disjunct – either through being committed to the consequent, or rejecting the antecedent, of the relevant material conditional – we should be struck by the fact that the speaker has communicated an apparent contradiction. In the case of the Tigers above, the speaker who utters both the premises and conclusion of this argument communicates that they believe the consequent, and also utters the conditional which conversationally implicates that the speaker does not believe the consequent. So the hearer has a contradiction communicated to them – little wonder, then, that these sentences sound so bad! But – and this is the crucial part – nothing in this explanation of the badness of the conditional is \emph{semantic}. Nothing in this explanation undermines the idea that the truth conditions of the English conditional construction are the same as those for the material conditional. This is all to do with what we standardly take ourselves to be able to infer from an utterance of a sentence with those semantics, and what goes wrong here is that inference, not the semantics.

\paragraph{Trickier Cases}

All is not smooth sailing for the material conditional account of `If $\phi$, $\psi$'. I mention two trickier cases: \begin{enumerate}
	\item There seem to be cases where conditional proof goes wrong – most noticeably, in the case of assumptions made that explicitly contradict other things we believe. This is especially apparent in so called \emph{counterfactual} conditionals – those (at least at first glance) that involve conditional reasoning about circumstances that do not obtain. So consider the (apparently true) counterfactual `If Univ had never existed, Exeter would have been the third oldest college'. The conditional proof rationale for this is: if we can derive the consequent from the assumption that Univ never existed and other things, then we can derive the conditional from those other things alone. Yet since Univ's existence is known to us, adding that assumption to our prior beliefs gives an inconsistent set of assumptions, from which any consequent whatever follows. But `If Univ had never existed, Catz would have been the third oldest college' is clearly false, though it should be derivable on exactly the same grounds. What we need to do, apparently, is modify which of our prior beliefs can be retained when reasoning from this contrary-to-fact assumption – there is a large literature on how best to do this, starting with \citet{lewcount}.
\item There seem to be cases where the Gricean explanation should make a conditional sound bad, but in which it does not. Consider: `You won't fail EDL. Even if you do, you will pass on the resit.' The first sentence of this little speech is a denial of the antecedent of the following conditional, so the conditional should sound bad. But it sounds fine. The word `even' may be held responsible; but it is now up to the Gricean to explain how it interferes with the ordinary Gricean mechanisms.
\end{enumerate}


\section{Entailment}

Another problematic feature of \lone\ (somewhat related to the foregoing by the deduction theorem) are the so-called \emph{paradoxes of entailment}. These are the following valid sequents: \begin{itemize}
	\item $\phi, \neg\phi \vDash \psi$;
	\item $\phi \vDash \psi \vee \neg \psi$.
\end{itemize}
But, some have objected, these are terrible inferences – we should never conclude, on the basis of contradictory premises, anything; and while it may be safe to conclude a tautology from any premises, that is hardly a good argument to persuade someone that the conclusion is a tautology. 

\paragraph{Relevance} Critics of these sequents have maintained that the failure here is one of \emph{relevance}. The arguments may be valid in the narrow sense, because there is no structure which satisfies the premises without also satisfying the conclusion (in the former case, because no structure satisfies the premises; in the latter, because every structure satisfies the conclusion). But validity in this narrow sense is too narrow – what we ordinarily understand to be a successful valid argument excludes these trivial cases of validity. The conclusion in a \emph{genuinely} valid argument – these critics say – should follow from, and be relevant to, the premises.

\paragraph{The Lewis Argument} Yet there is a very simple argument, using ordinary English reasoning which is apparently impeccable, to support these `problematic' sequents. It is known as the \emph{Lewis argument}, after C. I. Lewis, one of its modern rediscoverers. 
\begin{quote}
	\begin{exe}
	\ex $\phi$ \hfill Assumption
	\ex $\phi$ or $\psi$ \hfill Disjunction introduction, 1
	\ex $\neg \phi$ \hfill Assumption 
	\ex $\psi$ \hfill  Disjunctive syllogism, 2, 3
\end{exe}
\end{quote}
This argument seems to show that, even ordinarily, this form of apparently irrelevant argument is valid. And who could object to disjunction introduction or disjunctive syllogism, which seem like about the most basic things one can safely say about disjunction.

\paragraph{The relevantist response – reject Disjunctive Syllogism} The relevantist has a response to this argument – reject disjunctive syllogism. For, they say, it goes wrong in precisely this circumstance. The inference works normally because, given a true disjunction with one false disjunct, we can infer to the truth of the other disjunct \emph{without} checking to see whether it is true. (Disjunctive syllogism was called `the dog' by its Stoic inventors, because `even a dog uses this form of inference when it comes to a fork in the road, sniffs down one branch, and not finding the scent there immediately takes off down the other branch, without stopping to sniff' \citep[99--100]{burphilo}.) But in this case, $\phi$  is also an assumption – so the fact that $\phi \vee \psi$ has a disjunct whose denial is an assumption doesn't entail $\psi$. If there are cases where we can have inconsistent assumptions, like this, we should expect disjunctive syllogism to go wrong.

\paragraph{Dialethism} Some truly radical support for this line of argument can be found from the \emph{dialethist} position that some sentences are \emph{both} true and false. If $\phi$ is one such sentence, then both assumptions can be true ($\phi$ because it is true, $\neg\phi$ because $\phi$ is false). But $\psi$, since it is arbitrary, can be chosen to be a plain falsehood, giving a situation  which is a counterexample to disjunctive syllogism. Dialethists motivate their non-standard semantics by appeal to particular problem cases that don't seem to admit of any other solution. The best known is \begin{description}
	\item [L] L is false.
\end{description} The sentence L displayed above says of itself that it is false. If it is true, then L is false (as that is what it says). But if it is false, then what it says is false, so it must be true. So L is true iff it is false. The dialethist takes this proof at face value – L is both true and false. But, despite the great difficulties more standard approaches to the Liar paradox L have, few have found dialethism compelling (though it has surprisingly able defenders). 

 \paragraph{Inconsistent belief} Dialethism is too incredible to convincingly motivate the rejection of disjunctive syllogism. A weaker motivation is found in reasoning about what we believe. If you are like me, you probably have some inconsistent beliefs that you are not aware of. It is absurd to say that I therefore believe everything, just because I have beliefs that entail everything. So it seems, the set of my beliefs is not plausibly closed under classical entailment (that is, the set of my beliefs does not contain everything which is  entailed by some things in my set of beliefs). Disjunctive syllogism seems to fail for my beliefs; just because $\phi\vee\psi$ and $\neg\phi$ are among my beliefs does not mean that $\psi$ is one of my beliefs; for $\phi$ may be one of my beliefs as well. The logic of belief, it may be said, is relevant. (The original application of this kind of idea was to simple inference-drawing computers – if they are not aware of when the body of information input contains a contradiction, you should not allow them to draw conclusions using full classical logic.) The problem with all this is that it doesn't really seem like logic any more. There are no counterexamples to disjunctive syllogism, that is, structures where $\phi\vee\psi$ and $\neg\phi$ are true but $\psi$ is false. Rather, there are cases where \emph{according to my beliefs} $\phi \vee \psi$ and $\neg\phi$ are true, and also \emph{according to my beliefs} $\psi$ is not true. But it is already well-known that, when we have a non-truth-functional operator like `according to my beliefs', it won't generally be true that classically valid sequents remain correct when each sentence in the sequent is in the scope of the operator. (Think of the non-truth-functional operator `possibly': $\phi, \psi \vDash \phi \wedge \psi$ is correct, `possibly $\phi$, possibly $\psi \vDash$ possibly $\phi \wedge \psi$' is not correct – think of the case where $\psi$ = `the coin toss lands heads', $\phi$ = `the coin toss lands tails'.) 



\section{Designators}
\paragraph{Designators in \ltwo\ and English}

The formal properties of \ltwo\ have been established; now we begin a relatively informal investigation into the possible connections between the semantics of \ltwo\ and the semantics of English.
We begin in this section with a discussion of \emph{designation} in both languages.

\paragraph{Direct Reference}

\ltwo\ embodies a particularly simple view of the meaning of a constant.
 \begin{definition}[Direct Reference]
 	A referring expression \emph{directly refers} iff the meaning of the name is just its referent, the object denoted by the expression.
 \end{definition}	
Constants in \ltwo\ directly refer, because the meaning (semantic value) of $a$ in $\mathscr{A}$ is $\val{a}{A}$, which is just an item in the domain of $\mathscr{A}$.


This is to be contrasted with any thesis of \emph{indirect} reference, according to which the semantic value of a denoting expression is something other than the referent. The primary example here are definite descriptions; if Russell's account (recall chapter \ref{c8}) is correct, for example,  the meaning of a description is somehow to be extracted from a complex quantified sentence; the referent (if any) is determined by the meaning, but is not the meaning. This can clearly be seen if we accept Russell's treatment of ‘The present king of France is bald’, where the sentence is meaningful even though the definite description fails to denote anything. Given that the meaning of the sentence is determined by the meanings of its constituents, the definite description noun phrase must have a meaning, even though it lacks a referent, and hence cannot be a directly referential expression.



\paragraph{Rigid Designation} Direct reference is related to, but distinct from, the following notion:
\begin{definition}[Rigid Designator]
	A referring expression $a$ is a \emph{rigid designator} iff the referent of $a$ is the same object in every possible situation (possible world) in which that object exists.\footnote{Be careful not to confuse this with the claim that every possible use of ‘$a$’ refers to the same thing – this would be false. The idea is rather, when we actually evaluate the possibility of various things about $a$ in some possible world $w$, we continue to hold fixed the actual meaning of $a$, rather than (for example) what the residents of $w$ themselves would mean by their own utterance of ‘$a$’. Suppose we consider the possibility in which Aristotle hadn't been called ’Aristotle’: we are still talking about the person actually called ‘Aristotle’, even though residents of that possibility do not use that name to denote anyone.}
\end{definition}
It may seem that this doesn't hold in \ltwo, because if $\mathscr{A}\neq\mathscr{B}$, then it may be that  $I_{\mathscr{A}}(a)\neq I_{\mathscr{B}}(a)$. But for all that it may hold; the crucial fact to recognise is whether, \emph{in a given situation}, when we consider what is possible for $a$, we consider what is possible for what $a$ refers to – in every situation in which that thing exists. Since \ltwo\ has no resources for discussing possibility, it is a moot point whether or not rigid designation holds for it. 

If an expression is directly referential, in a language with resources for discussing possibility and necessity, then it looks like it will also be a rigid designator. Assume that we evaluate ‘Possibly, $F(a)$’ by seeing whether there is a possible world in which $F(a)$. If $a$ directly refers, the evaluation of ‘Possibly $F(a)$’ will proceed by evaluating whether actual meaning of $a$ – which is just some object $\mathbf{a}$, given direct reference – satisfies $F$ at each possibility in which $\mathbf{a}$ exists. So the actual meaning of $a$, evaluated at each world, will remain constant.

The classic example of a non-rigid designator is also a simple definite description, like ‘the tallest person’. Suppose Jacques is the tallest person. Still, I can perfectly well say ‘Possibly, Gill is the tallest person’, which is \emph{not} synonymous with ‘Possibly, Gill is Jacques’. At the possible world in which Gill is the tallest person, the expression ‘the tallest person’, with its actual meaning, denotes her, though it doesn't denote her actually. 

Things are complicated by the existence of \emph{rigidified descriptions}. As the name suggests, they are rigid designators, but they are also descriptions which refer indirectly. Consider ‘the actual tallest person’: intuitively, the meaning of this description evaluated at every possibility determines it to have a constant referent, of Jacques. Because of this, ‘Possibly, Gill is the actual tallest person’ does seem to entail that possibly, Gill is Jacques. 
	


\paragraph{Names and Direct Reference in English}

For names in English, like `Antony Eagle' or `Aristotle', there is no straightforward semantics for names as there is for constants in \ltwo.

On the one hand, there is considerable syntactic and semantic evidence that names in English function rather like the constants of \ltwo. Consider \begin{exe}
	\ex Antony Eagle is lecturing now.\label{s}
\end{exe}  \ref{s} is true  iff the thing \textbf{Antony Eagle}, designated by `Antony Eagle', has the property \textbf{lecturing} denoted by `is lecturing', \textbf{now}, i.e., at the time designated by (this use of) `now'. These seem remarkably like the clauses for the satisfaction of atomic formulae of \ltwo; in that sense it looks like the meaning of `Antony Eagle' is just the thing it refers to.

\paragraph{Objection to Direct Reference: Informative Identities}
 If `Hesperus' denotes Hesperus, and `Phosphorus' denotes Phosphorus, then, since Hesperus \emph{is} Phosphorus (both are names for Venus), then the meanings of `Hesperus' and `Phosphorus' are the same. So then why aren't these sentences synonymous?: \begin{exe}
		\ex Hesperus is Hesperus.
		\ex Hesperus is Phosphorus.
	\end{exe}



This objection trades on the fact that we receive information when told the second claim, but we do not when told the first; hence they must yield different information, and not be synonymous after all. One possible response is to say that \emph{for all we know}, we are in a situation where `Phosphorus' names something other than Venus; and what we are told by the second sentence is a contingent fact about English, namely, that `Phosphorus' means Phosphorus!

\paragraph{Objection to Direct Reference: Empty Names}
If the meaning of a name is its referent, then `Santa Claus is jolly' is meaningless, yet we judge it to be true. And don't say it is true `in a fiction' – for fictional English is the same as non-fictional English, so if it is meaningless outside of the fiction, it is just as meaningless inside. But set that aside: \emph{negated existentials} show that `Santa Claus' is meaningful, because `Santa Claus does not exist' is straightforwardly true.



To be more explicit: `Santa Claus does not exist' apparently expresses a truth. This can be rendered in the language \lequ\ by this translation: `$\neg \exists x (x= a)$', where `$a$' denotes Santa Claus. But since `$a=a$' expresses a logical truth, it follows that `$\exists x (x = a)$' expresses a logical truth of \lequ, denying the apparent truth which we translated. But this looks like a problem in English, as well as in formal languages: since `Santa Claus is identical to Santa Claus' also expresses a truth, the English claim `Something is identical to Santa Claus' is also true. But if something is identical to Santa Claus, Santa Claus does exist. To deny this is to deny claims that seem tautologous in English. 

This is a very serious objection to direct reference theories. Many have chosen to bite the bullet, and accept that such sentences are in fact meaningless. Others have chosen to accept some secondary notion of meaning, like a canonical description associated with the empty name, to play a name-like role in these kind of claims. But there is no consensus on how to deal with this problem.

\paragraph{Objection to Direct Reference: The Role of Descriptions}
Many names are simply introduced by descriptions; and certainly what most of us aim to communicate when we use a name is a certain individual satisfying a certain canonical description that the hearer is familiar with – if there were no such description, `how do people ever use names to refer to things at all?' \citep[28]{krinamne}. 

\begin{description}
	\item [Reference Fixing] Some descriptions are used to establish the reference: `Julius' may be introduced into the language as `the inventor of the zipper'. But it is no part of the meaning: consider the counterfactual situation where Julius gets pipped by his rival.
	\item [Background knowledge] If one knows a lot about Julius, and one's hearers do too, then one will expect that background knowledge to be useable when one uses the name `Julius'. It may be difficult if not impossible for a speaker themselves to separate the meaning of `Julius' from what they know about Julius.
\end{description}

\paragraph{Opaque Contexts}

 Consider \begin{itemize}
	\item Lois Lane believes that Superman is Superman;
	\item Lois Lane believes that Superman is Clark Kent.
\end{itemize}
It is fairly clear that in the context `Lois Lane beleives that Superman is $x$', one cannot straightforwardly substitute co-referring expressions into $x$ and get a true sentence. Such \emph{opaque contexts} – those which do not permit substitution \emph{salva veritate} – are a problem for the direct reference theory, of course, but also for other theories that think the objects of belief are propositions concerning the entities one has the belief about. 

Obviously \ltwo\ lacks opaque contexts, as theorems about substitution of co-referring constants proved last time show. This is an expressive weakness of \ltwo\ compared with English.



\paragraph{Other Designators in English}

There are several other natural language expressions other than names which have a designator role in English. \begin{description}
	\item [Indexicals] such as `I', `you', `here', `now', `actually'. This class may also include \emph{demonstratives} like `this', `that'.
	\item [Count Nouns] such as `chair', `boy', `piece of furniture'.
	\item [Non-Count Nouns] such as `gold', `dirt', `intelligence'.
	\item [Descriptions] such as `the inventor of the zipper', `the president of the United States'. 
\end{description}

Count nouns can be handled by the apparatus of quantification in \ltwo, rather than as designating expressions directly. Descriptions we talked about  more fully in Chapter 7.

\paragraph{Indexicals}

The characteristic of an indexical expression is \emph{context sensitivity}. `I' refers to different people when uttered by different people: to me when I say `I am tired', to you if you utter that same string of words. Similarly with other expressions in this category, as they are sensitive to features of context – where the context of an utterance includes the speaker, time, place, etc., of the utterance – and thus may have different referents in different contexts.

Despite the context-sensitivity of indexical \emph{reference}, their meaning appears constant. Distinguish (following Kaplan):  \begin{description}
	\item [Content] The content of an expression is its semantic value;
	\item [Character] The character of an expression is a function from context to content.
\end{description} The thesis is that indexicals have a constant character, but a variable content; and character is plausibly a kind of meaning.

Pronouns – `she', `they', `it' – may perhaps be analysed similarly.




\paragraph{Non-Count Nouns}

Non-count nouns are a certain type of \emph{non-plural} referring expression. One reasonable test is whether `$\mu$'  can be grammatically substituted into the expression `I want some $\mu$'. 
 These fall into at least two kinds: \begin{description}
	\item [Mass] Mass nouns, like `gold' or `dirt', denote a kind of stuff or substance, rather than a particular thing. They typically display \emph{cumulative reference}: if `being gold' is true of $a$ and $b$, then it is true of the \emph{sum} $a+b$ which has just $a$ and $b$ as parts.
	\item [Abstract] Abstract nouns, like `intelligence' or `generosity', which seem to denote \emph{qualities} that something may possess, but can themselves have things predicated of them.  
\end{description}
A full treatment of these nouns cannot be carried out in \ltwo; constants referring to qualities, that we naturally translate as predicates like `being gold', may require a \emph{second order} logic, which permits quantification over predicate variables as well as individual variables. 





\section{Binary Relations}




\paragraph{Binary Relations}

Binary relations – any subset of $D_{\mathscr{A}}\times D_{\mathscr{A}}$ – have many interesting properties. We can often represent these properties \emph{graphically}, if the domain on which the relation is defined is finite.

\begin{definition}[Finite Graph]
	A \emph{finite graph} on $D_{\mathscr{A}}$ is a pair $\langle V, E\rangle$ where the set of \emph{vertices} $V$ is $D_{\mathscr{A}}$, $D_{\mathscr{A}}$ is finite, and $E$ is a set of \emph{edges}, directed `arrows' from members of $V$ to members of $V$.
\end{definition} A graph on $D_{\mathscr{A}}$ \emph{corresponds to} a binary relation $R$ on $D_{\mathscr{A}}$ just in case \begin{enumerate}
	\item If $\langle x,y\rangle \in R$, then $x\in V$ and $y \in V$.
	\item If $\langle x,y\rangle \in R$, then there is an arrow from $x$ to $y$ in $E$.
\end{enumerate}

\paragraph{Reflexivity and Transitivity Represented Graphically}

Consider the domain $D_{\mathscr{A}}=\{1,2,3,4\}$.

\begin{definition}[Reflexive]
	A relation on $D_{\mathscr{A}}$ is \emph{reflexive} iff for all $x\in D_{\mathscr{A}}$, $\langle x,x\rangle\in R$. 
\end{definition}Graphically, that means, as in the figure on the left in Figure \ref{fone}, that each vertex has an arrow pointing to itself. A relation is \emph{irreflexive} iff no vertex has an arrow pointing to itself.

\begin{definition}[Transitive]
	A relation on $D_{\mathscr{A}}$ is \emph{transitive} iff for all $x,y,z\in D_{\mathscr{A}}$, if both $\langle x,y\rangle\in R$ and $\langle y,z\rangle\in R$, then also $\langle x,z\rangle\in R$.
\end{definition} Graphically, that means, as in the figure on the right in Figure \ref{fone}, that whenever there is an indirect sequence of arrows between two vertices, there is also a `shortcut'. A relation is \emph{intransitive} iff there is never a shortcut.

\begin{figure}
\begin{center}
	{~\xymatrix{1 \ar@(ur,dr)[lr]  & 2 \ar@(ur,dr)[lr] \\
	3 \ar@(ur,dr)[lr] & 4 \ar@(ur,dr)[lr] }\qquad\qquad}{\xymatrix{1  & 2 \ar[l] \\ 3 \ar[ur]
		\ar[u] & 4 \ar[l] \ar[ul] \ar[u] 
		}}
\end{center}	\caption{Reflexivity and Transitivity\label{fone}}
\end{figure}


\paragraph{Symmetry Represented Graphically}

\begin{definition}[Symmetric]
	A relation $R$ on $D_{\mathscr{A}}$ is \emph{symmetric} iff whenever $\langle x,y\rangle \in R$, then $\langle y,x\rangle \in R$.
\end{definition} Graphically, this means (as on the left in Figure \ref{ftwo}) that whenever there is an arrow from one vertex to another, there will be a return arrow. A relation is \emph{asymmetric} iff there are no returning arrows at all.

\begin{definition}[Anti-symmetric]
	A relation is \emph{anti-symmetric} iff whenever $\langle x,y\rangle \in R$ and $\langle y,x\rangle \in R$, then $x=y$ (sometimes this is known as \emph{weak asymmetry}).
\end{definition} Graphically, this means (as on the right in Figure \ref{ftwo}) that the only `returning' arrows are loops.

\begin{figure}
\begin{center}
	~{\xymatrix{1 \ar@/^/[r]  & 2 \ar@/^/[l] \ar@/^/[d] \\
	3 \ar@/^/[r] & 4 \ar@/^/[l] \ar@/^/[u] }\qquad\qquad}{\xymatrix{1 \ar@(ul,dl)[lr]  & 2 \ar[l] \\ 3 \ar[ur]
		\ar[u] & 4 \ar@(dl,dr)[lr] \ar[l] \ar[ul] \ar[u] 
		}}
\end{center}	\caption{Symmetry\label{ftwo}}
\end{figure}

We need to clearly distinguish non-reflexivity from irreflexivity: the former is just the failure of reflexivity, i.e., at least one object in the domain doesn't bear $R$ to itself, while the latter involves every object in the domain not bearing $R$ to itself. Likewise, we need to distinguish non-transitivity from intransitivity, and non-symmetry from asymmetry \emph{and} antisymmetry. 

The observant reader will have noticed that, while we defined \emph{reflexive on $D$}, \emph{symmetric on $D$}, etc., the only definition which actually mentions $D$ is the definition of reflexivity. All the other definitions are \emph{conditional} in form: they say, if certain pairs are in the relation, then certain other pairs will be too. We needn't specify a domain to check whether these conditionals hold of any relation. But to check whether a relation is reflexive, we need not only the ordered pairs of the relation, but also what the domain is, so we can see if any members of the domain are missing from the relation. (Reflexivity is an extrinsic property of a relation, relative to a domain; the others are intrinsic properties.)


\paragraph{Equivalence Relations}

\begin{definition}[Equivalence Relation]
	If $R$ on $D_{\mathscr{A}}$ is reflexive, transitive, and symmetric, then it is an \emph{equivalence relation}. 
\end{definition}

An equivalence relation divides up the domain into \emph{equivalence classes}, every member of which bears the relation to every other member of it. The relation `is the same height as' is an equivalence relation on the domain of all people; we say that it \emph{induces a partition} on the domain of people, sorting them into groups which are uniform with respect to height.

The most obvious equivalence relation is the identity relation; it is reflexive by definition, and moreover symmetric and transitive; since there are \emph{never} arrows from any $x$ to any distinct $y$, it follows that every time there are such arrows, there are return arrows and shortcuts. (Notice that the identity relation is both symmetric and anti-symmetric.)

Any relation on the empty set is also trivially an equivalence relation. While every empty relation is trivially symmetric and transitive, it will generally fail to be reflexive unless the domain is empty too.

\paragraph{Connectedness}

First, we define a (directed) path: \begin{definition}[Path]
	There is a \emph{path} from $x$ to $y$ in $R$ iff there exists a sequence $z_{1},\ldots,z_{n}$ such that $\langle x,z_{1}\rangle \in R$ and
	$\langle z_{1},z_{2}\rangle \in R$ and \ldots and $\langle z_{n},y\rangle \in R$.
\end{definition}
Now we may define: \begin{definition}[Strong Connectedness]
 	$R$ on $D$ is \emph{connected} iff for any $x \in D$ and $y \in D$, such that $x$ and $y$ are distinct, then there is a path from $x$ to $y$ in $R$. 
 \end{definition}
 There is another property of binary relations, sometimes called \emph{weak
connectedness}, which holds iff for any nodes on the graph of $R$, $x$ and $y$, there is
some path that joins $x$ to $y$ moving only along arrows (perhaps in
the `backwards' direction). For any binary relation $S$, let $S^{*}$ be the relation defined by $\langle x,y\rangle \in S^{*}$ iff $\langle x,y\rangle \in S$ or $\langle y,x\rangle \in S$. \begin{definition}[Weak Connectedness]
	A relation $R$ is weakly connected on $D$ iff  $R^{*}$ is connected on $D$.
\end{definition} 

There is an even stronger condition than connectedness: \begin{definition}[Totality]
  A relation $R$ on $D$ is \emph{total} iff for any $x$ and $y$, either $\langle x,y\rangle \in R$ or $\langle y,x\rangle \in R$.
\end{definition}\begin{theorem}\label{tot}
  If $R$ is total, then $R$ is reflexive.
  \begin{proof}
     If $R$ is total, then for any $x$ and $y$, either $\langle x,y\rangle \in R$ or $\langle y,x\rangle \in R$. Suppose $x=y$; then either either $\langle x,x\rangle \in R$ or $\langle x,x\rangle \in R$. Therefore, $\langle x,x\rangle \in R$, i.e., $R$ is reflexive.
   \end{proof} 
\end{theorem}

\paragraph{Inverse and Complement Relations}

Given a relation $R$ on $D$, we can define two interesting further relations: \begin{definition}[Inverse]
Given a relation $R$, its \emph{inverse} $R^{-1}$ is defined to be this set: $\left\{\langle y,x\rangle:\langle x,y\rangle \in R\right\}$. \end{definition}
\begin{definition}[Complement]
Given a relation $R$ on $D$, its \emph{complement} $R'$ on $D$ is defined to be this set: $D \times D \setminus R$ (i.e., the set of all ordered pairs of elements of the domain \emph{not} in $R$.)
 \end{definition} Note that complement is always with respect to the underlying domain, whereas we can define an inverse relation just with reference to the original relation.

\begin{theorem}[Property Preservation]
	If $R$ is reflexive then $R^{-1}$ is reflexive and $R'$ is irreflexive; if $R$ is symmetric then $R^{-1}$ and $R'$ are symmetric; if $R$ is transitive then $R^{-1}$ is transitive; if $R$ is connected then $R^{-1}$ is connected.
\begin{proof}
	I prove one case: reflexivity. If $R$ is reflexive, then for all $x \in D_{\mathscr{A}}$, $\langle x,x\rangle \in R$. By definition of inverse, then, all $\langle x,x\rangle \in R^{-1}$; i.e., $R^{-1}$ is also reflexive.  By definition of complement, \emph{no} $\langle x,x\rangle \in R'$; so $R'$ is irreflexive.
\end{proof}\end{theorem}



\paragraph{Orderings}
\begin{definition}[Ordering]
	A relation $R$ on $D_{\mathscr{A}}$ is a \begin{description}
		\item [Partial order] iff it is reflexive, transitive, and anti-symmetric (e.g., $\subseteq$ on a set of sets; see figure \ref{fthree});
		\item [Strict Partial Order] iff it is transitive and asymmetric (i.e., $\subset$);
		\item [Total Order] iff it is a total partial ordering (i.e., $\leqslant$ on $\mathbb{N}$). Also sometimes known as a \emph{linear order};
		\item [Strict Total Order] iff it is a total strict partial order (i.e., $<$).
	\end{description} 
\end{definition}

\begin{figure}
	\centering ~{\xymatrix{ \{A,B\} \ar@(ul,dl)[]_{\subseteq}   &
	\{A\}\ar[l]^{\subseteq} \ar@(dr,ru)[]_{\subseteq}\\
	 \{B\}
	\ar[u]_{\subseteq} \ar@(ul,dl)[]_{\subseteq}
	& \emptyset \ar@(dr,ru)[]_{\subseteq} \ar[u]^{\subseteq} \ar[l]_{\subseteq}
	\ar[ul]_{\subseteq} 
	}
	}\caption{Partial ordering of a set of sets by $\subseteq$\label{fthree}}
\end{figure}

As defined above, a total order is reflexive, transitive, antisymmetric, and total. By Theorem \ref{tot}, a total order is reflexive. So we could have defined a total order as one that is total, transitive, and antisymmetric.

\paragraph{Well-Orderings}

Suppose $R$ is a total order on $D_{\mathscr{A}}$. If $\langle x,y\rangle\in R$, then we say $x$ \emph{precedes} $y$. If $\langle x,y\rangle \in R$ and $x\neq y$ and there does not exist a $z$ such that $\langle x,z\rangle \in R$ and $\langle z,y\rangle \in R$, then $x$ \emph{immediately precedes} $y$. (We can give exactly similar definitions for \emph{succeeds}.)

An element $x$ of $D_{\mathscr{A}}$ is \emph{minimal} iff there is no distinct $y$ that precedes $x$. An element $x$ is \emph{least} if it precedes \emph{every distinct} $y \in D_{\mathscr{A}}$. (Corresponding definitions can be given for maximality and greatest.)

\begin{definition}[Well-ordering]
  $R$ is a well-ordering of $D$ iff it is total and every non-empty subset $d$ of $D$ has a least member under $R$.
\end{definition} Example: $\leqslant$ is a well-ordering of the natural numbers $\mathbb{N}$. Here is another well-ordering of $\mathbb{N}$: \begin{equation}
  x \preccurlyeq y \quad\text{ iff }\quad \begin{cases}
    \text{$x$ is even and $y$ is odd; or}\\
    \text{$x$ and $y$ are even and $x\leqslant y$; or}\\
    \text{$x$ and $y$ are odd and $x\leqslant y$}.\\
  \end{cases}
\end{equation} The order induced by $\preccurlyeq$ yields $0, 2, 4, \ldots, 1, 3, 5, \ldots$. In this ordering, every subset of $\mathbb{N}$ has a $\preccurlyeq$-least element. Notice, however, that $1$ is not least, but it has no immediate predecessor. 

While $\leqslant$ well-orders $\mathbb{N}$, it does not well-order the positive and negative integers $\mathbb{Z}$. Here is a well-ordering of $\mathbb{Z}$: $x \leqslant_{\text{abs}} y$ iff either (i) $|x|\leqslant |y|$ or (ii) $|x|=|y|$ and $x\leqslant y$, where `$|x|$' here denotes the absolute value of $x$.  This induces the order $0,-1,1,-2,2,\ldots$.


 
 \paragraph{Expressing Properties of Binary Relations in \ltwo$_{=}$}
 
 We can use formulae of \ltwo\ to express various properties of relations on an entire domain. For example \begin{description}
 	\item [Reflexivity] A relation $R$ is reflexive on $D_{\mathscr{A}}$ iff $R = \val{P^{2}}{A}$ and\\ $\val{\forall x P^{2}xx}{A}=T$. 
\item [Transitivity] A relation $R$ is transitive on $D_{\mathscr{A}}$ iff $R = \val{P^{2}}{A}$ and $$\val{\forall x \forall y \forall z (P^{2}xy \wedge P^{2}yz) \to P^{2}(xz)}{A}=T.$$ 
\item [Symmetry] A relation $R$ is symmetric on $D_{\mathscr{A}}$ iff $R = \val{P^{2}}{A}$ and\\ $\val{\forall x \forall y  (P^{2}xy \to P^{2}yz}{A}=T$. 
 \end{description}

\paragraph{Relations in English and \ltwo: Intension and Extension}

Is the semantic value of an English predicate expression a property or relation as those are interpreted in \ltwo-structures?

The property `is a person' on the domain of Oxford students is the same property as `is a student'; but even though they coincide in their members, we do not think these predicates are synonymous. We may distinguish the property given \begin{itemize}\item \emph{in extension}, by the list of things which satisfy it; or \item \emph{in intension}, by a rule.\end{itemize} On this domain, two different rules give the same extension; and intuitively in English the meaning is the property given intensionally.

\paragraph{Relations in English and \ltwo: Sparse and Abundant Properties}


 \emph{Any} set of pairs is a binary relation. Yet this means that `is identical to' is just as legitimate a relation as `is within 21 metres of'; whereas we think the former is a more fundamental and non-accidental relation than the latter. We may think that the `real' properties are sparse and fundamental, and the abundant predicates like `grue' and `within 21 metres of' are less so, and do not correspond to genuine properties – but \ltwo\ hardly allows us to draw that distinction.

So despite the fact that `grue' and `green' are both meaningful predicates of English, we want to give quite different things to be their semantic values – perhaps just a set of things for `grue', but a real property \textbf{greenness} to be the semantic value of `green'.

Despite these problems, the case against English having a similar semantics for predicates to \ltwo\ is rather weaker than that against English having a similar semantics for designators.


{\small
\subsection*{Further Reading}
\addcontentsline{toc}{subsection}{Further Reading}


\citet{bevpospa} and \citet{burphilo}, cited in the last chapter, provide useful material on relevant and conditional logics. On the topic of the English conditional, both indicative and counterfactual, a good guide is \citet{benphiguc}. In addition to the work of Grice cited above, \citet{sep-implicature} is a useful source. The best defence of dialethism is \citet{priinco}; the application of relevant logic to inconsistent data sets is \citet{belusefov}.  

A wonderful book on names and description is \citet{krinamne}. A modern classic: amazingly, it is more or less a transcription of a series of lectures originally delivered without notes. A useful source on empty names is \citet{capempna}. The topic of empty names is one main topic of another famous series of lectures by \citet{krirefex}.



\subsection*{Exercises}
\addcontentsline{toc}{subsection}{Exercises}


\begin{enumerate}
	
	\item 	\begin{enumerate} \item Show that the following rules of inference are acceptable in our framework: \begin{enumerate}
		\item \emph{Modus Ponens}: If $\vDash \phi$ and $\vDash \phi \to \psi$, then $\vDash \psi$.
				\item \emph{Disjunctive Syllogism}: If $\vDash \phi \vee \psi$ and $\vDash \neg \phi$, then $\vDash \psi$.
				\item \emph{Reductio ad Absurdum}: If $\phi \vDash \psi$ and $\phi \vDash \neg \psi$ then $\vDash \neg \phi$. \end{enumerate}
	\item
		 In our system each acceptable rule of inference has a corresponding correct semantic sequent (for example, the sequent $\phi, (\phi \to \psi) \vDash \psi$ corresponds to  \emph{modus ponens}). Consider now this rule of inference in English: if $\phi$ is a truth of logic, then `Necessarily, $\phi$' is too.  Is this rule intuitively correct? Is the corresponding sequent `$\phi \vDash_{\text{English}} \text{Necessarily } \phi$' intuitively correct?
		\end{enumerate}

		\item \begin {enumerate} \item John argues that, since $\phi$ and $\psi$ together entail (in English)
		$\psi$, if follows that $\psi$ entails `if $\phi$, $\psi$'. Evaluate John's argument.

		\item Mary claims that `if $\phi$, $\psi$' entails `if $\neg \psi$, $\neg
		\phi$'. Evaluate Mary's claim.
		\item Show how it would be possible to use John's conclusion and
		Mary's claim to argue that `$\phi \to \psi$' entails `if $\phi$,
		$\psi$'.
		\item Give an argument that `if $\phi$,$\psi$' is a truth-functional connective in English. Do you see any difficulties with your argument?

		\end{enumerate}	
	
\item Say that $\phi \vDash \psi$ is a \emph{perfect} sequent iff it is correct, and $\phi$ is satisfiable, and	$\psi$ is not a tautology. Say that a sequent is \emph{perfectible} iff it is a substitution instance of a perfect sequent. \begin{enumerate}
	\item Show that there must be a sentence letter in common between $\phi$ and $\psi$ is a perfectible sequent. (Hint: use the Craig Interpolation Theorem (page \pageref{thmcraig}), plus properties of substitution.)
	\item Give examples to show that perfectible entailment is not transitive (i.e., that there are cases where $\phi$ perfectibly entails $\psi$, where $\psi$ perfectibly entails $\chi$, but where $\phi$ does not perfectibly entail $\chi$.)
	\item Which of the structural rules of \S\ref{twostruct} (page \pageref{twostruct}) are perfectibly inappropriate (i.e., take perfectible sequents to non-perfectible sequents.)?
\end{enumerate}
	
	
\item Which of the relations expressed by the following English predicates are equivalence relations: \begin{enumerate}
	\item `$x$ and $y$ attend the same lectures', on the domain of Oxford students.
	\item `$x$ is studying the same subject as $y$', on the domain of Oxford students.
	\item `$x$ entails $y$', on the domain of sentences of \ltwo.
	\item `$x$ is logically equivalent to $y$', on the domain of sentences of \ltwo.
	\item `If $\{x\}$ is consistent, then $\{x\}\cup \{y\}$ is consistent' on the domain of sentences of \ltwo.
\end{enumerate}
\item Prove the following: \begin{enumerate}
\item If $R$ is irreflexive, then $R'$ is reflexive.
	\item If $R$ is symmetric, then $R^{-1}$ is symmetric.
	\item If $R$ is symmetric, then $R'$ is symmetric.
	\item If $R$ is transitive, then $R^{-1}$ is transitive.
	\item If $R$ is connected, then $R^{-1}$ is connected.
	\item If $R$ is asymmetric and non-empty, then $R'$ is non-symmetric.
\end{enumerate}
\item \begin{enumerate}
	\item If $R$ is transitive, is $R'$ transitive?
	\item If $R$ is connected, is $R'$ connected?
	\item If $R$ is antisymmetric, is $R'$ symmetric?
	\item If $R$ is transitive and non-empty, is $R'$ is transitive?
\end{enumerate}
\item If $R$ is defined on the empty domain, 
    can it be reflexive? Can it be irreflexive? What about
    transitivity and symmetry?
\item What is wrong with the following argument that reflexivity is a consequence of symmetry and transitivity? \begin{quote}
	If $\langle x,y\rangle \in R$, then $\langle y,x\rangle\in R$ since we assume $R$ is symmetric. If both $\langle x,y\rangle\in R$ and $\langle y,x\rangle \in R$, then since $R$ is transitive, $\langle x,x\rangle\in R$ – so $R$ is reflexive.
\end{quote} (After \citealt[p.\ 52]{pmwmatmel}.)
\item \begin{enumerate}
	\item Can a relation be asymmetric and reflexive?
	\item Can a relation 
	    be transitive, non-symmetric and irreflexive?
	\item 	Can a relation 
		    be connected and irreflexive?
\end{enumerate}  
\item \begin{enumerate}
	\item Show that $\forall x \forall y (\neg x=y\to (Pxy \vee Pyx) \equiv \forall x \forall y (x=y \vee Pxy \vee Pyx)$. 
	\item A relation $R$ satisfies \emph{trichotomy} iff, for all $x$ and $y$, at most one of these three holds: $Rxy$, $Ryx$, or $x=y$. Is every trichotomous relation connected? Under what circumstances is a connected relation trichotomous?
\end{enumerate} 
\item
 Show by suitable reasoning that
    in a finite domain, for any \emph{partial order} $R$,  $\exists x \forall y (Ryx \to
    x=y)$. Give a counterexample to this condition in an infinite domain. 

\item Show that a well-ordering of $D$ is a total ordering
    of $D$, but not \emph{vice versa}.
\item Give a graph, on some non-empty domain, of a relation $R$
	which satisfies this condition:
	$\forall x \forall y (Rxy \to \exists z ((x \neq z) \wedge (y \neq
	z) \wedge Rxz))$. Can you give an example of a relation which
	satisfies this condition (being sure to specify the domain)?
	
	



\item Prove that \begin{enumerate}
	\item If $x$ is a least member under an ordering $R$, then it is the unique least member.
	\item The set of natural numbers $\mathbb{N}$ is well-ordered by $<$.
	\item The set of (positive and negative) integers $\mathbb{Z}$ is \emph{not} well-ordered by $<$.
	\item There exists orderings with no maximal elements.
\end{enumerate}

\item A relation $R$ is \emph{dense} iff whenever $\langle x,y\rangle \in R$, there exists a $z$ such that $\langle x,z\rangle \in R$ and $\langle z,y\rangle \in R$. Prove that on the domain of the natural numbers $\mathbb{N}$, the greater-than relation $>$ is not dense; but that on the domain of the positive rationals (i.e., numbers of the form $\frac{n}{m}$ where $n,m \in \mathbb{N}$), it is dense.

\item Let $D = \{1,2,3,5,6,10,15,30\}$. Let $R$ be the relation on $D$ defined by \begin{equation*}
	R = \{\langle x,y\rangle: x\text{ divides $y$ without remainder}\}.
\end{equation*}\begin{enumerate}
	\item Show that $R$ is a weak partial order but not a total order.
	\item Draw the graph of $R$, and identify any minimal, maximal, least or greatest elements.
	\item Do the same for the set $\wp \{a,b,c\}$, using the relation $\subseteq$ on that domain.
\end{enumerate}

\item	We might normally expect `is similar to' to be a symmetric relation: after all, if there is a respect in which $a$ is similar to $b$, then $b$ must be similar to $a$ in that very same respect. But many people seem to judge that similarity is \emph{not} symmmetric: \begin{quote}
		When people are asked to make comparisons between a highly familiar object and a less familiar one, their responses reveal a systematic asymmetry: The unfamiliar object is judged as more similar to the familiar one than vice versa. For example, people who know more about the USA than about Mexico judge Mexico to be more similar to the USA than the USA is to Mexico. \citep[520]{kunsocco}
	\end{quote} (You might think also of the fact that it is much more natural to say that children resemble their parents, than that parents resemble their children.)

	Can you provide a rationale for these psychological results? Do they indicate that people are systematically mistaken about the meaning of the relational predicate `is similar to', or do they indicate that our theory of similarity in terms of matching respects of similarity is incorrect?

\end{enumerate}

}



	











\appendix

\chapter[Alternative Proof of Compactness]{An Alternative Proof of Compactness}\label{altcomp}
\paragraph{$n$-Acceptable structures}
Suppose that $\Gamma$ is a finitely satisfiable set of sentences, and $P_{1},P_{2},\ldots$ is an enumeration of the sentence letters occurring as subsentences of sentences in $\Gamma$. Let $\mathbf{S}_{n}$ be the set consisting of the first $n$ sentence letters in $\Gamma$ under the enumeration, i.e., $\mathbf{S}_{n} = \{P_{1},P_{2},\ldots,P_{n}\}$. ($\mathbf{S}_{0}$ is the empty set.)
    
     If $S$ is a set of \lone\ sentences, we say that a structure $\mathscr{C}$ is an \emph{$S$-agreeing variant} of a structure $\mathscr{B}$ iff $|\phi|_{\mathscr{C}}=|\phi|_{\mathscr{B}}$ for all $\phi \in S$.
    Call an \lone-structure $\mathscr{B}$ \emph{$n$-acceptable} iff, for each finite subset $\Delta \subseteq \Gamma$, $\mathscr{B}$ has a $\mathbf{S}_{n}$-agreeing variant $\mathscr{C}$ which satisfies $\Delta$.


Define a series of \lone-structures as follows: \begin{itemize}
      \item $\mathscr{A}_{0}$ is some arbitrarily chosen structure.
      \item If $\mathscr{B}$ is an $\mathbf{S}_{n}$-agreeing variant of $\mathscr{A}_{n}$ which assigns $T$ to $P_{n+1}$ and $\mathscr{B}$ is $(n+1)$-acceptable, let $\mathscr{A}_{n+1}$ be $\mathscr{B}$; otherwise let $\mathscr{A}_{n+1}$ be an $\mathbf{S}_{n}$-agreeing variant of $\mathscr{A}_{n}$ which assigns $F$ to $P_{n+1}$.
    \end{itemize} 
    
  \paragraph{Each $\mathscr{A}_{n}$ is $n$-acceptable}  
    
We now prove by induction on $n$ that each $\mathscr{A}_{n}$ is $n$-acceptable (i.e, $\mathscr{A}_{0}$ is $0$-acceptable, $\mathscr{A}_{1}$ is $1$-acceptable, and so on). 

Obviously $\mathscr{A}_{0}$ is $0$-acceptable; to be $0$-acceptable is for each finite subset of $\Gamma$ to agree with $\mathscr{A}_{0}$ on the set of sentence letters in $\mathbf{S}_{0}$; since the latter is the empty set, \emph{any} finite subset of $\Gamma$ agrees with $\mathscr{A}_{0}$ on that set.

For the induction step, suppose there is some $\mathscr{A}_{n}$ which is $n$-acceptable, but $\mathscr{A}_{n+1}$ is not. Since by construction, if there was an $(n+1)$-acceptable variant of $\mathscr{A}_{n}$ which assigned $T$ to $P_{n+1}$, $\mathscr{A}_{n+1}$ would be such. So $\mathscr{A}_{n+1}$ must assign $F$ to $P_{n+1}$. But from the induction hypothesis, some finite subset $\Delta\subset \Gamma$ is not satisfied by any $\mathbf{S}_{n+1}$-agreeing variant of $\mathscr{A}_{n+1}$; so $\Delta$ cannot be satisfied by \emph{any} $\mathbf{S}_{n}$-agreeing variant of $\mathscr{A}_{n}$ (since it can't be satisfied by a variant which assigns $T$ to $P_{n+1}$, or one that assigns $F$, and they are the only options), contrary to the assumption that $\mathscr{A}_{n}$ is $n$-acceptable. So $\mathscr{A}_{n+1}$ must in fact be $(n+1)$-acceptable.


\paragraph{Compactness Again}

If we let $\mathscr{A}$ be a structure such that, for all $i$, $|P_{i}|_{\mathscr{A}}=|P_{i}|_{\mathscr{A}_{i}}$ (that is, for all $i$, $\mathscr{A}$ agrees with the $i$-th structure $\mathscr{A}_{i}$ on the value it assigns to the $i$-th sentence letter $P_{i}$). 

Suppose $\mathscr{A}$ isn't a model for $\Gamma$. Then there is a $\phi \in \Gamma$ such that $|\phi|_{\mathscr{A}}=F$. Let $P_{k}$ be the highest numbered sentence letter occurring in $\phi$. Then $\mathscr{A}_{k}$ must make $\phi$ true, since $\{\phi\}$ is a finite subset of $\Gamma$ and $\mathscr{A}_{k}$ is $k$-acceptable. But then every $\mathbf{S}_{k}$-agreeing variant of $\mathscr{A}_{k}$ makes $\phi$ true; and $\mathscr{A}$ is one such. Contradiction – $\mathscr{A}$ must satisfy $\Gamma$.

But since $\Gamma$ was arbitrary, we've shown that if $\Gamma$ is finitely satisfiable, then $\Gamma$ is satisfiable. This is compactness.





\chapter{Greek letters}\label{greek}

\begin{table}[h] \begin{center}
	\begin{tabular}{llcll}
		\toprule
		$\alpha, A$ & Alpha && $\nu, N$ & Nu\\
		$\beta, B$ & Beta && $\xi, \Xi$ & Xi\\
		$\gamma, \Gamma$ & Gamma && $\omicron, O$ & Omicron\\
		$\delta, \Delta$ & Delta && $\pi, \Pi$ & Pi\\
		$\epsilon, E$ & Epsilon && $\rho, P$& Rho\\
		$\zeta, Z$ & Zeta && $\sigma, \Sigma$ & Sigma\\
		$\eta, H$ & Eta && $\tau, T$ & Tau\\
		$\theta, \Theta$ & Theta && $\upsilon, \Upsilon$ & Upsilon\\
		$\iota, I$ & Iota && $\phi, \Phi$ & Phi\\
		$\kappa, K$ & Kappa && $\chi, X$ & Chi\\
		$\lambda, \Lambda$ & Lambda && $\psi, \Psi$ & Psi\\
		$\mu, M$ & Mu && $\omega, \Omega$ & Omega \\ 
    \bottomrule
	\end{tabular}
\end{center} \caption{Table of Greek letters}
\end{table}

\twocolumn
\chapter{List of Definitions}\label{defns}
{\small  \listtheorems{definition}}



\newpage
\markright{bibliography}



{\small \addcontentsline{toc}{chapter}{Bibliography}\label{biblio}\markright{bibliography}
\begin{thebibliography}{39}
\newcommand{\enquote}[1]{‘#1’}
\expandafter\ifx\csname natexlab\endcsname\relax\def\natexlab#1{#1}\fi
\expandafter\ifx\csname url\endcsname\relax
  \def\url#1{{\tt #1}}\fi
\expandafter\ifx\csname urlprefix\endcsname\relax\def\urlprefix{\textsc{url} }\fi

\bibitem[{Beall and van Fraassen(2003)}]{bevpospa}
Beall, JC and van Fraassen, Bas~C (2003), \emph{Possibilities and Paradox}.
  Oxford: Oxford University Press.

\bibitem[Beall and Restall(2013)]{brlc} 
---\!\!---\!\!---\!\!--- and Restall, Greg, \enquote{Logical Consequence}.  In Edward~N Zalta (ed.), \emph{The Stanford Encyclopedia of Philosophy}. \urlprefix\href{http://plato.stanford.edu/entries/logical-consequence/}{\nolinkurl{plato.stanford.edu/entries/logical-consequence/}}.

\bibitem[{{Belnap}(1977)}]{belusefov}
{Belnap, Jr.}, Nuel~D (1977), \enquote{A Useful Four-Valued Logic}. In
 ~J~Michael Dunn and~G~Epstein (eds.), \emph{Modern Uses of Multiple-Valued
  Logic}, Dordrecht: D~Reidel, pp.~8–37.

\bibitem[{Bennett(2003)}]{benphiguc}
Bennett, Jonathan (2003), \emph{A Philosophical Guide to Conditionals}.
  Oxford: Oxford University Press.

\bibitem[{Boolos \emph{et~al.\/}(2007)Boolos, Burgess and Jeffrey}]{bbjcomlo}
Boolos, George~S, Burgess, John~P, and Jeffrey, Richard~C (2007), \emph{
  Computability and Logic}. Cambridge: Cambridge University Press, 5 ed.

\bibitem[Boolos(1971)]{boolos} Boolos, George (1971) ‘The Iterative Conception of Set’, \emph{Journal of Philosophy} \textbf{68}: 215–232.


\bibitem[{Borkowski and S{\l}upecki(1958)}]{boslogwol}
Borkowski,~L and S{\l}upecki,~J (1958), \enquote{The Logical Works of J~{\L}ukasiewicz}. \emph{Studia Logica}, vol.~8: pp.~7–56.

\bibitem[{Bostock(1997)}]{bosintlo}
Bostock, David (1997), \emph{Intermediate Logic}. Oxford: Oxford University
  Press.

\bibitem[{Burgess(2009)}]{burphilo}
Burgess, John~P (2009), \emph{Philosophical Logic}. Princeton: Princeton
  University Press.

\bibitem[{Caplan(2006)}]{capempna}
Caplan, Ben (2006), \enquote{Empty names}. In Robert Stainton and Alex Barber
  (eds.), \emph{The Encyclopedia of Language and Linguistics: Philosophy and
  Language}, vol.~4, Oxford: Elsevier, pp.~132–136.

\bibitem[{Craig(1957)}]{crathrush}
Craig, William (1957), \enquote{Three uses of the Herbrand-Gentzen theorem in
  relating model theory and proof theory}. \emph{Journal of Symbolic Logic},
  vol.~22: pp.~269–85.

\bibitem[{Davis(2014)}]{sep-implicature}
Davis, Wayne (2014), \enquote{Implicature}. In Edward~N Zalta (ed.), \emph{The Stanford Encyclopedia of Philosophy}. \urlprefix\href{http://plato.stanford.edu/entries/implicature/}{\nolinkurl{plato.stanford.edu/entries/implicature/}}.

\bibitem[{Dummett(1983)}]{dumphibai}
Dummett, Michael (1983), \enquote{The Philosophical Basis of Intuitionist
  Logic}. In Paul Benacerraf and Hilary Putnam (eds.), \emph{Philosophy of
  Mathematics: Selected Readings}, Cambridge: Cambridge University Press, 2
  ed., pp.~97–129.

\bibitem[{Gentzen(1969)}]{geninvinl}
Gentzen, Gerhard (1969), \enquote{Investigations into Logical Deduction}. In
 ~M~E Szabo (ed.), \emph{The Collected Papers of Gerhard Gentzen},
  Amsterdam: North-Holland, pp.~68–131.

\bibitem[{Gowers(2008)}]{fundta}
Gowers, Timothy (ed.) (2008), \emph{The Princeton Companion to Mathematics}. Princeton, NJ: Princeton University Press.

\bibitem[{Grice(1989)}]{grilogco}
Grice, Paul (1989), \enquote{Logic and Conversation}. In \emph{Studies in the
  Way of Words}, Cambridge, MA: Harvard University Press.

\bibitem[{Halbach(2010)}]{hallogma}
Halbach, Volker (2010), \emph{The Logic Manual}. Oxford: Oxford University
  Press.

\bibitem[{Harris(1982)}]{harwhasol}
Harris, John~H (1982), \enquote{What's so logical about the `logical' axioms?}
  \emph{Studia Logica}, vol.~41: pp.~159–71.

\bibitem[{Henkin(1949)}]{hencomfio}
Henkin, Leon (1949), \enquote{The Completeness of the First-Order Functional
  Calculus}. \emph{Journal of Symbolic Logic}, vol.~14: pp.~159–66.

\bibitem[{Henle, Garfield, and Tymoczko}(2011)]{sweetreas}
Henle, James~M, Garfield, Jay~L, and Tymoczko, Thomas (2011), \emph{Sweet Reason}. Chichester: Wiley-Blackwell, 2nd ed.


\bibitem[{Heyting(1956)}]{heyintin}
Heyting,~A (1956), \emph{Intuitionism: An Introduction}. Amsterdam:
  North-Holland.

\bibitem[Iemhoff(2015)]{int} Iemhoff, Rosalie (2015), \enquote{Intuitionism in the Philosophy of Mathematics}. In Edward N~Zalta (ed.), \emph{The Stanford Encyclopedia of Philosophy}. \href{http://plato.stanford.edu/entries/logic-intuitionistic/}{\nolinkurl{plato.stanford.edu/entries/logic-intuitionistic/}}.

\bibitem[{Jeffrey(2006)}]{jefforlos}
Jeffrey, Richard~C (2006), \emph{Formal Logic: Its Scope and Limits}. Indianapolis: Hackett, 4th ed, with a supplement by John P Burgess.

\bibitem[{Kripke(1976)}]{kriisthp}
Kripke, Saul (1976), \enquote{Is There a Problem About Substitutional
  Quantification?} In Gareth Evans and John McDowell (eds.), \emph{Truth and
  Meaning}, Oxford: Oxford University Press, pp.~324–419.

\bibitem[{Kripke(1980)}]{krinamne}
---\!\!---\!\!---\!\!--- (1980), \emph{Naming and Necessity}. Cambridge, MA:
  Harvard University Press.

\bibitem[{Kripke(2013)}]{krirefex}
---\!\!---\!\!---\!\!--- (2013), \emph{Reference and Existence}. Oxford:
  Oxford University Press.

\bibitem[{Kunda(1999)}]{kunsocco}
Kunda, Ziva (1999), \emph{Social Cognition: Making Sense of People}.
  Cambridge, MA: MIT Press.

\bibitem[{Lambert(2001)}]{lamfrelo}
Lambert, Karel (2001), \enquote{Free Logics}. In Lou Goble (ed.), \emph{The
  Blackwell Guide to Philosophical Logic}, Oxford: Blackwell.

\bibitem[{Lewis(1973)}]{lewcount}
Lewis, David (1973), \emph{Counterfactuals}. Oxford: Blackwell.

\bibitem[{Lewis(1984)}]{lewputpa}
---\!\!---\!\!---\!\!--- (1984), \enquote{Putnam's Paradox}. \emph{Australasian
  Journal of Philosophy}, vol.~62: pp.~221–36.

\bibitem[{Ludlow(2013)}]{sep-descriptions}
Ludlow, Peter (2013), \enquote{Descriptions}. In Edward~N Zalta (ed.), \emph{The Stanford Encyclopedia of Philosophy}. \urlprefix\href{http://plato.stanford.edu/entries/descriptions/}{\nolinkurl{plato.stanford.edu/entries/descriptions/}}.

\bibitem[{Machover(1996)}]{macsetthl}
Machover, Moshé (1996), \emph{Set Theory, Logic, and Their Limitations}.
  Cambridge: Cambridge University Press.

\bibitem[{McGee(2006)}]{mcgtherue}
McGee, Vann (2006), \enquote{There's a Rule for Everything}. In Agust\'{i}n
  Rayo and Gabriel Uzquiano (eds.), \emph{Absolute Generality}, Oxford: Oxford
  University Press, pp.~179–202.

\bibitem[{Partee \emph{et~al.\/}(1990)Partee, {ter Meulen} and Wall}]{pmwmatmel}
Partee, Barbara~H, {ter Meulen}, Alice and Wall, Robert~E (1990), \emph{Mathematical Methods in Linguistics}. Dordrecht: Kluwer.

\bibitem[{Potter(2004)}]{potsetthi}
Potter, Michael (2004), \emph{Set Theory and Its Philosophy}. Oxford: Oxford
  University Press.

\bibitem[{Prawitz(2006)}]{pranatde}
Prawitz, Dag (2006), \emph{Natural Deduction: A Proof-theoretic study}. New
  York: Dover.

\bibitem[{Priest(2006)}]{priinco}
Priest, Graham (2006), \emph{In Contradiction}. Oxford: Oxford University
  Press, 2nd ed.

\bibitem[{Prior(1960)}]{prirunint}
Prior,~A~N (1960), \enquote{The Runabout Inference-Ticket}. \emph{Analysis},
  vol.~21: pp.~38–9.

\bibitem[{Pullum(2008)}]{pulscolos}
Pullum, Geoffrey~K (2008), \enquote{Scooping the Loop Snooper: a proof that
  the Halting Problem is unsolvable}.
  \urlprefix\href{http://www.ling.ed.ac.uk/~gpullum/loopsnoop.pdf}{\nolinkurl{www.ling.ed.ac.uk/~gpullum/loopsnoop.pdf}}.

\bibitem[{Putnam(1980)}]{putmodre}
Putnam, Hilary (1980), \enquote{Models and Reality}. \emph{Journal of Symbolic
  Logic}, vol.~45: pp.~464–82.

\bibitem[Quine(1940)]{quine} Quine, W~V~O (1940) \emph{Mathematical Logic}. Boston, MA: Harvard University Press.

\bibitem[Restall(2000)]{restsub}
Restall, Greg (2000) \emph{An Introduction to Substructural Logics}. London: Routledge.

\bibitem[Richard(1986)]{richard} Richard, Mark (1986) `Quotation, grammar, and opacity', \emph{Linguistics and Philosophy} \textbf{9}: 383–403.

\bibitem[{Russell(1956)}]{rusonde}
Russell, Bertrand (1956), \enquote{On Denoting}. In \emph{Logic and
  Knowledge}, London and New York: Routledge, pp.~41–56.

\bibitem[Shapiro(2000)]{shap} Shapiro, Stewart (2000) \emph{Thinking About Mathematics}. Oxford: Oxford University Press.

\bibitem[{Shapiro(2013)}]{sep-logic-classical}
---\!\!---\!\!---\!\!---  (2013), \enquote{Classical Logic}. In Edward~N Zalta (ed.),
  \emph{The Stanford Encyclopedia of Philosophy}. \urlprefix\href{http://plato.stanford.edu/entries/logic-classical/}{\nolinkurl{plato.stanford.edu/entries/logic-classical/}}.

\bibitem[{Strawson(1950)}]{stronre}
Strawson,~P~F (1950), \enquote{On Referring}. \emph{Mind}, vol.~59:
  pp.~320–44.

\bibitem[{Tarski(1933)}]{tarcontrf}
Tarski, Alfred (1933), \enquote{The Concept of Truth in Formalized Languages}. 
  In \emph{Logic, Semantics, Metamathematics}, Indianapolis: Hackett, 2nd ed., 1983, pp.
  152–278.

\bibitem[{Tarski(1936)}]{tarski}
---\!\!---\!\!---\!\!--- (1936), \enquote{On the Concept of Logical Consequence}. 
  In \emph{Logic, Semantics, Metamathematics}, Indianapolis: Hackett, 2nd ed., 1983, pp. 409–420.



\bibitem[{Taylor(2006)}]{taymodtrr}
Taylor, Barry (2006), \emph{Models, Truth, and Realism}. Oxford: Oxford
  University Press.

\bibitem[{Turing(1937)}]{turoncon}
Turing,~A~M (1937), \enquote{On Computable Numbers, with an Application to
  the Entscheidungsproblem}. \emph{Proceedings of the London Mathematical
  Society}, vol.~42: pp.~230–65.

\end{thebibliography}
}






	
\end{document}